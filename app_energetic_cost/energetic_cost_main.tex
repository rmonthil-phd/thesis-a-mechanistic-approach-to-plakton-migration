\chapter{The energetic cost of surfing}\label{app:energetic_cost}

It has been shown in Chap.~\ref{chap:bio_discussion}, Sec.~\ref{sec:optimality}, that the metabolic power corresponds to the main source of power consumption in the context of the vertical navigation problem.
Therefore we deduced that the application of the surfing strategy would cause be particularly beneficial due to its low energetic cost.
However (1) this deduction relies on strong assumptions and (2) the surfing strategy can be applied in a broader context (such as the navigation of artificial microswimmers).
Therefore we consider below the case of low metabolic power consumption (relative to the power consumption of the active power): $P_{\mathrm{meta.}} = 0$.

\section{The cost of active reorientation}\label{sec:energetic_cost}

Considering the interesting limit for which $P_{\mathrm{swim}} \gg P_{\mathrm{meta}}$ and $P_{\mathrm{turn}} \gg P_{\mathrm{meta}}$, we evaluate numerically the value of $\left\langle \Performance \right\rangle$ and $\left\langle \SwimmingAngularVelocity^2 \right\rangle$ from our numerical simulations for any surfing parameter $\TimeHorizon$.
Figure \ref{fig:energy_efficiency_time_horizon}\textbf{(a)} shows that $P_{\mathrm{turn}}$ evolves as $\TimeHorizon^2$.
\begin{figure}%[H]
	\centering
	\begin{tikzpicture}
	\node[anchor=center] at (2.9,4.7) {$\SwimmingVelocity =$};
	\begin{groupplot}[
   		group style={
   			group size=2 by 1,
   			%y descriptions at=edge left,
   			%x descriptions at=edge bottom,
   			horizontal sep=0.14\linewidth,
   			%vertical sep=0.04\linewidth,
   		},
   		% size
   		width=0.45\textwidth,
		% x
		xlabel=$\TimeHorizon / \KolmogorovTimeScale$,
		xmin=0,
		xmax=6,
		% layers
		set layers,
		% legend
		legend style={draw=none, fill=none, /tikz/every even column/.append style={column sep=4pt}, at={(1.0, 1.05)}, anchor=south},
		%legend pos=north west,
   		legend cell align=left,
   		legend columns=-1,
   	]
	
	\nextgroupplot[
		axis on top,
		% y
		ylabel={$\left\langle \varOmega_{\mathrm{rel.}}^2 \right\rangle \, \KolmogorovTimeScale^2 / (1 + \SwimmingVelocity/\KolmogorovVelocityScale)$},
		ymin=0.1,
		ymax=100,
		ymode=log
	]
		\node[anchor=north west] at (axis cs:0.0,100) {\textbf{(a)}};
		%% us 1.0
		%%% 95 CI
		\addplot[name path=A, draw=none, forget plot] table [
			x index=4,
			y expr={(\thisrowno{1} - \thisrowno{2}) * 0.0424^2}, %tau_\eta = 0.0424
			col sep=comma,
			comment chars=\#,
			restrict expr to domain={\thisrowno{3}}{1.0:1.0},
			unbounded coords=discard,
		]{data/surfers/surfer__merge_rotating_power_consumption.csv};
		\addplot[name path=B, draw=none, forget plot] table [
			x index=4,
			y expr={(\thisrowno{1} + \thisrowno{2}) * 0.0424^2}, %u_\eta = 0.0424
			col sep=comma,
			comment chars=\#,
			restrict expr to domain={\thisrowno{3}}{1.0:1.0},
			unbounded coords=discard,
		]{data/surfers/surfer__merge_rotating_power_consumption.csv};
		\addplot[ColorSurf!66!ColorVs, opacity=0.25, forget plot, on layer=axis background] fill between[of=A and B];
		%%% average
		\addplot
		[
		color=ColorSurf!66!ColorVs,
		opacity=1.0,
		only marks,%solid
		mark=square
		]
		table[
			x index=4,
			y expr={\thisrowno{1} * 0.0424^2}, %u_\eta = 0.0424
			col sep=comma,
			comment chars=\#,
			restrict expr to domain={\thisrowno{3}}{1.0:1.0},
			unbounded coords=discard,
		]{data/surfers/surfer__merge_rotating_power_consumption.csv};
		\addlegendentry{$\KolmogorovVelocityScale$}
		%% us 4.0
		%%% 95 CI
		\addplot[name path=A, draw=none, forget plot] table [
			x index=4,
			y expr={(\thisrowno{1} - \thisrowno{2}) * 0.0424^2}, %tau_\eta = 0.0424
			col sep=comma,
			comment chars=\#,
			restrict expr to domain={\thisrowno{3}}{4.0:4.0},
			unbounded coords=discard,
		]{data/surfers/surfer__merge_rotating_power_consumption.csv};
		\addplot[name path=B, draw=none, forget plot] table [
			x index=4,
			y expr={(\thisrowno{1} + \thisrowno{2}) * 0.0424^2}, %u_\eta = 0.0424
			col sep=comma,
			comment chars=\#,
			restrict expr to domain={\thisrowno{3}}{4.0:4.0},
			unbounded coords=discard,
		]{data/surfers/surfer__merge_rotating_power_consumption.csv};
		\addplot[ColorSurf!33!ColorVs, opacity=0.25, forget plot, on layer=axis background] fill between[of=A and B];
		%%% average
		\addplot
		[
		color=ColorSurf!33!ColorVs,
		opacity=1.0,
		only marks,%solid
		mark=pentagon*
		]
		table[
			x index=4,
			y expr={\thisrowno{1} * 0.0424^2}, %u_\eta = 0.0424
			col sep=comma,
			comment chars=\#,
			restrict expr to domain={\thisrowno{3}}{4.0:4.0},
			unbounded coords=discard,
		]{data/surfers/surfer__merge_rotating_power_consumption.csv};
		\addlegendentry{$4\KolmogorovVelocityScale$}
		%% us 8.0
		%%% 95 CI
		\addplot[name path=A, draw=none, forget plot] table [
			x index=4,
			y expr={(\thisrowno{1} - \thisrowno{2}) * 0.0424^2}, %tau_\eta = 0.0424
			col sep=comma,
			comment chars=\#,
			restrict expr to domain={\thisrowno{3}}{8.0:8.0},
			unbounded coords=discard,
		]{data/surfers/surfer__merge_rotating_power_consumption.csv};
		\addplot[name path=B, draw=none, forget plot] table [
			x index=4,
			y expr={(\thisrowno{1} + \thisrowno{2}) * 0.0424^2}, %u_\eta = 0.0424
			col sep=comma,
			comment chars=\#,
			restrict expr to domain={\thisrowno{3}}{8.0:8.0},
			unbounded coords=discard,
		]{data/surfers/surfer__merge_rotating_power_consumption.csv};
		\addplot[ColorSurf!00!ColorVs, opacity=0.25, forget plot, on layer=axis background] fill between[of=A and B];
		%%% average
		\addplot
		[
		color=ColorSurf!00!ColorVs,
		opacity=1.0,
		only marks,%solid
		mark=o
		]
		table[
			x index=4,
			y expr={\thisrowno{1} * 0.0424^2}, %u_\eta = 0.0424
			col sep=comma,
			comment chars=\#,
			restrict expr to domain={\thisrowno{3}}{8.0:8.0},
			unbounded coords=discard,
		]{data/surfers/surfer__merge_rotating_power_consumption.csv};
		\addlegendentry{$8\KolmogorovVelocityScale$}
		% %% sqrt
		% \addplot
		% [
		% color=black,
		% opacity=1.0,
		% solid,
		% domain=0:8,
		% samples=20,
		% ] {x^2};

	\nextgroupplot[
		axis on top,
		% y
		ymin=0.01,
		ymax=1,
		ymode=log,
		ylabel={$\mu \KolmogorovScale \KolmogorovVelocityScale E$},
	]
		\pgfplotstableread[col sep=comma]{data/surfers/surfer__merge_average_velocity_axis_0.csv}\TableVelocity
		\pgfplotstableread[col sep=comma]{data/surfers/surfer__merge_rotating_power_consumption.csv}\TablePower
		\pgfplotstablegetcolumnnamebyindex{1}\of{\TablePower}\to\TablePowerColumnFour
		\pgfplotstablecreatecol[create col/copy column from table=\TablePower{\TablePowerColumnFour}]{omega_active^2}\TableVelocity
		\pgfplotstablegetcolumnnamebyindex{0}\of{\TableVelocity}\to\TablePowerColumnZero
		\pgfplotstablegetcolumnnamebyindex{1}\of{\TableVelocity}\to\TablePowerColumnOne
		\pgfplotstablegetcolumnnamebyindex{2}\of{\TableVelocity}\to\TablePowerColumnTwo
		\pgfplotstablegetcolumnnamebyindex{3}\of{\TableVelocity}\to\TablePowerColumnThree
		\pgfplotstablegetcolumnnamebyindex{4}\of{\TableVelocity}\to\TablePowerColumnFour
		\pgfplotstablegetcolumnnamebyindex{5}\of{\TableVelocity}\to\TablePowerColumnFive

		\node[anchor=north west] at (axis cs:0.0,1.0) {\textbf{(b)}};
		% us 1.0
		%% 95 CI
		\addplot[name path=A, draw=none, forget plot] table [
			x=\TablePowerColumnFour,
			y expr={(\thisrow{\TablePowerColumnOne} - \thisrow{\TablePowerColumnTwo}) / 0.066 / (9.4 * \thisrow{\TablePowerColumnThree}^2 * 0.5 + 3.1 * \thisrow{\TablePowerColumnFive} * 0.0424^2 * 0.5^3)}, %\eta = 0.00280, u_\eta = 0.066, tau_\eta = 0.0424
			restrict expr to domain={\thisrow{\TablePowerColumnThree}}{1.0:1.0},
			unbounded coords=discard,
		] {\TableVelocity};
		\addplot[name path=B, draw=none, forget plot] table [
			x=\TablePowerColumnFour,
			y expr={(\thisrow{\TablePowerColumnOne} + \thisrow{\TablePowerColumnTwo}) / 0.066 / (9.4 * \thisrow{\TablePowerColumnThree}^2 * 0.5 + 3.1 * \thisrow{\TablePowerColumnFive} * 0.0424^2 * 0.5^3)}, %\eta = 0.00280, u_\eta = 0.066, tau_\eta = 0.0424
			restrict expr to domain={\thisrow{\TablePowerColumnThree}}{1.0:1.0},
			unbounded coords=discard,
		] {\TableVelocity};
		\addplot[ColorSurf!66!ColorVs, opacity=0.25, forget plot, on layer=axis background] fill between[of=A and B];
		%%% average
		\addplot
		[
		color=ColorSurf!66!ColorVs,
		opacity=1.0,
		only marks,%solid
		mark=square
		]
		table[
			x=\TablePowerColumnFour, 
			y expr={\thisrow{\TablePowerColumnOne} / 0.066 / (9.4 * \thisrow{\TablePowerColumnThree}^2 * 0.5 + 3.1 * \thisrow{\TablePowerColumnFive} * 0.0424^2 * 0.5^3)}, %\eta = 0.00280, u_\eta = 0.066, tau_\eta = 0.0424
			restrict expr to domain={\thisrow{\TablePowerColumnThree}}{1.0:1.0},
			unbounded coords=discard,
		] {\TableVelocity};
		% us 4.0
		%% 95 CI
		\addplot[name path=A, draw=none, forget plot] table [
			x=\TablePowerColumnFour,
			y expr={(\thisrow{\TablePowerColumnOne} - \thisrow{\TablePowerColumnTwo}) / 0.066 / (9.4 * \thisrow{\TablePowerColumnThree}^2 * 0.5 + 3.1 * \thisrow{\TablePowerColumnFive} * 0.0424^2 * 0.5^3)}, %\eta = 0.00280, u_\eta = 0.066, tau_\eta = 0.0424
			restrict expr to domain={\thisrow{\TablePowerColumnThree}}{4.0:4.0},
			unbounded coords=discard,
		] {\TableVelocity};
		\addplot[name path=B, draw=none, forget plot] table [
			x=\TablePowerColumnFour,
			y expr={(\thisrow{\TablePowerColumnOne} + \thisrow{\TablePowerColumnTwo}) / 0.066 / (9.4 * \thisrow{\TablePowerColumnThree}^2 * 0.5 + 3.1 * \thisrow{\TablePowerColumnFive} * 0.0424^2 * 0.5^3)}, %\eta = 0.00280, u_\eta = 0.066, tau_\eta = 0.0424
			restrict expr to domain={\thisrow{\TablePowerColumnThree}}{4.0:4.0},
			unbounded coords=discard,
		] {\TableVelocity};
		\addplot[ColorSurf!33!ColorVs, opacity=0.25, forget plot, on layer=axis background] fill between[of=A and B];
		%%% average
		\addplot
		[
		color=ColorSurf!33!ColorVs,
		opacity=1.0,
		only marks,%solid
		mark=pentagon*
		]
		table[
			x=\TablePowerColumnFour, 
			y expr={\thisrow{\TablePowerColumnOne} / 0.066 / (9.4 * \thisrow{\TablePowerColumnThree}^2 * 0.5 + 3.1 * \thisrow{\TablePowerColumnFive} * 0.0424^2 * 0.5^3)}, %\eta = 0.00280, u_\eta = 0.066, tau_\eta = 0.0424
			restrict expr to domain={\thisrow{\TablePowerColumnThree}}{4.0:4.0},
			unbounded coords=discard,
		] {\TableVelocity};
		% us 8.0
		%% 95 CI
		\addplot[name path=A, draw=none, forget plot] table [
			x=\TablePowerColumnFour,
			y expr={(\thisrow{\TablePowerColumnOne} - \thisrow{\TablePowerColumnTwo}) / 0.066 / (9.4 * \thisrow{\TablePowerColumnThree}^2 * 0.5 + 3.1 * \thisrow{\TablePowerColumnFive} * 0.0424^2 * 0.5^3)}, %\eta = 0.00280, u_\eta = 0.066, tau_\eta = 0.0424
			restrict expr to domain={\thisrow{\TablePowerColumnThree}}{8.0:8.0},
			unbounded coords=discard,
		] {\TableVelocity};
		\addplot[name path=B, draw=none, forget plot] table [
			x=\TablePowerColumnFour,
			y expr={(\thisrow{\TablePowerColumnOne} + \thisrow{\TablePowerColumnTwo}) / 0.066 / (9.4 * \thisrow{\TablePowerColumnThree}^2 * 0.5 + 3.1 * \thisrow{\TablePowerColumnFive} * 0.0424^2 * 0.5^3)}, %\eta = 0.00280, u_\eta = 0.066, tau_\eta = 0.0424
			restrict expr to domain={\thisrow{\TablePowerColumnThree}}{8.0:8.0},
			unbounded coords=discard,
		] {\TableVelocity};
		\addplot[ColorSurf!00!ColorVs, opacity=0.25, forget plot, on layer=axis background] fill between[of=A and B];
		%%% average
		\addplot
		[
		color=ColorSurf!00!ColorVs,
		opacity=1.0,
		only marks,%solid
		mark=o
		]
		table[
			x=\TablePowerColumnFour, 
			y expr={\thisrow{\TablePowerColumnOne} / 0.066 / (9.4 * \thisrow{\TablePowerColumnThree}^2 * 0.5 + 3.1 * \thisrow{\TablePowerColumnFive} * 0.0424^2 * 0.5^3)}, %\eta = 0.00280, u_\eta = 0.066, tau_\eta = 0.0424
			restrict expr to domain={\thisrow{\TablePowerColumnThree}}{8.0:8.0},
			unbounded coords=discard,
		] {\TableVelocity};
	\end{groupplot}
\end{tikzpicture}

	\caption[Influence of the surfing parameter $\TimeHorizon$ and swimming velocity on rotating power consumption.]{
		Influence of the surfing parameter $\TimeHorizon$ and swimming velocity $\SwimmingVelocity$ on rotating energy consumption $P_{\mathrm{turn}} \propto \SwimmingAngularVelocity^2$ [Eq.~\eqref{eq:p_turn}].
		Shaded area represents the 95\% confidence interval.
	}
	\label{fig:energy_efficiency_time_horizon}
\end{figure}
This is expected due to the linear dependence on $\TimeHorizon$ of the swimming angular velocity $\SwimmingAngularVelocity \approx \TimeHorizon \norm{d(\Gradients)^T/dt}$.
This can be deduced from the expression of the surfing strategy given in Eq~\eqref{end:eq:surfing_swimming_direction_final} for which the derivative can be expressed as the following
\begin{equation}
	\frac{d \ControlDirectionOpt}{dt} = \TimeHorizon \left( \matr{\mathit{Id}} - \ControlDirectionOpt \otimes \ControlDirectionOpt \right) \cdot \frac{d \left( \Gradients \right)^T}{dt} \cdot \ControlDirectionOpt.
\end{equation}
Moreover, one observes a strong dependence on the swimming velocity $\SwimmingVelocity$ of the power consumption.
Passing faster through the flow, fast micro-swimmers increase the temporal derivative of their measure.
This force surfers to react faster to flow variations, leading to a higher energetic cost.

Using this computed value of $P_{\mathrm{turn.}}$, we then compare the energetic efficiency $E_{\mathrm{eff.}}$ of surfers of various swimming speed $\SwimmingVelocity$ in Fig.~\ref{fig:energy_efficiency_time_horizon}\textbf{(b)} (still assuming that $P_{\mathrm{meta.}} \approx 0$).
We observe that the maximal energetic performance is slightly shifted towards smaller values of the surfing parameter $\TimeHorizon$, for instance $\TimeHorizonOpt \approx 2 \KolmogorovTimeScale$ for $\SwimmingVelocity = 4 \KolmogorovVelocityScale$, compared to the case without taking energetic cost into account, for which $\TimeHorizonOpt \approx 3 \KolmogorovTimeScale$ for $\SwimmingVelocity = 4 \KolmogorovVelocityScale$ (Fig.~\ref{fig:surfing_parameter_tau_vs}).
Note furthermore that swimming speed itself, $\SwimmingVelocity$, has an important impact on energetic performance $E_{\mathrm{eff.}}$, the smaller the swimming velocity, the best the performance.


While surfing could enable plankton to migrate faster for higher values of $\TimeHorizon$, the energetic cost may not be worth it.
This discriminates the cases where power consumption is important.
However, if the metabolic power $P_{\mathrm{meta.}}$ is high enough, the problem is independent of the power consumption due to active orientation $P_{\mathrm{turn.}}$.

% \subsubsection{Optimal swimming velocity.}
% 
% As observed in Fig.~\ref{fig:energy_efficiency_time_horizon}, energetic performance is also function of swimming velocity $\SwimmingVelocity$.
% In this context, a optimal plankter swimming velocity exists, that maximizes energetic efficiency.
% The effective velocity of surfers evolves mostly lineary with swimming speed $\left\langle \Performance \right\rangle \approx \alpha_{\mathrm{\NameSurfShort}} \, \SwimmingVelocity$.
% This is especially true for $\SwimmingVelocity \lesssim \KolmogorovVelocityScale$, for which $\alpha_{\mathrm{\NameSurfShort}} \approx 1.7$ (Fig~\ref{fig:surfing_parameter_tau_vs}b) and for $\SwimmingVelocity \gg \KolmogorovVelocityScale$ for which $\alpha_{\mathrm{\NameSurfShort}} = 1$.
% Assuming $\alpha_{\mathrm{\NameSurfShort}}$ constant over the optimization range, $E$ is maximized for:
% \begin{align}
	% \SwimmingVelocity &= \sqrt{\frac{1}{3 \pi \mu d} \left( P_{\mathrm{turn}} + P_{\mathrm{meta.}} \right)} \\
	% &= \sqrt{\frac{1}{3} \left(d^2 \SwimmingAngularVelocity^2 + \frac{P_{\mathrm{meta.}}}{\pi\mu d} \right)}.
% \end{align}
% Replacing the metabolic power by the Kleiber estimate (Eq. \eqref{eq:kleiber}):
% \begin{equation}
% \SwimmingVelocity = \sqrt{\frac{1}{3} \left(d^2 \SwimmingAngularVelocity^2 + \frac{0.144}{6^{3/4} \pi^{1/4}} \frac{\rho_{\mathrm{plank.}}^{3/4}}{\mu} d^{5/4} \right)}.
% \end{equation}
% Then assuming that most plankton have roughly the same density as water $\rho_{\mathrm{plank.}} \approx \rho_{\mathrm{water}}$, the problem ends up depending of two last parameters: the rotation velocity $\SwimmingAngularVelocity$ and the plankter size (diamater) $d$.
% As $\SwimmingAngularVelocity$ depend on average solely of the surfing parameter $\TimeHorizon$, the latter and the plankter size $d$ remain the sole parameters of the problem.

\section{The cost of swimming: the on/off strategy}\label{sec:on_off}

Until now, the swimming velocity $\SwimmingVelocity$ of plankters was imposed constant. 
However, if a constant swimming speed $\SwimmingVelocity$ is not prescribed, an energy efficient plankter may modulate it to achieve energy efficient vertical migration. 
While the surfing strategy remains the solution of the \textit{average} swimming direction, one can ask how velocity could be distributed around this average to minimize energy consumption.

As the general problem may be too challenging to address, we consider in this section a spherical swimming plankter with no preferred reorientation.
The plankter is neither bottom-heavy, neither controls actively its orientation.
The evolution of its position $\ParticlePosition$ and swimming axis $\SwimmingDirection$ is then described by
\begin{subequations}\label{eq:spherical_motion}
	\begin{align}
		\frac{d \ParticlePosition}{dt} & =
		 \FlowVelocity (\ParticlePosition, t) + \SwimmingVelocity(t) \, \SwimmingDirection,\\
		\frac{d \SwimmingDirection}{dt} & = \frac{1}{2} \FlowVorticity \times \SwimmingDirection,
	\end{align}
\end{subequations}
with $\SwimmingVelocity(t)$ the time dependent swimming speed and $\FlowVorticity$ the flow vorticity at the plankter's position.
Furthermore, we restrict the possible swimming speed to an off/off behavior.
Swimming is then triggered when a measured quantity passes over a given threshold.
Commonly found in literature, it is often used to model plankton behavior \citep{wheeler2019not, yamazaki2003modeling, dodson1997daphnia}.
In our study, we consider the following threshold:
\begin{equation}\label{eq:on_off_behavior}
	\SwimmingVelocity(t) =
	\begin{cases}
		\SwimmingVelocity & \forall \, \theta(t) > \theta_{\mathrm{th.}} \\
		0 & \mathrm{else}
	\end{cases}
\end{equation}
with $\theta_{\mathrm{th.}}$ a threshold angle and $\theta = \theta_{\langle \ControlDirection \rangle_{\theta, \phi}, \SwimmingDirection}$ the angle between $\langle \ControlDirection \rangle_{\theta, \phi} (\Gradients, \Direction)$ and $\SwimmingDirection$ with $\langle \ControlDirection \rangle_{\theta, \phi}$ the preferred average swimming direction of the swimmer.

\begin{wrapfigure}[11]{L}[0.5\width]{0.35\textwidth}
	\centering
	\def\svgwidth{0.3\textwidth}
	\input{chap_end/schemes/optimal_threshold.pdf_tex}
	\captionsetup{width=0.3\textwidth}
  	\caption{Illustration of the optimal threshold angle $\theta_{\mathrm{th.}}^*$.}
  	\label{fig:optimal_threshold}
\end{wrapfigure}
We first asses the power consumption of such a swimmer.
As we consider a plankter with no reorientation torque, we can assume in turbulence its orientation is uniformly distributed in all possible orientations.
The average swimming power consumed is then obtained by integrating the swimming power consumed over all possible directions
\begin{align}
	\left\langle P_{\mathrm{swim}} \right\rangle_{\theta, \phi} &= \frac{1}{4\pi} \iint F_{\mathrm{swim}} \SwimmingVelocity(\theta) \sin \theta \, d\theta \, d\phi \nonumber \\
	&= \frac{1}{2} \int_0^{\theta_{\mathrm{th.}}} 3 \pi \mu d \SwimmingVelocity^2 \sin \theta \, d\theta \nonumber \\
	&= \frac{3}{2} \pi \mu d \SwimmingVelocity^2 \left( 1 - \cos \theta_{\mathrm{th.}} \right).
\end{align}
This results in a estimate of the swimming power as a function of both the swimming velocity $\SwimmingVelocity$ and the threshold angle $\theta_{\mathrm{th.}}$

The question is to swim as fast a possible in the direction $\Direction$ for a given, prescribed swimming power $P_{\mathrm{swim}}$.
The only parameter to optimize is the threshold angle $\theta_{\mathrm{th.}}$.
We first evaluate the swimming velocity $\SwimmingVelocity$ as a function of $\theta_{\mathrm{th.}}$ that ensures the same average power consumption $\left\langle P_{\mathrm{swim}} \right\rangle_{\theta, \phi}$
\begin{equation}\label{eq:energy_swimming_velocity}
	\SwimmingVelocity =  \sqrt{\frac{2 \left\langle P_{\mathrm{swim}} \right\rangle_{\theta, \phi}}{3 \pi \mu d (1 - \cos \theta_{\mathrm{th.}})}}.
\end{equation} 
Then, we assess how the vertical effective swimming velocity evolves with $\theta$.
For naive swimmers for which $\ControlDirection = \Direction$, we expect the average effective vertical velocity $\Performance$ being directly proportional to $\cos \theta$. 
It can then be expressed as the following
\begin{equation}\label{eq:perf_naive}
	\left\langle \Performance \right\rangle_{\mathrm{naive}} = \frac{\SwimmingVelocity}{2} \int_0^{\theta_{\mathrm{th.}}} \cos \theta \sin \theta \, d\theta = \frac{\SwimmingVelocity}{4} \sin^2 \theta_{\mathrm{th.}} = \sqrt{\frac{\left\langle P_{\mathrm{swim}} \right\rangle_{\theta, \phi}}{24 \pi \mu d}} \frac{\sin^2 \theta_{\mathrm{th.}}}{\sqrt{1 - \cos \theta_{\mathrm{th.}}}}
\end{equation}
that is maximized for $\theta_{\mathrm{th.}}^* = \arcos (1/3) \approx 0.39 \pi$ (Fig.~\ref{fig:optimal_threshold})

As for reorienting surfers, we expect on/off surfers ($\ControlDirection = \ControlDirectionOpt$) to perform better than naive on/off swimmers.
If the same improvement is obtained, we expect $\left\langle \Performance \right\rangle_{\mathrm{\NameSurfShort}} / \left\langle \Performance \right\rangle_{\mathrm{naive}} = \alpha_{\mathrm{\NameSurfShort}} \approx 1.7$ (Fig~\ref{fig:surfing_parameter_tau_vs}b) and
\begin{equation}\label{eq:perf_surf}
	\left\langle \Performance \right\rangle_{\mathrm{\NameSurfShort}} = \alpha_{\mathrm{\NameSurfShort}} \sqrt{\frac{\left\langle P_{\mathrm{swim}} \right\rangle_{\theta, \phi}}{24 \pi \mu d}} \frac{\sin^2 \theta_{\mathrm{th.}}}{\sqrt{1 - \cos \theta_{\mathrm{th.}}}}.
\end{equation}

To test this theory, we simulate numerical on/off plankton [Eq.~\eqref{eq:on_off_behavior}] in a turbulent environment of Reynolds number $\mathit{Re}_{\lambda} = 11$.
Micro-swimmers are characterizes by their strategy [$\ControlDirection(t) = \Direction$ or $\ControlDirection(t) = \ControlDirectionOpt$], their angular threshold [$\theta_{\mathrm{th.}}$, Eq.\eqref{eq:on_off_behavior}] and their average swimming power $\left\langle P_{\mathrm{swim}} \right\rangle_{\theta, \phi}$.
Both the naive upward strategy, $\ControlDirection = \Direction$, and the surfing strategy, $\ControlDirection = \ControlDirectionOpt$, are tested varying $\theta_{\mathrm{th.}}$ from $\pi/10$ to $\pi/2$.

To ensure the same swimming power is used by each micro-swimmer, their swimming velocity when swimming is given by Eq.~\eqref{eq:energy_swimming_velocity} with $\langle P_{\mathrm{swim}} \rangle_{\theta, \phi} = P_{\mathrm{swim}, \mathrm{target}}$.
This expression relies on the hypothesis of a uniform angular distribution.
This assumption may break down.
Simulated micro-swimmers rotate with flow vorticity $\FlowVorticity$. 
When swimming velocity increases, the correlation time of $\FlowVorticity$ sampled decreases.
This should lead to slightly smaller rotational diffusion, thus biasing the orientation distribution (Fig.~\ref{fig:swimming_velocity_rigidification}).
\begin{figure}
	\centering
	\begin{minipage}{0.5\textwidth}
		\centering
		\def\svgwidth{0.6\textwidth}
		\input{chap_end/schemes/swimming_velocity_rigidification.pdf_tex}
		\captionsetup{width=0.9\textwidth}
	  	\caption{Illustration of the rotational diffusion reduction when swimming cause a swimmer to swim more than expected.}
	  	\label{fig:swimming_velocity_rigidification}
	\end{minipage}%
	\begin{minipage}{0.5\textwidth}
		\centering
		\def\svgwidth{0.6\textwidth}
		\input{chap_end/schemes/swimming_velocity_surfing_flexification.pdf_tex}
		\captionsetup{width=0.9\textwidth}
	  	\caption{Illustration of the increase of rotational diffusion of the surfing direction causing a swimmer to swim less than expected.}
	  	\label{fig:swimming_velocity_surfing_flexification}
	\end{minipage}
\end{figure}
The swimming state being more probable than expected, the swimmers with the lowest threshold $\theta_{\mathrm{th.}}$ (and thus swimming the fastest instantaneously) should consume slightly more swimming power.
As observed in Fig.~\ref{fig:energy_efficiency_threshold_power}\textbf{(a)}, this decrease in rotational diffusion is negligible and does not cause any effect on the average swimming power consumption of the naive swimmer ($\ControlDirection = \Direction$).
\begin{figure}%[H]
	\centering
	\begin{tikzpicture}
	% gain as a function of the free parameter $\tau$
	\begin{groupplot}[
   		group style={
   			group size=2 by 1,
   			%y descriptions at=edge left,
   			horizontal sep=0.08\linewidth,
   		},
   		% size
   		width=0.5\textwidth,
		% x
		xlabel=$\theta_{\mathrm{th}}$,
		xmin=0.1,
		xmax=0.5,
		xtick={0.1, 0.2, 0.3, 0.4, 0.5},
		xticklabels={$\pi/10$, $\pi/5$, $3\pi/10$, $2\pi/5$, $\pi/2$},
		% y
		ylabel={$\left\langle \SwimmingVelocity^2 \right\rangle_{N, t} / \KolmogorovVelocityScale^2$},
		% layers
		set layers,
		% legend
		legend style={draw=none, fill=none, /tikz/every even column/.append style={column sep=4pt}, at={(1.0, 1.05)}, anchor=south},
		%legend pos=north west,
   		legend cell align=left,
   		legend columns=-1,
   	]	
		\nextgroupplot[
			axis on top,
			% y
			ymin=0,
			ymax=1.0,
			%ytick={0.0, 0.1, 0.2},
		]
			\node[anchor=north west] at (axis cs:0.1,1.0) {\textbf{(a)}: $P_{\mathrm{swim}, \mathrm{target}} = (3/2) \pi \mu d \KolmogorovVelocityScale^2$};
			%% bottom-heavy
			%%% 95 CI
			\addplot[name path=A, draw=none, forget plot] table [
				x index=4,
				y expr={(\thisrowno{1} - \thisrowno{2}) / 0.21^2}, %u_\eta = 0.21
				col sep=comma, 
				comment chars=\#,
				restrict expr to domain={\thisrowno{3}}{1.0:1.0},
				%unbounded coords=discard,
			]{data/on_off_risers__flow__n_128__re_250/on_off_riser__merge_swimming_power_consumption.csv};
			\addplot[name path=B, draw=none, forget plot] table [
				x index=4, 
				y expr={(\thisrowno{1} + \thisrowno{2}) / 0.21^2}, %u_\eta = 0.21
				col sep=comma,
				comment chars=\#,
				restrict expr to domain={\thisrowno{3}}{1.0:1.0},
				%unbounded coords=discard,
			]{data/on_off_risers__flow__n_128__re_250/on_off_riser__merge_swimming_power_consumption.csv};
			\addplot[ColorBh, opacity=0.25, forget plot, on layer=axis background] fill between[of=A and B];
			%%% average
			\addplot
			[
			color=ColorBh,
			opacity=1.0,
			only marks,%solid
			mark=o
			]
			table[
				x index=4, 
				y expr={\thisrowno{1} / 0.21^2}, %u_\eta = 0.21
				col sep=comma, 
				comment chars=\#,
				restrict expr to domain={\thisrowno{3}}{1.0:1.0},
				%unbounded coords=discard,
			]{data/on_off_risers__flow__n_128__re_250/on_off_riser__merge_swimming_power_consumption.csv};
			\addlegendentry{\NameBhShort}
			%% surf
			%%% 95 CI
			\addplot[name path=A, draw=none, forget plot] table [
				x index=4,
				y expr={(\thisrowno{1} - \thisrowno{2}) / 0.21^2}, %u_\eta = 0.21
				col sep=comma, 
				comment chars=\#,
				restrict expr to domain={\thisrowno{3}}{1.0:1.0},
				%unbounded coords=discard,
			]{data/on_off_surfers__flow__n_128__re_250/on_off_surfer__merge_swimming_power_consumption.csv};
			\addplot[name path=B, draw=none, forget plot] table [
				x index=4, 
				y expr={(\thisrowno{1} + \thisrowno{2}) / 0.21^2}, %u_\eta = 0.21
				col sep=comma,
				comment chars=\#,
				restrict expr to domain={\thisrowno{3}}{1.0:1.0},
				%unbounded coords=discard,
			]{data/on_off_surfers__flow__n_128__re_250/on_off_surfer__merge_swimming_power_consumption.csv};
			\addplot[ColorSurf, opacity=0.25, forget plot, on layer=axis background] fill between[of=A and B];
			%%% average
			\addplot
			[
			color=ColorSurf,
			opacity=1.0,
			only marks,%solid
			mark=square*
			]
			table[
				x index=4, 
				y expr={\thisrowno{1} / 0.21^2}, %u_\eta = 0.21
				col sep=comma, 
				comment chars=\#,
				restrict expr to domain={\thisrowno{3}}{1.0:1.0},
				%unbounded coords=discard,
			]{data/on_off_surfers__flow__n_128__re_250/on_off_surfer__merge_swimming_power_consumption.csv};
			\addlegendentry{\NameSurfShort}
			%% y = x
			\addplot
			[
				color=gray!50!white,
				opacity=1.0,
				%line width=1pt,
				solid,
				on layer=axis background,
				domain=0.1:0.5,
			]{0.5};




		\nextgroupplot[
			axis on top,
			% y
			ymin=0,
			ymax=10.0,
			ylabel={},
			%ytick={0.0, 0.1, 0.2},
		]
			\node[anchor=north west] at (axis cs:0.1,10.0) {\textbf{(b)}: $P_{\mathrm{swim}, \mathrm{target}} = (3/2) \pi \mu d (4\KolmogorovVelocityScale)^2$};
			%% bottom-heavy
			%%% 95 CI
			\addplot[name path=A, draw=none, forget plot] table [
				x index=4,
				y expr={(\thisrowno{1} - \thisrowno{2}) / 0.21^2}, %u_\eta = 0.21
				col sep=comma, 
				comment chars=\#,
				restrict expr to domain={\thisrowno{3}}{4.0:4.0},
				%unbounded coords=discard,
			]{data/on_off_risers_fast__flow__n_128__re_250/on_off_riser__merge_swimming_power_consumption.csv};
			\addplot[name path=B, draw=none, forget plot] table [
				x index=4, 
				y expr={(\thisrowno{1} + \thisrowno{2}) / 0.21^2}, %u_\eta = 0.21
				col sep=comma,
				comment chars=\#,
				restrict expr to domain={\thisrowno{3}}{4.0:4.0},
				%unbounded coords=discard,
			]{data/on_off_risers_fast__flow__n_128__re_250/on_off_riser__merge_swimming_power_consumption.csv};
			\addplot[ColorBh, opacity=0.25, forget plot, on layer=axis background] fill between[of=A and B];
			%%% average
			\addplot
			[
			color=ColorBh,
			opacity=1.0,
			only marks,%solid
			mark=o
			]
			table[
				x index=4, 
				y expr={\thisrowno{1} / 0.21^2}, %u_\eta = 0.21
				col sep=comma, 
				comment chars=\#,
				restrict expr to domain={\thisrowno{3}}{4.0:4.0},
				%unbounded coords=discard,
			]{data/on_off_risers_fast__flow__n_128__re_250/on_off_riser__merge_swimming_power_consumption.csv};
			%% surf
			%%% 95 CI
			\addplot[name path=A, draw=none, forget plot] table [
				x index=4,
				y expr={(\thisrowno{1} - \thisrowno{2}) / 0.21^2}, %u_\eta = 0.21
				col sep=comma,
				comment chars=\#,
				restrict expr to domain={\thisrowno{3}}{4.0:4.0},
				%unbounded coords=discard,
			]{data/on_off_surfers_fast__flow__n_128__re_250/on_off_surfer__merge_swimming_power_consumption.csv};
			\addplot[name path=B, draw=none, forget plot] table [
				x index=4,
				y expr={(\thisrowno{1} + \thisrowno{2}) / 0.21^2}, %u_\eta = 0.21
				col sep=comma,
				comment chars=\#,
				restrict expr to domain={\thisrowno{3}}{4.0:4.0},
				%unbounded coords=discard,
			]{data/on_off_surfers_fast__flow__n_128__re_250/on_off_surfer__merge_swimming_power_consumption.csv};
			\addplot[ColorSurf, opacity=0.25, forget plot, on layer=axis background] fill between[of=A and B];
			%%% average
			\addplot
			[
			color=ColorSurf,
			opacity=1.0,
			only marks,%solid
			mark=square*
			]
			table[
				x index=4,
				y expr={\thisrowno{1} / 0.21^2}, %u_\eta = 0.21
				col sep=comma,
				comment chars=\#,
				restrict expr to domain={\thisrowno{3}}{4.0:4.0},
				%unbounded coords=discard,
			]{data/on_off_surfers_fast__flow__n_128__re_250/on_off_surfer__merge_swimming_power_consumption.csv};
			%% y = x
			\addplot
			[
				color=gray!50!white,
				opacity=1.0,
				%line width=1pt,
				solid,
				on layer=axis background,
				domain=0.1:0.5,
			]{8};
	\end{groupplot}
\end{tikzpicture}

	\caption[Influence of the angular threshold on swimming power consumption.]{
		Influence of the angular threshold $\theta_{\mathrm{th.}}$ [Eq.~\eqref{eq:on_off_behavior}] on swimming power consumption $P_{\mathrm{swim}} \propto \left\langle \SwimmingVelocity(t)^2 \right\rangle$ for naive swimmers and surfers and two target swimming powers.
		Shaded area represents the 95\% confidence interval.
	}
	\label{fig:energy_efficiency_threshold_power}
\end{figure}

However, the decrease of the correlation time of the flow velocity gradient $\Gradients$ impacts the surfers.
Indeed, their flow measure controls their preferred direction $\ControlDirection = \ControlDirectionOpt$.
When swimming $\ControlDirectionOpt$ varies faster, thus reducing the time for which $\theta$ [Eq.~\eqref{eq:energy_swimming_velocity}] remains below the swimming threshold $\theta_{\mathrm{th.}}$ (Fig.~\ref{fig:swimming_velocity_surfing_flexification}).
This decreases the overall time spent in the swimming state compared to the waiting state, thus biasing orientation out of a uniform distribution of orientation of $\SwimmingDirection$ with respect to $\ControlDirection$.
Eq.~\eqref{eq:energy_swimming_velocity} thus underestimates the necessary swimming velocity to keep a constant average swimming power $P_{\mathrm{swim}, \mathrm{target}}$.
% The lower the threshold $\theta_{\mathrm{th.}}$, the higher is the swimming velocity and the stronger this effect reducing the swimming power consumption becomes.

We can then compare the effective vertical swimming velocity $\Performance$ reached by on/off naive swimmers and surfers for various swimming thresholds $\theta_{\mathrm{th.}}$ (Fig.~\ref{fig:energy_efficiency_threshold}).
\begin{figure}%[H]
	\centering
	\begin{tikzpicture}
	% gain as a function of the free parameter $\tau$
	\begin{groupplot}[
   		group style={
   			group size=2 by 1,
   			y descriptions at=edge left,
   			horizontal sep=0.08\linewidth,
   		},
   		% size
   		width=0.5\textwidth,
		% x
		xlabel=$\theta_{\mathrm{th}}$,
		xmin=0.1,
		xmax=0.5,
		xtick={0.1, 0.2, 0.3, 0.4, 0.5},
		xticklabels={$\pi/10$, $\pi/5$, $3\pi/10$, $2\pi/5$, $\pi/2$},
		% y
		ymin=0,
		ymax=0.6,
		ylabel={$\left\langle \Performance \right\rangle_N / \sqrt{\left\langle P_{\mathrm{swim}} \right\rangle_{N, t}/24 \pi \mu d}$},
		% layers
		set layers,
		% legend
		legend style={draw=none, fill=none, /tikz/every even column/.append style={column sep=4pt}, at={(1.0, 1.05)}, anchor=south},
		%legend pos=north west,
   		legend cell align=left,
   		legend columns=-1,
   	]
		\nextgroupplot[
			axis on top,
		]

			\pgfplotstableread[col sep=comma]{data/on_off_risers__flow__n_128__re_250/on_off_riser__merge_average_velocity_axis_0.csv}\TableVelocity
			\pgfplotstableread[col sep=comma]{data/on_off_risers__flow__n_128__re_250/on_off_riser__merge_swimming_power_consumption.csv}\TablePower
			\pgfplotstablegetcolumnnamebyindex{1}\of{\TablePower}\to\TablePowerColumnFour
			\pgfplotstablecreatecol[create col/copy column from table=\TablePower{\TablePowerColumnFour}]{velocity^2}\TableVelocity
			\pgfplotstablegetcolumnnamebyindex{0}\of{\TableVelocity}\to\TablePowerColumnZero
			\pgfplotstablegetcolumnnamebyindex{1}\of{\TableVelocity}\to\TablePowerColumnOne
			\pgfplotstablegetcolumnnamebyindex{2}\of{\TableVelocity}\to\TablePowerColumnTwo
			\pgfplotstablegetcolumnnamebyindex{3}\of{\TableVelocity}\to\TablePowerColumnThree
			\pgfplotstablegetcolumnnamebyindex{4}\of{\TableVelocity}\to\TablePowerColumnFour
			\pgfplotstablegetcolumnnamebyindex{5}\of{\TableVelocity}\to\TablePowerColumnFive
		
			\node[anchor=north west] at (axis cs:0.1,0.6) {\textbf{(a)}: $P_{\mathrm{swim}, \mathrm{target}} = (3/2) \pi \mu d \KolmogorovVelocityScale^2$};
			%% riser
			%%% 95 CI
			\addplot[name path=A, draw=none, forget plot] table [
				x index=\TablePowerColumnFour,
				y expr={(\thisrow{\TablePowerColumnOne} - \thisrow{\TablePowerColumnTwo}) / sqrt(2*\thisrow{\TablePowerColumnFive})}, %u_\eta = 0.21
				col sep=comma,
				comment chars=\#,
				restrict expr to domain={\thisrow{\TablePowerColumnThree}}{1.0:1.0},
				unbounded coords=discard,
			]{\TableVelocity};
			\addplot[name path=B, draw=none, forget plot] table [
				x index=\TablePowerColumnFour,
				y expr={(\thisrow{\TablePowerColumnOne} + \thisrow{\TablePowerColumnTwo}) / sqrt(2*\thisrow{\TablePowerColumnFive})}, %u_\eta = 0.21
				col sep=comma,
				comment chars=\#,
				restrict expr to domain={\thisrow{\TablePowerColumnThree}}{1.0:1.0},
				unbounded coords=discard,
			]{\TableVelocity};
			\addplot[ColorBh, opacity=0.25, forget plot, on layer=axis background] fill between[of=A and B];
			%% average
			\addplot
			[
			color=ColorBh,
			opacity=1.0,
			only marks,%solid
			mark=o
			]
			table[
				x index=\TablePowerColumnFour,
				y expr={\thisrow{\TablePowerColumnOne} / sqrt(2*\thisrow{\TablePowerColumnFive})}, %u_\eta = 0.21
				col sep=comma,
				comment chars=\#,
				restrict expr to domain={\thisrow{\TablePowerColumnThree}}{1.0:1.0},
				unbounded coords=discard,
			]{\TableVelocity};
			\addlegendentry{\NameBhShort}

			\pgfplotstableread[col sep=comma]{data/on_off_surfers__flow__n_128__re_250/on_off_surfer__merge_average_velocity_axis_0.csv}\TableVelocity
			\pgfplotstableread[col sep=comma]{data/on_off_surfers__flow__n_128__re_250/on_off_surfer__merge_swimming_power_consumption.csv}\TablePower
			\pgfplotstablegetcolumnnamebyindex{1}\of{\TablePower}\to\TablePowerColumnFour
			\pgfplotstablecreatecol[create col/copy column from table=\TablePower{\TablePowerColumnFour}]{velocity^2}\TableVelocity
			\pgfplotstablegetcolumnnamebyindex{0}\of{\TableVelocity}\to\TablePowerColumnZero
			\pgfplotstablegetcolumnnamebyindex{1}\of{\TableVelocity}\to\TablePowerColumnOne
			\pgfplotstablegetcolumnnamebyindex{2}\of{\TableVelocity}\to\TablePowerColumnTwo
			\pgfplotstablegetcolumnnamebyindex{3}\of{\TableVelocity}\to\TablePowerColumnThree
			\pgfplotstablegetcolumnnamebyindex{4}\of{\TableVelocity}\to\TablePowerColumnFour
			\pgfplotstablegetcolumnnamebyindex{5}\of{\TableVelocity}\to\TablePowerColumnFive

			%% surf
			%%% 95 CI
			\addplot[name path=A, draw=none, forget plot] table [
				x index=\TablePowerColumnFour,
				y expr={(\thisrow{\TablePowerColumnOne} - \thisrow{\TablePowerColumnTwo}) / sqrt(2*\thisrow{\TablePowerColumnFive})}, %u_\eta = 0.21
				col sep=comma,
				comment chars=\#,
				restrict expr to domain={\thisrow{\TablePowerColumnThree}}{1.0:1.0},
				unbounded coords=discard,
			]{\TableVelocity};
			\addplot[name path=B, draw=none, forget plot] table [
				x index=\TablePowerColumnFour,
				y expr={(\thisrow{\TablePowerColumnOne} + \thisrow{\TablePowerColumnTwo}) / sqrt(2*\thisrow{\TablePowerColumnFive})}, %u_\eta = 0.21
				col sep=comma,
				comment chars=\#,
				restrict expr to domain={\thisrow{\TablePowerColumnThree}}{1.0:1.0},
				unbounded coords=discard,
			]{\TableVelocity};
			\addplot[ColorSurf, opacity=0.25, forget plot, on layer=axis background] fill between[of=A and B];
			%%% average
			\addplot
			[
			color=ColorSurf,
			opacity=1.0,
			only marks,%solid
			mark=square*
			]
			table[
				x index=\TablePowerColumnFour,
				y expr={\thisrow{\TablePowerColumnOne} / sqrt(2*\thisrow{\TablePowerColumnFive})}, %u_\eta = 0.21
				col sep=comma,
				comment chars=\#,
				restrict expr to domain={\thisrow{\TablePowerColumnThree}}{1.0:1.0},
				unbounded coords=discard,
			]{\TableVelocity};
			\addlegendentry{\NameSurfShort \quad\quad\quad\quad $\alpha_{\mathrm{\NameSurfShort}}=$}

			%%% models
			\addplot
			[
			color=ColorBh,
			opacity=1.0,
			solid,
			domain=0.1:0.5,
			]{0.25 * sin(deg(x * pi))^2/sqrt(1 - cos(deg(x * pi)))};
			\addlegendentry{$1$}
			\addplot
			[
			color=ColorSurf,
			opacity=1.0,
			solid,
			domain=0.1:0.5,
			]{1.7 * 0.25 * sin(deg(x * pi))^2/sqrt(1 - cos(deg(x * pi)))};
			\addlegendentry{$1.7$}
			
			% %% y = x
			% \addplot
			% [
			% color=gray!50!white,
			% opacity=1.0,
			% %line width=1pt,
			% solid,
			% on layer=axis background,
			% domain=0:16,
			% ]{1};


		\nextgroupplot[
			axis on top,
			% legend
			legend style={at={(0.52, 1.05)}, anchor=south},
		]
			\node[anchor=north west] at (axis cs:0.1,0.6) {\textbf{(b)}: $P_{\mathrm{swim}, \mathrm{target}} = (3/2) \pi \mu d (4\KolmogorovVelocityScale)^2$};

			\pgfplotstableread[col sep=comma]{data/on_off_risers_fast__flow__n_128__re_250/on_off_riser__merge_average_velocity_axis_0.csv}\TableVelocity
			\pgfplotstableread[col sep=comma]{data/on_off_risers_fast__flow__n_128__re_250/on_off_riser__merge_swimming_power_consumption.csv}\TablePower
			\pgfplotstablegetcolumnnamebyindex{1}\of{\TablePower}\to\TablePowerColumnFour
			\pgfplotstablecreatecol[create col/copy column from table=\TablePower{\TablePowerColumnFour}]{velocity^2}\TableVelocity
			\pgfplotstablegetcolumnnamebyindex{0}\of{\TableVelocity}\to\TablePowerColumnZero
			\pgfplotstablegetcolumnnamebyindex{1}\of{\TableVelocity}\to\TablePowerColumnOne
			\pgfplotstablegetcolumnnamebyindex{2}\of{\TableVelocity}\to\TablePowerColumnTwo
			\pgfplotstablegetcolumnnamebyindex{3}\of{\TableVelocity}\to\TablePowerColumnThree
			\pgfplotstablegetcolumnnamebyindex{4}\of{\TableVelocity}\to\TablePowerColumnFour
			\pgfplotstablegetcolumnnamebyindex{5}\of{\TableVelocity}\to\TablePowerColumnFive
		
			\node[anchor=north west] at (axis cs:0.6,0.0) {\textbf{(a)}: $P_{\mathrm{swim}, \mathrm{target}} = (3/2) \pi \mu d \KolmogorovVelocityScale^2$};
			%% riser
			%%% 95 CI
			\addplot[name path=A, draw=none, forget plot] table [
				x index=\TablePowerColumnFour,
				y expr={(\thisrow{\TablePowerColumnOne} - \thisrow{\TablePowerColumnTwo}) / sqrt(2*\thisrow{\TablePowerColumnFive})}, %u_\eta = 0.21
				col sep=comma,
				comment chars=\#,
				restrict expr to domain={\thisrow{\TablePowerColumnThree}}{4.0:4.0},
				unbounded coords=discard,
			]{\TableVelocity};
			\addplot[name path=B, draw=none, forget plot] table [
				x index=\TablePowerColumnFour,
				y expr={(\thisrow{\TablePowerColumnOne} + \thisrow{\TablePowerColumnTwo}) / sqrt(2*\thisrow{\TablePowerColumnFive})}, %u_\eta = 0.21
				col sep=comma,
				comment chars=\#,
				restrict expr to domain={\thisrow{\TablePowerColumnThree}}{4.0:4.0},
				unbounded coords=discard,
			]{\TableVelocity};
			\addplot[ColorBh, opacity=0.25, forget plot, on layer=axis background] fill between[of=A and B];
			%% average
			\addplot
			[
			color=ColorBh,
			opacity=1.0,
			only marks,%solid
			mark=o,
			forget plot
			]
			table[
				x index=\TablePowerColumnFour,
				y expr={\thisrow{\TablePowerColumnOne} / sqrt(2*\thisrow{\TablePowerColumnFive})}, %u_\eta = 0.21
				col sep=comma,
				comment chars=\#,
				restrict expr to domain={\thisrow{\TablePowerColumnThree}}{4.0:4.0},
				unbounded coords=discard,
			]{\TableVelocity};
			%\addlegendentry{\NameBhShort}

			\pgfplotstableread[col sep=comma]{data/on_off_surfers_fast__flow__n_128__re_250/on_off_surfer__merge_average_velocity_axis_0.csv}\TableVelocity
			\pgfplotstableread[col sep=comma]{data/on_off_surfers_fast__flow__n_128__re_250/on_off_surfer__merge_swimming_power_consumption.csv}\TablePower
			\pgfplotstablegetcolumnnamebyindex{1}\of{\TablePower}\to\TablePowerColumnFour
			\pgfplotstablecreatecol[create col/copy column from table=\TablePower{\TablePowerColumnFour}]{velocity^2}\TableVelocity
			\pgfplotstablegetcolumnnamebyindex{0}\of{\TableVelocity}\to\TablePowerColumnZero
			\pgfplotstablegetcolumnnamebyindex{1}\of{\TableVelocity}\to\TablePowerColumnOne
			\pgfplotstablegetcolumnnamebyindex{2}\of{\TableVelocity}\to\TablePowerColumnTwo
			\pgfplotstablegetcolumnnamebyindex{3}\of{\TableVelocity}\to\TablePowerColumnThree
			\pgfplotstablegetcolumnnamebyindex{4}\of{\TableVelocity}\to\TablePowerColumnFour
			\pgfplotstablegetcolumnnamebyindex{5}\of{\TableVelocity}\to\TablePowerColumnFive

			%% surf
			%%% 95 CI
			\addplot[name path=A, draw=none, forget plot] table [
				x index=\TablePowerColumnFour,
				y expr={(\thisrow{\TablePowerColumnOne} - \thisrow{\TablePowerColumnTwo}) / sqrt(2*\thisrow{\TablePowerColumnFive})}, %u_\eta = 0.21
				col sep=comma,
				comment chars=\#,
				restrict expr to domain={\thisrow{\TablePowerColumnThree}}{4.0:4.0},
				unbounded coords=discard,
			]{\TableVelocity};
			\addplot[name path=B, draw=none, forget plot] table [
				x index=\TablePowerColumnFour,
				y expr={(\thisrow{\TablePowerColumnOne} + \thisrow{\TablePowerColumnTwo}) / sqrt(2*\thisrow{\TablePowerColumnFive})}, %u_\eta = 0.21
				col sep=comma,
				comment chars=\#,
				restrict expr to domain={\thisrow{\TablePowerColumnThree}}{4.0:4.0},
				unbounded coords=discard,
			]{\TableVelocity};
			\addplot[ColorSurf, opacity=0.25, forget plot, on layer=axis background] fill between[of=A and B];
			%%% average
			\addplot
			[
			color=ColorSurf,
			opacity=1.0,
			only marks,%solid
			mark=square*,
			forget plot
			]
			table[
				x index=\TablePowerColumnFour,
				y expr={\thisrow{\TablePowerColumnOne} / sqrt(2*\thisrow{\TablePowerColumnFive})}, %u_\eta = 0.21
				col sep=comma,
				comment chars=\#,
				restrict expr to domain={\thisrow{\TablePowerColumnThree}}{4.0:4.0},
				unbounded coords=discard,
			]{\TableVelocity};
			%\addlegendentry{\NameSurfShort \quad\quad\quad\quad $\alpha_{\mathrm{\NameSurfShort}}=$}

			%%% models
			\addplot
			[
			color=ColorBh,
			opacity=1.0,
			solid,
			domain=0.1:0.5,
			forget plot,
			]{0.25 * sin(deg(x * pi))^2/sqrt(1 - cos(deg(x * pi)))};
			\addplot
			[
			color=ColorSurf,
			opacity=1.0,
			dashed,
			domain=0.1:0.5,
			]{1.2 * 0.25 * sin(deg(x * pi))^2/sqrt(1 - cos(deg(x * pi)))};
			\addlegendentry{$1.2$}
			
			% %% y = x
			% \addplot
			% [
			% color=gray!50!white,
			% opacity=1.0,
			% %line width=1pt,
			% solid,
			% on layer=axis background,
			% domain=0:16,
			% ]{1};
	\end{groupplot}
\end{tikzpicture}

	\caption[Influence of the angular threshold on effective upward velocity.]{
		Influence of the angular threshold $\theta_{\mathrm{th.}}$ [Eq.~\eqref{eq:on_off_behavior}] on effective upward velocity $\Performance$ for naive swimmers and surfers and two target swimming powers.
		Shaded area represents the 95\% confidence interval.
		Solid lines represent the theory [Eq.~\eqref{eq:perf_surf}].
	}
	\label{fig:energy_efficiency_threshold}
\end{figure}
We first notice that performance of naive swimmers is correctly predicted by the model described in Eq.~\eqref{eq:perf_naive}.
The performance of surfers is however not completely captured by Eq.~\eqref{eq:perf_surf} for the smallest values of $\theta_{\mathrm{th.}}$.
This discrepancy is due to the dependence of surfing performance with swimming velocity (which also caused low surfing performance in Fig.~\ref{fig:energy_efficiency_threshold}\textbf{(b)} corresponding to high swimming power).
Indeed, $\alpha_{\mathrm{surf}}$ was considered constant to derive the model but in practice it varies and decreases with the swimming velocity $\SwimmingVelocity$.
As a consequence, compared to the model, performance slightly drops for the lowest thresholds $\theta_{\mathrm{th.}}$ corresponding to the highest swimming speeds.

Overall, the surfing framework can also be used to find efficient strategies in the context of minimal power consumption.
