\chapter{Surfing on turbulence}\label{chap:surfing_on_turbulence}

Turbulence is an important feature of the flow environment relevant to many plankton species \citep{fuchs2016seascape}.
For instance, turbulent flow velocity fluctuations are known to increase plankton contact rates \citep{rothschild1988small}.
On the other hand, flow disturbances generated by turbulent flows might also alter plankton sensing and, as a consequence, influence prey/predator interactions \citep{saiz1995predatory, pecseli2019feeding} and mate finding \citep{michalec2020efficient}.
Inertial and active particles are known to preferentially concentrate \citep{monchaux2012analyzing, gustavsson2016preferential} in turbulence, then displaying macroscopic density distribution patterns.
Overall, turbulence lead to important consequences on the development of plankters, such as altering their growth rate \citep{peters2000effects}.
Note however that some recent studies demonstrated that plankton may experience weaker turbulence than previously considered \citep{franks2022oceanic}.

Despite these numerous studies, the role of turbulence in plankton dynamics remains to be fully understood, in particular in regard of plankton \textbf{active response} to turbulence \citep{franks2022oceanic}.
In this chapter, we discuss the surfing strategy, presented in Chap.~\ref{chap:the_surfing_strategy}, in the context of navigation in turbulent flows.

We first discuss the features of turbulent flows and explain how the turbulent environments are simulated in this study.
Then we assess surfing performance in homogeneous isotropic turbulence for various turbulence intensities.
Finally we propose an estimate of surfing performance based upon the analysis of surfing performance restricted to various parts of the flow. 
Some of the results presented in this chapter have been published in \citet{monthiller2022surfing} but original results are also included.

\section{Modeling the turbulent environment}\label{sec:simulating_turbulent_flows}

\subsection{Turbulent flows}

\begin{wrapfigure}[16]{R}[0.4\width]{0.3\textwidth}
	\centering
	\vspace{-25pt}
	\def\svgwidth{0.25\textwidth}
	\input{chap_numeric/schemes/lagrangian_specification.pdf_tex}
	\captionsetup{width=0.25\textwidth}
  	\caption{
  		Illustration of the differences of Eulerian and Lagrangian specification.
  	}
  	\label{fig:flow_specifications}
\end{wrapfigure}
As previously described in Chap.~\ref{chap:the_surfing_strategy}, we consider the plankter environment as an \textbf{single phase} \textbf{incompressible} fluid flow of constant kinematic viscosity $\nu$ and density $\rho$.
The flow velocity $\FlowVelocity$ and pressure $p$ of flow are described by the the Navier-Stokes equations,
\begin{subequations}\label{turb:eq:navier_stokes}
	\begin{align}
		\frac{\partial \FlowVelocity}{\partial t} + \left( \FlowVelocity \cdot \vec{\nabla} \right) \, \FlowVelocity & =
		- \frac{1}{\rho} \vec{\nabla} p + \nu \nabla^2 \FlowVelocity + \frac{1}{\rho} \vec{f}_{\mathrm{ext}}\label{turb:eq:navier_stokes_momentum} \\
		\vec{\nabla} \cdot \FlowVelocity & = 0.\label{turb:eq:navier_stokes_mass}
	\end{align}
\end{subequations}

Flows can be studied through two different specifications: The Eulerian one or the Lagrangian one.
In the Eulerian approach, the flow field is described as a function of position $\vec{x}$ and time $t$. 
In the Lagrangian approach however, the observer follows the motion of fluid particles along their trajectory (Fig.~\ref{fig:flow_specifications}).
The flow field is then represented as a function of time and the particle selected.
One can link flow kinematics in both specifications through the material derivative
\begin{equation}
	\frac{D \FlowVelocity}{D t} = \frac{\partial \FlowVelocity}{\partial t} + \left( \FlowVelocity \cdot \vec{\nabla} \right)\ \, \FlowVelocity.
\end{equation}
The operator $D \mathord{\cdot} / D t$ describes the time derivative in the Lagrangian specification (following a fluid particle trajectory) and $\partial \mathord{\cdot} / \partial t$ describes the time derivative in the Eulerian specification (at a fixed position in space).

Since we are interested in particles advected by a turbulent flow, each approach are useful and advantageous depending on the context.
In this study, the turbulent velocity fields are solely the result of Eulerian simulations.
However as our problem is inherently Lagrangian (as plankters are advected by the flow), some Lagrangian properties of these flows are also considered in this study.

In fluid dynamics, turbulence is the state of a fluid flow characterized by its strong irregularity.
Contrary to the smooth nature of laminar flows, turbulent flows present small scales fluctuations on top of large scale motion (Fig.~\ref{fig:laminar_turbulent_channels}).
\begin{figure}
	\centering
	\def\svgwidth{\textwidth}
	\input{chap_numeric/schemes/channel.pdf_tex}
	%\captionsetup{width=0.3\textwidth}
  	\caption[Illustration of differences between a laminar flow and turbulent flow.]{
  		Illustration of differences between a laminar flow and turbulent flow.
  		Colors indicate the norm of planar flow velocity normalized by the bulk velocity (average velocity).
  		The turbulent case is plotted from data of the Johns Hopkins Turbulence Database turbulent channel flow \citep{li2008public, perlman2007data}.
  	}
  	\label{fig:laminar_turbulent_channels}
\end{figure}

The turbulent nature of a flow is quantified by the Reynolds number $\mathit{Re}$.
This number is the result of the ratio of the intensity of the inertial and viscous terms of the Navier-Stokes momentum equation \eqref{turb:eq:navier_stokes}
\begin{equation}\label{eq:reynolds_number}
	\mathit{Re} \sim \frac{\norm{\left( \FlowVelocity \cdot \vec{\nabla} \right) \, \FlowVelocity}}{\norm{\nu \nabla^2 \FlowVelocity}} \sim \frac{U D}{\nu}.
\end{equation}
This number is generally evaluated using a flow velocity scale $U$ and a flow length scale $D$.
This choice is however not unique and depends on the problem.
Contrary to the particle Reynolds number $\mathit{Re}_p$ introduced in Chap.~\ref{chap:the_surfing_strategy}, Sec.~\ref{sec:the_surfing_strategy_problem}, defined at the scale of the plankter, this Reynolds number characterizes the large scale flow motion, independent of the plankter movements.

For a large Reynolds number ($\mathit{Re} \gg 1$), inertial effects prevail over viscous effects.
In that case, viscous effects are too weak to dissipate large structures directly.
This cause large structures to continuously break down into smaller ones down until the smallest possible scale is reached, where viscous dissipation occurs.
This phenomenon, discussed below, is characteristic of turbulent flows and is generally called the "inertial cascade" described below.

\subsection{Homogeneous isotropic turbulence}\label{sec:numeric_hit}

We can characterize the small scales of the flow through the description of the inertial cascade \citep{richardson1922weather}.
This phenomenon is well characterized by the Kolmogorov theory \citep{kolmogorov1941dissipation, kolmogorov1941degeneration, kolmogorov1941logarithmically, kolmogorov1941local}.
In addition to the \textbf{single phase} and \textbf{incompressible} properties of the flow, this theory relies on important additional assumptions:
\begin{itemize}
	\item The flow is \textbf{isotropic}, meaning the flow must be, on average, the same in every direction. 
	\item The flow is \textbf{homogeneous}, meaning the flow must be, on average, the same at any point in space.
	\item The \textbf{energy transfer must occurs locally}, meaning eddies only interact with other eddies of similar size.
\end{itemize}
While turbulent flows are rarely isotropic at large scales, small scales tend towards isotropy and homogeneity in turbulent flows \citep{frisch1995turbulence}.
As a consequence, for large enough Reynolds number flows $\mathit{Re} \gg 1$, small enough scales exist in the flow for which this assumption is valid.

We note the turbulent dissipation rate of the kinetic energy
\begin{equation}\label{eq:dissipation_rate}
	\epsilon = 2 \nu \norm{\GradientsSym}^2,
\end{equation}
that describes the energy dissipated by the smallest scales of the flow.
Due to the energy conservation, for the process to be stationary, the small scales of the flow must draw the same amount of energy from larger scales of the flow as the amount of energy dissipated.
Thus as transfers occur locally (assumption previously stated), $\epsilon$ then also corresponds to the energy transfer rate through flow scales and the energy rate injected in the flow at large scales.

Assuming then $\epsilon$ is independent of the flow viscosity $\nu$, one can assume by dimensional analysis that $\epsilon = \FlowVelocityScalar_l^3 / l$ for each flow length scale $l \ll L$, with $L$ the integral scale.
The integral scale $L$ is defined as the scale for which the energy is injected in the system.
Note that, flow behavior at integral scale $L$ is highly dependent of the problem and is not described by this theory.
One can then deduce the scale of the flow for which viscous effects take over inertial effects [the length scale $\KolmogorovScale$ so that $\mathit{Re}_{\KolmogorovScale} = 1$, Eq.~\eqref{eq:reynolds_number}]
\begin{equation}
	\label{eq:kolmogorov_scale}
	\KolmogorovScale = \left( \frac{\nu^3}{\epsilon} \right)^{1/4},
\end{equation}
from which we can also deduce a time scale and a velocity scale
\begin{equation}
	\label{eq:kolmogorov}
	\KolmogorovTimeScale = \sqrt{\frac{\nu}{\epsilon}}, \quad \KolmogorovVelocityScale = \left(\nu \epsilon \right)^{1/4}.
\end{equation}
These scales, called the Kolmogorov microscales, describe the scales of the smallest flow features of turbulence.

While the scale of Kolmogorov $\KolmogorovScale$ corresponds to the smallest scale of turbulence, dissipation starts to occur at larger scales.
The scale for which gradients starts to contribute to viscous dissipation is characterized by the Taylor microscale \citep{taylor1935statistical}
\begin{equation}
	\label{eq:taylor_microscale}
	\lambda = \sqrt{15 \frac{\nu}{\epsilon}} \FlowVelocityScalar_{\mathrm{rms}},
\end{equation}
with $\FlowVelocityScalar_{\mathrm{rms}} = \sqrt{ \langle \norm{\FlowVelocity}^2 \rangle_{x,t} }$ the root mean square velocity of the flow.
Defined by the statistics of the flow, this length scale is often used to define the Taylor Reynolds number
\begin{equation}\label{eq:taylor_scale_reynolds}
	\mathit{Re}_{\lambda} = \frac{\FlowVelocityScalar_{\mathrm{rms}} \lambda}{\nu}.
\end{equation}
This definition removes the ambiguity of the choice of the length scale $D$ and velocity $U$ of the Reynolds number in Eq~\eqref{eq:reynolds_number}.

\begin{figure}
	\centering
	\def\svgwidth{0.7\textwidth}
	\input{chap_numeric/schemes/turbulence_spectrum.pdf_tex}
	%\captionsetup{width=0.3\textwidth}
  	\caption[Illustration of the dynamics of 3D homogeneous isotropic turbulence.]{
  		Illustration of the dynamics of 3D homogeneous isotropic turbulence.
  		Energy is injected in the system through a large scale forcing.
  		Turbulent energy is then transferred through the energy cascade to ever smalle scales until the Kolmogorov scale $\KolmogorovScale$ is reached: scale for chich viscouss dissipation occurs.
  	}
  	\label{fig:turbulence_spectrum}
\end{figure}
The Kolmogorov theory also leads to the description of the spectral energy density $E(k)$, associated to a wavenumber $k$, of turbulence in the inertial cascade.
The spectral energy density is defined as
\begin{equation}\label{eq:spectral_density}
	\mathcal{E} = \frac{1}{2} \int \norm*{\frac{d\FlowVelocity}{dk}}^2 4\pi k^2 dk = \int E(k) dk,
\end{equation}
with $\mathcal{E}$ the total kinetic energy of the flow, $\norm*{d\FlowVelocity/dk}$ the module of the spatial Fourier transform of the flow velocity and $k = 2\pi/l$ the norm of the wavenumber considered.
Based on the same hypothesis of independence of $\nu$ and locality, by dimensional analysis, Kolmogorov theory predicts the spectral energy density to follow
\begin{equation}\label{eq:kolmogorov_spectrum}
	E(k) = C_{K} \epsilon^{2/3} k^{-5/3}.
\end{equation}
$C_{K}$ is the Kolmogorov constant, supposedly independent of the flow kinematic viscosity and the large scale forcing.
Experimentally, this constant has been determined to be of the order of unity, $C_{K} \approx 1.5$ \citep{sreenivasan1995universality}.
The basics of 3D turbulence dynamics are summed up in Fig~\ref{fig:turbulence_spectrum}.

Note that this theory is only valid for 3D turbulence, case of interest here. 
2D flows have different properties (for example due to the absence of vortex stretching) that generate different dynamics .
This causes 2D turbulence to behave rather differently. 
For instance, we can observe an inverse energy cascade where energy is transferred to larger scales than the scale of energy injection.

The description of turbulence by Kolmogorov theory is however limited to this spectral energy density $E(k)$.
To model and capture all the complexity of small scale turbulence relevant for planktonic organisms, one must rely on direct numerical simulations.

\subsection{Direct Numerical Simulations}

\subsubsection{\textit{Snoopy} simulations}

In computational fluid dynamics, \textit{direct numerical simulations} are used to solve a discrete version of the Navier-Stokes equations [Eq.~\eqref{eq:navier_stokes}] directly.
On the contrary, \textit{Reynolds average numerical simulations} or \textit{large eddy simulations} do not solve turbulence down to the smallest scales and require a turbulence model \citep{lesieur2014turbulence}.

To capture the complex small scale dynamics of plankton-turbulence interaction, we rely on direct numerical simulations to simulate the plankter environment.
We use the pseudo-spectral, open-source solver \textit{Snoopy} \citep{lesur2005relevance, lesur2007impact} to simulate homogeneous isotropic turbulence.
We solved the Navier-Stokes equations for an incompressible fluid with kinematic viscosity $\nu$, varying from 0.002 to 0.02 (arbitrary unit), in a tri-periodic domain of size $l = 1$ (arbitrary unit) with resolution of $n \times n \times n$ with $n=64$ or $n=128$.
The flows were made statistically steady thanks to an external forcing delta-correlated in time and localized in spectral space ($3/2 < \norm{\vec{k}} l < 5/2$, with $\vec{k}$ the wavevector).
These simulated turbulent flows are characterized by their root mean square velocity $\FlowVelocityScalar_{\mathrm{rms}}$ and their integral length scale $L$, computed as follows
\begin{equation}\label{eq:integral_scale}
	L = \frac{\pi}{2 \FlowVelocityScalar_{\mathrm{rms}}} \int \frac{E(k)}{k} \, dk,
\end{equation}
where $E$ is the spectral energy density [Eq.~\eqref{eq:spectral_density}].

The parameters of the simulations performed with \textit{Snoopy} are summed up in Tab.~\ref{tab:snoopy_simulation_parameters}.
\begin{table}
	\center
	\begin{tabular}{w{c}{0.11\linewidth}w{c}{0.11\linewidth}w{c}{0.11\linewidth}w{c}{0.11\linewidth}w{c}{0.11\linewidth}w{c}{0.11\linewidth}}
		\rowcolor{ColorTabularParameters}
		$Re_{\lambda}$ & $n$ & $k_{\mathrm{max}} \eta$ & $L / \KolmogorovScale$ & $T_{L} / \KolmogorovTimeScale$ & $\FlowVelocityScalar_{\mathrm{rms}} / \KolmogorovVelocityScale$\\
		\rowcolor{ColorTabularValues}
		1.6 & 64 & 1.9 & 11 & 17 & 0.6 \\
		\rowcolor{ColorTabularValues}
		3.6 & 64 &  1.2 & 18 & 17 & 1.0 \\
		\rowcolor{ColorTabularValues}
		11 & 128 & 1.1 & 35 & 19 & 1.8 \\
		\rowcolor{ColorTabularValues}
		21 & 128 & 0.66 & 56 & 22 & 2.4 \\
	\end{tabular}
	\caption{
		Flow parameters and characteristics of the homogeneous isotropic turbulence simulations performed using \textit{Snoopy}.
	}
	\label{tab:snoopy_simulation_parameters}
\end{table}
Note that capturing the smallest features of the flow requires the spatial and the temporal resolution to be small enough $dx = L / n \lesssim \KolmogorovScale$ and $dt \lesssim \KolmogorovTimeScale$.
As the size of the Kolmogorov scales decrease with the turbulence intensity ($\KolmogorovScale/L \propto \mathit{Re}_{\lambda}^{-3/2}$, $\KolmogorovTimeScale/T_L \propto \mathit{Re}_{\lambda}^{-1}$), the computation time increases with $\mathit{Re}_{\lambda}^{5/2}$.

However, since our approach to compute plankter dynamics is inherently Lagrangian (cf. Chap.~\ref{chap:the_surfing_strategy}, Sec.~\ref{sec:the_surfing_strategy_problem}), we have to store all of the velocity components over many snapshots in order to be able to reconstruct the particle dynamics.
As the required memory size scales as $L^3 T_L/\KolmogorovScale^3 \KolmogorovTimeScale \propto \mathit{Re}_{\lambda}^{17/8}$ to save all necessary snapshots, limitations in terms of storage is therefore the main reason behind these low resolutions instead of actual computational costs of the simulations, hence the modest resolution used here (compared to modern standards).
Strictly speaking, for the lowest values of $\mathit{Re}_{\lambda}$, the simulations do not result in actual turbulent flows but rather in 3D chaotic flows.
Nevertheless, navigating in these flows remains (1) challenging and (2) relevant for plankton as the do not always experience very turbulent flows.

\subsubsection{Johns Hopkins Turbulence Database}

\begin{figure}
	\centering
	\def\svgwidth{0.6\textwidth}
	\input{chap_numeric/schemes/jhtdb_turbulent_homogeneous_isotropic.pdf_tex}
	%\captionsetup{width=0.3\textwidth}
  	\caption[Visualization of the vertical centerplane velocity field of the 3D forced homogeneous isotropic turbulence simulation of the Johns Hopkins Turbulence Database.]{
  		Visualization of the vertical centerplane velocity field of the 3D forced homogeneous isotropic turbulence simulation of the Johns Hopkins Turbulence Database.
  		Colors indicate the norm of the flow velocity normalized by the root mean square velocity.
  		The visualisation is adapted from data of the Johns Hopkins Turbulence Database homogeneous isotropic turbulence simulation \citep{li2008public, perlman2007data}.
  	}
  	\label{fig:jhtdb_visualization}
\end{figure}
To complete our study and reach a higher Reynolds number $Re$, we use the Johns Hopkins turbulence database \citep{li2008public, perlman2007data}.
This database provides open access to high quality, high Reynolds number simulated turbulent flows.

The forced homogeneous isotropic case has been generated from a direct numerical simulation using a pseudo-spectral code.
It reproduced the case of isotropic turbulence forced by keeping constant the total energy of the lowest modes of flow ($k \le 2$).
The problem has been solved on a $1024^3$ periodic grid of dimension $2 \pi \times 2 \pi \times 2 \pi$.
The database gives access to five large-eddy turnover times $T_{L} = L / \FlowVelocityScalar_{\mathrm{rms}}$, with $\FlowVelocityScalar_{\mathrm{rms}}$ the root mean square velocity and $L$ the integral scale.

The parameters of the simulation are summed up in Tab.~\ref{tab:jhtdb_flows_simulation_parameters}.
\begin{table}
	\center
	\begin{tabular}{w{c}{0.11\linewidth}w{c}{0.11\linewidth}w{c}{0.11\linewidth}w{c}{0.11\linewidth}w{c}{0.11\linewidth}w{c}{0.11\linewidth}}
		\rowcolor{ColorTabularParameters}
		$Re_{\lambda}$ & $n$ & $k_{\mathrm{max}} \eta$ & $L / \KolmogorovScale$ & $T_{L} / \KolmogorovTimeScale$ & $\FlowVelocityScalar_{\mathrm{rms}} / \KolmogorovVelocityScale$\\
		\rowcolor{ColorTabularValues}
		418 & 1024 & 1.35 & 487 & 46.9 & 10.4 \\
	\end{tabular}
	\caption[Flow parameters and characteristics of the forced homogeneous isotropic turbulence simulation of the Johns Hopkins Turbulence Database.]{
		Flow parameters and characteristics of the forced homogeneous isotropic turbulence simulation of the Johns Hopkins Turbulence Database \citep{li2008public, perlman2007data}.
	}
	\label{tab:jhtdb_flows_simulation_parameters}
\end{table}
More details are provided at \url{http://turbulence.pha.jhu.edu/Forced_isotropic_turbulence.aspx}.
A visualization of the flow is provided in Fig.~\ref{fig:jhtdb_visualization}.

\section{Problem description and surfing strategy}

We remind briefly the description of the navigation problem addressed in this study (see Chap.~\ref{chap:the_surfing_strategy}, Sec.~\ref{sec:the_surfing_strategy_problem} for details).

We consider a plankter whose task is to go as fast as possible in a target direction, which is chosen to be $\Direction$, the vertical.
We model the plankter as an active particle with position $\ParticlePosition (t)$, swimming in direction $\SwimmingDirection(t)$ at \textbf{constant swimming speed} $\SwimmingVelocity$ in a flow velocity field $\FlowVelocity (\vec{x}, t)$ of vorticity $\FlowVorticity (\vec{x}, t) = \vec{\nabla} \times \FlowVelocity$.
The plankter is assumed to be \textbf{inertialess}, \textbf{neutrally buoyant}, and \textbf{small compared to the Kolmogorov scale} $\eta$ [the scale of the smallest turbulent features \citep{frisch1995turbulence}].
It actively controls its orientation by choosing a preferred direction $\ControlDirection$.
We start by assuming that the swimming direction $\SwimmingDirection$ is always aligned with this preferred direction $\ControlDirection$ (assumption of \textbf{instantaneous reorientation}).
We will lift this assumption and examine the effects of a finite reorientation time below.
Under these assumptions, the equations of motion are
\begin{subequations}\label{turb:eq:motion}
	\begin{align}
		\frac{d \ParticlePosition}{dt} & =
		 \FlowVelocity (\ParticlePosition, t) + \SwimmingVelocity \, \SwimmingDirection, \label{turb:eq:x_motion}\\
		\SwimmingDirection(t) & = \ControlDirection(t) .\label{turb:eq:p_motion}
	\end{align}
\end{subequations}

We assume that the plankter \textbf{senses the local flow velocity gradient} $\Gradients$ and the \textbf{vertical direction} $\Direction$.
It responds to this information by choosing its preferred direction $\ControlDirection(\Gradients,\Direction)$, \textbf{without any memory}.

The metric used to quantify the performance of the plankters is the effective velocity, $\Performance$, defined as the long-time average velocity along $\Direction$ (Fig.~\ref{fig:performance_description}).
\begin{equation}
	\label{turb:eq:performance}
	\Performance = \lim_{\FinalTime\to\infty} \frac{\ParticlePosition (\FinalTime) - \ParticlePosition (0)}{\FinalTime} \cdot \Direction.
\end{equation}

As derived in Chap.~\ref{chap:the_surfing_strategy}, Sec.~\ref{sec:the_surfing_strategy_derivation}, this navigation problem has an approximate solution based on a local optimization.
This solution, called the surfing strategy, is formulated as follows
\begin{equation}
	\label{turb:eq:surfing_swimming_direction_final}
	\ControlDirectionOpt = \frac{\ControlDirectionOptNN}{\norm{\ControlDirectionOptNN}}, \quad \text{with} \quad \ControlDirectionOptNN = \left[ \exp \left( \TimeHorizon \Gradients \right) \right]^T \cdot \Direction,
\end{equation}
with $\ControlDirectionOpt$ the chosen preferred direction, $\TimeHorizon$ the sole free parameter of the surfing strategy, $\Gradients$ the measured flow velocity gradients and $\Direction$ the target direction.

To assess the relevance of this strategy to real life planktonic organisms and determine its possible benefit, we need to demonstrate its efficiency in biologically relevant environments such as turbulent flows.
To this end, we first evaluate the performance of surfers in homogeneous isotropic turbulence, compared to bottom-heavy swimmers that are always aligned with the vertical.
We thus compare $\SwimmingDirection = \ControlDirectionOpt$ [Eq.~\eqref{turb:eq:surfing_swimming_direction_final}] for surfers and $\SwimmingDirection=\ControlDirection_{\mathrm{\NameBhShort}}=\Direction$ for bottom-heavy swimmers in Eq.~\eqref{turb:eq:motion}.
Then, based on the previous observation that surfer behave differently in pure strain than in pure vortices (Chap.~\ref{chap:the_surfing_strategy}, Sec.~\ref{sec:the_surfing_strategy_linear}), we push the analysis further to assess which component of the flow contributes the most to surfing performance and attempt to estimate surfing performance based on this analysis.

\section{Evaluation of surfing performance}\label{sec:surfing_on_turbulence_IHT}

In the following, the performance of surfers is assessed in homogeneous isotropic turbulence.
Despite turbulence rarely being rigorously isotropic and homogeneous in the oceans, we expect nevertheless the insight gained in this model flow to be of great interest due to the universality of turbulence at small scales \citep{frisch1995turbulence}.
Homogeneous isotropic turbulence is a simplification of real turbulent flows but still captures most of their complexity.

To assess their navigation performance, surfers are simulated in numerical simulations of turbulence.
Initially placed randomly in the virtual domain, the trajectories of plankters are integrated over time (illustrated in Fig.~\ref{fig:turbulence_trajectories}).
\begin{figure}[t]
	\centering
	\def\svgwidth{0.65\textwidth}
	\input{chap_turbulence/schemes/trajectories.pdf_tex}
	\caption[Visualization of 3D trajectories obtained in the simulations of plankters migrating vertically in turbulence.]{
  		Visualization of 3D trajectories obtained in the simulations of plankters migrating vertically in turbulence.
  		The gray line shows the depth of the initial positions and circles show the average final vertical position for the same turbulent flow.
  		For the same simulation time, surfers migrated further upwards in average compared to bottom-heavy swimmers.
  	}
	\label{fig:turbulence_trajectories}
\end{figure}
Due to the complexity of the expression of the flow velocity field $\FlowVelocity$ in 3D turbulence, the equations of motion Eq.~\eqref{turb:eq:motion} must be integrated numerically. 
To this end we use our own in-house open-source code \textit{Sheld0n}\footnote{Our in-house code is available at \url{http://www.github.com/C0PEP0D/sheld0n}.}.
The code can be setup to integrate plankter trajectories by automatically querying various flow fields of John Hopkins turbulence database at plankton positions through simulations.
It is also able to use local flow databases generated from \textit{Snoopy} simulations (or from any fluid solver with the same output format) by interpolating the flow field at plankter positions.
In this study, we used a fourth-order Lagrange polynomial interpolation to integrate trajectories in our own flow databases while a sixth-order interpolation is performed on query when using the Johns Hopkins turbulence database.

Unless mentioned otherwise, the performance is evaluated using Eq.~\eqref{eq:surfing_performance} after a time $T$ larger than five large eddy turnover times $T \gtrsim 5 T_L$, and averaged over $N$ plankton with random initial positions.
This average is noted $\left\langle \cdot \right\rangle_N$.
The number of plankters $N$ varies from $10$ for $\SwimmingVelocity = 20 \KolmogorovVelocityScale$ to $16384$ for $\SwimmingVelocity = \KolmogorovVelocityScale/2$, so that uncertainties on performance are independent of swimming velocity $\SwimmingVelocity$.

\begin{figure}[p]
	\centering
	\begin{tikzpicture}
	% peformance as a function of swimming velocity
	\begin{groupplot}[
   		group style={
   			group size=2 by 2,
   			y descriptions at=edge left,
   			x descriptions at=edge bottom,
   			horizontal sep=0.04\linewidth,
   			vertical sep=0.04\linewidth,
   		},
   		axis on top ,
		% size
		width=0.5\linewidth,
		%height=0.62\linewidth,
		% x
		xlabel=$\SwimmingVelocity / \KolmogorovVelocityScale$,
		xmode=log,
		xmin=0.5,
   		% legend
   		legend style={draw=none, fill=none, /tikz/every even column/.append style={column sep=4pt}, at={(1.0, 1.15)}, anchor=south},
   		%legend pos=north west,
  		legend cell align=left,
  		legend columns=-1,
   	]
   		\nextgroupplot[
			axis on top,
			% x
			xmax=20,
			extra x ticks={0.5, 5, 20},
			extra x tick labels={,,},
			% y
			ylabel={$\left\langle \Performance \right\rangle_N / \KolmogorovVelocityScale$},
			ymode=log,
			ymin=0.5,
			ymax=30,
			yticklabels={0.1,1,10},
			extra y ticks={0.5, 5, 20, 30},
			extra y tick labels={0.5,5,20,30},
		]
			\node[anchor=north west] at (axis cs:0.5,30.0) {\textbf{(a):} $\mathit{Re}_{\lambda} = 418$};
			%% tss
			%%% average
			\addplot
			[
			color=ColorSurf,
			opacity=1.0,
			%line width=1pt,
			only marks,%solid,
			mark=square*
			]
			table[
				x index=2,
				y expr={\thisrowno{0} / 0.066}, %u_\eta = 0.066
				col sep=comma,
				comment chars=\#,
			]{chap_turbulence/data/main_results/max.csv};
			\addlegendentry{\NameSurf}
			%%% 95 CLI
			\addplot[name path=A, draw=none, forget plot] table [
				x index=2,
				y expr={(\thisrowno{0} - \thisrowno{1}) / 0.066}, %u_\eta = 0.066
				col sep=comma,
				comment chars=\#,
			]{chap_turbulence/data/main_results/max.csv};
			\addplot[name path=B, draw=none, forget plot] table [
				x index=2,
				y expr={(\thisrowno{0} + \thisrowno{1}) / 0.066}, %u_\eta = 0.066
				col sep=comma,
				comment chars=\#,
			]{chap_turbulence/data/main_results/max.csv};
			\addplot[ColorSurf, opacity=0.25, forget plot, on layer=axis background] fill between[of=A and B];
			%\addlegendentry{95 CI}
			%% straight
			%%% average
			\addplot
			[
			color=ColorBh,
			opacity=1.0,
			%line width=1pt,
			only marks,%solid,
			mark=o
			]
			table[
				x index=2,
				y expr={\thisrowno{0} / 0.066}, %u_\eta = 0.066
				col sep=comma,
				comment chars=\#,
				restrict expr to domain={\thisrowno{3}}{0.0:0.0},
				unbounded coords=discard,
			]{chap_turbulence/data/main_results/merge.csv};
			\addlegendentry{\NameBh}
			%%% 95 CLI
			\addplot[name path=A, draw=none, forget plot] table [
				x index=2,
				y expr={(\thisrowno{0} - \thisrowno{1}) / 0.066}, %u_\eta = 0.066
				col sep=comma,
				comment chars=\#,
				restrict expr to domain={\thisrowno{3}}{0.0:0.0},
				unbounded coords=discard,
			]{chap_turbulence/data/main_results/merge.csv};
			\addplot[name path=B, draw=none, forget plot] table [
				x index=2,
				y expr={(\thisrowno{0} + \thisrowno{1}) / 0.066}, %u_\eta = 0.066
				col sep=comma,
				comment chars=\#,
				restrict expr to domain={\thisrowno{3}}{0.0:0.0},
				unbounded coords=discard,
			]{chap_turbulence/data/main_results/merge.csv};
			\addplot[ColorBh, opacity=0.25, forget plot, on layer=axis background] fill between[of=A and B];
			%\addlegendentry{95 CI}
			%% y = x
			\addplot
			[
			color=black,
			opacity=1.0,
			%line width=1pt,
			solid,
			on layer=axis background,
			domain=0.5:20,
			]
			{x};
			\addlegendentry{$\left\langle \Performance \right\rangle = \SwimmingVelocity$}
			% %%% model
			% \def\moddelta{0.02}
			% \def\modtimesym{0.3 / \moddelta}
			% \def\modomega{0.65}
			% \def\modtimeasym{4.25}
			% \def\modparam{0.5}
			% \addplot
			% [
				% color=black,
				% opacity=1.0,
				% dashed,
				% domain=0.5:20.0,
				% samples=10,
			% ]{x * (3.0 * pi / 4.0) * (1.0 + exp(-3.0 * \moddelta * (\modtimesym + \modtimeasym ) / ((1.0 - \modparam) + \modparam * x))) / ( (1.0 + exp(-3.0 * \moddelta * \modtimesym / ((1.0 - \modparam) + \modparam * x))) * exp(-(1.0/8.0) * (\modomega * \modtimeasym / ((1.0 - \modparam) + \modparam * x) )^2.0) + 1)};

		\nextgroupplot[
			axis on top,
			% x
			xmax=10,
			extra x ticks={0.5, 5},
			extra x tick labels={,},
			% y
			ymode=log,
			ymin=0.5,
			ymax=30,
			extra y ticks={0.5, 5, 20, 30},
			extra y tick labels={,,,},
		]
			\node[anchor=north west] at (axis cs:0.5,30.0) {\textbf{(b):} $\mathit{Re}_{\lambda} = 11$};
			%% tss
			%%% average
			\addplot
			[
			color=ColorSurf,
			opacity=1.0,
			%line width=1pt,
			only marks,%solid,
			mark=square*
			]
			table[
				x index=3,
				y expr={\thisrowno{1} / 0.21}, %u_\eta = 0.21
				col sep=comma,
				comment chars=\#,
			]{data/surfers__flow__n_128__re_250/surfer__max_average_velocity_axis_0.csv};
			%%% 95 CLI
			\addplot[name path=A, draw=none, forget plot] table [
				x index=3,
				y expr={(\thisrowno{1} - \thisrowno{2}) / 0.21}, %u_\eta = 0.21
				col sep=comma,
				comment chars=\#,
			]{data/surfers__flow__n_128__re_250/surfer__max_average_velocity_axis_0.csv};
			\addplot[name path=B, draw=none, forget plot] table [
				x index=3,
				y expr={(\thisrowno{1} + \thisrowno{2}) / 0.21}, %u_\eta = 0.21
				col sep=comma,
				comment chars=\#,
			]{data/surfers__flow__n_128__re_250/surfer__max_average_velocity_axis_0.csv};
			\addplot[ColorSurf, opacity=0.25, forget plot, on layer=axis background] fill between[of=A and B];
			%\addlegendentry{95 CI}
			%% straight
			%%% average
			\addplot
			[
			color=ColorBh,
			opacity=1.0,
			%line width=1pt,
			only marks,%solid,
			mark=o
			]
			table[
				x index=3,
				y expr={\thisrowno{1} / 0.21}, %u_\eta = 0.21
				col sep=comma,
				comment chars=\#,
				restrict expr to domain={\thisrowno{4}}{0.0:0.0},
				unbounded coords=discard,
			]{data/surfers__flow__n_128__re_250/surfer__merge_average_velocity_axis_0.csv};
			%%% 95 CLI
			\addplot[name path=A, draw=none, forget plot] table [
				x index=3,
				y expr={(\thisrowno{1} - \thisrowno{2}) / 0.21}, %u_\eta = 0.21
				col sep=comma,
				comment chars=\#,
				restrict expr to domain={\thisrowno{4}}{0.0:0.0},
				unbounded coords=discard,
			]{data/surfers__flow__n_128__re_250/surfer__merge_average_velocity_axis_0.csv};
			\addplot[name path=B, draw=none, forget plot] table [
				x index=3,
				y expr={(\thisrowno{1} + \thisrowno{2}) / 0.21}, %u_\eta = 0.21
				col sep=comma,
				comment chars=\#,
				restrict expr to domain={\thisrowno{4}}{0.0:0.0},
				unbounded coords=discard,
			]{data/surfers__flow__n_128__re_250/surfer__merge_average_velocity_axis_0.csv};
			\addplot[ColorBh, opacity=0.25, forget plot, on layer=axis background] fill between[of=A and B];
			%\addlegendentry{95 CI}
			%% y = x
			\addplot
			[
			color=black,
			opacity=1.0,
			%line width=1pt,
			solid,
			on layer=axis background,
			domain=0.5:10,
			]
			{x};
			% %%% model
			% \def\moddelta{0.015}
			% \def\modtimesym{0.45 / \moddelta}
			% \def\modomega{1.0}
			% \def\modtimeasym{2.5}
			% \def\modparam{0.5}
			% \addplot
			% [
				% color=black,
				% opacity=1.0,
				% dashed,
				% domain=0.5:20.0,
				% samples=10,
			% ]{x * (3.0 * pi / 4.0) * (1.0 + exp(-3.0 * \moddelta * (\modtimesym + \modtimeasym ) / ((1.0 - \modparam) + \modparam * x))) / ( (1.0 + exp(-3.0 * \moddelta * \modtimesym / ((1.0 - \modparam) + \modparam * x))) * exp(-(1.0/8.0) * (\modomega * \modtimeasym / ((1.0 - \modparam) + \modparam * x) )^2.0) + 1)};
		
		\nextgroupplot[
			axis on top,
			% x
			xmax=20,
			xticklabels={0.1,1,10},
			extra x ticks={0.5, 5, 20},
			extra x tick labels={0.5,5,20},
			% y
			ylabel={$\left\langle \Performance \right\rangle_N / \SwimmingVelocity$},
			ymin=0.0,
			ymax=2.5,
		]
			\node[anchor=north west] at (axis cs:0.5,2.5) {\textbf{(c):} $\mathit{Re}_{\lambda} = 418$};
			% straight
			\addplot[name path=A, draw=none, forget plot] table [
				x index=2,
				y expr={(\thisrowno{0} - \thisrowno{1}) / (\thisrowno{2} * 0.066)}, %u_\eta = 0.066
				col sep=comma,
				comment chars=\#,
				restrict expr to domain={\thisrowno{3}}{0.0:0.0},
				unbounded coords=discard,
			]{chap_turbulence/data/main_results/merge.csv};
			\addplot[name path=B, draw=none, forget plot] table [
				x index=2,
				y expr={(\thisrowno{0} + \thisrowno{1}) / (\thisrowno{2} * 0.066)}, %u_\eta = 0.066
				col sep=comma,
				comment chars=\#,
				restrict expr to domain={\thisrowno{3}}{0.0:0.0},
				unbounded coords=discard,
			]{chap_turbulence/data/main_results/merge.csv};
			\addplot[ColorBh, opacity=0.25, forget plot, on layer=axis background] fill between[of=A and B];
			\addplot
			[
			color=ColorBh,
			opacity=1.0,
			%line width=1pt,
			only marks,%solid,
			mark=o
			]
			table[
				x index=2,
				y expr={\thisrowno{0} / (\thisrowno{2} * 0.066)}, %u_\eta = 0.066
				col sep=comma,
				comment chars=\#,
				restrict expr to domain={\thisrowno{3}}{0.0:0.0},
				unbounded coords=discard,
			]{chap_turbulence/data/main_results/merge.csv};
			% tss
			\addplot[name path=A, draw=none, forget plot] table [
				x index=2,
				y expr={(\thisrowno{0} - \thisrowno{1}) / (\thisrowno{2} * 0.066)}, %u_\eta = 0.066
				col sep=comma,
				comment chars=\#,
			]{chap_turbulence/data/main_results/max.csv};
			\addplot[name path=B, draw=none, forget plot] table [
				x index=2,
				y expr={(\thisrowno{0} + \thisrowno{1}) / (\thisrowno{2} * 0.066)}, %u_\eta = 0.066
				col sep=comma,
				comment chars=\#,
			]{chap_turbulence/data/main_results/max.csv};
			\addplot[ColorSurf, opacity=0.25, forget plot, on layer=axis background] fill between[of=A and B];
			\addplot
			[
			color=ColorSurf,
			opacity=1.0,
			%line width=1pt,
			only marks,%solid,
			mark=square*
			]
			table[
				x index=2,
				y expr={\thisrowno{0} / (\thisrowno{2} * 0.066)}, %u_\eta = 0.066
				col sep=comma,
				comment chars=\#,
			]{chap_turbulence/data/main_results/max.csv};
			%% y = x
			\addplot
			[
			color=black,
			opacity=1.0,
			%line width=1pt,
			solid,
			on layer=axis background,
			domain=0.5:20,
			]{1};
			% %%% model
			% \def\moddelta{0.02}
			% \def\modtimesym{0.3 / \moddelta}
			% \def\modomega{0.65}
			% \def\modtimeasym{4.25}
			% \def\modparam{0.5}
			% \addplot
			% [
				% color=black,
				% opacity=1.0,
				% dashed,
				% domain=0.5:20.0,
				% samples=10,
			% ]{(3.0 * pi / 4.0) * (1.0 + exp(-3.0 * \moddelta * (\modtimesym + \modtimeasym ) / ((1.0 - \modparam) + \modparam * x))) / ( (1.0 + exp(-3.0 * \moddelta * \modtimesym / ((1.0 - \modparam) + \modparam * x))) * exp(-(1.0/8.0) * (\modomega * \modtimeasym / ((1.0 - \modparam) + \modparam * x) )^2.0) + 1)};

		\nextgroupplot[
			axis on top,
			% x
			xmax=10,
			xticklabels={0.1,1,10},
			extra x ticks={0.5, 5},
			extra x tick labels={0.5,5},
			% y
			ymin=0.0,
			ymax=2.5,
		]
			\node[anchor=north west] at (axis cs:0.5,2.5) {\textbf{(d):} $\mathit{Re}_{\lambda} = 11$};
			%% tss
			%%% average
			\addplot
			[
			color=ColorSurf,
			opacity=1.0,
			%line width=1pt,
			only marks,%solid,
			mark=square*
			]
			table[
				x index=3,
				y expr={\thisrowno{1} / (\thisrowno{3} * 0.21)}, %u_\eta = 0.21
				col sep=comma,
				comment chars=\#,
			]{data/surfers__flow__n_128__re_250/surfer__max_average_velocity_axis_0.csv};
			%%% 95 CLI
			\addplot[name path=A, draw=none, forget plot] table [
				x index=3,
				y expr={(\thisrowno{1} - \thisrowno{2}) / (\thisrowno{3} * 0.21)}, %u_\eta = 0.21
				col sep=comma,
				comment chars=\#,
			]{data/surfers__flow__n_128__re_250/surfer__max_average_velocity_axis_0.csv};
			\addplot[name path=B, draw=none, forget plot] table [
				x index=3,
				y expr={(\thisrowno{1} + \thisrowno{2}) / (\thisrowno{3} * 0.21)}, %u_\eta = 0.21
				col sep=comma,
				comment chars=\#,
			]{data/surfers__flow__n_128__re_250/surfer__max_average_velocity_axis_0.csv};
			\addplot[ColorSurf, opacity=0.25, forget plot, on layer=axis background] fill between[of=A and B];
			%\addlegendentry{95 CI}
			%% straight
			%%% average
			\addplot
			[
			color=ColorBh,
			opacity=1.0,
			%line width=1pt,
			only marks,%solid,
			mark=o
			]
			table[
				x index=3,
				y expr={\thisrowno{1} / (\thisrowno{3} * 0.21)}, %u_\eta = 0.21
				col sep=comma,
				comment chars=\#,
				restrict expr to domain={\thisrowno{4}}{0.0:0.0},
				unbounded coords=discard,
			]{data/surfers__flow__n_128__re_250/surfer__merge_average_velocity_axis_0.csv};
			%%% 95 CLI
			\addplot[name path=A, draw=none, forget plot] table [
				x index=3,
				y expr={(\thisrowno{1} - \thisrowno{2}) / (\thisrowno{3} * 0.21)}, %u_\eta = 0.21
				col sep=comma,
				comment chars=\#,
				restrict expr to domain={\thisrowno{4}}{0.0:0.0},
				unbounded coords=discard,
			]{data/surfers__flow__n_128__re_250/surfer__merge_average_velocity_axis_0.csv};
			\addplot[name path=B, draw=none, forget plot] table [
				x index=3,
				y expr={(\thisrowno{1} + \thisrowno{2}) / (\thisrowno{3} * 0.21)}, %u_\eta = 0.21
				col sep=comma,
				comment chars=\#,
				restrict expr to domain={\thisrowno{4}}{0.0:0.0},
				unbounded coords=discard,
			]{data/surfers__flow__n_128__re_250/surfer__merge_average_velocity_axis_0.csv};
			\addplot[ColorBh, opacity=0.25, forget plot, on layer=axis background] fill between[of=A and B];
			%\addlegendentry{95 CI}
			%% y = x
			\addplot
			[
			color=black,
			opacity=1.0,
			%line width=1pt,
			solid,
			on layer=axis background,
			domain=0.5:10,
			]
			{1};
			% %%% model
			% \def\moddelta{0.015}
			% \def\modtimesym{0.45 / \moddelta}
			% \def\modomega{1.0}
			% \def\modtimeasym{2.5}
			% \def\modparam{0.5}
			% \addplot
			% [
				% color=black,
				% opacity=1.0,
				% dashed,
				% domain=0.5:20.0,
				% samples=10,
			% ]{(3.0 * pi / 4.0) * (1.0 + exp(-3.0 * \moddelta * (\modtimesym + \modtimeasym ) / ((1.0 - \modparam) + \modparam * x))) / ( (1.0 + exp(-3.0 * \moddelta * \modtimesym / ((1.0 - \modparam) + \modparam * x))) * exp(-(1.0/8.0) * (\modomega * \modtimeasym / ((1.0 - \modparam) + \modparam * x) )^2.0) + 1)};
	\end{groupplot}







	
	\begin{groupplot}[
   		group style={
   			group size=2 by 2,
   			y descriptions at=edge left,
   			x descriptions at=edge top,
   			horizontal sep=0.04\linewidth,
   			vertical sep=0.04\linewidth,
   		},
   		axis on top,
		% size
		width=0.5\linewidth,
		%height=0.62\linewidth,
		% x
		xlabel=$\SwimmingVelocity / \FlowVelocityScalar_{\mathrm{rms}}$,
		xmode=log,
		xmin=0.05,
		xtick={0.1,1,2},
		xticklabels={0.1,1,2},
		axis x line*=top,
		% y
		axis y line=none,
   	]
   		\nextgroupplot[
			axis on top,
			% x
			xmax=2,
			% y
			ymode=log,
			ymin=0.5,
			ymax=30,
		]
			%% y = x
			\addplot
			[
			color=black,
			opacity=1.0,
			%line width=1pt,
			solid,
			on layer=axis background,
			domain=0.05:2,
			]
			{10 * x};

		\nextgroupplot[
			axis on top,
			% x
			xmax=1,
			xtick={0.1,0.5,1},
			xticklabels={0.6,3,6},
			% y
			ymode=log,
			ymin=0.5,
			ymax=30,
		]
			%% y = x
			\addplot
			[
			color=black,
			opacity=1.0,
			%line width=1pt,
			solid,
			on layer=axis background,
			domain=0.05:1,
			]
			{10 * x};
		
		\nextgroupplot[
			axis on top,
			% x
			xmax=2,
			% y
			ymin=0.0,
			ymax=2.5,
		]
			%% y = x
			\addplot
			[
			color=black,
			opacity=1.0,
			%line width=1pt,
			solid,
			on layer=axis background,
			domain=0.05:2,
			]{1};

		\nextgroupplot[
			axis on top,
			% x
			xmax=1,
			% y
			ymin=0.0,
			ymax=2.5,
		]
			%% y = x
			\addplot
			[
			color=black,
			opacity=1.0,
			%line width=1pt,
			solid,
			on layer=axis background,
			domain=0.05:1,
			]
			{1};
	\end{groupplot}
\end{tikzpicture}

	\caption[The surfing strategy may double migration speed in turbulence.]{
		The surfing strategy may double migration speed in turbulence.
		Effective upward velocity [$\Performance$, Eq.~\eqref{eq:surfing_performance}] as a function of the swimming velocity ($\SwimmingVelocity$) for a surfer ($\SwimmingDirection=\ControlDirectionOpt$ with optimal time horizon $\TimeHorizon \approx \TimeHorizonOpt$) and for a bottom-heavy swimmer that always swims upwards ($\SwimmingDirection=\ControlDirection_{\mathrm{\NameBhShort}}=\Direction$) for (a) $\mathit{Re}_{\lambda} = 418$ and (b) $\mathit{Re}_{\lambda} = 11$.
		Velocities are normalized either by the Kolmogorov velocity [$\KolmogorovVelocityScale$, Eq.~\eqref{eq:kolmogorov}] (bottom x-axis) or by the root-mean-square velocity $u_{\mathrm{rms}}$ (top x-axis).
		The same data is presented in (c) and (d), but effective upward velocity is normalized by the swimming velocity.
		The solid line represents $\Performance = \SwimmingVelocity$. Shaded areas correspond to 95\% confidence intervals.
	}
	\label{fig:surfing_main_results}
\end{figure}
The vertical migration performance of plankters is then evaluated as the effective velocity, $\Performance$ [Eq.~\eqref{turb:eq:performance}], evaluated at the end of the time span available ($\FinalTime \approx 5 T_L$).
Plankter performance is evaluated in simulations of variable turbulence intensity throughout the study.
The simulations are referred to using the value of their corresponding Reynolds number $\mathit{Re}_{\lambda}$ (see Sec.~\ref{sec:simulating_turbulent_flows} and Tabs.~\ref{tab:snoopy_simulation_parameters} and \ref{tab:jhtdb_flows_simulation_parameters} for more details).

We plot in Fig.~\ref{fig:surfing_main_results} the maximal performance of surfers, evaluated as the effective vertical velocity $\Performance$ for the optimal value of $\TimeHorizon$, compared to that of bottom-heavy swimmers in two simulations of turbulence ($\mathit{Re}_{\lambda} = 418$ and $\mathit{Re}_{\lambda} = 11$).
For the most turbulent case, we show that surfers can reach an effective velocity as large as twice their swimming speed ($\Performance \approx 2 \SwimmingVelocity$) when $\SwimmingVelocity \lesssim \KolmogorovVelocityScale$.
Therefore, they systematically outperform bottom-heavy swimmers, for which turbulence only acts as a random noise of zero mean and whose performance is $\Performance = \SwimmingVelocity$.
In contrast, surfers exploit the turbulent flow by biasing the sampling of vertical flow velocities.

This bias is illustrated in Fig.~\ref{fig:surfing_velocity_sampled}\textbf{(a)}, 
\begin{figure}[t]
	\centering
	\begin{tikzpicture}
	\begin{groupplot}[
		group style={
			group size=2 by 2,
			y descriptions at=edge left,
			%x descriptions at=edge bottom,
			horizontal sep=0.06\linewidth,
			vertical sep=0.06\linewidth,
		},
		% size
		width=0.5\textwidth,
		% y
		ymode=log,
		% layers
		set layers,
		% legend
		legend style={draw=none, fill=none, /tikz/every even column/.append style={column sep=4pt}, at={(1.0, 1.05)}, anchor=south},
		%legend pos=north west,
   		legend cell align=left,
   		legend columns=-1,
	]
		\nextgroupplot[
			axis on top,
			% x
			xmin=-30,
			xmax=30,
			xtick={-30,-15,0,15,30},
			% y
			ylabel={$p(U = \FlowVelocityScalar)$},
			ymin=0.001,
			ymax=0.1,
		]
			\node[anchor=north west] at (axis cs:-30,0.1) {\textbf{(a)}: $\mathit{Re}_{\lambda}$ = 418};
			%% passive
			\addplot
			[
			color=black,
			opacity=1.0,
			%only marks,%solid
			mark=asterisk,
			mark repeat=10,
			]
			table[
				x expr={\thisrowno{0} / 0.066}, %u_\eta = 0.066
				y expr={\thisrowno{1} * 0.066},
				col sep=comma, 
				comment chars=\#,
			]{chap_turbulence/data/flow_sampled/pdf_u_0__pagent.csv};
			\addlegendentry{passive}
			%% us 1.0 straight
			\addplot
			[
			color=ColorBh,
			opacity=1.0,
			%only marks,%solid
			mark=o,
			mark repeat=10,
			]
			table[
				x expr={\thisrowno{0} / 0.066}, %u_\eta = 0.066
				y expr={\thisrowno{1} * 0.066},
				col sep=comma, 
				comment chars=\#,
			]{chap_turbulence/data/flow_sampled/pdf_u_0__agent__us_1o0__surftimeconst_0o0.csv};
			\addlegendentry{\NameBh}
			%% us 1.0 surf
			\addplot
			[
			color=ColorSurf,
			opacity=1.0,
			%only marks,%solid
			mark=square*,
			mark repeat=10,
			]
			table[
				x expr={\thisrowno{0} / 0.066}, %u_\eta = 0.066
				y expr={\thisrowno{1} * 0.066},
				col sep=comma, 
				comment chars=\#,
			]{chap_turbulence/data/flow_sampled/pdf_u_0__agent_full__us_1o0__surftimeconst_5o0.csv};
			\addlegendentry{\NameSurf}



		\nextgroupplot[
			axis on top,
			% x
			xmin=-30,
			xmax=30,
			xtick={-30,-15,0,15,30},
			% y
			ymin=0.001,
			ymax=0.1,
		]
			\node[anchor=north west] at (axis cs:-30,0.1) {\textbf{(b)}: $\mathit{Re}_{\lambda}$ = 418};
			%% passive
			\addplot
			[
			color=black,
			opacity=1.0,
			%only marks,%solid
			mark=asterisk,
			mark repeat=10,
			]
			table[
				x expr={\thisrowno{0} / 0.066}, %u_\eta = 0.066
				y expr={\thisrowno{1} * 0.066},
				col sep=comma, 
				comment chars=\#,
			]{chap_turbulence/data/flow_sampled/pdf_u_2__pagent.csv};
			%\addlegendentry{passive}
			%% us 1.0 straight
			\addplot
			[
			color=ColorBh,
			opacity=1.0,
			%only marks,%solid
			mark=o,
			mark repeat=10,
			]
			table[
				x expr={\thisrowno{0} / 0.066}, %u_\eta = 0.066
				y expr={\thisrowno{1} * 0.066},
				col sep=comma, 
				comment chars=\#,
			]{chap_turbulence/data/flow_sampled/pdf_u_2__agent__us_1o0__surftimeconst_0o0.csv};
			%\addlegendentry{\NameBh}
			%% us 1.0 surf
			\addplot
			[
			color=ColorSurf,
			opacity=1.0,
			%only marks,%solid
			mark=square*,
			mark repeat=10,
			]
			table[
				x expr={\thisrowno{0} / 0.066}, %u_\eta = 0.066
				y expr={\thisrowno{1} * 0.066},
				col sep=comma, 
				comment chars=\#,
			]{chap_turbulence/data/flow_sampled/pdf_u_2__agent_full__us_1o0__surftimeconst_5o0.csv};
			%\addlegendentry{\NameSurf}




		\nextgroupplot[
			axis on top,
			% x
			xlabel=$\FlowVelocity_\DirectionScalar / \KolmogorovVelocityScale$,
			xmin=-4,
			xmax=4,
			% y
			ylabel={$p(U = \FlowVelocityScalar)$},
			ymin=0.01,
			ymax=1,
		]
			\node[anchor=north west] at (axis cs:-4,1) {\textbf{(c)}: $\mathit{Re}_{\lambda}$ = 11};
			%% passive
			\addplot[
				ColorPassive,
				%only marks,
				mark=star,
			] table [
				x expr={\thisrowno{0} / 0.21},
				y expr={\thisrowno{1} * 0.21},
				col sep=comma, 
				comment chars=\#,
				unbounded coords=discard,
			] {data/tracers__flow__n_128__re_250/tracer__flow_velocity_sampled_pdfs.csv};
			%\addlegendentry{passive}
			%% bh
			\addplot[
				ColorBh,
				%only marks,
				mark=o,
			] table [
				x expr={\thisrowno{0} / 0.21},
				y expr={\thisrowno{1} * 0.21},
				col sep=comma, 
				comment chars=\#,
				unbounded coords=discard,
			] {data/surfers__flow__n_128__re_250/surfer__vs_1o0__surftimeconst_0o0__flow_velocity_sampled_pdfs.csv};
			%\addlegendentry{\NameBh}
			%% surfer
			\addplot[
				ColorSurf,
				%only marks,
				mark=square*,
			] table [
				x expr={\thisrowno{0} / 0.21},
				y expr={\thisrowno{1} * 0.21},
				col sep=comma, 
				comment chars=\#,
				unbounded coords=discard,
			] {data/surfers__flow__n_128__re_250/surfer__vs_1o0__surftimeconst_2o0__flow_velocity_sampled_pdfs.csv};
			%\addlegendentry{\NameSurf}




		
		\nextgroupplot[
			axis on top,
			% x
			xlabel=$\FlowVelocityScalar_y / \KolmogorovVelocityScale$,
			xmin=-4,
			xmax=4,
			% y
			ymin=0.01,
			ymax=1,
		]
			\node[anchor=north west] at (axis cs:-4,1) {\textbf{(d)}: $\mathit{Re}_{\lambda}$ = 11};
			%% passive
			\addplot[
				ColorPassive,
				%only marks,
				mark=star,
			] table [
				x expr={\thisrowno{0} / 0.21},
				y expr={\thisrowno{3} * 0.21},
				col sep=comma, 
				comment chars=\#,
				unbounded coords=discard,
			] {data/tracers__flow__n_128__re_250/tracer__flow_velocity_sampled_pdfs.csv};
			%\addlegendentry{passive}
			%% bh
			\addplot[
				ColorBh,
				%only marks,
				mark=o,
			] table [
				x expr={\thisrowno{0} / 0.21},
				y expr={\thisrowno{3} * 0.21},
				col sep=comma, 
				comment chars=\#,
				unbounded coords=discard,
			] {data/surfers__flow__n_128__re_250/surfer__vs_1o0__surftimeconst_0o0__flow_velocity_sampled_pdfs.csv};
			%\addlegendentry{\NameBh}
			%% surfer
			\addplot[
				ColorSurf,
				%only marks,
				mark=square*,
			] table [
				x expr={\thisrowno{0} / 0.21},
				y expr={\thisrowno{3} * 0.21},
				col sep=comma, 
				comment chars=\#,
				unbounded coords=discard,
			] {data/surfers__flow__n_128__re_250/surfer__vs_1o0__surftimeconst_2o0__flow_velocity_sampled_pdfs.csv};
			%\addlegendentry{\NameSurf}
	\end{groupplot}
\end{tikzpicture}

	\caption[The surfing strategy induces preferential sampling of the flow velocity.]{
		The surfing strategy induces preferential sampling of the flow velocity.
		Probability density function of the vertical flow velocity sampled along trajectories of passives particles, bottom-heavy swimmers and surfers. 
		The swimming speed of simulated plankters is set to $\SwimmingVelocity = \KolmogorovVelocityScale$ and the surfing parameter is set to its optimal value $\TimeHorizon \approx \TimeHorizonOpt$ ($\TimeHorizon = 5 \KolmogorovTimeScale$ for $\mathit{Re}_{\lambda} = 418$ and $\TimeHorizon = 2 \KolmogorovTimeScale$ for $\mathit{Re}_{\lambda} = 11$).
	}
	\label{fig:surfing_velocity_sampled}
\end{figure}
where we show the distribution of the vertical velocity component of the turbulent flow sampled by surfers, bottom-heavy swimmers, and passive particles.
One can see that the Gaussian distribution of the vertical component is not centered on zero but shifted towards positive values for surfers.
The positive shift is approximately equal to the Kolmogorov velocity scale $\KolmogorovVelocityScale$.
The velocity being less widely distributed in weak turbulence ($\mathit{Re}_{\lambda} = 11$), this effect is even more apparent in Fig.~\ref{fig:surfing_velocity_sampled}\textbf{(c)}.

On the contrary, as illustrated in Fig.~\ref{fig:surfing_velocity_sampled}\textbf{(b)} and Fig.~\ref{fig:surfing_velocity_sampled}\textbf{(d)}, the horizontal component of the sampled flow velocity is mainly unchanged.
We still notice a slightly wider spreading of the sampled horizontal flow velocity distribution for surfers in Fig.~\ref{fig:surfing_velocity_sampled}\textbf{(d)}.
As noted in Chap.~\ref{chap:the_surfing_strategy}, Sec.~\ref{sec:the_surfing_strategy_interpretation}, surfers also seek to exploit horizontal currents to reach vertical ones faster.
This would lead to a slight accumulation of surfers at the maxima of horizontal velocity, hence the wider distribution of the sampled horizontal velocity.
This shows that sensing flow gradients is beneficial for navigation in turbulence and that surfing allows plankton to exploit this information in order to sample beneficial currents.

The preferential sampling of the vertical flow velocity is the desired effect. 
However, surfing also causes weak yet significant preferential sampling of the gradients (Fig.~\ref{fig:surfing_gradient_sampled}).
\begin{figure}[t]
	\centering
	% Reynolds
\begin{tikzpicture}
	\begin{groupplot}[
			group style={
				group size=3 by 3,
				y descriptions at=edge left,
				horizontal sep=0.05\linewidth,
				vertical sep=0.08\linewidth,
			},
			% size
			width=0.35\textwidth,
			% y
			ylabel={pdf},
			ymin=0.25,
			ymax=2,
			log basis y=2,
			ymode=log,
			% x
			xmin=-0.5,
			xmax=0.5,
			xtick={-0.5,-0.25,0.0,0.25,0.5},
			xticklabels={-0.5,,0,,0.5},
			% layers
			set layers,
			% legend
			legend style={draw=none, fill=none, /tikz/every even column/.append style={column sep=4pt}, at={(1.5, 1.05)}, anchor=south},
			%legend pos=north west,
	   		legend cell align=left,
	   		legend columns=-1,
		]
		% noisy measure of target direction
		\nextgroupplot[
			axis on top,
			% x
			xlabel=$\partial \FlowVelocityScalar_x / \partial x \, \KolmogorovTimeScale$,
		]
			\node[anchor=north west] at (axis cs:-0.5,2) {\textbf{(a)}};
			%% passive
			\addplot[
				ColorPassive,
				%only marks,
				mark=star,
			] table [
				x expr={\thisrowno{0} * 0.088},
				y expr={\thisrowno{5} / 0.088},
				col sep=comma, 
				comment chars=\#,
				unbounded coords=discard,
			] {data/tracers__flow__n_128__re_250/tracer__flow_gradients_sampled_pdfs.csv};
			\addlegendentry{passive}
			%% bh
			\addplot[
				ColorBh,
				%only marks,
				mark=o,
			] table [
				x expr={\thisrowno{0} * 0.088},
				y expr={\thisrowno{5} / 0.088},
				col sep=comma, 
				comment chars=\#,
				unbounded coords=discard,
			] {data/surfers__flow__n_128__re_250/surfer__vs_1o0__surftimeconst_0o0__flow_gradients_sampled_pdfs.csv};
			\addlegendentry{\NameBh}
			%% surfer
			\addplot[
				ColorSurf,
				%only marks,
				mark=square*,
			] table [
				x expr={\thisrowno{0} * 0.088},
				y expr={\thisrowno{5} / 0.088},
				col sep=comma, 
				comment chars=\#,
				unbounded coords=discard,
			] {data/surfers__flow__n_128__re_250/surfer__vs_1o0__surftimeconst_2o0__flow_gradients_sampled_pdfs.csv};
			\addlegendentry{\NameSurf}
		% noisy measure of target direction
		\nextgroupplot[
			axis on top,
			% x
			xlabel=$\partial \FlowVelocityScalar_x / \partial y \, \KolmogorovTimeScale$,
		]
			\node[anchor=north west] at (axis cs:-0.5,2) {\textbf{(b)}};
			%% passive
			\addplot[
				ColorPassive,
				%only marks,
				mark=star,
			] table [
				x expr={\thisrowno{0} * 0.088},
				y expr={\thisrowno{6} / 0.088},
				col sep=comma, 
				comment chars=\#,
				unbounded coords=discard,
			] {data/tracers__flow__n_128__re_250/tracer__flow_gradients_sampled_pdfs.csv};
			%\addlegendentry{passive}
			%% bh
			\addplot[
				ColorBh,
				%only marks,
				mark=o,
			] table [
				x expr={\thisrowno{0} * 0.088},
				y expr={\thisrowno{6} / 0.088},
				col sep=comma, 
				comment chars=\#,
				unbounded coords=discard,
			] {data/surfers__flow__n_128__re_250/surfer__vs_1o0__surftimeconst_0o0__flow_gradients_sampled_pdfs.csv};
			%\addlegendentry{\NameBh}
			%% surfer
			\addplot[
				ColorSurf,
				%only marks,
				mark=square*,
			] table [
				x expr={\thisrowno{0} * 0.088},
				y expr={\thisrowno{6} / 0.088},
				col sep=comma, 
				comment chars=\#,
				unbounded coords=discard,
			] {data/surfers__flow__n_128__re_250/surfer__vs_1o0__surftimeconst_2o0__flow_gradients_sampled_pdfs.csv};
			%\addlegendentry{\NameSurf}
		% noisy measure of target direction
		\nextgroupplot[
			axis on top,
			% x
			xlabel=$\partial \FlowVelocityScalar_x / \partial z \, \KolmogorovTimeScale$,
		]
			\node[anchor=north west] at (axis cs:-0.5,2) {\textbf{(c)}};
			%% passive
			\addplot[
				ColorPassive,
				%only marks,
				mark=star,
			] table [
				x expr={\thisrowno{0} * 0.088},
				y expr={\thisrowno{4} / 0.088},
				col sep=comma, 
				comment chars=\#,
				unbounded coords=discard,
			] {data/tracers__flow__n_128__re_250/tracer__flow_gradients_sampled_pdfs.csv};
			%\addlegendentry{passive}
			%% bh
			\addplot[
				ColorBh,
				%only marks,
				mark=o,
			] table [
				x expr={\thisrowno{0} * 0.088},
				y expr={\thisrowno{4} / 0.088},
				col sep=comma, 
				comment chars=\#,
				unbounded coords=discard,
			] {data/surfers__flow__n_128__re_250/surfer__vs_1o0__surftimeconst_0o0__flow_gradients_sampled_pdfs.csv};
			%\addlegendentry{\NameBh}
			%% surfer
			\addplot[
				ColorSurf,
				%only marks,
				mark=square*,
			] table [
				x expr={\thisrowno{0} * 0.088},
				y expr={\thisrowno{4} / 0.088},
				col sep=comma, 
				comment chars=\#,
				unbounded coords=discard,
			] {data/surfers__flow__n_128__re_250/surfer__vs_1o0__surftimeconst_2o0__flow_gradients_sampled_pdfs.csv};
			%\addlegendentry{\NameSurf}
		% noisy measure of target direction
		\nextgroupplot[
			axis on top,
			% x
			xlabel=$\partial \FlowVelocityScalar_y / \partial x \, \KolmogorovTimeScale$,
		]
			\node[anchor=north west] at (axis cs:-0.5,2) {\textbf{(d)}};
			%% passive
			\addplot[
				ColorPassive,
				%only marks,
				mark=star,
			] table [
				x expr={\thisrowno{0} * 0.088},
				y expr={\thisrowno{8} / 0.088},
				col sep=comma, 
				comment chars=\#,
				unbounded coords=discard,
			] {data/tracers__flow__n_128__re_250/tracer__flow_gradients_sampled_pdfs.csv};
			%\addlegendentry{passive}
			%% bh
			\addplot[
				ColorBh,
				%only marks,
				mark=o,
			] table [
				x expr={\thisrowno{0} * 0.088},
				y expr={\thisrowno{8} / 0.088},
				col sep=comma, 
				comment chars=\#,
				unbounded coords=discard,
			] {data/surfers__flow__n_128__re_250/surfer__vs_1o0__surftimeconst_0o0__flow_gradients_sampled_pdfs.csv};
			%\addlegendentry{\NameBh}
			%% surfer
			\addplot[
				ColorSurf,
				%only marks,
				mark=square*,
			] table [
				x expr={\thisrowno{0} * 0.088},
				y expr={\thisrowno{8} / 0.088},
				col sep=comma, 
				comment chars=\#,
				unbounded coords=discard,
			] {data/surfers__flow__n_128__re_250/surfer__vs_1o0__surftimeconst_2o0__flow_gradients_sampled_pdfs.csv};
			%\addlegendentry{\NameSurf}
		% noisy measure of target direction
		\nextgroupplot[
			axis on top,
			% x
			xlabel=$\partial \FlowVelocityScalar_y / \partial y \, \KolmogorovTimeScale$,
		]
			\node[anchor=north west] at (axis cs:-0.5,2) {\textbf{(e)}};
			%% passive
			\addplot[
				ColorPassive,
				%only marks,
				mark=star,
			] table [
				x expr={\thisrowno{0} * 0.088},
				y expr={\thisrowno{9} / 0.088},
				col sep=comma, 
				comment chars=\#,
				unbounded coords=discard,
			] {data/tracers__flow__n_128__re_250/tracer__flow_gradients_sampled_pdfs.csv};
			%\addlegendentry{passive}
			%% bh
			\addplot[
				ColorBh,
				%only marks,
				mark=o,
			] table [
				x expr={\thisrowno{0} * 0.088},
				y expr={\thisrowno{9} / 0.088},
				col sep=comma, 
				comment chars=\#,
				unbounded coords=discard,
			] {data/surfers__flow__n_128__re_250/surfer__vs_1o0__surftimeconst_0o0__flow_gradients_sampled_pdfs.csv};
			%\addlegendentry{\NameBh}
			%% surfer
			\addplot[
				ColorSurf,
				%only marks,
				mark=square*,
			] table [
				x expr={\thisrowno{0} * 0.088},
				y expr={\thisrowno{9} / 0.088},
				col sep=comma, 
				comment chars=\#,
				unbounded coords=discard,
			] {data/surfers__flow__n_128__re_250/surfer__vs_1o0__surftimeconst_2o0__flow_gradients_sampled_pdfs.csv};
			%\addlegendentry{\NameSurf}
		\nextgroupplot[
			axis on top,
			% x
			xlabel=$\partial \FlowVelocityScalar_y / \partial z \, \KolmogorovTimeScale$,
		]
			\node[anchor=north west] at (axis cs:-0.5,2) {\textbf{(f)}};
			%% passive
			\addplot[
				ColorPassive,
				%only marks,
				mark=star,
			] table [
				x expr={\thisrowno{0} * 0.088},
				y expr={\thisrowno{7} / 0.088},
				col sep=comma, 
				comment chars=\#,
				unbounded coords=discard,
			] {data/tracers__flow__n_128__re_250/tracer__flow_gradients_sampled_pdfs.csv};
			%\addlegendentry{passive}
			%% bh
			\addplot[
				ColorBh,
				%only marks,
				mark=o,
			] table [
				x expr={\thisrowno{0} * 0.088},
				y expr={\thisrowno{7} / 0.088},
				col sep=comma, 
				comment chars=\#,
				unbounded coords=discard,
			] {data/surfers__flow__n_128__re_250/surfer__vs_1o0__surftimeconst_0o0__flow_gradients_sampled_pdfs.csv};
			%\addlegendentry{\NameBh}
			%% surfer
			\addplot[
				ColorSurf,
				%only marks,
				mark=square*,
			] table [
				x expr={\thisrowno{0} * 0.088},
				y expr={\thisrowno{7} / 0.088},
				col sep=comma, 
				comment chars=\#,
				unbounded coords=discard,
			] {data/surfers__flow__n_128__re_250/surfer__vs_1o0__surftimeconst_2o0__flow_gradients_sampled_pdfs.csv};
			%\addlegendentry{\NameSurf}
		\nextgroupplot[
			axis on top,
			% x
			xlabel=$\partial \FlowVelocityScalar_z / \partial x \, \KolmogorovTimeScale$,
		]
			\node[anchor=north west] at (axis cs:-0.5,2) {\textbf{(g)}};
			%% passive
			\addplot[
				ColorPassive,
				%only marks,
				mark=star,
			] table [
				x expr={\thisrowno{0} * 0.088},
				y expr={\thisrowno{2} / 0.088},
				col sep=comma, 
				comment chars=\#,
				unbounded coords=discard,
			] {data/tracers__flow__n_128__re_250/tracer__flow_gradients_sampled_pdfs.csv};
			%\addlegendentry{passive}
			%% bh
			\addplot[
				ColorBh,
				%only marks,
				mark=o,
			] table [
				x expr={\thisrowno{0} * 0.088},
				y expr={\thisrowno{2} / 0.088},
				col sep=comma, 
				comment chars=\#,
				unbounded coords=discard,
			] {data/surfers__flow__n_128__re_250/surfer__vs_1o0__surftimeconst_0o0__flow_gradients_sampled_pdfs.csv};
			%\addlegendentry{\NameBh}
			%% surfer
			\addplot[
				ColorSurf,
				%only marks,
				mark=square*,
			] table [
				x expr={\thisrowno{0} * 0.088},
				y expr={\thisrowno{2} / 0.088},
				col sep=comma, 
				comment chars=\#,
				unbounded coords=discard,
			] {data/surfers__flow__n_128__re_250/surfer__vs_1o0__surftimeconst_2o0__flow_gradients_sampled_pdfs.csv};
			%\addlegendentry{\NameSurf}
		\nextgroupplot[
			axis on top,
			% x
			xlabel=$\partial \FlowVelocityScalar_z / \partial y \, \KolmogorovTimeScale$,
		]
			\node[anchor=north west] at (axis cs:-0.5,2) {\textbf{(h)}};
			%% passive
			\addplot[
				ColorPassive,
				%only marks,
				mark=star,
			] table [
				x expr={\thisrowno{0} * 0.088},
				y expr={\thisrowno{3} / 0.088},
				col sep=comma, 
				comment chars=\#,
				unbounded coords=discard,
			] {data/tracers__flow__n_128__re_250/tracer__flow_gradients_sampled_pdfs.csv};
			%\addlegendentry{passive}
			%% bh
			\addplot[
				ColorBh,
				%only marks,
				mark=o,
			] table [
				x expr={\thisrowno{0} * 0.088},
				y expr={\thisrowno{3} / 0.088},
				col sep=comma, 
				comment chars=\#,
				unbounded coords=discard,
			] {data/surfers__flow__n_128__re_250/surfer__vs_1o0__surftimeconst_0o0__flow_gradients_sampled_pdfs.csv};
			%\addlegendentry{\NameBh}
			%% surfer
			\addplot[
				ColorSurf,
				%only marks,
				mark=square*,
			] table [
				x expr={\thisrowno{0} * 0.088},
				y expr={\thisrowno{3} / 0.088},
				col sep=comma, 
				comment chars=\#,
				unbounded coords=discard,
			] {data/surfers__flow__n_128__re_250/surfer__vs_1o0__surftimeconst_2o0__flow_gradients_sampled_pdfs.csv};
			%\addlegendentry{\NameSurf}
		\nextgroupplot[
			axis on top,
			% x
			xlabel=$\partial \FlowVelocityScalar_z / \partial z \, \KolmogorovTimeScale$,
		]
			\node[anchor=north west] at (axis cs:-0.5,2) {\textbf{(i)}};
			%% passive
			\addplot[
				ColorPassive,
				%only marks,
				mark=star,
			] table [
				x expr={\thisrowno{0} * 0.088},
				y expr={\thisrowno{1} / 0.088},
				col sep=comma, 
				comment chars=\#,
				unbounded coords=discard,
			] {data/tracers__flow__n_128__re_250/tracer__flow_gradients_sampled_pdfs.csv};
			%\addlegendentry{passive}
			%% bh
			\addplot[
				ColorBh,
				%only marks,
				mark=o,
			] table [
				x expr={\thisrowno{0} * 0.088},
				y expr={\thisrowno{1} / 0.088},
				col sep=comma, 
				comment chars=\#,
				unbounded coords=discard,
			] {data/surfers__flow__n_128__re_250/surfer__vs_1o0__surftimeconst_0o0__flow_gradients_sampled_pdfs.csv};
			%\addlegendentry{\NameBh}
			%% surfer
			\addplot[
				ColorSurf,
				%only marks,
				mark=square*,
			] table [
				x expr={\thisrowno{0} * 0.088},
				y expr={\thisrowno{1} / 0.088},
				col sep=comma, 
				comment chars=\#,
				unbounded coords=discard,
			] {data/surfers__flow__n_128__re_250/surfer__vs_1o0__surftimeconst_2o0__flow_gradients_sampled_pdfs.csv};
			%\addlegendentry{\NameSurf}
	\end{groupplot}
	% stuff
	%\node[anchor=north west] at (rel axis cs:0.83,1) {\textbf{(a)}};
	%\node[anchor=north west] at (rel axis cs:1.98,1) {\textbf{(b)}};
\end{tikzpicture}

	\caption[The surfing strategy induces preferential sampling of the flow velocity gradients.]{
		The surfing strategy induces preferential sampling of the flow velocity gradients.
		Probability density function of the flow velocity gradients components, for bottom-heavy swimmers and surfers. 
		The flow simulation corresponds to the case $\mathit{Re}_{\lambda} = 11$. The swimming speed of simulated plankters is set to $\SwimmingVelocity = \KolmogorovVelocityScale$ and the surfing parameter is set to its optimal value $\TimeHorizon \approx \TimeHorizonOpt$ ($\TimeHorizon = 2 \KolmogorovTimeScale$).
	}
	\label{fig:surfing_gradient_sampled}
\end{figure}
First we observe narrower distributions of the horizontal gradients of $\FlowVelocityScalar_{z}$ [Fig.~\ref{fig:surfing_gradient_sampled}\textbf{(g,h)}]. 
As maxima of the vertical velocity are located where its gradient is minimum, the preferential sampling of the vertical velocity leads to the preferential sampling of smaller values of $\partial \FlowVelocityScalar_z/\partial x$ and $\partial \FlowVelocityScalar_z/\partial y$.
In a similar way, we also note a slight preferential sampling of smaller values of $\partial \FlowVelocityScalar_x/\partial y$ [Fig.~\ref{fig:surfing_gradient_sampled}\textbf{(b)}] and $\partial \FlowVelocityScalar_y/\partial x$ [Fig.~\ref{fig:surfing_gradient_sampled}\textbf{(d)}] that result directly of the slight preferential sampling of high horizontal velocity observable in Fig.~\ref{fig:surfing_velocity_sampled}\textbf{(d)}.

We also observe a positive shift of the distributions of $\partial \FlowVelocityScalar_x/\partial x$ [Fig.~\ref{fig:surfing_gradient_sampled}\textbf{(a)}] and $\partial \FlowVelocityScalar_y/\partial y$ [Fig.~\ref{fig:surfing_gradient_sampled}\textbf{(e)}] that indicates a preferential sampling.
To explain these shifts, one may consider the behavior of the surfing strategy in a linear symmetric flow (component of the flow responsible for compression and extension axes).
As described in Chap.~\ref{chap:the_surfing_strategy}, Sec.~\ref{sec:the_surfing_strategy_linear_sym}, in such a flow, surfers swim horizontally to exploit the maximal extension axis of the flow by tilting in its direction.
This preferential swimming direction in the horizontal plane certainly leads to this preferential sampling of $\partial \FlowVelocityScalar_x/\partial x$ [Fig.~\ref{fig:surfing_gradient_sampled}\textbf{(a)}] and $\partial \FlowVelocityScalar_y/\partial y$ [Fig.~\ref{fig:surfing_gradient_sampled}\textbf{(e)}].
Due to incompressibility, the vertical compression axis is also preferentially sampled causing a negative shift of $\partial \FlowVelocityScalar_z/\partial z$ [Fig.~\ref{fig:surfing_gradient_sampled}\textbf{(i)}].

Overall, the surfing behavior may double migration speed in turbulence. 
This effect is explained by the preferential sampling of upward vertical flow velocity that also causes a weak preferential sampling of the velocity gradients.

\subsection{Surfing time horizon $\TimeHorizon$}\label{sec:surfing_time_horizon}

We recall that the surfing strategy depends on a free parameter: the time horizon $\TimeHorizon$.
All previous results correspond to the surfing performance of surfers using the optimal horizon time, noted $\TimeHorizonOpt$.
This value has been determined numerically: for a given swimming speed $\SwimmingVelocity$, we looked for the value that maximizes performance when $\TimeHorizon$ varied in the range $[0, 8\KolmogorovTimeScale]$ [Fig.~\ref{fig:surfing_parameter_tau_vs}].
\begin{figure}%[H]
	\centering
	\begin{tikzpicture}
	\node[anchor=center] at (3.6,5.3) {$\SwimmingVelocity =$};
	\begin{groupplot}[
		group style={
			group size=2 by 1,
			y descriptions at=edge left,
			%x descriptions at=edge bottom,
			horizontal sep=0.04\linewidth,
			%vertical sep=0.06\linewidth,
		},
		% size
		width=0.5\textwidth,
		% y
		ymin=0,
		ymax=2.5,
		ylabel={$\left\langle \Performance \right\rangle_N / \SwimmingVelocity$},
		% x
		xlabel=$\TimeHorizon / \KolmogorovTimeScale$,
		% layers
		set layers,
		% legend
		legend style={draw=none, fill=none, /tikz/every even column/.append style={column sep=4pt}, at={(1.0, 1.05)}, anchor=south},
		%legend pos=north west,
   		legend cell align=left,
   		legend columns=-1,
	]
		\nextgroupplot[
			axis on top,
			% y
			extra y ticks={0.5, 1.5, 2.5},
			% x
			xmin=0,
			xmax=8,
		]
			\node[anchor=north west] at (axis cs:0.0,2.5) {\textbf{(a):} $\mathit{Re}_{\lambda} = 418$};
			%% us 1.0
			%%% 95 CI
			\addplot[name path=A, draw=none, forget plot] table [
				x index=3,
				y expr={(\thisrowno{0} - \thisrowno{1}) / (\thisrowno{2} * 0.066)}, %u_\eta = 0.066
				col sep=comma, 
				comment chars=\#,
				restrict expr to domain={\thisrowno{2}}{1.0:1.0},
				unbounded coords=discard,
			]{chap_turbulence/data/main_results/merge.csv};
			\addplot[name path=B, draw=none, forget plot] table [
				x index=3, 
				y expr={(\thisrowno{0} + \thisrowno{1}) / (\thisrowno{2} * 0.066)}, %u_\eta = 0.066
				col sep=comma,
				comment chars=\#,
				restrict expr to domain={\thisrowno{2}}{1.0:1.0},
				unbounded coords=discard,
			]{chap_turbulence/data/main_results/merge.csv};
			\addplot[ColorSurf!100!ColorVs, opacity=0.25, forget plot, on layer=axis background] fill between[of=A and B];
			%%% average
			\addplot
			[
			color=ColorSurf!100!ColorVs,
			opacity=1.0,
			only marks,%solid
			mark=square*
			]
			table[
				x index=3, 
				y expr={\thisrowno{0} / (\thisrowno{2} * 0.066)}, %u_\eta = 0.066
				col sep=comma, 
				comment chars=\#,
				restrict expr to domain={\thisrowno{2}}{1.0:1.0},
				unbounded coords=discard,
			]{chap_turbulence/data/main_results/merge.csv};
			\addlegendentry{$\KolmogorovVelocityScale$}
			%%% fit
			\addplot
			[
			color=ColorSurf!100!ColorVs,
			opacity=1.0,
			solid,
			forget plot
			]
			table[
				x index=0, 
				y expr={\thisrowno{1} / (1.0 * 0.066)}, %u_\eta = 0.066
				col sep=comma, 
				comment chars=\#,
				unbounded coords=discard,
			]{chap_turbulence/data/main_results/fits_low.csv};
			%%%% model
			%\addplot
			%[
			%color=colortss!33!colorus,
			%opacity=1.0,
			%dashed,
			%forget plot
			%]
			%table[
			%    x index=0, 
			%    y expr={cos(deg(0.24 * (4.4 - \thisrowno{0}))) / cos(deg(0.24 * 4.4))}, %u_\eta = 0.066
			%    %y expr={cos(deg(\thisrowno{0}))}, %u_\eta = 0.066
			%    col sep=comma, 
			%    comment chars=\#,
			%    unbounded coords=discard,
			%]{data/jhtdb_more/fits_average_velocity_axis_0__agent.csv};
			%% us 4.0
			%%% 95 CI
			\addplot[name path=A, draw=none, forget plot] table [
				x index=3, 
				y expr={(\thisrowno{0} - \thisrowno{1}) / (\thisrowno{2} * 0.066)}, %u_\eta = 0.066
				col sep=comma, 
				comment chars=\#,
				restrict expr to domain={\thisrowno{2}}{4.0:4.0},
				unbounded coords=discard,
			]{chap_turbulence/data/main_results/merge.csv};
			\addplot[name path=B, draw=none, forget plot] table [
				x index=3, 
				y expr={(\thisrowno{0} + \thisrowno{1}) / (\thisrowno{2} * 0.066)}, %u_\eta = 0.066
				col sep=comma, 
				comment chars=\#,
				restrict expr to domain={\thisrowno{2}}{4.0:4.0},
				unbounded coords=discard,
			]{chap_turbulence/data/main_results/merge.csv};
			\addplot[ColorSurf!50!ColorVs, opacity=0.25, forget plot, on layer=axis background] fill between[of=A and B];
			%%% average
			\addplot
			[
			color=ColorSurf!50!ColorVs,
			opacity=1.0,
			only marks,%solid
			mark=pentagon
			]
			table[
				x index=3, 
				y expr={\thisrowno{0} / (\thisrowno{2} * 0.066)}, %u_\eta = 0.066
				col sep=comma, 
				comment chars=\#,
				restrict expr to domain={\thisrowno{2}}{4.0:4.0},
				unbounded coords=discard,
			]{chap_turbulence/data/main_results/merge.csv};
			\addlegendentry{$4 \KolmogorovVelocityScale$}
			%%% fit
			\addplot
			[
			color=ColorSurf!50!ColorVs,
			opacity=1.0,
			solid,
			forget plot
			]
			table[
				x index=0, 
				y expr={\thisrowno{2} / (4.0 * 0.066)}, %u_\eta = 0.066
				col sep=comma, 
				comment chars=\#,
				unbounded coords=discard,
			]{chap_turbulence/data/main_results/fits_more.csv};
			%%%% model
			%\addplot
			%[
			%color=colortss!66!colorus,
			%opacity=1.0,
			%dashed,
			%forget plot
			%]
			%table[
			%    x index=0, 
			%    y expr={cos(deg(0.24 * (3.37 - \thisrowno{0}))) / cos(deg(0.24 * 3.37))}, %u_\eta = 0.066
			%    %y expr={cos(deg(\thisrowno{0}))}, %u_\eta = 0.066
			%    col sep=comma, 
			%    comment chars=\#,
			%    unbounded coords=discard,
			%]{data/jhtdb_more/fits_average_velocity_axis_0__agent.csv};
			%% us 8.0
			%%% 95 CI
			\addplot[name path=A, draw=none, forget plot] table [
				x index=3, 
				y expr={(\thisrowno{0} - \thisrowno{1}) / (\thisrowno{2} * 0.066)}, %u_\eta = 0.066
				col sep=comma, 
				comment chars=\#,
				restrict expr to domain={\thisrowno{2}}{8.0:8.0},
				unbounded coords=discard,
			]{chap_turbulence/data/main_results/merge.csv};
			\addplot[name path=B, draw=none, forget plot] table [
				x index=3, 
				y expr={(\thisrowno{0} + \thisrowno{1}) / (\thisrowno{2} * 0.066)}, %u_\eta = 0.066
				col sep=comma, 
				comment chars=\#,
				restrict expr to domain={\thisrowno{2}}{8.0:8.0},
				unbounded coords=discard,
			]{chap_turbulence/data/main_results/merge.csv};
			\addplot[ColorSurf!0!ColorVs, opacity=0.25, forget plot, on layer=axis background] fill between[of=A and B];
			%%% average
			\addplot
			[
			color=ColorSurf!0!ColorVs,
			opacity=1.0,
			only marks,%solid
			mark=*
			]
			table[
				x index=3, 
				y expr={\thisrowno{0} / (\thisrowno{2} * 0.066)}, %u_\eta = 0.066
				col sep=comma, 
				comment chars=\#,
				restrict expr to domain={\thisrowno{2}}{8.0:8.0},
				unbounded coords=discard,
			]{chap_turbulence/data/main_results/merge.csv};
			\addlegendentry{$8 \KolmogorovVelocityScale$}
			%%% fit
			\addplot
			[
			color=ColorSurf!0!ColorVs,
			opacity=1.0,
			solid,
			forget plot
			]
			table[
				x index=0, 
				y expr={\thisrowno{6} / (8.0 * 0.066)}, %u_\eta = 0.066
				col sep=comma, 
				comment chars=\#,
				unbounded coords=discard,
			]{chap_turbulence/data/main_results/fits_even_more.csv};
			%%%% model
			%\addplot
			%[
			%color=colortss!100!colorus,
			%opacity=1.0,
			%dashed,
			%forget plot
			%]
			%table[
			%    x index=0, 
			%    y expr={cos(deg(0.24 * (2.57 - \thisrowno{0}))) / cos(deg(0.24 * 2.57))}, %u_\eta = 0.066
			%    %y expr={cos(deg(\thisrowno{0}))}, %u_\eta = 0.066
			%    col sep=comma, 
			%    comment chars=\#,
			%    unbounded coords=discard,
			%]{data/jhtdb_more/fits_average_velocity_axis_0__agent.csv};
			%% y = x
			\addplot
			[
			color=gray!50!white,
			opacity=1.0,
			%line width=1pt, 
			solid, 
			on layer=axis background,
			domain=0:10,
			]{1};

		\nextgroupplot[
			axis on top,
			% x
			xmin=0,
			xmax=8,
		]
			\node[anchor=north west] at (axis cs:0.0,2.5) {\textbf{(b):} $\mathit{Re}_{\lambda} = 11$};
			% %% us 0.5
			% %%% 95 CI
			% \addplot[name path=A, draw=none, forget plot] table [
				% x index=4,
				% y expr={(\thisrowno{1} - \thisrowno{2}) / (\thisrowno{3} * 0.21)}, %u_\eta = 0.21
				% col sep=comma,
				% comment chars=\#,
				% restrict expr to domain={\thisrowno{3}}{0.5:0.5},
				% unbounded coords=discard,
			% ]{data/surfers__flow__n_128__re_250/surfer__merge_average_velocity_axis_0.csv};
			% \addplot[name path=B, draw=none, forget plot] table [
				% x index=4,
				% y expr={(\thisrowno{1} + \thisrowno{2}) / (\thisrowno{3} * 0.21)}, %u_\eta = 0.21
				% col sep=comma,
				% comment chars=\#,
				% restrict expr to domain={\thisrowno{3}}{0.5:0.5},
				% unbounded coords=discard,
			% ]{data/surfers__flow__n_128__re_250/surfer__merge_average_velocity_axis_0.csv};
			% \addplot[ColorSurf!100!ColorVs, opacity=0.25, forget plot, on layer=axis background] fill between[of=A and B];
			% %%% average
			% \addplot
			% [
			% color=ColorSurf!100!ColorVs,
			% opacity=1.0,
			% only marks,%solid
			% mark=triangle*
			% ]
			% table[
				% x index=4,
				% y expr={\thisrowno{1} / (\thisrowno{3} * 0.21)}, %u_\eta = 0.21
				% col sep=comma,
				% comment chars=\#,
				% restrict expr to domain={\thisrowno{3}}{0.5:0.5},
				% unbounded coords=discard,
			% ]{data/surfers__flow__n_128__re_250/surfer__merge_average_velocity_axis_0.csv};
			% \addlegendentry{$\KolmogorovVelocityScale/2$}
			% %%% fit
			% \addplot
			% [
			% color=ColorSurf!100!ColorVs,
			% opacity=1.0,
			% solid,
			% forget plot
			% ]
			% table[
				% x index=0,
				% y expr={\thisrowno{3} / (0.5 * 0.21)}, %u_\eta = 0.21
				% col sep=comma,
				% comment chars=\#,
				% unbounded coords=discard,
			% ]{data/surfers__flow__n_128__re_250/surfer__fits_average_velocity_axis_0.csv};
			%% us 1.0
			%%% 95 CI
			\addplot[name path=A, draw=none, forget plot] table [
				x index=4,
				y expr={(\thisrowno{1} - \thisrowno{2}) / (\thisrowno{3} * 0.21)}, %u_\eta = 0.21
				col sep=comma, 
				comment chars=\#,
				restrict expr to domain={\thisrowno{3}}{1.0:1.0},
				unbounded coords=discard,
			]{data/surfers__flow__n_128__re_250/surfer__merge_average_velocity_axis_0.csv};
			\addplot[name path=B, draw=none, forget plot] table [
				x index=4, 
				y expr={(\thisrowno{1} + \thisrowno{2}) / (\thisrowno{3} * 0.21)}, %u_\eta = 0.21
				col sep=comma,
				comment chars=\#,
				restrict expr to domain={\thisrowno{3}}{1.0:1.0},
				unbounded coords=discard,
			]{data/surfers__flow__n_128__re_250/surfer__merge_average_velocity_axis_0.csv};
			\addplot[ColorSurf!100!ColorVs, opacity=0.25, forget plot, on layer=axis background] fill between[of=A and B];
			%%% average
			\addplot
			[
			color=ColorSurf!100!ColorVs,
			opacity=1.0,
			only marks,%solid
			mark=square*
			]
			table[
				x index=4, 
				y expr={\thisrowno{1} / (\thisrowno{3} * 0.21)}, %u_\eta = 0.21
				col sep=comma, 
				comment chars=\#,
				restrict expr to domain={\thisrowno{3}}{1.0:1.0},
				unbounded coords=discard,
			]{data/surfers__flow__n_128__re_250/surfer__merge_average_velocity_axis_0.csv};
			%\addlegendentry{$\KolmogorovVelocityScale$}
			%%% fit
			\addplot
			[
			color=ColorSurf!100!ColorVs,
			opacity=1.0,
			solid,
			forget plot
			]
			table[
				x index=0, 
				y expr={\thisrowno{1} / (1.0 * 0.21)}, %u_\eta = 0.21
				col sep=comma, 
				comment chars=\#,
				unbounded coords=discard,
			]{data/surfers__flow__n_128__re_250/surfer__fits_average_velocity_axis_0.csv};
			%% us 4.0
			%%% 95 CI
			\addplot[name path=A, draw=none, forget plot] table [
				x index=4, 
				y expr={(\thisrowno{1} - \thisrowno{2}) / (\thisrowno{3} * 0.21)}, %u_\eta = 0.21
				col sep=comma, 
				comment chars=\#,
				restrict expr to domain={\thisrowno{3}}{4.0:4.0},
				unbounded coords=discard,
			]{data/surfers__flow__n_128__re_250/surfer__merge_average_velocity_axis_0.csv};
			\addplot[name path=B, draw=none, forget plot] table [
				x index=4, 
				y expr={(\thisrowno{1} + \thisrowno{2}) / (\thisrowno{3} * 0.21)}, %u_\eta = 0.21
				col sep=comma, 
				comment chars=\#,
				restrict expr to domain={\thisrowno{3}}{4.0:4.0},
				unbounded coords=discard,
			]{data/surfers__flow__n_128__re_250/surfer__merge_average_velocity_axis_0.csv};
			\addplot[ColorSurf!50!ColorVs, opacity=0.25, forget plot, on layer=axis background] fill between[of=A and B];
			%%% average
			\addplot
			[
			color=ColorSurf!50!ColorVs,
			opacity=1.0,
			only marks,%solid
			mark=pentagon
			]
			table[
				x index=4, 
				y expr={\thisrowno{1} / (\thisrowno{3} * 0.21)}, %u_\eta = 0.21
				col sep=comma, 
				comment chars=\#,
				restrict expr to domain={\thisrowno{3}}{4.0:4.0},
				unbounded coords=discard,
			]{data/surfers__flow__n_128__re_250/surfer__merge_average_velocity_axis_0.csv};
			%\addlegendentry{$4 \KolmogorovVelocityScale$}
			%%% fit
			\addplot
			[
			color=ColorSurf!50!ColorVs,
			opacity=1.0,
			solid,
			forget plot
			]
			table[
				x index=0, 
				y expr={\thisrowno{2} / (4.0 * 0.21)}, %u_\eta = 0.21
				col sep=comma, 
				comment chars=\#,
				unbounded coords=discard,
			]{data/surfers__flow__n_128__re_250/surfer__fits_average_velocity_axis_0.csv};
			%% us 8.0
			%%% 95 CI
			\addplot[name path=A, draw=none, forget plot] table [
				x index=4, 
				y expr={(\thisrowno{1} - \thisrowno{2}) / (\thisrowno{3} * 0.21)}, %u_\eta = 0.21
				col sep=comma, 
				comment chars=\#,
				restrict expr to domain={\thisrowno{3}}{8.0:8.0},
				unbounded coords=discard,
			]{data/surfers__flow__n_128__re_250/surfer__merge_average_velocity_axis_0.csv};
			\addplot[name path=B, draw=none, forget plot] table [
				x index=4, 
				y expr={(\thisrowno{1} + \thisrowno{2}) / (\thisrowno{3} * 0.21)}, %u_\eta = 0.21
				col sep=comma, 
				comment chars=\#,
				restrict expr to domain={\thisrowno{3}}{8.0:8.0},
				unbounded coords=discard,
			]{data/surfers__flow__n_128__re_250/surfer__merge_average_velocity_axis_0.csv};
			\addplot[ColorSurf!0!ColorVs, opacity=0.25, forget plot, on layer=axis background] fill between[of=A and B];
			%%% average
			\addplot
			[
			color=ColorSurf!0!ColorVs,
			opacity=1.0,
			only marks,%solid
			mark=*
			]
			table[
				x index=4, 
				y expr={\thisrowno{1} / (\thisrowno{3} * 0.21)}, %u_\eta = 0.21
				col sep=comma, 
				comment chars=\#,
				restrict expr to domain={\thisrowno{3}}{8.0:8.0},
				unbounded coords=discard,
			]{data/surfers__flow__n_128__re_250/surfer__merge_average_velocity_axis_0.csv};
			%\addlegendentry{$8 \KolmogorovVelocityScale$}
			%%% fit
			\addplot
			[
			color=ColorSurf!0!ColorVs,
			opacity=1.0,
			solid,
			forget plot
			]
			table[
				x index=0, 
				y expr={\thisrowno{4} / (8.0 * 0.21)}, %u_\eta = 0.21
				col sep=comma, 
				comment chars=\#,
				unbounded coords=discard,
			]{data/surfers__flow__n_128__re_250/surfer__fits_average_velocity_axis_0.csv};
			%% y = x
			\addplot
			[
			color=gray!50!white,
			opacity=1.0,
			%line width=1pt, 
			solid, 
			on layer=axis background,
			domain=0:8,
			]{1};
	\end{groupplot}
\end{tikzpicture}

	\caption[Influence of the time horizon on the surfing strategy.]{
		Influence of the time horizon on the surfing strategy.
		Effect of the time horizon [$\TimeHorizon$, Eq.~\eqref{turb:eq:surfing_swimming_direction_final}] on the effective velocity [$\Performance$, Eq.~\eqref{turb:eq:performance}], for different swimming velocities $\SwimmingVelocity$ and Reynolds numbers $\mathit{Re}_{\lambda}$.
		Shaded area represents the 95\% confidence interval.
		Solid lines represent a fit with Chebyshev polynomials of degree 3.
	}
	\label{fig:surfing_parameter_tau_vs}
\end{figure}
For all swimming velocities $\SwimmingVelocity$, the performance $\Performance$ has a clear maximum at $\TimeHorizonOpt(\SwimmingVelocity) = O(\KolmogorovTimeScale)$.
When $\TimeHorizon \ll \KolmogorovTimeScale$, surfers do not use gradient sensing and swim upwards [Eq.~\eqref{turb:eq:surfing_swimming_direction_final}].
Acting as bottom-heavy swimmers, their performance is $\Performance = \SwimmingVelocity$.
When $\TimeHorizon \gg \KolmogorovTimeScale$, the steady linear approximation of the flow, given in Eq.~\eqref{eq:surfing_linear}, breaks down and the planned route becomes irrelevant.
The optimal value $\TimeHorizonOpt$ can thus be interpreted as the duration over which the steady linear approximation of the flow is reasonable.
For $\SwimmingVelocity = \KolmogorovVelocityScale$, the optimal time horizon is $\TimeHorizonOpt \approx 4 \KolmogorovTimeScale$ for $\mathit{Re}_{\lambda} = 418$ and $\TimeHorizonOpt \approx 3 \KolmogorovTimeScale$ for $\mathit{Re}_{\lambda} = 11$.
Note the similitude of these results with those obtained in Taylor-Green vortices (Chap.~\ref{chap:the_surfing_strategy}, Sec.~\ref{sec:the_surfing_strategy_taylor}, Fig.~\ref{fig:taylor_green_vortex_tau_performance_vs_and_rdir}).
A similar phenomena occurs here but rather than surfing on Taylor-Green cells, plankters surf on Kolmogorov eddies.

Despite Fig.~\ref{fig:surfing_parameter_tau_vs}\textbf{(b)} showing a weak Reynolds dependence of both surfing performance and the value of the optimal parameter $\TimeHorizonOpt$, we expect our conclusions to be qualitatively independent of $\mathit{Re}_{\lambda}$, because of the universality of turbulence at small scale in the limit of large $\mathit{Re}_{\lambda}$ \citep{frisch1995turbulence}.
This assumption is discussed below.

To characterize the dependence of the time evolution of the flow sampled by plankters, we plot in Fig.~\ref{fig:surfing_correlation_time}\textbf{(a)} the square root of the module of the temporal Fourier transform $I(\omega_f)$ of $\tr([ \Gradients ]^2)$, an invariant of the flow velocity gradients
\begin{equation}\label{eq:invar_corr}
	I(\omega_f) = \sqrt{ \abs*{ \frac{d}{d\omega_f} \tr ( [ \Gradients ]^2 )} },
\end{equation}
measured along trajectories of plankters with various swimming velocities. 
The symbol $\omega_f$ denotes the temporal pulsation associated to the module $I(\omega_f)$.
The derivative $d/d\omega_f$ then denotes that it is the Fourier transform of the invariant that is considered (independent on time but dependant on $\omega_f$).
\begin{figure}%[H]
	\centering
	\begin{tikzpicture}
	\node[anchor=center] at (0.7,5.35) {$\SwimmingVelocity =$};
	\begin{groupplot}[
		group style={
			group size=2 by 1,
			%y descriptions at=edge left,
			%x descriptions at=edge bottom,
			horizontal sep=0.08\linewidth,
			%vertical sep=0.06\linewidth,
		},
		% size
		width=0.5\textwidth,
		% x
		xlabel=$\TimeHorizon / \KolmogorovTimeScale$,
		% layers
		set layers,
		% legend
		legend style={draw=none, fill=none, /tikz/every even column/.append style={column sep=4pt}, at={(0.5, 1.05)}, anchor=south},
		legend cell align=left,
		legend columns=3,
	]
		% spectrum
		\nextgroupplot[
			axis on top,
			% y
			ymode=log,
			ymin=0.5,
			ymax=10,
			ylabel={$\left\langle I(\omega_f) \right\rangle_N$},
			% x
			xmode=log,
			xlabel=$\omega_f / \omega_{\KolmogorovTimeScale}$,
			xmin=0.001,
			xmax=10,
		]
			\node[anchor=north west] at (axis cs:0.001,10) {\textbf{(a)}};
			% us 1.0
			\addplot[name path=A, draw=none, forget plot] table [
				x expr={\thisrowno{0} * 0.0424 / (2.0 * pi)}, %\tau_\eta = 0.0424
				y expr={sqrt(\thisrowno{1} - \thisrowno{2})}, 
				col sep=comma, 
				comment chars=\#,
				%skip coords between index={0}{1},
			]{chap_turbulence/data/correlation_time/average_fft_tr(J^2)__agent__us_1o0__surftimeconst_0o0.csv};
			\addplot[name path=B, draw=none, forget plot] table [
				x expr={\thisrowno{0} * 0.0424 / (2.0 * pi)}, %\tau_\eta = 0.0424
				y expr={sqrt(\thisrowno{1} + \thisrowno{2})}, 
				col sep=comma, 
				comment chars=\#,
				%skip coords between index={0}{1},
			]{chap_turbulence/data/correlation_time/average_fft_tr(J^2)__agent__us_1o0__surftimeconst_0o0.csv};
			\addplot[ColorSurf!100!ColorVs, opacity=0.25, forget plot] fill between[of=A and B];
			% us 4.0
			\addplot[name path=A, draw=none, forget plot] table [
				x expr={\thisrowno{0} * 0.0424 / (2.0 * pi)}, %\tau_\eta = 0.0424
				y expr={sqrt(\thisrowno{1} - \thisrowno{2})}, 
				col sep=comma, 
				comment chars=\#,
				%skip coords between index={0}{1},
			]{chap_turbulence/data/correlation_time/average_fft_tr(J^2)__agent__us_4o0__surftimeconst_0o0.csv};
			\addplot[name path=B, draw=none, forget plot] table [
				x expr={\thisrowno{0} * 0.0424 / (2.0 * pi)}, %\tau_\eta = 0.0424
				y expr={sqrt(\thisrowno{1} + \thisrowno{2})}, 
				col sep=comma, 
				comment chars=\#,
				%skip coords between index={0}{1},
			]{chap_turbulence/data/correlation_time/average_fft_tr(J^2)__agent__us_4o0__surftimeconst_0o0.csv};
			\addplot[ColorSurf!50!ColorVs, opacity=0.25, forget plot] fill between[of=A and B];
			% us 8.0
			\addplot[name path=A, draw=none, forget plot] table [
				x expr={\thisrowno{0} * 0.0424 / (2.0 * pi)}, %\tau_\eta = 0.0424
				y expr={sqrt(\thisrowno{1} - \thisrowno{2})}, 
				col sep=comma, 
				comment chars=\#,
				%skip coords between index={0}{1},
			]{chap_turbulence/data/correlation_time/average_fft_tr(J^2)__agent__us_8o0__surftimeconst_0o0.csv};
			\addplot[name path=B, draw=none, forget plot] table [
				x expr={\thisrowno{0} * 0.0424 / (2.0 * pi)}, %\tau_\eta = 0.0424
				y expr={sqrt(\thisrowno{1} + \thisrowno{2})}, 
				col sep=comma, 
				comment chars=\#,
				%skip coords between index={0}{1},
			]{chap_turbulence/data/correlation_time/average_fft_tr(J^2)__agent__us_8o0__surftimeconst_0o0.csv};
			\addplot[ColorSurf!0!ColorVs, opacity=0.25, forget plot] fill between[of=A and B];

			% us 1.0
			\addplot
			[
			color=ColorSurf!100!ColorVs,
			opacity=1.0,
			only marks,
			mark=square*
			]
			table[
				x expr={\thisrowno{0} * 0.0424 / (2.0 * pi)}, %\tau_\eta = 0.0424
				y expr={sqrt(\thisrowno{1})}, 
				col sep=comma, 
				comment chars=\#,
				%restrict expr to domain={\thisrowno{0} * 0.0424}{0.0:20.0},
				%unbounded coords=discard,
				%skip coords between index={0}{1},
			]{chap_turbulence/data/correlation_time/average_fft_tr(J^2)__agent__us_1o0__surftimeconst_0o0.csv};
			\addlegendentry{$\KolmogorovVelocityScale$}
			% us 4.0
			\addplot
			[
			color=ColorSurf!50!ColorVs,
			opacity=1.0,
			only marks,
			mark=pentagon
			]
			table[
				x expr={\thisrowno{0} * 0.0424 / (2.0 * pi)}, %\tau_\eta = 0.0424
				y expr={sqrt(\thisrowno{1})}, 
				col sep=comma, 
				comment chars=\#,
				%restrict expr to domain={\thisrowno{0} * 0.0424}{0.0:20.0},
				%unbounded coords=discard,
				%skip coords between index={0}{1},
			]{chap_turbulence/data/correlation_time/average_fft_tr(J^2)__agent__us_4o0__surftimeconst_0o0.csv};
			\addlegendentry{$4 \KolmogorovVelocityScale$}
			% us 8.0
			\addplot
			[
			color=ColorSurf!0!ColorVs,
			opacity=1.0,
			only marks,
			mark=*
			]
			table[
				x expr={\thisrowno{0} * 0.0424 / (2.0 * pi)}, %\tau_\eta = 0.0424
				y expr={sqrt(\thisrowno{1})}, 
				col sep=comma, 
				comment chars=\#,
				%restrict expr to domain={\thisrowno{0} * 0.0424}{0.0:20.0},
				%unbounded coords=discard,
				%skip coords between index={0}{1},
			]{chap_turbulence/data/correlation_time/average_fft_tr(J^2)__agent__us_8o0__surftimeconst_0o0.csv};
			\addlegendentry{$8 \KolmogorovVelocityScale$}
			%% exp
			%\addplot
			%[
			%color=colorsym,
			%opacity=1.0,
			%dashed, 
			%]
			%table[
			%    x expr={\thisrowno{0} * 0.0424 / (2.0 * pi)}, %\tau_\eta = 0.0424
			%    y expr={1/pi * 0.2/((\thisrowno{0} * 0.0424 / (2.0 * pi))^2 + 0.04)}, 
			%    col sep=comma, 
			%    comment chars=\#,
			%    %skip coords between index={0}{1},
			%]{data/jhtdb_even_more/average_fft_||J.z||__agent__us_8o0__surftimeconst_0o0.csv};
			%\addlegendentry{exp}
			% more


		
		% correlation time
		\nextgroupplot[
			axis on top,
			% y
			ylabel={$\TimeHorizon/\KolmogorovTimeScale$},
			ymin=0,
			ymax=6,
			% x
			xlabel=$\SwimmingVelocity / \KolmogorovVelocityScale$,
			xmin=0,
			xmax=20,
		]
			\node[anchor=north west] at (axis cs:0,6) {\textbf{(b)}};
			% tau estimated
			\addplot
			[
			color=black,
			opacity=1.0,
			%only marks,
			mark=asterisk,
			]
			table[
				x index=0,
				y expr={0.55 * \thisrowno{1} / 0.0424}, % \tau_\eta = 0.0424
				col sep=comma, 
				comment chars=\#,
			]{chap_turbulence/data/correlation_time/time_sq_tr_j_2.csv};
			\addlegendentry{0.55 $\CorrelationTime$}
			% tau estimated
			\addplot
			[
			color=ColorSym,
			opacity=1.0,
			mark=o,
			%only marks,
			]
			table[
				x index=0,
				y expr={0.35 * \thisrowno{1} / 0.0424}, % \tau_\eta = 0.0424
				col sep=comma, 
				comment chars=\#,
			]{chap_turbulence/data/correlation_time/time_direction.csv};
			\addlegendentry{0.35 $\CorrelationTime^b$}
			% tau estimated
			\addplot
			[
			color=ColorAsym,
			opacity=1.0,
			mark=triangle,
			%only marks,
			]
			table[
				x index=0,
				y expr={0.11 * \thisrowno{1} / 0.0424}, % \tau_\eta = 0.0424
				col sep=comma,
				comment chars=\#,
			]{chap_turbulence/data/correlation_time/time_norm.csv};
			\addlegendentry{0.11 $\CorrelationTime^c$}
			% tau opt (fit)
			\addplot
			[
			color=ColorSurf,
			opacity=1.0,
			solid, 
			only marks,
			mark=square*,
			]
			table[
				x index=0, 
				y expr={\thisrowno{2}},
				col sep=comma, 
				comment chars=\#,
			]{chap_turbulence/data/main_results/fits_max.csv};
			\addlegendentry{$\TimeHorizonOpt$}
			% model
			\addplot
			[
			color=gray,
			dashed, 
			domain=0.5:20,
			samples=100,
			]{4.368 / (1 + 0.08 * x)};
			\addlegendentry{Eq.~(3.18)}
	\end{groupplot}
\end{tikzpicture}

	\caption[Influence of the swimming speed on the time correlations of the flow sampled by plankters.]{
		Influence of the swimming speed on the time correlations of the flow sampled by plankters.
		\textbf{(a)} Invariant intensity [$I$, Eq.~\eqref{eq:invar_corr}] as a function of the pulsation $\omega_f$ for various plankter swimming speed. \textbf{(b)} Correlation time $\CorrelationTime$, defined in Eq.~\eqref{eq:corr}, and optimal time horizon $\TimeHorizonOpt$ as a function of swimming velocity ($\TimeHorizonOpt$ is evaluated using the fitted polynomial).
		The optimal time horizon $\TimeHorizonOpt$ is compared to the correlation time $\CorrelationTime$ [Eq.~\eqref{eq:corr}] for various definitions of $I(\omega_f)$ (Tab.~\ref{tab:invariants}) and to the model given by Eq.~\eqref{eq:time_horizon_model}.
		The simulation case is $\mathit{Re}_{\lambda} = 418$.
	}
	\label{fig:surfing_correlation_time}
\end{figure}
As expected, the intensity $I(\omega_f)$ shifts from low to high frequencies as the swimming velocity increases.

Supported by this observation, we hypothesize that the optimal time horizon $\TimeHorizonOpt$ scales as a correlation time $\CorrelationTime$.
We define $\CorrelationTime$ as the average of the period $2\pi/\omega_f$, weighted by $I(\omega_f)$ (averaged over all trajectories of plankters: $\langle I(\omega_f) \rangle_N$)
\begin{equation}
	\label{eq:corr}
	\CorrelationTime (\SwimmingVelocity) = \frac{\int \left\langle I(\omega_f) \right\rangle_N (2\pi/\omega_f) \, d\omega_f}{\int \left\langle I(\omega_f) \right\rangle_N \, d\omega_f}.
\end{equation}
Figure \ref{fig:surfing_correlation_time}\textbf{(b)} shows that, up to a multiplicative constant, $\CorrelationTime$ is a good predictor of the optimal time horizon with $\TimeHorizonOpt \approx 0.55 \CorrelationTime$.

The choice of $I(\omega_f)$ in Eq.~\eqref{eq:corr} is not unique, any invariant of the gradient matrix could be used.
Figure \ref{fig:surfing_correlation_time}\textbf{(b)} shows the differences obtained using the alternative definitions of $I(\omega_f)$ given in Tab.~\ref{tab:invariants}.
\begin{table}
	\center
	%\setlength{\tabcolsep}{2pt}
	\begin{tabular}{w{c}{0.25\linewidth}w{c}{0.25\linewidth}w{c}{0.25\linewidth}}
		\rowcolor{ColorTabularParameters}
		$I(\omega_f)$ & $I^b(\omega_f)$ & $I^c(\omega_f)$ \\
		\rowcolor{ColorTabularValues}
		$\displaystyle \sqrt{ \abs*{ \frac{d}{d\omega_f} \tr \left[ \Gradients \right]^2 } }$ & $\displaystyle \abs*{ \frac{d}{d\omega_f} \Direction \cdot \Gradients \cdot \Direction}$ & $\displaystyle \abs*{ \frac{d}{d\omega_f} \norm {\Gradients \cdot \Direction} }$ \\
	\end{tabular}
	\caption{Various possible definitions of $I(\omega_f)$ based on different invariants of the flow velocity gradients.}
	\label{tab:invariants}
\end{table}
While we observe differences on $\CorrelationTime$ for alternative invariants, $\TimeHorizonOpt$ is always approximately proportional to any definition of the correlation time.

To attempt to capture the dependence of the optimal time horizon $\TimeHorizonOpt \propto \CorrelationTime$ on swimming velocity $\SwimmingVelocity$ we start from the observation that, passing faster through the flow, fast micro-swimmers increase the temporal derivative of the velocity gradients they measure $d\Gradients/dt$.
The correlation of the flow sampled, and thus $\TimeHorizonOpt$, should then scale as $\TimeHorizonOpt \propto \norm{\Gradients} / \norm{d\Gradients/dt}$.
We expect the evolution of this lagrangian measure to be controlled by a viscous diffusive flux and an advection flux controlled by the swimming speed.
The time scale of the diffusion process should scale with $\KolmogorovTimeScale$ while the advection process is controlled by $\SwimmingVelocity \norm{ \vec{\nabla} (\Gradients) }$.
Assuming $\Gradients \propto 1/\KolmogorovTimeScale$ and $\vec{\nabla} \propto 1/\KolmogorovScale$, $\norm{d(\Gradients)^T/dt}$ should then scale as follows 
\begin{equation}
	\norm*{\frac{d\Gradients}{dt}} \propto \frac{1}{\KolmogorovTimeScale^2} + \alpha_{\mathrm{swim}} \frac{\SwimmingVelocity}{\KolmogorovScale \KolmogorovTimeScale} = \frac{1}{\KolmogorovTimeScale^2} \left(1 + \alpha_{\mathrm{swim}} \frac{\SwimmingVelocity}{\KolmogorovVelocityScale} \right)
\end{equation}
with $\alpha_{\mathrm{swim}}$ a dimensionless parameter that characterises the importance of the swimming advection process over the viscous diffusion.
Note that this model does not account for any of the complex spatial and temporal correlations prescribed by turbulence.
Due to the apparent complexity it would imply, we favor simplicity here. 
Note well more advanced models exist that describe the temporal correlations of the Lagrangian gradient tensor \citep{fang2015short, yu2010lagrangian, chevillard2011lagrangian}. 
However, these models do not account for an active swimming velocity $\SwimmingVelocity$.
The optimal time horizon should then scale as 
\begin{equation}\label{eq:time_horizon_model}
	\TimeHorizonOpt \appropto \frac{\KolmogorovTimeScale}{1 + \alpha_{\mathrm{swim}} (\SwimmingVelocity / \KolmogorovVelocityScale)}.
\end{equation}
The value of $\alpha_{\mathrm{swim}} \approx 0.08$ and the prefactor $4.368$ are fitted to our data.
Plotted in Fig.~\ref{fig:surfing_correlation_time}\textbf{(b)}, this model provides a fair prediction in the range of parameters explored here but the simplicity of the model fails to capture all features observed: the concavity of the function for small swimming velocities for instance.
Therefore models that account for more effects, such as the complex spatial distribution of flow features in turbulence, should be considered in the future.

Overall, the optimal horizon time $\TimeHorizonOpt$ has then two properties: (1) $\TimeHorizonOpt$ is always of the order of the Kolmogorov time scale and (2) increasing the swimming velocity tends to reduce $\TimeHorizonOpt$ (due to shorter Lagrangian decorrelation time induced by Doppler-shifting of frequencies).

\subsection{Influence of the Reynolds number $\mathit{Re}_{\lambda}$}

Oceanic turbulence intensity varies widely with ocean regions and weather \citep{fuchs2016seascape}.
Thus, plankton may experience very different Reynolds numbers depending on their habitat.
While for large Reynolds numbers ($\mathit{Re}_{\lambda} \gg 1$), the universality of turbulence at small scales ensures weak dependence of performance on the Reynolds number \citep{frisch1995turbulence}, this is not the case for small Reynolds numbers $\mathit{Re}_{\lambda} \lesssim 1$.
If there is no mean flow $\langle \FlowVelocity \rangle_{x,t} = \vec{0}$, temporal and spatial flow fluctuations decrease directly with $\mathit{Re}_{\lambda}$. 
Surfing performance is then expected to drop as $\mathit{Re}_{\lambda}$ decreases, down to the performance of bottom-heavy swimmers for $\mathit{Re}_{\lambda} = 0$ for which the fluid would be quiescent.

To assess the effect of the Reynolds number on performance, surfers with swimming speed $\SwimmingVelocity = \KolmogorovVelocityScale$, are simulated in flows of various $\mathit{Re}_{\lambda}$ (Fig.~\ref{fig:surfing_parameter_tau_reynolds}).
\begin{figure}%[H]
	\centering
	\begin{tikzpicture}
	% gain as a function of the free parameter $\tau$
	\begin{axis} [
		axis on top,
		% size
		width=0.66\textwidth,
		height=0.62\textwidth,
		% y
		ymin=0,
		ymax=2.5,
		ylabel={$\left\langle \Performance \right\rangle_N / \SwimmingVelocity$},
		y label style={yshift=-4pt},
		%extra y ticks={0.5, 1.5, 2.5},
		% x
		xlabel=$\TimeHorizon / \KolmogorovTimeScale$,
		x label style={yshift=4pt},
		xmin=0,
		xmax=8,
		% layers
		set layers,
		% legend
		legend style={
			draw=none, 
			fill=none, 
			%/tikz/every even column/.append style={column sep=8pt},
			xshift=25pt,
		},
		legend pos=south west,
		legend cell align=left,
		legend columns=-1,
	]
		\node[anchor=south west, yshift=6pt] at (rel axis cs:0.0,0.0) {$\mathit{Re}_{\lambda} = $};
		%% re 1.4
		%%% 95 CI
		\addplot[name path=A, draw=none, forget plot] table [
			x index=4,
			y expr={(\thisrowno{1} - \thisrowno{2}) / (\thisrowno{3} * 0.314)}, %u_\eta = 0.21
			col sep=comma, 
			comment chars=\#,
			restrict expr to domain={\thisrowno{3}}{1.0:1.0},
			unbounded coords=discard,
		]{data/surfers__flow__n_64__re_50/surfer__merge_average_velocity_axis_0.csv};
		\addplot[name path=B, draw=none, forget plot] table [
			x index=4, 
			y expr={(\thisrowno{1} + \thisrowno{2}) / (\thisrowno{3} * 0.314)}, %u_\eta = 0.21
			col sep=comma,
			comment chars=\#,
			restrict expr to domain={\thisrowno{3}}{1.0:1.0},
			unbounded coords=discard,
		]{data/surfers__flow__n_64__re_50/surfer__merge_average_velocity_axis_0.csv};
		\addplot[ColorSurf!0!ColorRe, opacity=0.25, forget plot, on layer=axis background] fill between[of=A and B];
		%%% average
		\addplot
		[
		color=ColorSurf!0!ColorRe,
		only marks,%solid
		mark=star
		]
		table[
			x index=4, 
			y expr={\thisrowno{1} / (\thisrowno{3} * 0.314)}, %u_\eta = 0.21
			col sep=comma, 
			comment chars=\#,
			restrict expr to domain={\thisrowno{3}}{1.0:1.0},
			unbounded coords=discard,
		]{data/surfers__flow__n_64__re_50/surfer__merge_average_velocity_axis_0.csv};
		\addlegendentry{$1.4$}
		%% re 3.6
		%%% 95 CI
		\addplot[name path=A, draw=none, forget plot] table [
			x index=4,
			y expr={(\thisrowno{1} - \thisrowno{2}) / (\thisrowno{3} * 0.262)}, %u_\eta = 0.21
			col sep=comma, 
			comment chars=\#,
			restrict expr to domain={\thisrowno{3}}{1.0:1.0},
			unbounded coords=discard,
		]{data/surfers__flow__n_64__re_100/surfer__merge_average_velocity_axis_0.csv};
		\addplot[name path=B, draw=none, forget plot] table [
			x index=4, 
			y expr={(\thisrowno{1} + \thisrowno{2}) / (\thisrowno{3} * 0.262)}, %u_\eta = 0.21
			col sep=comma,
			comment chars=\#,
			restrict expr to domain={\thisrowno{3}}{1.0:1.0},
			unbounded coords=discard,
		]{data/surfers__flow__n_64__re_100/surfer__merge_average_velocity_axis_0.csv};
		\addplot[ColorSurf!25!ColorRe, opacity=0.25, forget plot, on layer=axis background] fill between[of=A and B];
		%%% average
		\addplot
		[
		color=ColorSurf!25!ColorRe,
		only marks,%solid
		mark=*
		]
		table[
			x index=4, 
			y expr={\thisrowno{1} / (\thisrowno{3} * 0.262)}, %u_\eta = 0.262
			col sep=comma, 
			comment chars=\#,
			restrict expr to domain={\thisrowno{3}}{1.0:1.0},
			unbounded coords=discard,
		]{data/surfers__flow__n_64__re_100/surfer__merge_average_velocity_axis_0.csv};
		\addlegendentry{$3.6$}
		%% re 11
		%%% 95 CI
		\addplot[name path=A, draw=none, forget plot] table [
			x index=4,
			y expr={(\thisrowno{1} - \thisrowno{2}) / (\thisrowno{3} * 0.21)}, %u_\eta = 0.21
			col sep=comma, 
			comment chars=\#,
			restrict expr to domain={\thisrowno{3}}{1.0:1.0},
			unbounded coords=discard,
		]{data/surfers__flow__n_128__re_250/surfer__merge_average_velocity_axis_0.csv};
		\addplot[name path=B, draw=none, forget plot] table [
			x index=4, 
			y expr={(\thisrowno{1} + \thisrowno{2}) / (\thisrowno{3} * 0.21)}, %u_\eta = 0.21
			col sep=comma,
			comment chars=\#,
			restrict expr to domain={\thisrowno{3}}{1.0:1.0},
			unbounded coords=discard,
		]{data/surfers__flow__n_128__re_250/surfer__merge_average_velocity_axis_0.csv};
		\addplot[ColorSurf!50!ColorRe, opacity=0.25, forget plot, on layer=axis background] fill between[of=A and B];
		%%% average
		\addplot
		[
		color=ColorSurf!50!ColorRe,
		only marks,%solid
		mark=pentagon
		]
		table[
			x index=4, 
			y expr={\thisrowno{1} / (\thisrowno{3} * 0.21)}, %u_\eta = 0.21
			col sep=comma, 
			comment chars=\#,
			restrict expr to domain={\thisrowno{3}}{1.0:1.0},
			unbounded coords=discard,
		]{data/surfers__flow__n_128__re_250/surfer__merge_average_velocity_axis_0.csv};
		\addlegendentry{$11$}
		%% re 21
		%%% 95 CI
		\addplot[name path=A, draw=none, forget plot] table [
			x index=4,
			y expr={(\thisrowno{1} - \thisrowno{2}) / (\thisrowno{3} * 0.18)}, %u_\eta = 0.18
			col sep=comma, 
			comment chars=\#,
			restrict expr to domain={\thisrowno{3}}{1.0:1.0},
			unbounded coords=discard,
		]{data/surfers__flow__n_128__re_500/surfer__merge_average_velocity_axis_0.csv};
		\addplot[name path=B, draw=none, forget plot] table [
			x index=4, 
			y expr={(\thisrowno{1} + \thisrowno{2}) / (\thisrowno{3} * 0.18)}, %u_\eta = 0.18
			col sep=comma,
			comment chars=\#,
			restrict expr to domain={\thisrowno{3}}{1.0:1.0},
			unbounded coords=discard,
		]{data/surfers__flow__n_128__re_500/surfer__merge_average_velocity_axis_0.csv};
		\addplot[ColorSurf!75!ColorRe, opacity=0.25, forget plot, on layer=axis background] fill between[of=A and B];
		%%% average
		\addplot
		[
		color=ColorSurf!75!ColorRe,
		only marks,%solid
		mark=square*
		]
		table[
			x index=4, 
			y expr={\thisrowno{1} / (\thisrowno{3} * 0.18)}, %u_\eta = 0.18
			col sep=comma, 
			comment chars=\#,
			restrict expr to domain={\thisrowno{3}}{1.0:1.0},
			unbounded coords=discard,
		]{data/surfers__flow__n_128__re_500/surfer__merge_average_velocity_axis_0.csv};
		\addlegendentry{$21$}
		%% re 418
		%%% 95 CI
		\addplot[name path=A, draw=none, forget plot] table [
			x index=3,
			y expr={(\thisrowno{0} - \thisrowno{1}) / (\thisrowno{2} * 0.066)}, %u_\eta = 0.066
			col sep=comma, 
			comment chars=\#,
			restrict expr to domain={\thisrowno{2}}{1.0:1.0},
			unbounded coords=discard,
		]{chap_turbulence/data/main_results/merge.csv};
		\addplot[name path=B, draw=none, forget plot] table [
			x index=3, 
			y expr={(\thisrowno{0} + \thisrowno{1}) / (\thisrowno{2} * 0.066)}, %u_\eta = 0.066
			col sep=comma,
			comment chars=\#,
			restrict expr to domain={\thisrowno{2}}{1.0:1.0},
			unbounded coords=discard,
		]{chap_turbulence/data/main_results/merge.csv};
		\addplot[ColorSurf!100!ColorVs, opacity=0.25, forget plot, on layer=axis background] fill between[of=A and B];
		%%% average
		\addplot
		[
		color=ColorSurf!100!ColorVs,
		only marks,%solid
		mark=triangle
		]
		table[
			x index=3, 
			y expr={\thisrowno{0} / (\thisrowno{2} * 0.066)}, %u_\eta = 0.066
			col sep=comma, 
			comment chars=\#,
			restrict expr to domain={\thisrowno{2}}{1.0:1.0},
			unbounded coords=discard,
		]{chap_turbulence/data/main_results/merge.csv};
		\addlegendentry{$418$}
		%% y = x
		\addplot
		[
		color=gray!50!white,
		%line width=1pt, 
		solid, 
		on layer=axis background,
		domain=0:10,
		]{1};
	\end{axis}
\end{tikzpicture}

	\caption[Surfing performance increases with the Reynolds number $\mathit{Re}_{\lambda}$.]{
		Surfing performance increases with the Reynolds number $\mathit{Re}_{\lambda}$.
		Effective velocity [$\Performance$, Eq.~\eqref{turb:eq:performance}] as function of the surfing time horizon $\TimeHorizon$, for various Reynolds numbers $\mathit{Re}_{\lambda}$.
		The swimming speed of simulated plankters is set to $\SwimmingVelocity = \KolmogorovVelocityScale$.
		Shaded area represents the 95\% confidence interval.
		Evaluated for $\SwimmingVelocity = \KolmogorovVelocityScale$.
	}
	\label{fig:surfing_parameter_tau_reynolds}
\end{figure}
As expected, starting from a low (yet significant) maximal performance of $+8\%$ at $\mathit{Re}_{\lambda} = 1.4$, surfing performance increases with $\mathit{Re}_{\lambda}$ for low turbulence levels and seems to plateau for larger values of $\mathit{Re}_{\lambda}$.
Furthermore, we observe that the Reynolds number $\mathit{Re}_{\lambda}$ has little influence on the optimal surfing time horizon $\TimeHorizonOpt$ that remains of the order of $\KolmogorovTimeScale$: it ranges from $\TimeHorizonOpt \approx \KolmogorovTimeScale$ for $\mathit{Re}_{\lambda} = 1.4$ to $\TimeHorizonOpt \approx 4\KolmogorovTimeScale$ for $\mathit{Re}_{\lambda} = 418$.
We note particularly that very little differences are observed between our most extreme simulations with the Snoopy solver ($\mathit{Re}_{\lambda} = 21$) and the Johns Hopkins turbulence databases ($\mathit{Re}_{\lambda} = 418$) in regard of the large Reynolds number difference.

\section{Estimating surfing performance}\label{sec:estimating_surfing_performance}

Now that we demonstrated the surfing strategy to be effective, we look for a proper quantification of surfing performance.
To do so, we push further our analysis to determine the role of the symmetric component of the flow and of the skew symmetric one.
This lets us first assess which component of the flow is responsible for most of the surfing performance.
Build upon this assessment, we then estimate surfing performance.
Based on the simplified formulation of the surfing strategy in both a simple vortex flow and a pure strain flow (cf. Chap.~\ref{chap:the_surfing_strategy}, Sec.~\ref{sec:the_surfing_strategy_linear}) we then estimate surfing performance.

\subsection{Partial surfing performance}\label{sec:partial}

Based on the observation that the surfing strategy behaves differently in a vortex flow than in a pure strain flow, we can expect that different components of the flow contributes differently to surfing performance.
In linear flows (cf. Chap.~\ref{chap:the_surfing_strategy}, Sec.~\ref{sec:the_surfing_strategy_linear}), when the flow corresponds to pure strain (symmetric), $\Gradients = \GradientsSym$, surfers perform only slightly better than bottom-heavy swimmers.
On the contrary, in a vortex flow (skew symmetric), $\Gradients = \GradientsAsym$, bottom-heavy swimmers get trapped into the vortex. 
Surfers are however able to escape from it and perform much better.
Based on the observation made for linear flows, we already expect the rotation rate $\GradientsAsym$ to contribute more to surfing performance. 
But how does these observations translate to a turbulent flow?
To answer this question, the surfing strategy can be adapted by directly replacing the gradients $\Gradients$ by one of these components in Eq.~\eqref{turb:eq:surfing_swimming_direction_final}.

\subsubsection{Performance of symmetric surfers: $\GradientsSym$}

Starting with the strain rate tensor $\GradientsSym = \sym \Gradients$, surfing can be adapted
\begin{equation}\label{turb:eq:surf_partial_sym}
	\ControlDirectionOptSym = \frac{\ControlDirectionOptSymNN}{\norm{\ControlDirectionOptSymNN}}, \quad \text{with} \quad \ControlDirectionOptSymNN = \left[ \exp \left( \TimeHorizon \, \GradientsSym \right) \right]^T \cdot \Direction.
\end{equation}
This limited strategy is referred to as \textit{symmetric surfing} below.

In Fig. \ref{fig:surfing_partial}, we show that even though performance drops drastically compared to surfers using the full velocity gradients (for which $\left\langle \Performance \right\rangle_{N}(\TimeHorizonOpt) \approx 2 \SwimmingVelocity$), such limited surfers are still able to perform $25\%$ better than bottom-heavy swimmers for both the turbulence intensities presented.
\begin{figure}%[H]
	\centering
	% Reynolds
\begin{tikzpicture}[
	declare function={erf(\x)=%
	      (1+(e^(-(\x*\x))*(-265.057+abs(\x)*(-135.065+abs(\x)%
	      *(-59.646+(-6.84727-0.777889*abs(\x))*abs(\x)))))%
	      /(3.05259+abs(\x))^5)*(\x>0?1:-1);},
]
	\begin{groupplot}[
			group style={
				group size=2 by 1,
				y descriptions at=edge left,
				horizontal sep=0.04\linewidth,
			},
			% size
			width=0.5\textwidth,
			%height=0.62\textwidth,
			% y
			ylabel={$\left\langle \Performance \right\rangle_N / \SwimmingVelocity$},
			y label style={yshift=-4pt},
			ymin=0.0,
			ymax=2.5,
			% x
			x label style={yshift=4pt},
			xmin=0.0,
			% layers
			set layers,
			% legend
			legend style={draw=none, fill=none, /tikz/every even column/.append style={column sep=4pt}, at={(1.0, 1.05)}, anchor=south},
			%legend pos=north west,
	   		legend cell align=left,
	   		legend columns=-1,
		]
	\nextgroupplot[
		axis on top,
		% size
		extra y ticks={0.5,1.5,2.5},
		% x
		xlabel=$\TimeHorizon / \KolmogorovTimeScale$,
		xmax=30,
	]
		\node[anchor=north west] at (axis cs:0.0,2.5) {\textbf{(a):} $\mathit{Re}_{\lambda} = 418$};
		%% full
		\addplot[name path=A, draw=none, forget plot] table [
			x index=3, 
			y expr={(\thisrowno{0} - \thisrowno{1}) / (\thisrowno{2} * 0.066)}, %u_\eta = 0.066
			col sep=comma, 
			comment chars=\#,
		]{chap_turbulence/data/partial/merge_average_velocity_axis_0__agent_full.csv};
		\addplot[name path=B, draw=none, forget plot] table [
			x index=3, 
			y expr={(\thisrowno{0} + \thisrowno{1}) / (\thisrowno{2} * 0.066)}, %u_\eta = 0.066
			col sep=comma, 
			comment chars=\#,
		]{chap_turbulence/data/partial/merge_average_velocity_axis_0__agent_full.csv};
		\addplot[ColorSurf, opacity=0.25, forget plot] fill between[of=A and B];
		%% sym
		\addplot[name path=A, draw=none, forget plot] table [
			x index=3, 
			y expr={(\thisrowno{0} - \thisrowno{1}) / (\thisrowno{2} * 0.066)}, %u_\eta = 0.066
			col sep=comma, 
			comment chars=\#,
		]{chap_turbulence/data/partial/merge_average_velocity_axis_0__agent_sym.csv};
		\addplot[name path=B, draw=none, forget plot] table [
			x index=3, 
			y expr={(\thisrowno{0} + \thisrowno{1}) / (\thisrowno{2} * 0.066)}, %u_\eta = 0.066
			col sep=comma,
			comment chars=\#,
		]{chap_turbulence/data/partial/merge_average_velocity_axis_0__agent_sym.csv};
		\addplot[ColorSym, opacity=0.25, forget plot] fill between[of=A and B];
		%% asym
		\addplot[name path=A, draw=none, forget plot] table [
			x index=3, 
			y expr={(\thisrowno{0} - \thisrowno{1}) / (\thisrowno{2} * 0.066)}, %u_\eta = 0.066
			col sep=comma, 
			comment chars=\#,
		]{chap_turbulence/data/partial/merge_average_velocity_axis_0__agent_asym.csv};
		\addplot[name path=B, draw=none, forget plot] table [
			x index=3, 
			y expr={(\thisrowno{0} + \thisrowno{1}) / (\thisrowno{2} * 0.066)}, %u_\eta = 0.066
			col sep=comma,
			comment chars=\#,
		]{chap_turbulence/data/partial/merge_average_velocity_axis_0__agent_asym.csv};
		\addplot[ColorAsym, opacity=0.25, forget plot] fill between[of=A and B];
		% full
		\addplot
		[
		color=ColorSurf,
		opacity=1.0,
		only marks,%solid
		mark=pentagon*
		]
		table[
			x index=3, 
			y expr={\thisrowno{0} / (\thisrowno{2} * 0.066)}, %u_\eta = 0.066
			col sep=comma, 
			comment chars=\#,
		]{chap_turbulence/data/partial/merge_average_velocity_axis_0__agent_full.csv};
		\addlegendentry{$\Gradients$}
		% sym
		\addplot
		[
		color=ColorSym,
		opacity=1.0,
		only marks,%solid
		mark=square
		]
		table[
			x index=3, 
			y expr={\thisrowno{0} / (\thisrowno{2} * 0.066)}, %u_\eta = 0.066
			col sep=comma, 
			comment chars=\#,
		]{chap_turbulence/data/partial/merge_average_velocity_axis_0__agent_sym.csv};
		\addlegendentry{$\GradientsSym$}
		% asym
		\addplot
		[
		color=ColorAsym,
		opacity=1.0,
		only marks,%solid
		mark=triangle
		]
		table[
			x index=3, 
			y expr={\thisrowno{0} / (\thisrowno{2} * 0.066)}, %u_\eta = 0.066
			col sep=comma, 
			comment chars=\#,
		]{chap_turbulence/data/partial/merge_average_velocity_axis_0__agent_asym.csv};
		\addlegendentry{$\GradientsAsym$}
		% %%% model sym
		% \def\moddelta{0.02}
		% \def\modtimesym{0.3 / \moddelta}
		% \addplot
		% [
		% color=black,
		% opacity=1.0,
		% dashdotted,
		% domain=0.0:30.0,
		% samples=100,
		% ]{(3.0 * pi / 4.0) * (exp(\moddelta * \modtimesym) + exp(-2.0 * \moddelta * \modtimesym - 3.0 * \moddelta * x)) / (2.0 * exp(\moddelta * \modtimesym) + exp(-2 * \moddelta * \modtimesym))};
		% \addlegendentry{Eq.~(4.21)}
		% %%% model asym cos
		% \def\modomega{0.65}
		% \def\modtimeasym{4.25}
		% \addplot
		% [
		% color=black,
		% opacity=1.0,
		% dotted,
		% domain=0.0:30.0,
		% samples=100,
		% ]{(2 * cos(deg(0.5 * \modomega * (\modtimeasym - x))) + 1) / (2 * cos(deg(0.5 * \modomega * \modtimeasym)) + 1)};
		% \addlegendentry{Eq.~(4.27)}
		% %%% model asym exp
		% \def\modomega{0.65}
		% \def\modtimeasym{4.25}
		% \addplot
		% [
		% color=black,
		% opacity=1.0,
		% dashed,
		% domain=0.0:30.0,
		% samples=100,
		% ]{(2.0 * exp(-(1.0/8.0) * (\modomega * abs(\modtimeasym - x))^2.0) + 1) / (2.0 * exp(-(1.0/8.0) * (\modomega * \modtimeasym)^2.0) + 1)};
		% \addlegendentry{Eq.~(4.30)}
		% %%% model full
		% \addplot
		% [
		% color=black,
		% opacity=1.0,
		% solid,
		% domain=0.0:30.0,
		% samples=100,
		% ]{(3.0 * pi / 8.0) * ( (1.0 + 2.0 * exp(-3.0 * \moddelta * (\modtimesym + x))) * exp(-(1.0/8.0) * (\modomega * abs(\modtimeasym - x))^2.0) + 1) / ( (1.0 + exp(-3.0 * \moddelta * \modtimesym)) * exp(-(1.0/8.0) * (\modomega * \modtimeasym)^2.0) + 1)};
		% \addlegendentry{Eq.~(4.37)}
		
		%% y = x
		\addplot
		[
		color=gray!50!white,
		opacity=1.0,
		%line width=1pt, 
		solid, 
		on layer=axis background,
		domain=0:30,
		]{1};




	\nextgroupplot[
		axis on top,
		% size
		extra y ticks={0.5,1.5,2.5},
		extra y tick labels={,,},
		% x
		xlabel=$\TimeHorizon / \KolmogorovTimeScale$,
		xmax=15,
		% legend
		legend pos = north east,
		legend style={fill opacity=0.5, text opacity=1},
	]
		\node[anchor=north west] at (axis cs:0.0,2.5) {\textbf{(b):} $\mathit{Re}_{\lambda} = 11$};
		%% full
		\addplot[name path=A, draw=none, forget plot] table [
			x index=4, 
			y expr={(\thisrowno{1} - \thisrowno{2}) / (\thisrowno{3} * 0.21)}, %u_\eta = 0.21
			col sep=comma, 
			comment chars=\#,
			restrict expr to domain={\thisrowno{3}}{1.0:1.0},
			unbounded coords=discard,
		]{data/surfers__flow__n_128__re_250/surfer__merge_average_velocity_axis_0.csv};
		\addplot[name path=B, draw=none, forget plot] table [
			x index=4, 
			y expr={(\thisrowno{1} + \thisrowno{2}) / (\thisrowno{3} * 0.21)}, %u_\eta = 0.21
			col sep=comma, 
			comment chars=\#,
			restrict expr to domain={\thisrowno{3}}{1.0:1.0},
			unbounded coords=discard,
		]{data/surfers__flow__n_128__re_250/surfer__merge_average_velocity_axis_0.csv};
		\addplot[ColorSurf, opacity=0.25, forget plot] fill between[of=A and B];
		
		%% sym
		\addplot[name path=A, draw=none, forget plot] table [
			x index=4,
			y expr={(\thisrowno{1} - \thisrowno{2}) / (\thisrowno{3} * 0.21)}, %u_\eta = 0.21
			col sep=comma,
			comment chars=\#,
			restrict expr to domain={\thisrowno{3}}{1.0:1.0},
			unbounded coords=discard,
		]{data/partial_surfers__flow__n_128__re_250/surfer_sym__merge_average_velocity_axis_0.csv};
		\addplot[name path=B, draw=none, forget plot] table [
			x index=4,
			y expr={(\thisrowno{1} + \thisrowno{2}) / (\thisrowno{3} * 0.21)}, %u_\eta = 0.21
			col sep=comma,
			comment chars=\#,
			restrict expr to domain={\thisrowno{3}}{1.0:1.0},
			unbounded coords=discard,
		]{data/partial_surfers__flow__n_128__re_250/surfer_sym__merge_average_velocity_axis_0.csv};
		\addplot[ColorSym, opacity=0.25, forget plot] fill between[of=A and B];

		%% asym
		\addplot[name path=A, draw=none, forget plot] table [
			x index=4,
			y expr={(\thisrowno{1} - \thisrowno{2}) / (\thisrowno{3} * 0.21)}, %u_\eta = 0.21
			col sep=comma,
			comment chars=\#,
		]{data/partial_surfers__flow__n_128__re_250/surfer_skew__merge_average_velocity_axis_0.csv};
		\addplot[name path=B, draw=none, forget plot] table [
			x index=4,
			y expr={(\thisrowno{1} + \thisrowno{2}) / (\thisrowno{3} * 0.21)}, %u_\eta = 0.21
			col sep=comma,
			comment chars=\#,
		]{data/partial_surfers__flow__n_128__re_250/surfer_skew__merge_average_velocity_axis_0.csv};
		\addplot[ColorAsym, opacity=0.25, forget plot] fill between[of=A and B];

		
		% full
		\addplot
		[
		color=ColorSurf,
		opacity=1.0,
		only marks,%solid
		mark=pentagon*
		]
		table[
			x index=4,
			y expr={\thisrowno{1} / (\thisrowno{3} * 0.21)}, %u_\eta = 0.21
			col sep=comma,
			comment chars=\#,
			restrict expr to domain={\thisrowno{3}}{1.0:1.0},
			unbounded coords=discard,
		]{data/surfers__flow__n_128__re_250/surfer__merge_average_velocity_axis_0.csv};
		%\addlegendentry{$\Gradients$}
		
		% sym
		\addplot
		[
		color=ColorSym,
		opacity=1.0,
		only marks,%solid
		mark=square
		]
		table[
			x index=4,
			y expr={\thisrowno{1} / (\thisrowno{3} * 0.21)}, %u_\eta = 0.21
			col sep=comma,
			comment chars=\#,
		]{data/partial_surfers__flow__n_128__re_250/surfer_sym__merge_average_velocity_axis_0.csv};
		%\addlegendentry{$\sym \Gradients$}

		% asym
		\addplot
		[
		color=ColorAsym,
		opacity=1.0,
		only marks,%solid
		mark=triangle
		]
		table[
			x index=4,
			y expr={\thisrowno{1} / (\thisrowno{3} * 0.21)}, %u_\eta = 0.21
			col sep=comma,
			comment chars=\#,
		]{data/partial_surfers__flow__n_128__re_250/surfer_skew__merge_average_velocity_axis_0.csv};
		%\addlegendentry{$\asym \Gradients$}
		% %%% model
		% \def\moddelta{0.015}
		% \def\modtimesym{0.45 / \moddelta}
		% \addplot
		% [
		% color=black,
		% opacity=1.0,
		% dashdotted,
		% domain=0.0:15.0,
		% samples=100,
		% ]{(3.0 * pi / 4.0) * (exp(\moddelta * \modtimesym) + exp(-2.0 * \moddelta * \modtimesym - 3.0 * \moddelta * x)) / (2.0 * exp(\moddelta * \modtimesym) + exp(-2 * \moddelta * \modtimesym))};
		% %%% model
		% \def\modomega{1.0}
		% \def\modtimeasym{2.5}
		% \addplot
		% [
		% color=black,
		% opacity=1.0,
		% dotted,
		% domain=0.0:15.0,
		% samples=100,
		% ]{(2 * cos(deg(0.5 * \modomega * (\modtimeasym - x))) + 1) / (2 * cos(deg(0.5 * \modomega * \modtimeasym)) + 1)};
		% %]{(exp(-(1.0/8.0) * (\modomega * abs(\modtime - x))^2.0) + 1) / (exp(-(1.0/8.0) * (\modomega * \modtime)^2.0) + 1)};
		% %%% model
		% \def\modomega{1.0}
		% \def\modtimeasym{2.5}
		% \addplot
		% [
		% color=black,
		% opacity=1.0,
		% dashed,
		% domain=0.0:15.0,
		% samples=100,
		% ]{(2.0 * exp(-(1.0/8.0) * (\modomega * abs(\modtimeasym - x))^2.0) + 1) / (2.0 * exp(-(1.0/8.0) * (\modomega * \modtimeasym)^2.0) + 1)};
		% %%% model full
		% \addplot
		% [
		% color=black,
		% opacity=1.0,
		% solid,
		% domain=0.0:15.0,
		% samples=100,
		% ]{(3.0 * pi / 8.0) * ( (1.0 + 2.0 * exp(-3.0 * \moddelta * (\modtimesym + x))) * exp(-(1.0/8.0) * (\modomega * abs(\modtimeasym - x))^2.0) + 1) / ( (1.0 + exp(-3.0 * \moddelta * \modtimesym)) * exp(-(1.0/8.0) * (\modomega * \modtimeasym)^2.0) + 1)};

		%% y = x
		\addplot
		[
		color=gray!50!white,
		opacity=1.0,
		%line width=1pt, 
		solid, 
		on layer=axis background,
		domain=0:15,
		]{1};
	\end{groupplot}
\end{tikzpicture}

	\caption[The use of the rotation rate tensor $\GradientsAsym$ is enough to capture most of surfing performance.]{
		The use of the rotation rate tensor $\GradientsAsym$ is enough to capture most of surfing performance.
		Performance of surfers compared to surfers limited to the measure of components of the velocity gradients tensor.
		%Performance is also compared to estimators derived in Sec.\ref{sec:perf_estimation}.
		The swimming speed of plankters is set to $\SwimmingVelocity = \KolmogorovVelocityScale$.
		Shaded area represents the 95\% confidence interval.
	}
	\label{fig:surfing_partial}
\end{figure}
Note how the optimal surfing parameter $\TimeHorizonOpt$ for which surfing performance is maximal is shifted towards larger values compared to a fully informed surfer.
For example, in the case $\mathit{Re}_{\lambda} = 418$, $\TimeHorizonOpt_{\GradientsSym} \approx 7 \KolmogorovTimeScale$ for symmetric surfers whereas $\TimeHorizonOpt \approx 4 \KolmogorovTimeScale$ for fully informed surfers.
Moreover, for $\TimeHorizon \gtrsim \KolmogorovTimeScale$ the dependence on the free parameter $\TimeHorizon$ is weak: for the case $\mathit{Re}_{\lambda} = 418$, performance does not vary much from $\TimeHorizon \approx 2 \KolmogorovTimeScale$ to $\TimeHorizon \approx 15 \KolmogorovTimeScale$.
Similarly for $\mathit{Re}_{\lambda} = 11$, performance of symmetric surfers is weakly dependant of $\TimeHorizon$ from $\TimeHorizon \approx 2 \KolmogorovTimeScale$ to $\TimeHorizon \approx 8 \KolmogorovTimeScale$.
An explanation for this effect is provided below.

This shows that even limited to the measure of the strain rate tensor, the flow remains exploitable.
Less sensitive to $\TimeHorizon$, the parametrization of the surfing strategy even becomes easier.
However, the overall performance is much smaller than the fully informed surfing, further highlighting the importance of the rotation rate component of the flow to navigate.

\subsubsection{Performance of skew symmetric surfers: $\GradientsAsym$}

Now investigating the impact of the rotation rate $\GradientsAsym = \asym \Gradients$, the surfing direction can be adapted to
\begin{equation}\label{turb:eq:surf_partial_asym}
	\ControlDirectionOptAsym = \frac{\ControlDirectionOptAsymNN}{\norm{\ControlDirectionOptAsymNN}}, \quad \text{with} \quad \ControlDirectionOptAsymNN = \left[ \exp \left( \TimeHorizon \, \GradientsAsym \right) \right]^T \cdot \Direction.
\end{equation}
This is referred to \textit{skew-symmetric surfing} below.
Surfers using the rotational part of the velocity gradients are almost performing as well as fully-informed surfers: for $\mathit{Re}_{\lambda} = 418$, $\langle \PerformanceAsym \rangle_{N}(\TimeHorizonOpt) \approx 1.7 \SwimmingVelocity$ (Fig.~\ref{fig:surfing_partial}).
However, contrary to symmetric surfers, performance drastically drops when the surfing time horizon $\TimeHorizon$ exceeds its optimal value for wich performance is maximal.
For $\mathit{Re}_{\lambda} = 11$, large values of $\TimeHorizon \gtrsim 5\KolmogorovTimeScale$ can lead to poor performance to the point where $\langle \PerformanceBh \rangle_{N} < \SwimmingVelocity$. 
Such surfers would be better off not reacting to the flow at all.

This shows that most of the surfing performance can be captured by reacting to the rotation rate tensor (flow vorticity).
However in that case, the choice of the surfing time horizon $\TimeHorizon$ is critical as it strongly influences performance.

\section{Performance estimation}\label{sec:perf_estimation}

We now look for an estimate of surfing performance.
As it is shown below, the previously described partial surfing behaviors enable to simplify the expression of the surfing directions.
We first account for these simplifications to deduce models of surfing performance of symmetric surfers and skew-symmetric surfers.
We then build upon these models to estimate the performance of fully informed surfers.

\subsubsection{The strain rate tensor: $\GradientsSym$}

In the following, we look for an estimate of symmetric surfing performance as a function of $\TimeHorizon$.
We remind the reader that, in a linear flow, the position of a swimmer can be integrated as follows (cf. Chap.~\ref{chap:the_surfing_strategy}, Sec.~\ref{sec:the_surfing_strategy_derivation})
\begin{multline}
	\ParticlePosition(\FinalTime) =
	\left[ \exp \left( \FinalTime \Gradients \right) - \matr{Id} \right] \cdot \Gradients^{-1} \cdot \left[ \FlowVelocity \, + \Gradients^{-1} \cdot \left(\frac{\partial \FlowVelocity}{\partial t} \right) \, \right] \\
	- \, \FinalTime \Gradients^{-1} \cdot \left(\frac{\partial \FlowVelocity}{\partial t} \right)
	+ \, \SwimmingVelocity \int_{0}^{\FinalTime} \exp \left[ (\FinalTime - t) \Gradients \right] \cdot\ControlDirection(t) \, dt.
\end{multline}
We model the flow as a succession of random linear flows that do not change over a time $\FinalTime$.
Then averaging over all possible values of $\FlowVelocity$ and $\partial \FlowVelocity / \partial t$, the last term is the only term that does not cancel out.
Based on this hypothesis, the expected effective swimming velocity can be estimated as
\begin{equation}\label{turb:eq:performance}
	\left\langle \Performance \right\rangle_{\FlowVelocity, \partial \FlowVelocity / \partial t} = \SwimmingVelocity \iint p(\Gradients, \TimeHorizon_{\Gradients}) \left( \left[ \exp \left( \TimeHorizon_{\Gradients} \Gradients \right) \cdot \ControlDirection(\TimeHorizon) \right] \cdot \Direction \right) \,\, d \Gradients \, d \TimeHorizon_{\Gradients},
\end{equation}
with $p(\Gradients, \TimeHorizon_{\Gradients})$ the joint probability density function of the gradient tensor $\Gradients$ and the duration $\TimeHorizon_{\Gradients} = \FinalTime - t$ corresponding to the time left before $\Gradients$ changes.

We now want to model the term $[ \exp ( \TimeHorizon_{\Gradients} \Gradients ) \cdot \ControlDirection(\TimeHorizon) ] \cdot \Direction$ in the case $\ControlDirection = \ControlDirectionOptSym$.
We further assume that the symmetric surfing direction $\ControlDirectionOptSym$ is independent of the rotational part of the flow velocity gradients $\GradientsAsym$.
In practice, both parts of the symmetric decomposition of the gradient tensor are known to be correlated \citep{buaria2022vorticity} but here our ambition is to obtain a tractable model of surfing performance.

After averaging over possible values of $\GradientsAsym$, the exponential term is expected to be proportional to its value in a pure strain flow
\begin{equation}\label{eq:partial_sym_prop}
	\left\langle \left[ \exp \left( \TimeHorizon_{\Gradients} \Gradients \right) \cdot \ControlDirectionOptSym(\TimeHorizon) \right] \cdot \Direction \right\rangle_\GradientsAsym \propto \left[ \exp \left( \TimeHorizon_{\GradientsSym} \GradientsSym \right) \cdot \ControlDirectionOptSym(\TimeHorizon) \right] \cdot \Direction,
\end{equation}
where $\TimeHorizon_{\GradientsSym}$ is the time left before $\GradientsSym$ changes.
We can then write everything in the eigen orthonormal basis of $\GradientsSym = \sym \Gradients$, noted $(\hat{\vec{e}}_{\alpha}, \hat{\vec{e}}_{\beta}, \hat{\vec{e}}_{\gamma})$ with its respective eigenvalues $\alpha \ge \beta \ge \gamma$ (We have $\gamma = -(\alpha + \beta) \le 0$ due to flow incompressibility).
Symmetric surfing can then be formulated as follows
\begin{equation}\label{turb:eq:surf_sym}
	\ControlDirectionOptSym(\TimeHorizon) = \frac{e^{\alpha \TimeHorizon} \DirectionScalar_\alpha \hat{\vec{e}}_\alpha + e^{\beta \TimeHorizon} \DirectionScalar_\beta \hat{\vec{e}}_\beta + e^{-(\alpha + \beta) \TimeHorizon} \DirectionScalar_\gamma \hat{\vec{e}}_\gamma}{\sqrt{e^{2 \alpha \TimeHorizon} \DirectionScalar_\alpha^2 + e^{2 \beta \TimeHorizon} \DirectionScalar_\beta^2 + e^{-2 (\alpha + \beta) \TimeHorizon} \DirectionScalar_\gamma^2}},
\end{equation}
with $\DirectionScalar_\alpha = \Direction \cdot \hat{\vec{e}}_\alpha$, $\DirectionScalar_\beta = \Direction \cdot \hat{\vec{e}}_\beta$ and $\DirectionScalar_\gamma = \Direction \cdot \hat{\vec{e}}_\gamma$.
Then injecting Eq.~\eqref{turb:eq:surf_sym} into Eq.~\eqref{eq:partial_sym_prop}, we obtain the following proportionality relation
\begin{equation}
	\left\langle \left[ \exp \left( \TimeHorizon_{\Gradients} \Gradients \right) \cdot \ControlDirectionOptSym(\TimeHorizon) \right] \cdot \Direction \right\rangle_\GradientsAsym \propto \frac{e^{\alpha (\TimeHorizon_{\GradientsSym} + \TimeHorizon)} \DirectionScalar_\alpha^2 + e^{\beta (\TimeHorizon_{\GradientsSym} + \TimeHorizon)} \DirectionScalar_\beta^2 + e^{-(\alpha + \beta) (\TimeHorizon_{\GradientsSym} + \TimeHorizon)} \DirectionScalar_\gamma^2}{\sqrt{e^{2 \alpha \TimeHorizon} \DirectionScalar_\alpha^2 + e^{2 \beta \TimeHorizon} \DirectionScalar_\beta^2 + e^{-2 (\alpha + \beta)
		\TimeHorizon} \DirectionScalar_\gamma^2}}
\end{equation}

In turbulence, the eigenvalue $\beta$ is on average positive \citep{lund1994improved}, thus $\hat{\vec{e}}_\beta$ is most likely an extension axis.
Furthermore, the most likely state is $\delta \equiv \alpha = \beta$ and $\gamma = -2 \delta$.
Evaluating $\langle [ \exp ( \TimeHorizon_{\Gradients} \GradientsSym ) \cdot \ControlDirectionOptSym(\TimeHorizon) ] \cdot \Direction \rangle_\GradientsAsym$ in this mostly likely state reduces the previous expression to
\begin{equation}\label{eq:performation_proportional_estimation_sym}
	\left\langle \left[ \exp \left( \TimeHorizon_{\Gradients} \Gradients \right) \cdot \ControlDirectionOptSym(\TimeHorizon) \right] \cdot \Direction \right\rangle_\GradientsAsym \propto \frac{e^{\delta (\TimeHorizon_{\GradientsSym} + \TimeHorizon)} \left[ \DirectionScalar_\alpha^2 + \DirectionScalar_\beta^2 \right] + e^{-2 \delta (\TimeHorizon_{\GradientsSym} + \TimeHorizon)} \DirectionScalar_\gamma^2}{\sqrt{e^{2 \delta \TimeHorizon} \left[ \DirectionScalar_\alpha^2 + \DirectionScalar_\beta^2 \right] + e^{-4 \delta \TimeHorizon} \DirectionScalar_\gamma^2}}
\end{equation}

To continue the analysis, one needs to make yet another assumption: $e^{-4 \delta \TimeHorizon} \ll e^{2 \delta \TimeHorizon}$.
This assumption reduces the model to large enough values of the surfing time horizon, $\delta \TimeHorizon \gg 1$.
In this case, the expression further simplifies to
\begin{equation}
	\left\langle \left[ \exp \left( \TimeHorizon_{\Gradients} \Gradients \right) \cdot \ControlDirectionOptSym(\TimeHorizon) \right] \cdot \Direction \right\rangle_\GradientsAsym \propto \frac{e^{\delta \TimeHorizon_{\GradientsSym}} \left[ \DirectionScalar_\alpha^2 + \DirectionScalar_\beta^2 \right] + e^{-2 \delta \TimeHorizon_{\GradientsSym} - 3 \delta \TimeHorizon} \DirectionScalar_\gamma^2}{\sqrt{\left[ \DirectionScalar_\alpha^2 + \DirectionScalar_\beta^2 \right]}}
\end{equation}
One may then average this value over all possible orientations of the target direction $\Direction$ in the basis $(\hat{\vec{e}}_{\alpha}, \hat{\vec{e}}_{\beta}, \hat{\vec{e}}_{\gamma})$.
Writing $\Direction$ in spherical coordinates, we obtain
\begin{align}
	\left\langle \left[ \exp \left( \TimeHorizon_{\Gradients} \Gradients \right) \cdot \ControlDirectionOptSym(\TimeHorizon) \right] \cdot \Direction \right\rangle_{\GradientsAsym, \Direction} &\propto \frac{1}{4 \pi} \int_{0}^{2 \pi} \int_{0}^{\pi} \left( e^{\delta \TimeHorizon_{\GradientsSym}} \sin^2 \theta + e^{-2 \delta \TimeHorizon_{\GradientsSym} - 3 \delta \TimeHorizon} \cos^2 \theta \right) \, d\theta \, d\phi \\
	&\propto \frac{\pi}{4} \left( e^{\delta \TimeHorizon_{\GradientsSym}} + e^{-2 \delta \TimeHorizon_{\GradientsSym} - 3 \delta \TimeHorizon} \right)
\end{align}
Coming back to the estimation of the effective vertical velocity, if we inject this expression into Eq.~\eqref{turb:eq:performance}, we obtain
\begin{equation}
	\left\langle \PerformanceSym \right\rangle_{\FlowVelocity, \partial \FlowVelocity / \partial t, \Direction} \propto \SwimmingVelocity \frac{\pi}{4} \iint p(\delta, \TimeHorizon_{\GradientsSym}) \left( e^{\delta \TimeHorizon_{\GradientsSym}} + e^{-2 \delta \TimeHorizon_{\GradientsSym} - 3 \delta \TimeHorizon} \right) \,\, d \delta \, d \TimeHorizon_{\GradientsSym},
\end{equation}
where $\delta \approx \norm{\Gradients} / 2$ is a characteristic stretching value of the flow sampled by the microswimmers and $\TimeHorizon_{\GradientsSym}$ is the duration for which a specific gradient can be considered constant, linked to the correlation time of the measured velocity gradient tensor.

The joint probability density function $p(\delta, \TimeHorizon_{\GradientsSym})$ is not obvious to model.
First, $\delta$ and $\TimeHorizon_{\GradientsSym}$ might be correlated.
Indeed, one can actually expect $\TimeHorizon_{\GradientsSym}$ to decrease when $\delta$ increase as the local dissipation increases [cf. the Navier-Stokes equations, Eq.~\eqref{eq:navier_stokes}].
Then the statistics of the velocity gradient tensor are known to display complex non-Gaussian dynamics in turbulence \citep{li2005origin}.
Finally, the distribution of the stretching value $\delta$ might be affected by the surfer behavior.

In practice, one could evaluate the probability density function $p(\delta, \TimeHorizon_{\GradientsSym})$ numerically in simulations.
But, given the already strong assumptions made previously, we look for the simplest model that would capture an estimation of performance.
Thus, constant characteristic values of $\delta$ and $\TimeHorizon_{\GradientsSym}$ are considered.
Both values then become free parameters of our model.
Even though these parameters are free in practice, these are linked to the properties of the flow sampled by swimmers:
$\delta$ characterizes the intensity of flow stretching, while $\TimeHorizon_{\GradientsSym}$ characterizes the time correlation of measured velocity gradient tensor.
Under this assumption, symmetric surfing performance follows
\begin{equation}\label{turb:eq:performance_estimation_sym}
	\left\langle \PerformanceSym \right\rangle_{\FlowVelocity, \partial \FlowVelocity / \partial t, \Direction} \propto \SwimmingVelocity \frac{\pi}{4} \left( e^{\delta \TimeHorizon_{\GradientsSym}} + e^{-2 \delta \TimeHorizon_{\GradientsSym} - 3 \delta \TimeHorizon} \right).
\end{equation}

Now, the proportionality coefficient remains to be evaluated.
To this end, we use the fact that for $\TimeHorizon = 0$, the bottom-heavy strategy is retrieved $\ControlDirectionOptSym(0) = \ControlDirection_{\NameBhShort} = \Direction$.
If continuous reorientation is assumed, the swimming direction of a bottom-heavy swimmer is completely uncorrelated with the flow, meaning $\left\langle \Performance^{\mathrm{\NameBhShort}} \right\rangle_{\FlowVelocity, \partial \FlowVelocity / \partial t, \Direction} = \SwimmingVelocity$.
Using the same approach, one may evaluate the surfing performance as a function of $\delta$ and $\TimeHorizon_{\GradientsSym}$.
Note however that Eq.~\eqref{turb:eq:performance_estimation_sym} is only valid for $\delta \TimeHorizon \gg 1$.
To evaluate $\left\langle \PerformanceBh \right\rangle_{\FlowVelocity, \partial \FlowVelocity / \partial t, \Direction}$, one needs to start back from Eq.~\eqref{eq:performation_proportional_estimation_sym}.
Evaluating for $\TimeHorizon = 0$ and similarly averaging all possible orientations of $\Direction$, we obtain
\begin{equation}\label{turb:eq:performance_estimation_sym_bh}
	 \left\langle \PerformanceBh \right\rangle_{\FlowVelocity, \partial \FlowVelocity / \partial t, \Direction} \propto \frac{\SwimmingVelocity}{3} \left[ 2 e^{\delta \TimeHorizon_{\GradientsSym}} + e^{-2 \delta \TimeHorizon_{\GradientsSym}} \right]
\end{equation}

Now, if the rotational (skew symmetric) part of the flow, $\GradientsAsym = \asym \Gradients$, impacts bottom-heavy swimmers and symmetric surfers in the same way, the proportionality coefficient should be the same in Eqs.~\eqref{turb:eq:performance_estimation_sym} and \eqref{turb:eq:performance_estimation_sym_bh}.
As surfing may induce preferential sampling of $\GradientsAsym$, this assumption is not obvious but as shown in Sec.~\ref{sec:surfing_on_turbulence_IHT}, this preferential sampling is weak.
This leads to this final estimation of performance for symmetric surfers
\begin{equation}\label{turb:eq:performance_estimation_sym_final}
	 \left\langle \PerformanceSym \right\rangle_{\FlowVelocity, \partial \FlowVelocity / \partial t, \Direction} \approx \SwimmingVelocity \frac{3\pi}{4} \frac{1 + e^{-3 \delta ( \TimeHorizon_{\GradientsSym} + \TimeHorizon )}}{ 2 + e^{-3 \delta \TimeHorizon_{\GradientsSym}}},
\end{equation}
This model is shown in Figs.~\ref{fig:surfing_partial_and_models}\textbf{(a)} with $\delta \approx 0.02/\KolmogorovTimeScale$ and $\TimeHorizon_{\GradientsSym} \approx 15 \KolmogorovTimeScale$. 
It shows good agreement with the simulations results for symmetric surfers performance in the limit $\delta \TimeHorizon \gg 1$.

\subsubsection{Rotation rate tensor: $\GradientsAsym$}

In the case of skew-symmetric surfers, the arguments are similar to the symmetric case.
Starting from Eq.~\eqref{turb:eq:performance}, we look for an estimate of the same exponential in the case $\ControlDirection = \ControlDirectionOptAsym$.

Similarly to the symmetric case, one may average over all possible values of the pure strain part, $\GradientsSym$, to obtain the following proportionality relation
\begin{equation}\label{eq:partial_asym_prop}
	\left\langle \left[ \exp \left( \TimeHorizon_{\Gradients} \, \Gradients \right) \cdot \ControlDirectionOptAsym(\TimeHorizon) \right] \cdot \Direction \right\rangle_\GradientsSym \propto \left[ \exp \left( \TimeHorizon_{\GradientsAsym} \GradientsAsym \right) \cdot \ControlDirectionOptAsym(\TimeHorizon) \right] \cdot \Direction,
\end{equation}
where $\TimeHorizon_{\GradientsAsym}$ corresponds the time left before $\GradientsAsym$ changes.
We can then write our equations in a basis, noted $(\hat{\vec{e}}_{\alpha}, \hat{\vec{e}}_{\beta}, \hat{\vec{e}}_{\gamma})$, so that $\hat{\vec{e}}_{\beta}$ is aligned with the vorticity.
In this basis, the effective velocity can be expressed as
\begin{equation}
	\left\langle \left[ \exp \left( \TimeHorizon_{\Gradients} \, \Gradients \right) \cdot \ControlDirectionOptAsym(\TimeHorizon) \right] \cdot \Direction \right\rangle_\GradientsSym \propto \SwimmingVelocity \cos \left( \frac{\omega}{2} \, [\TimeHorizon_{\GradientsAsym} - \TimeHorizon] \right) \left( \DirectionScalar_{\alpha}^2 + \DirectionScalar_{\gamma}^2 \right) + \DirectionScalar_{\beta}^2
\end{equation}
with $\FlowVorticityScalar$ the norm of the flow vorticity.
If a constant vorticity norm $\FlowVorticityScalar$ is considered, this expression is directly linked to the estimated effective velocity $\PerformanceAsym$.
Moreover, averaging over all possible orientations of the direction $\Direction$ leads to the expression
\begin{equation}\label{turb:eq:performance_surf_asym}
	\left\langle \PerformanceAsym \right\rangle \propto \frac{\SwimmingVelocity}{3} \left[ 2 \cos \left( \frac{\omega}{2} \, [\TimeHorizon_{\GradientsAsym} - \TimeHorizon] \right) + 1 \right].
\end{equation}
Similarly to the symmetric case, the proportionality coefficient can be found by evaluating this expression for $\TimeHorizon = 0$ (corresponding to bottom-heavy swimmers)
\begin{equation}\label{turb:eq:performance_bh_asym}
	\left\langle \PerformanceBh \right\rangle \propto \frac{\SwimmingVelocity}{3} \left[ 2 \cos \left( \frac{\omega}{2} \TimeHorizon_{\GradientsAsym} \right) + 1 \right]
\end{equation}
This leads to the expression of performance for skew-symmetric surfers
\begin{equation}\label{turb:eq:performance_estimation_asym_final}
	 \left\langle \PerformanceAsym \right\rangle \approx \SwimmingVelocity \frac{2 \cos \left( \omega [\TimeHorizon_{\GradientsAsym} - \TimeHorizon]/2 \right) + 1}{2 \cos \left( \omega \TimeHorizon_{\GradientsAsym} / 2 \right) + 1}
\end{equation}
This model is plotted in Figs.~\ref{fig:surfing_partial_and_models}\textbf{(a)} with $\TimeHorizon_{\GradientsAsym} \approx 4.25 \KolmogorovTimeScale$ and $\FlowVorticityScalar \approx 0.65 / \KolmogorovTimeScale$.
While this model captures fairly well the qualitative behavior of performance for small values of $\TimeHorizon$ however its oscillating nature differs from the actual observed performance of skew symmetric surfers.

To overcome the simplicity of this model, one may consider an non-trivial probability density function of the vorticity norm $\FlowVorticityScalar$
\begin{equation}
	\left\langle \PerformanceAsym \right\rangle \propto \frac{\SwimmingVelocity}{3} \int p(\FlowVorticityScalar) \left[ 2 \cos \left( \frac{\omega}{2} \, [\TimeHorizon_{\GradientsAsym} - \TimeHorizon] \right) + 1 \right] \,\, d \FlowVorticityScalar.
\end{equation}
Again, to obtain the simplest model possible, we consider the time $\TimeHorizon_{\GradientsAsym}$ constant and independent of $\FlowVorticityScalar$, even though this is not expected to be as simple in practice.
In addition, the distribution $p(\FlowVorticityScalar)$ of vorticity is then chosen to be Gaussian.
In practice, the distribution of vorticity is known for its non-Gaussian properties \citep{meneveau2011lagrangian} but this assumption ensures tractability.
This assumption then leads to
\begin{equation}
	\left\langle \PerformanceAsym \right\rangle \propto \frac{\SwimmingVelocity}{3} \left[ 2 \exp \left( -\frac{1}{8} [ \TimeHorizon_{\GradientsAsym} - \TimeHorizon ]^2 \sigma_{\FlowVorticityScalar}^2 \right) + 1 \right],
\end{equation}
with $\sigma_{\FlowVorticityScalar}$ the standard deviation of the vorticity.

Again, this expression can be evaluated for $\TimeHorizon = 0$, corresponding to bottom-heavy swimmers, and effective velocity of skew-symmetric surfers is estimated to be
\begin{equation}\label{turb:eq:performance_surf_asym_final}
	\left\langle \PerformanceAsym \right\rangle \approx \SwimmingVelocity \, \frac{ 2 \exp \left( -[ \TimeHorizon_{\GradientsAsym} - \TimeHorizon ]^2 \sigma_{\FlowVorticityScalar}^2 / 8 \right) + 1}{2 \exp \left( -{\TimeHorizon_{\GradientsAsym}}^2 \sigma_{\FlowVorticityScalar}^2 / 8 \right) + 1},
\end{equation}
where $\TimeHorizon_{\GradientsAsym} \approx 4.25 \KolmogorovTimeScale$ and $\sigma_{\FlowVorticityScalar} \approx 0.65 / \KolmogorovTimeScale$ are fitted on the numerical data (Fig.~\ref{fig:surfing_partial_and_models}\textbf{(a)}).
The model could certainly be improved further by using a better choice of the probability density function.

\subsubsection{Full velocity gradient tensor: $\Gradients$}

\begin{figure}%[H]
	\centering
	% Reynolds
\begin{tikzpicture}[
	declare function={erf(\x)=%
	      (1+(e^(-(\x*\x))*(-265.057+abs(\x)*(-135.065+abs(\x)%
	      *(-59.646+(-6.84727-0.777889*abs(\x))*abs(\x)))))%
	      /(3.05259+abs(\x))^5)*(\x>0?1:-1);},
]
	\begin{groupplot}[
			group style={
				group size=2 by 2,
				y descriptions at=edge left,
				horizontal sep=0.04\linewidth,
			},
			% size
			width=0.5\textwidth,
			% layers
			set layers,
			% legend
			legend style={
				draw=none, 
				fill=none, 
				/tikz/every even column/.append style={column sep=4pt}, 
				at={(0.5, 1.05)}, 
				anchor=south,
				yshift=4pt,
			},
			%legend pos=north east,
	   		legend cell align=left,
	   		legend columns=4,
		]
	\nextgroupplot[
		axis on top,
		% size
		ylabel={$\left\langle \Performance \right\rangle_N / \SwimmingVelocity$},
		y label style={yshift=-4pt},
		ymin=0.0,
		ymax=2.5,
		extra y ticks={0.5,1.5,2.5},
		% x
		xlabel=$\TimeHorizon / \KolmogorovTimeScale$,
		xmin=0.0,
		xmax=30,
		% legend
		legend style={
			at={(0.4, 1.05)}, 
		},
	]
		\node[anchor=north west] at (axis cs:0.0,2.5) {\textbf{(a)}};
		%% full
		\addplot[name path=A, draw=none, forget plot] table [
			x index=3, 
			y expr={(\thisrowno{0} - \thisrowno{1}) / (\thisrowno{2} * 0.066)}, %u_\eta = 0.066
			col sep=comma, 
			comment chars=\#,
		]{chap_turbulence/data/partial/merge_average_velocity_axis_0__agent_full.csv};
		\addplot[name path=B, draw=none, forget plot] table [
			x index=3, 
			y expr={(\thisrowno{0} + \thisrowno{1}) / (\thisrowno{2} * 0.066)}, %u_\eta = 0.066
			col sep=comma, 
			comment chars=\#,
		]{chap_turbulence/data/partial/merge_average_velocity_axis_0__agent_full.csv};
		\addplot[ColorSurf, opacity=0.25, forget plot] fill between[of=A and B];
		%% sym
		\addplot[name path=A, draw=none, forget plot] table [
			x index=3, 
			y expr={(\thisrowno{0} - \thisrowno{1}) / (\thisrowno{2} * 0.066)}, %u_\eta = 0.066
			col sep=comma, 
			comment chars=\#,
		]{chap_turbulence/data/partial/merge_average_velocity_axis_0__agent_sym.csv};
		\addplot[name path=B, draw=none, forget plot] table [
			x index=3, 
			y expr={(\thisrowno{0} + \thisrowno{1}) / (\thisrowno{2} * 0.066)}, %u_\eta = 0.066
			col sep=comma,
			comment chars=\#,
		]{chap_turbulence/data/partial/merge_average_velocity_axis_0__agent_sym.csv};
		\addplot[ColorSym, opacity=0.25, forget plot] fill between[of=A and B];
		%% asym
		\addplot[name path=A, draw=none, forget plot] table [
			x index=3, 
			y expr={(\thisrowno{0} - \thisrowno{1}) / (\thisrowno{2} * 0.066)}, %u_\eta = 0.066
			col sep=comma, 
			comment chars=\#,
		]{chap_turbulence/data/partial/merge_average_velocity_axis_0__agent_asym.csv};
		\addplot[name path=B, draw=none, forget plot] table [
			x index=3, 
			y expr={(\thisrowno{0} + \thisrowno{1}) / (\thisrowno{2} * 0.066)}, %u_\eta = 0.066
			col sep=comma,
			comment chars=\#,
		]{chap_turbulence/data/partial/merge_average_velocity_axis_0__agent_asym.csv};
		\addplot[ColorAsym, opacity=0.25, forget plot] fill between[of=A and B];
		% full
		\addplot
		[
		color=ColorSurf,
		opacity=1.0,
		only marks,%solid
		mark=pentagon*
		]
		table[
			x index=3, 
			y expr={\thisrowno{0} / (\thisrowno{2} * 0.066)}, %u_\eta = 0.066
			col sep=comma, 
			comment chars=\#,
		]{chap_turbulence/data/partial/merge_average_velocity_axis_0__agent_full.csv};
		\addlegendentry{$\Gradients$}
		% sym
		\addplot
		[
		color=ColorSym,
		opacity=1.0,
		only marks,%solid
		mark=square
		]
		table[
			x index=3, 
			y expr={\thisrowno{0} / (\thisrowno{2} * 0.066)}, %u_\eta = 0.066
			col sep=comma, 
			comment chars=\#,
		]{chap_turbulence/data/partial/merge_average_velocity_axis_0__agent_sym.csv};
		\addlegendentry{$\GradientsSym$}
		% asym
		\addplot
		[
		color=ColorAsym,
		opacity=1.0,
		only marks,%solid
		mark=triangle
		]
		table[
			x index=3, 
			y expr={\thisrowno{0} / (\thisrowno{2} * 0.066)}, %u_\eta = 0.066
			col sep=comma, 
			comment chars=\#,
		]{chap_turbulence/data/partial/merge_average_velocity_axis_0__agent_asym.csv};
		\addlegendentry{$\GradientsAsym$}
		\addlegendimage{empty legend}\addlegendentry{}
		%%% model sym
		\def\moddelta{0.02}
		\def\modtimesym{0.3 / \moddelta}
		\addplot
		[
		color=black,
		opacity=1.0,
		dashdotted,
		domain=0.0:30.0,
		samples=100,
		]{(3.0 * pi / 4.0) * (exp(\moddelta * \modtimesym) + exp(-2.0 * \moddelta * \modtimesym - 3.0 * \moddelta * x)) / (2.0 * exp(\moddelta * \modtimesym) + exp(-2 * \moddelta * \modtimesym))};
		\addlegendentry{(3.33)}
		%%% model asym cos
		\def\modomega{0.65}
		\def\modtimeasym{4.25}
		\addplot
		[
		color=black,
		opacity=1.0,
		dotted,
		domain=0.0:30.0,
		samples=100,
		]{(2 * cos(deg(0.5 * \modomega * (\modtimeasym - x))) + 1) / (2 * cos(deg(0.5 * \modomega * \modtimeasym)) + 1)};
		\addlegendentry{(3.38)}
		%%% model asym exp
		\def\modomega{0.65}
		\def\modtimeasym{4.25}
		\addplot
		[
		color=black,
		opacity=1.0,
		dashed,
		domain=0.0:30.0,
		samples=100,
		]{(2.0 * exp(-(1.0/8.0) * (\modomega * abs(\modtimeasym - x))^2.0) + 1) / (2.0 * exp(-(1.0/8.0) * (\modomega * \modtimeasym)^2.0) + 1)};
		\addlegendentry{(3.41)}
		%%% model full
		\addplot
		[
		color=black,
		opacity=1.0,
		solid,
		domain=0.0:30.0,
		samples=100,
		]{(3.0 * pi / 8.0) * ( (1.0 + 2.0 * exp(-3.0 * \moddelta * (\modtimesym + x))) * exp(-(1.0/8.0) * (\modomega * abs(\modtimeasym - x))^2.0) + 1) / ( (1.0 + exp(-3.0 * \moddelta * \modtimesym)) * exp(-(1.0/8.0) * (\modomega * \modtimeasym)^2.0) + 1)};
		\addlegendentry{(3.48)}
		
		%% y = x
		\addplot
		[
		color=gray!50!white,
		opacity=1.0,
		%line width=1pt, 
		solid, 
		on layer=axis background,
		domain=0:30,
		forget plot,
		]{1};




	\nextgroupplot[
		axis on top,
		% x
		% x
		xlabel=$\SwimmingVelocity / \KolmogorovVelocityScale$,
		xmode=log,
		xmin=0.5,
		xmax=20,
		xticklabels={0.1,1,10},
		extra x ticks={0.5, 5, 20},
		extra x tick labels={0.5,5,20},
		% y
		ymin=0.0,
		ymax=2.5,
	]
		\node[anchor=north west] at (axis cs:0.5,2.5) {\textbf{(c)}};
		% straight
		\addplot[name path=A, draw=none, forget plot] table [
			x index=2,
			y expr={(\thisrowno{0} - \thisrowno{1}) / (\thisrowno{2} * 0.066)}, %u_\eta = 0.066
			col sep=comma,
			comment chars=\#,
			restrict expr to domain={\thisrowno{3}}{0.0:0.0},
			unbounded coords=discard,
		]{chap_turbulence/data/main_results/merge.csv};
		\addplot[name path=B, draw=none, forget plot] table [
			x index=2,
			y expr={(\thisrowno{0} + \thisrowno{1}) / (\thisrowno{2} * 0.066)}, %u_\eta = 0.066
			col sep=comma,
			comment chars=\#,
			restrict expr to domain={\thisrowno{3}}{0.0:0.0},
			unbounded coords=discard,
		]{chap_turbulence/data/main_results/merge.csv};
		\addplot[ColorBh, opacity=0.25, forget plot, on layer=axis background] fill between[of=A and B];
		\addplot
		[
		color=ColorBh,
		opacity=1.0,
		%line width=1pt,
		only marks,%solid,
		mark=o
		]
		table[
			x index=2,
			y expr={\thisrowno{0} / (\thisrowno{2} * 0.066)}, %u_\eta = 0.066
			col sep=comma,
			comment chars=\#,
			restrict expr to domain={\thisrowno{3}}{0.0:0.0},
			unbounded coords=discard,
		]{chap_turbulence/data/main_results/merge.csv};
		\addlegendentry{\NameBh}
		% tss
		\addplot[name path=A, draw=none, forget plot] table [
			x index=2,
			y expr={(\thisrowno{0} - \thisrowno{1}) / (\thisrowno{2} * 0.066)}, %u_\eta = 0.066
			col sep=comma,
			comment chars=\#,
		]{chap_turbulence/data/main_results/max.csv};
		\addplot[name path=B, draw=none, forget plot] table [
			x index=2,
			y expr={(\thisrowno{0} + \thisrowno{1}) / (\thisrowno{2} * 0.066)}, %u_\eta = 0.066
			col sep=comma,
			comment chars=\#,
		]{chap_turbulence/data/main_results/max.csv};
		\addplot[ColorSurf, opacity=0.25, forget plot, on layer=axis background] fill between[of=A and B];
		\addplot
		[
		color=ColorSurf,
		opacity=1.0,
		%line width=1pt,
		only marks,%solid,
		mark=square*
		]
		table[
			x index=2,
			y expr={\thisrowno{0} / (\thisrowno{2} * 0.066)}, %u_\eta = 0.066
			col sep=comma,
			comment chars=\#,
		]{chap_turbulence/data/main_results/max.csv};
		\addlegendentry{\NameSurf}
		%% y = x
		\addplot
		[
		color=gray!50!white,
		opacity=1.0,
		%line width=1pt,
		solid,
		on layer=axis background,
		domain=0.5:20,
		forget plot,
		]{1};
		%%% model
		\def\moddelta{0.02}
		\def\modtimesym{0.3 / \moddelta}
		\def\modomega{0.65}
		\def\modtimeasym{4.25}
		\def\modparam{0.08}
		\addplot
		[
			color=black,
			opacity=1.0,
			dashed,
			domain=0.5:20.0,
			samples=10,
		]{(3.0 * pi / 4.0) * (1.0 + exp(-3.0 * \moddelta * (\modtimesym + \modtimeasym ) / ((1.0 - \modparam) + \modparam * (x-1)))) / ( (1.0 + exp(-3.0 * \moddelta * \modtimesym / ((1.0 - \modparam) + \modparam * (x-1)))) * exp(-(1.0/8.0) * (\modomega * \modtimeasym / ((1.0 - \modparam) + \modparam * (x-1)) )^2.0) + 1)};
		\addlegendentry{(3.49)}
	\end{groupplot}
\end{tikzpicture}

	\caption[The influence of parameters on surfing performance are not yet fully understood.]{
		The influence of parameters on surfing performance are not yet fully understood.
		Performance of surfers compared to surfers limited to the measure of components of the velocity gradients tensor.
		Performance is also compared to the estimators derived in Sec.\ref{sec:perf_estimation}.
		The swimming speed of plankters is set to $\SwimmingVelocity = \KolmogorovVelocityScale$.
		Shaded area represents the 95\% confidence interval.
	}
	\label{fig:surfing_partial_and_models}
\end{figure}

Building upon the previous models, we now look for an estimate performance in the case a fully-informed surfer.
To this end, we restrict our analysis to a particular case: we assume that vorticity is aligned with the second eigenvector $\hat{\vec{e}}_\beta$ of the gradient tensor $\Gradients$ and that the eigenvalues of $\GradientsSym = \sym \Gradients$ are $\delta$, $\delta$ and $-2\delta$.
This preferential alignment is a property of 3D turbulence, partly due to vortex stretching \citep{ashurst1987alignment, tsinober1992experimental, gulitski2007velocity}.
With this assumption the gradient tensor, $\Gradients$, has the form
\begin{equation}\label{turb:eq:gradient_tensor_model}
	\Gradients \approx \begin{pmatrix}
		\delta & 0 & \FlowVorticityScalar\\
		0 & \delta & 0\\
		-\FlowVorticityScalar & 0 & -2\delta
	\end{pmatrix},
\end{equation}
in the basis $(\hat{\vec{e}}_\alpha, \hat{\vec{e}}_\beta, \hat{\vec{e}}_\gamma)$ formed by the normed eigenvectors of $\GradientsSym$, with $\FlowVorticityScalar$ the norm of vorticity.
The basis is chosen so that $\hat{\vec{e}}_\beta \cdot \FlowVorticity > 0$.

As in the previous models, we look for an estimation of performance as a function of the time horizon $\TimeHorizon$.
We start by an estimation of the term $[ \exp \left( \TimeHorizon_{\Gradients} \Gradients \right) \cdot \ControlDirectionOpt(\TimeHorizon)] \cdot \Direction$.
To compute this estimation  we use the following approximation for the matrix exponential
\begin{equation}\label{turb:eq:splitting}
	\exp \left( \TimeHorizon_{\GradientsAsym} \Gradients \right) \approx \exp \left( \TimeHorizon_{\Gradients} \GradientsSym \right) \, \exp \left( \TimeHorizon_{\Gradients} \GradientsAsym \right).
\end{equation}
This expression is exact if $\sym \Gradients$ and $\asym \Gradients$ commute, which is not the case here, and is a first order approximation otherwise [$O(\TimeHorizon_{\Gradients} \norm{ \Gradients})$].
Note that higher order splitting methods could be used, such as the \citet{strang1968construction} splitting method, to obtain a higher order approximation.

Note that as already observed above, nothing guarantees that the duration $\TimeHorizon_{\Gradients}$ is the same for both the components $\GradientsSym$ and $\GradientsAsym$ of the full velocity gradient tensor $\Gradients$.
As a consequence, we consider two different values of this time: $\TimeHorizon_{\GradientsSym}$ and $\TimeHorizon_{\GradientsAsym}$ corresponding to each part of the flow velocity gradient.
Using both Eq.~\eqref{turb:eq:gradient_tensor_model} and Eq.~\eqref{turb:eq:splitting}, we obtain
\begin{multline}\label{turb:eq:term_full_grad}
	\left[ \exp \left( \TimeHorizon_{\Gradients} \Gradients \right) \cdot \ControlDirectionOpt(\TimeHorizon) \right] \cdot \Direction \propto \left[ e^{\delta (\TimeHorizon_{\GradientsSym} + \TimeHorizon)} \left[ \DirectionScalar_\alpha \left( \DirectionScalar_\alpha \cos \theta - \DirectionScalar_\gamma \sin \theta \right) + \DirectionScalar_\beta^2 \right] + \right. \\
	\left. e^{-2 \delta (\TimeHorizon_\GradientsSym + \TimeHorizon)} \DirectionScalar_\gamma \left( \DirectionScalar_\alpha \sin \theta + \DirectionScalar_\gamma \cos \theta \right) \right] / \sqrt{e^{2 \delta \TimeHorizon} \left[ \DirectionScalar_\alpha^2 + \DirectionScalar_\beta^2 \right] + e^{-4 \delta \TimeHorizon} \DirectionScalar_\gamma^2},
\end{multline}
with $\theta \equiv [ \FlowVorticityScalar (\TimeHorizon_\GradientsAsym - \TimeHorizon)/2 ]$.
Note the introduction of two different times $\TimeHorizon_\GradientsSym$ and $\TimeHorizon_\GradientsAsym$, both linked to their corresponding part of $\Gradients$.

We then use the same simplification as in the symmetric case ($e^{2\delta \TimeHorizon} \gg e^{-4 \delta \TimeHorizon}$).
The previous expression then reduces to
\begin{multline}
	\left[ \exp \left( \TimeHorizon_{\Gradients} \Gradients \right) \cdot \ControlDirectionOpt(\TimeHorizon) \right] \cdot \Direction \propto \left[ e^{\delta \TimeHorizon_\GradientsSym} \left[ \DirectionScalar_\alpha \left( \DirectionScalar_\alpha \cos \theta - \DirectionScalar_\gamma \sin \theta \right) + \DirectionScalar_\beta^2 \right] + \right. \\
	\left. e^{-2 \delta \TimeHorizon_\GradientsSym - 3\delta \TimeHorizon} \DirectionScalar_\gamma \left( \DirectionScalar_\alpha \sin \theta + \DirectionScalar_\gamma \cos \theta \right) \right] / \sqrt{\DirectionScalar_\alpha^2 + \DirectionScalar_\beta^2}.
\end{multline}
Then, we average over all possible orientation of $\Direction$ in the basis $(\hat{\vec{e}}_\alpha, \hat{\vec{e}}_\beta, \hat{\vec{e}}_\gamma)$
\begin{equation}
	\left\langle \left[ \exp \left( \TimeHorizon_{\Gradients} \Gradients \right) \cdot \ControlDirectionOpt(\TimeHorizon) \right] \cdot \Direction \right\rangle_{\Direction} \propto e^{\delta \TimeHorizon_\GradientsSym} \frac{\pi}{8} \left[ \left( 1 + 2 e^{-3 \delta (\TimeHorizon_\GradientsSym + \TimeHorizon)} \right) \cos \left( \omega \left[ \TimeHorizon_\GradientsAsym - \TimeHorizon \right]/2 \right) + 1 \right].
\end{equation}
Similarly to the previous cases, we obtain a similar expression for bottom-heavy swimmers.
Note that $e^{2\delta \TimeHorizon} \gg e^{-4 \delta \TimeHorizon}$ cannot be assumed here, and we use Eq.~\eqref{turb:eq:term_full_grad} evaluated for $\TimeHorizon = 0$
\begin{equation}
	\left\langle \left[ \exp \left( \TimeHorizon_{\Gradients} \Gradients \right) \cdot \ControlDirection_{\mathrm{\NameBhShort}} \right] \cdot \Direction \right\rangle_{\Direction} \propto \frac{e^{\delta \TimeHorizon_\GradientsSym}}{3} \left[ \left( 1 + e^{-3 \delta \TimeHorizon_\GradientsSym} \right) \cos \left( \omega \TimeHorizon_\GradientsAsym/2 \right) + 1 \right].
\end{equation}

Finally, as for the case of skew symmetric surfers, we integrate this expression over a Gaussian distribution of the vorticity norm $\FlowVorticityScalar$ to obtain the following estimation of performance
\begin{equation}\label{turb:eq:performance_surf_full_final}
	\left\langle \Performance \right\rangle \approx \SwimmingVelocity \, \frac{3\pi}{8} \, \frac{ \left( 1 + 2 e^{-3 \delta (\TimeHorizon_\GradientsSym + \TimeHorizon)} \right) \exp \left( -\left( \TimeHorizon_\GradientsAsym - \TimeHorizon \right)^2 \sigma_{\FlowVorticityScalar}^2 / 8 \right) + 1}{\left( 1 + e^{-3 \delta \TimeHorizon_\GradientsSym} \right) \exp \left( -{\TimeHorizon_\GradientsAsym}^2 \sigma_{\FlowVorticityScalar}^2 / 8 \right) + 1}.
\end{equation}
Using the same values of $\delta$, $\TimeHorizon_\GradientsSym$, $\sigma_{\FlowVorticityScalar}$ and $\TimeHorizon_\GradientsAsym$ fitted for the previous models, this model is plotted in Fig.~\ref{fig:surfing_partial_and_models}\textbf{(a-b)}.
Despite its inability to capture correctly performance for small and large values of the surfing time horizon $\TimeHorizon$, the model captures fairly well the maximal performance.
This maximal performance is obtained for $\TimeHorizon = \TimeHorizonOpt \approx \TimeHorizon_\GradientsAsym$ for which performance is estimated as
\begin{equation}\label{turb:eq:performance_surf_full_optimal}
	\left\langle \Performance \right\rangle(\TimeHorizonOpt) \approx \SwimmingVelocity \, \frac{3\pi}{4} \, \frac{ 1 + e^{-3 \delta (\TimeHorizon_\GradientsSym + \TimeHorizon_\GradientsAsym)}}{\left( 1 + e^{-3 \delta \TimeHorizon_\GradientsSym} \right) \exp \left( -{\TimeHorizon_\GradientsAsym}^2 \sigma_{\FlowVorticityScalar}^2 / 8 \right) + 1}.
\end{equation}
Note well that this expression is only valid for $\TimeHorizon_\GradientsAsym \delta \gg 1$.

Overall, depending on four parameters that require fitting, the model is not very satisfactory to predict performance prior to the evaluation of surfers in turbulence. 
But the final expression [Eq.~\eqref{turb:eq:performance_surf_full_optimal}] can be used to guess qualitatively how maximal surfing performance varies with flow properties.
Based on this expression, one may expect surfing performance to vary mainly with $\TimeHorizon_\GradientsAsym \sigma_{\FlowVorticityScalar}$.

Now assuming the swimming speed dependence of $\TimeHorizon_\GradientsSym$ and $\TimeHorizon_\GradientsAsym$ is given by the model described by Eq.~\ref{eq:time_horizon_model}, we may be able to capture the velocity dependence of maximal surfing performance.
Replacing $\TimeHorizon_\GradientsSym$ and $\TimeHorizon_\GradientsAsym$ by $\TimeHorizon_\GradientsSym / [1 + \alpha_{\mathrm{swim}} (\SwimmingVelocity / \KolmogorovVelocityScale - 1)]$ and $\TimeHorizon_\GradientsAsym / [1 + \alpha_{\mathrm{swim}} (\SwimmingVelocity / \KolmogorovVelocityScale - 1)]$, we obtain the model plotted in Fig~\ref{fig:surfing_partial_and_models}\textbf{(b)}.
The model overestimates surfing performance. 
This suggests that the influence of swimming velocity $\SwimmingVelocity$ on the problem is not completely understood.
For instance, by inducing preferential sampling of the gradients, $\SwimmingVelocity$ could influence both parameters $\delta$ and $\sigma_{\FlowVorticityScalar}$.
Moreover, the evolution of $\TimeHorizon_\GradientsSym$ and $\TimeHorizon_\GradientsAsym$ with swimming velocity $\SwimmingVelocity$ might not be same as opposed the assumption above.
This highlights the importance for future research to investigate how the statistics of the flow sampled by micro-swimmers are influenced by their swimming velocity $\SwimmingVelocity$ and behavior.

\section{Summary}

We conclude this chapter by summing up key elements discussed previously
\begin{itemize}
	\item the flow can be exploited for navigation even in a turbulent flow regardless of turbulence intensity
	\item surfing performance is due to the preferential flow sampling of the flow
	\item this strategy can lead to a vertical effective speed that is twice the actual swimming speed: $\Performance \approx 2 \SwimmingVelocity$
	\item the component of the flow that is responsible for most of the performance is the rotational component $\GradientsAsym$, which is the most difficult to sense for organisms (using setae) because it does not yield body stretching (c.f. Chap.~\ref{chap:intro}, App.~\ref{sec:intro_flow_sensing})
\end{itemize}
