\chapter{Expected plankter orientation}\label{app:orientation_statistics}

\todo{Bien vérifier que mes calculs/raisonements sont corrects ici. Jusqu'à présent ça ne l'était pas.}

\section{Expected orientation with respect to horizontal vorticity}\label{sec:expected_orientation_vorticity}

\subsection{The expected orientation of bottom-heavy plankters}

\begin{wrapfigure}[15]{L}[0.5\width]{0.42\textwidth}
	\centering
	\vspace{-15pt}
	\def\svgwidth{0.38\textwidth}
	\input{chap_end/schemes/bottom_heavy_coordinate_system.pdf_tex}
	\captionsetup{width=0.38\textwidth}
  	\caption{
  		Illustration of the coordinate system considered in Sec.~\ref{sec:expected_orientation_vorticity} to describe the orientation of the swimming direction $\SwimmingDirection$ of plankters.
  		The flow vorticity is noted $\FlowVorticity$ and the vertical is noted $\Direction$.
  	}
  	\label{fig:bottom_heavy_coordinate_system}
\end{wrapfigure}
The orientation of a small spherical bottom-heavy plankter can be modeled as \citep{Pedley1992}
\begin{equation}
	\frac{d \SwimmingDirection}{d t}  =
	\frac{1}{2} \FlowVorticity (\ParticlePosition, t) \times \SwimmingDirection + \frac{1}{2 \ReorientationTime} \left[ \Direction - (\Direction \cdot \SwimmingDirection) \SwimmingDirection \right],
\end{equation}
with $\SwimmingDirection$ the swimming axis of the plankter, $\FlowVorticity = \vec{\nabla} \times \FlowVelocity$ the flow vorticity and $\ReorientationTime$ the characteristic alignment time towards the vertical $\Direction$ due to bottom-heaviness.
We consider here the case for which $\ReorientationTime \ll \KolmogorovTimeScale$, meaning the orientation dynamics only mostly depends on the local rotation rate $\GradientsAsym$ and not from the past rotation rate sampled.

In this section we describe the orientation of the plankter swimming direction $\SwimmingDirection$ in the orthonormal basis defined by the direction of horizontal vorticity $\hat{\FlowVorticity}_{\perp \Direction} = \FlowVorticity_{\perp \Direction} / \norm{\FlowVorticity_{\perp \Direction}}$ (with the horizontal vorticity defined as $\FlowVorticity_{\perp \Direction} = \FlowVorticity - \FlowVorticity_\DirectionScalar \Direction$), $\hat{\FlowVorticity}_{\perp \Direction} \times \Direction$ and the vertical $\Direction$ (Fig.~\ref{fig:bottom_heavy_coordinate_system}).
We note $\theta$ the angle between the vertical $\Direction$ and the swimming direction $\SwimmingDirection$.
We note $\phi$ the signed angle along the vertical axis with $\phi = 0$ corresponding to the projection of the swimming direction on the horizontal plane being aligned with the horizontal vorticity (Illustrated in Fig.~\ref{fig:bottom_heavy_coordinate_system}).
We now derive the orientation of bottom-heavy plankters in this basis.

As bottom-heaviness does not play any role in the horizontal dynamics of the dynamics, for a given vertical vorticity $\FlowVorticityScalar_{\DirectionScalar}$ we have directly
\begin{equation}\label{app:eq:spherical_bottom_phi}
	\phi_{\mathrm{\NameBhShort}} = \frac{1}{2} \FlowVorticityScalar_{\DirectionScalar} t + \phi_{\mathrm{\NameBhShort}, 0}
\end{equation}
with $\phi_{\mathrm{\NameBhShort}} = \phi_{\mathrm{\NameBhShort}, 0}$ for $t = 0$.
We further assume that the vertical rotation rate is small compared to the horizontal dynamics: $\FlowVorticityScalar_{\DirectionScalar} \ReorientationTime / 2 \ll 1$.
This means that the plankter has the time to reached the equilibrium position for each value of $\phi(t)$ as it evolves in time.
In that case the vertical orientation of the plankter $\theta_{\mathrm{\NameBhShort}}$ reads directly
\begin{equation}\label{app:eq:spherical_bottom_heavy_theta}
	\sin \theta_{\mathrm{\NameBhShort}} = \ReorientationTime \FlowVorticityScalar_{\perp \Direction} \cos \phi_{\mathrm{\NameBhShort}}
\end{equation}
with $\FlowVorticityScalar_{\perp \Direction} = \norm{\FlowVorticity_{\perp \Direction}}$.
Note that in that case $\phi_{\mathrm{\NameBhShort}}$ does not describe but the whole range $\phi_{\mathrm{\NameBhShort}} \in [0, 2\pi]$ but rather $\phi_{\mathrm{\NameBhShort}} \in [0, \pi/4]$.
This describes the trajectory of the orientation of a strongly bottom-heavy plankter in flow vorticity.

We are now interested in how this translates in term of statistics projected on a given plane.
To do so we consider an horizontal plane formed by a given horizontal vector $\hat{\vec{e}}_x$ and the vertical $\Direction$ of normal $\hat{\vec{e}}_{y} = \Direction \times \hat{\vec{e}}_{x}$.
This direction is orthogonal to the vertical $\Direction$.
Its orientation in the plane $(\hat{\FlowVorticity}_{\perp \Direction}, \hat{\FlowVorticity}_{\perp \Direction} \times \Direction)$ is described by the angle $\phi_{x}$.
The \textbf{signed} angle that can be measured in such experiment corresponds to $\theta_{\mathrm{\NameBhShort}, y}$.
This angle is the projection of the angle $\theta_{\mathrm{\NameBhShort}}$ in the plane $(\hat{\vec{e}}_x, \Direction)$.
This angle reads
\begin{equation}\label{app:eq:spherical_bottom_heavy_orientation}
	\tan \theta_{\mathrm{\NameBhShort}, y} = \frac{\cos \left( \phi_{x} - \phi_{\mathrm{\NameBhShort}} \right) \sin \theta_{\mathrm{\NameBhShort}}}{\cos \theta_{\mathrm{\NameBhShort}}}
\end{equation}
Finally, the only measure of vorticity accessible through the experiment is $\FlowVorticityScalar_{y} = \FlowVorticity \cdot \hat{\vec{e}}_{y}$, the component along $\hat{\vec{e}}_{y}$.
Thus we need to assess the expected influence of $\FlowVorticityScalar_{y}$ on $\theta_{\mathrm{\NameBhShort}, y}$.
To do so, we can decompose $\FlowVorticityScalar_{\perp \Direction}$ as a function of $\FlowVorticityScalar_{y}$ and $\FlowVorticityScalar_{x} = \FlowVorticity \cdot \hat{\vec{e}}_{x}$.
\begin{equation}\label{app:eq:horizontal_vorticity_expression}
	\FlowVorticityScalar_{\perp \Direction} = \sqrt{\FlowVorticityScalar_{x}^2 + \FlowVorticityScalar_{y}^2}.
\end{equation}
That said if the angle $\theta_{\mathrm{\NameBhShort}, y}$ is measured from experimental data and $\tan \theta_{\mathrm{\NameBhShort}, y}$ is evaluated and averaged over $N$ plankters and over time, and further assuming that $\FlowVorticityScalar_{x}$ and $\FlowVorticityScalar_{y}$ are independent, it should result in
\begin{equation}\label{app:eq:final_correlation_vort_bh}
	\left\langle \tan \theta_{\mathrm{\NameBhShort}, y} \right\rangle_{N, t} = \frac{1}{4 \pi^2} \int_{-\infty}^{+\infty} \int_0^{2\pi} \int_0^{\pi/4} p(\FlowVorticityScalar_x) \frac{\cos \left( \phi_{x} - \phi_{\mathrm{\NameBhShort}} \right) \sin \theta_{\mathrm{\NameBhShort}}}{\cos \theta_{\mathrm{\NameBhShort}}} \, d \phi_{\mathrm{\NameBhShort}} \, d\phi_{x} \, d\FlowVorticityScalar_x
\end{equation}
with $p(\FlowVorticityScalar_yx$ the normed probability function of the vorticity along $\hat{\vec{e}}_x$ sampled along the trajectory of plankters and with $\phi_{\mathrm{\NameBhShort}}$ given by Eq.~\eqref{app:eq:spherical_bottom_heavy_theta} and $\FlowVorticityScalar_{\perp \Direction}$ given by Eq.~\eqref{app:eq:horizontal_vorticity_expression}.
In practice if the flow vorticity sampled by bottom-heavy swimmers is assumed to be isotropic in the horizontal plane, we can deduce this probability density function $p(\FlowVorticityScalar_x)$ from the statistics of $\FlowVorticityScalar_{y}$ [$p(\FlowVorticityScalar_x) \approx p(\FlowVorticityScalar_y)$].

Using the experimental data to compute $p(\FlowVorticityScalar_x)$, this previous expression of $\left\langle \tan \theta_{\mathrm{\NameBhShort}, y} \right\rangle_{N, t}$ could be evaluated for various values of $\ReorientationTime$ and compared to the actual value $\left\langle \tan \theta_{\mathrm{\NameBhShort}, y} \right\rangle_{N, t}$ measured in the experiment.
It would highlight if the plankers exhibit a bottom-heavy behavior or not.

\subsection{The expected orientation of surfing plankters}

As presented throughout the study, in contrast to bottom-heavy swimmers, surfers tend to oppose to vorticity when reorienting.
If surfers only react to horizontal vorticity, the average surfing direction corresponds the rotation of the vertical direction opposed to vorticity, expressed as
\begin{equation}
	\ControlDirectionOpt(\TimeHorizon) = \matr{R}_{\hat{\FlowVorticity}}(-\omega \TimeHorizon/2) \, \Direction,
\end{equation}
with $\matr{R}_{\hat{\FlowVorticity}}(-\FlowVorticityScalar \TimeHorizon/2)$ the rotation matrix of angle $-\FlowVorticityScalar \TimeHorizon/2$ with $\omega = \norm{\vec{\nabla} \times \FlowVelocity}$ and of axis $\hat{\FlowVorticity} = \FlowVorticity/\FlowVorticityScalar$, the normalized vorticity.

Performing a computation similar to the bottom-heavy case presented above, we obtain
\begin{subequations}
	\begin{align}
		\left\langle \tan \theta_{\mathrm{\NameSurfShort}, y} \right\rangle_{N, t} &= \frac{1}{4 \pi^2} \int_{-\infty}^{+\infty} \int_0^{2\pi} \int_{-\pi/4}^{0} p(\FlowVorticityScalar_x) \frac{\cos \left( \phi_{x} - \phi_{\mathrm{\NameSurfShort}} \right) \sin \theta_{\mathrm{\NameSurfShort}}}{\cos \theta_{\mathrm{\NameSurfShort}}} \, d \phi_{\mathrm{\NameSurfShort}} \, d\phi_{x} \, d\FlowVorticityScalar_x\\
		\text{with} \quad \sin \theta_{\mathrm{\NameSurfShort}} &= \TimeHorizon \sqrt{\FlowVorticityScalar_{x}^2 + \FlowVorticityScalar_{y}^2} \cos \phi_{\mathrm{\NameSurfShort}}
	\end{align}
\end{subequations}

\todo{If I have time, reduce these expressions and assess the validity of these results with simulations. Also illustrate these expressions with Gaussian probability functions.}

\section{Expected with respect to the horizontal gradient of vertical velocity}

As discussed above in Chap.~\ref{chap:surfing_robustness}, Sec.~\ref{sec:computational_power}, most of the surfing performance is captured through the first order computation of the exponential.
In that case, if plankters are able to measure more than horizontal vorticity, the surfing direction could be directly linked to horizontal vertical velocity gradients $\vec{\nabla}_{xy} \FlowVelocityScalar_{\DirectionScalar}$.
This provide another differentiating cue to highlight surfing-like behavior in experimental data.

\subsection{Expected orientation of bottom-heavy plankters}

The orientation statistics of plankters are still described by Eq.~\eqref{app:eq:final_correlation_vort_bh}.
We now want to express the angle $\theta_{\mathrm{\NameBhShort}, x}$ as a function of $\partial \FlowVelocityScalar_z / \partial x$ that can be directly obtained as $\FlowVorticityScalar_y = \partial \FlowVelocityScalar_x / \partial z - \partial \FlowVelocityScalar_z / \partial x$.
We obtain then
\begin{multline}
	\left\langle \tan \theta_{\mathrm{\NameBhShort}, y} \right\rangle_{N, t} = \\ 
	\frac{1}{4 \pi^2} \int_{-\infty}^{+\infty} \int_{-\infty}^{+\infty} \int_0^{2\pi} \int_0^{2\pi} p(\FlowVorticityScalar_x) p \left( \frac{\partial \FlowVelocityScalar_x}{\partial z} \right) \frac{\cos \left( \phi_{x} - \phi_{\mathrm{\NameBhShort}} \right) \sin \theta_{\mathrm{\NameBhShort}}}{\cos \theta_{\mathrm{\NameBhShort}}} \, d \phi_{\mathrm{\NameBhShort}} \, d\phi_{x} \, d\FlowVorticityScalar_x \, d\left( \frac{\partial \FlowVelocityScalar_x}{\partial z} \right) \\
	\text{with} \quad \sin \theta_{\mathrm{\NameBhShort}} = \TimeHorizon \cos \phi_{\mathrm{\NameBhShort}} \sqrt{\left( \frac{\FlowVelocityScalar_x}{\partial z} - \frac{\partial \FlowVelocityScalar_z}{\partial x} \right)^2 + \FlowVorticityScalar_{y}^2}
\end{multline}

\subsection{Expected orientation of surfing plankters}

As showed above, at first order, the surfing strategy reduces to
\begin{equation}
	\ControlDirectionOpt(\TimeHorizon) = \frac{\ControlDirectionOptNN}{\norm{\ControlDirectionOptNN}} \quad \text{with} \quad \ControlDirectionOptNN = \Direction + \TimeHorizon \vec{\nabla} \FlowVelocityScalar_{\DirectionScalar}.
\end{equation}
The angle to the vertical in the plane $(\Direction, \hat{\vec{e}}_x)$ can then directly be expressed as
\begin{equation}\label{app:eq:shear_surf}
	\tan \theta_{\mathrm{\NameSurfShort}} = \frac{\TimeHorizon (\partial\FlowVelocityScalar_{\DirectionScalar}/\partial x)}{1 + \TimeHorizon (\partial \FlowVelocityScalar_{\DirectionScalar} / \partial \DirectionScalar)}.
\end{equation}
Similarly to the computations performed above, we average this expression and obtain
\begin{equation}\label{app:eq:shear_surf}
	\left\langle \tan \theta_{\mathrm{\NameSurfShort}} \right\rangle_{N, t} = \int_{-\infty}^{+\infty} p\left( \frac{\partial \FlowVelocityScalar_\DirectionScalar}{\partial \DirectionScalar} \right) \frac{\TimeHorizon (\partial\FlowVelocityScalar_{\DirectionScalar}/\partial x)}{1 + \TimeHorizon (\partial \FlowVelocityScalar_{\DirectionScalar} / \partial \DirectionScalar)} \, d \left( \frac{\partial \FlowVelocityScalar_\DirectionScalar}{\partial \DirectionScalar} \right).
\end{equation}
