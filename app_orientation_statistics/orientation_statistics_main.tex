\chapter{Expected plankter orientation}\label{app:orientation_statistics}

For the sake of simplicity, apart from the diagonal components that are linked through incompressibility, we consider here that the components of the gradient tensor, $\Gradients$, are independant.
In practice, this is known not to be true in turbulence however we do not expect the interdependance of the coefficients to influence much the following results.

\section{Bottom-heavy plankters}

\begin{wrapfigure}[15]{L}[0.5\width]{0.42\textwidth}
	\centering
	\vspace{-20pt}
	\def\svgwidth{0.38\textwidth}
	\input{chap_end/schemes/bottom_heavy_coordinate_system.pdf_tex}
	\captionsetup{width=0.38\textwidth}
  	\caption{
  		Illustration of the coordinate system considered to describe the orientation of the swimming direction $\SwimmingDirection$ of plankters.
  		The flow vorticity is noted $\FlowVorticity$ and the vertical is noted $\Direction$.
  	}
  	\label{fig:bottom_heavy_coordinate_system}
\end{wrapfigure}
The orientation of a small spherical bottom-heavy plankter can be modeled as \citep{Pedley1992}
\begin{equation}
	\frac{d \SwimmingDirection}{d t}  =
	\frac{1}{2} \FlowVorticity (\ParticlePosition, t) \times \SwimmingDirection + \frac{1}{2 \ReorientationTime} \left[ \Direction - (\Direction \cdot \SwimmingDirection) \SwimmingDirection \right],
\end{equation}
with $\SwimmingDirection$ the swimming axis of the plankter, $\FlowVorticity = \vec{\nabla} \times \FlowVelocity$ the flow vorticity and $\ReorientationTime$ the characteristic alignment time towards the vertical $\Direction$ due to bottom-heaviness.
We consider here the case for which $\ReorientationTime \ll \KolmogorovTimeScale$, meaning the orientation dynamics only mostly depends on the local rotation rate $\GradientsAsym$ and not from the past rotation rate sampled.

In this section we describe the orientation of the plankter swimming direction $\SwimmingDirection$ in the orthonormal basis defined by the direction of horizontal vorticity $\hat{\FlowVorticity}_{\perp \Direction} = \FlowVorticity_{\perp \Direction} / \norm{\FlowVorticity_{\perp \Direction}}$ (with the horizontal vorticity defined as $\FlowVorticity_{\perp \Direction} = \FlowVorticity - \FlowVorticity_\DirectionScalar \Direction$), $\hat{\FlowVorticity}_{\perp \Direction} \times \Direction$ and the vertical $\Direction$ (Fig.~\ref{fig:bottom_heavy_coordinate_system}).
We note $\theta$ the angle between the vertical $\Direction$ and the swimming direction $\SwimmingDirection$.
We note $\phi$ the signed angle along the vertical axis with $\phi = 0$ corresponding to the projection of the swimming direction on the horizontal plane being aligned with the horizontal vorticity (Illustrated in Fig.~\ref{fig:bottom_heavy_coordinate_system}).
We now derive the orientation of bottom-heavy plankters in this basis.

As bottom-heaviness does not play any role in the horizontal dynamics of the dynamics, for a given vertical vorticity $\FlowVorticityScalar_{\DirectionScalar}$ we have directly
\begin{equation}\label{app:eq:spherical_bottom_phi}
	\frac{d \phi_{\mathrm{\NameBhShort}} }{dt} = \frac{1}{2} \FlowVorticityScalar_{\DirectionScalar} t
\end{equation}
with $\phi_{\mathrm{\NameBhShort}} = \phi_{\mathrm{\NameBhShort}, 0}$ for $t = 0$.
We further assume that the vertical rotation rate is small compared to the horizontal dynamics: $\FlowVorticityScalar_{\DirectionScalar} \ReorientationTime / 2 \ll 1$.
This means that the plankter has the time to reached the equilibrium position for each value of $\phi_{\NameBhShort}(t)$ as it evolves in time.
In that case the vertical orientation of the plankter $\theta_{\mathrm{\NameBhShort}}$ reads directly
\begin{equation}\label{app:eq:spherical_bottom_heavy_theta}
	\sin \theta_{\mathrm{\NameBhShort}} = \ReorientationTime \FlowVorticityScalar_{\perp \Direction} \cos \phi_{\mathrm{\NameBhShort}}
\end{equation}
with $\FlowVorticityScalar_{\perp \Direction} = \norm{\FlowVorticity_{\perp \Direction}}$.
Note first that the equilibrium position can only be reached if $\ReorientationTime \FlowVorticityScalar_{\perp \Direction} \cos \phi_{\mathrm{\NameBhShort}} < 1$.
Note further that in the case $\ReorientationTime \FlowVorticityScalar_{\perp \Direction} < 1$, the equilibrium position can always be reached independently of $\phi_{\mathrm{\NameBhShort}}$ and in that case $\phi_{\mathrm{\NameBhShort}}$ does not describe the whole range $[0, 2\pi]$ as the plankter orients back vertically when $\phi_{\mathrm{\NameBhShort}} = \pi/2$ [Eq.~\eqref{app:eq:spherical_bottom_heavy_theta}].
Therefore it rather takes periodically the values $\phi_{\mathrm{\NameBhShort}} \in [0, \pi/2]$.
These equation are then enough to describe the trajectory of the orientation of a strongly bottom-heavy plankter in flow vorticity.

We are now interested in how this translates in term of statistics projected on a given plane.
To do so we consider an horizontal plane formed by a given horizontal vector $\hat{\vec{e}}_x$ and the vertical $\Direction$ of normal $\hat{\vec{e}}_{y} = \Direction \times \hat{\vec{e}}_{x}$.
The direction $\hat{\vec{e}}_x$ is orthogonal to the vertical $\Direction$ and
its orientation in the plane $(\hat{\FlowVorticity}_{\perp \Direction} \times \Direction, \hat{\FlowVorticity}_{\perp \Direction})$ is described by the angle $\phi_{x}$.

Our aim is to characterize the average swimming direction $\langle \SwimmingDirection_{\mathrm{\NameBhShort}} \rangle_{t, \FlowVorticityScalar_x, \FlowVorticityScalar_z} (\FlowVorticity_y)$ as a function of the measurable component of vorticity: $\FlowVorticity_y$.
For $-1/\ReorientationTime < \FlowVorticityScalar_y < 1/\ReorientationTime$, the vertical component of this swimming direction can be easily computed as
\begin{subequations}
	\begin{align}
		\left\langle \SwimmingDirection_{\mathrm{\NameBhShort}, z} \right\rangle_{t, \FlowVorticityScalar_x, \FlowVorticityScalar_z} &= \frac{\int_{-\omega_{\max}}^{+\omega_{\max}} \int_0^{\pi/2} p(\FlowVorticityScalar_x) \cos \theta_{\mathrm{\NameBhShort}} \sin \theta_{\mathrm{\NameBhShort}} \, d \phi_{\mathrm{\NameBhShort}} \, d\FlowVorticityScalar_x}{\int_0^{\pi/2} \sin \theta_{\mathrm{\NameBhShort}} \, d \phi_{\mathrm{\NameBhShort}}},\\
		\text{with} \quad \omega_{\max} &= \sqrt{1/\ReorientationTime^2 - \FlowVorticityScalar_y^2}
	\end{align}
\end{subequations}
with $p(\FlowVorticityScalar_x)$ the normalized probability function of the vorticity component $\FlowVorticityScalar_x$ sampled by bottom-heavy swimmers.
Using Eq.~\eqref{app:eq:spherical_bottom_heavy_theta}, this expression simplifies to
\begin{subequations}
	\begin{align}
		\left\langle \SwimmingDirection_{\mathrm{\NameBhShort}, z} \right\rangle_{t, \FlowVorticityScalar_x, \FlowVorticityScalar_z} &= \frac{1}{2} \int_{-\omega_{\max}}^{+\omega_{\max}} p(\FlowVorticityScalar_x) \left[ 1 + \tanh^{-1} \left( \ReorientationTime \FlowVorticityScalar_{\perp \Direction} \right) \frac{1 - \ReorientationTime^2 \FlowVorticityScalar_{\perp \Direction}^2}{\ReorientationTime \FlowVorticityScalar_{\perp \Direction}} \right]  \, d\FlowVorticityScalar_x\\
		\text{with} \quad \omega_{\max} &= \sqrt{1/\ReorientationTime^2 - \FlowVorticityScalar_y^2} \quad \text{and} \quad \FlowVorticityScalar_{\perp \Direction} = \sqrt{\FlowVorticityScalar_{x}^2 + \FlowVorticityScalar_{y}^2},
	\end{align}
\end{subequations}
with $\tanh^{-1}$ the inverse hyperbolic tangent function.
Note that in practice if the flow vorticity sampled by bottom-heavy swimmers is assumed to be isotropic in the horizontal plane, we can deduce this probability density function $p(\FlowVorticityScalar_x)$ from the statistics of $\FlowVorticityScalar_{y}$ [$p(\FlowVorticityScalar_x) \approx p(\FlowVorticityScalar_y)$].

Similarly, one may then express the expected horizontal component of the swimming direction $\langle \SwimmingDirection_{\mathrm{\NameBhShort}, x} \rangle_{t, \FlowVorticityScalar_x, \FlowVorticityScalar_z}$
\begin{subequations}
	\begin{align}
		\left\langle \SwimmingDirection_{\mathrm{\NameBhShort}, x} \right\rangle_{t, \FlowVorticityScalar_x, \FlowVorticityScalar_z} &= \frac{\int_{-\omega_{\max}}^{+\omega_{\max}} \int_0^{\pi/2} p(\FlowVorticityScalar_x) \sin^2 \theta_{\mathrm{\NameBhShort}} \cos \left( \phi_{\mathrm{\NameBhShort}} - \phi_x \right) \, d \phi_{\mathrm{\NameBhShort}} \, d\FlowVorticityScalar_x}{\int_0^{\pi/2} \sin \theta_{\mathrm{\NameBhShort}} \, d \phi_{\mathrm{\NameBhShort}}},\\
		\text{with} \quad \tan \left( \phi_x \right) &= \frac{\FlowVorticityScalar_x}{\FlowVorticityScalar_y},
	\end{align}
\end{subequations}
and using again Eq.~\eqref{app:eq:spherical_bottom_heavy_theta}
\begin{equation}
	\left\langle \SwimmingDirection_{\mathrm{\NameBhShort}, x} \right\rangle_{t, \FlowVorticityScalar_x, \FlowVorticityScalar_z} = \frac{1}{3} \int_{-\omega_{\max}}^{+\omega_{\max}} p(\FlowVorticityScalar_x) \ReorientationTime ( \omega_x + 2 \omega_y) \, d\FlowVorticityScalar_x.
\end{equation}
Finally, assuming $p(\FlowVorticityScalar_x)$ is centered on $0$, we obtain
\begin{equation}
	\left\langle \SwimmingDirection_{\mathrm{\NameBhShort}, x} \right\rangle_{t, \FlowVorticityScalar_x, \FlowVorticityScalar_z} = \frac{2}{3} \ReorientationTime \FlowVorticityScalar_y.
\end{equation}

Using experimental data to estimate $p(\FlowVorticityScalar_x)$ one can use these expressions to evaluate the expected orientation of bottom-heavy plankters for various reorientation times $\ReorientationTime$.
It would highlight if the plankter considered in experiments exhibits a bottom-heavy behavior.
\begin{figure}
	\centering
	\begin{tikzpicture}
	\begin{groupplot}[
			group style={
				group size=2 by 1,
				%y descriptions at=edge left,
				horizontal sep=0.14\linewidth,
				%vertical sep=0.08\linewidth,
			},
			axis on top,
			% size
			width=0.47\textwidth,
			%ymode=log,
			% layers
			set layers ,
			% legend
			legend style={draw=none, fill=none, /tikz/every even column/.append style={column sep=4pt}, at={(1.0, 1.1)}, anchor=south},
			%legend pos=north east,
			legend cell align=right,
			legend columns=4,
		]
		% n_{surf, z}
		\nextgroupplot[
			% x
			xlabel=$\FlowVorticityScalar_y \KolmogorovTimeScale$,
			xmin=-2,
			xmax=2,
			% y
			ylabel=$\langle \SwimmingDirection_{\DirectionScalar} \rangle$,
			ymin=-1,
			ymax=1,
			ytick={-1,0,1},
		]
			\addlegendimage{empty legend}\addlegendentry{\NameSurf, $\TimeHorizon =$}
			% tau 1.0
			\addplot[
				ColorSurf!50!ColorDuration,
				mark=triangle*,
				only marks ,
			] table [
				x expr={\thisrowno{0}},
				y expr={\thisrowno{1}},
				col sep=comma, 
				comment chars=\#,
				unbounded coords=discard,
			] {data/surfers__flow__n_128__re_250/surfer__vs_1o0__surftimeconst_1o0__vorticity_z__p_x.csv};
			\addlegendentry{$\KolmogorovTimeScale$}
			\addplot[
				ColorSurf!50!ColorDuration,
				forget plot,
			] table [
				x expr={\thisrowno{0}},
				y expr={\thisrowno{1}},
				col sep=comma, 
				comment chars=\#,
				unbounded coords=discard,
			] {data/surfers__flow__n_128__re_250/surfer__vs_1o0__surftimeconst_1o0__vorticity_z__model_p_x.csv};
			% tau 2.0
			\addplot[
				ColorSurf!75!ColorDuration,
				mark=square*,
				only marks,
			] table [
				x expr={\thisrowno{0}},
				y expr={\thisrowno{1}},
				col sep=comma, 
				comment chars=\#,
				unbounded coords=discard,
			] {data/surfers__flow__n_128__re_250/surfer__vs_1o0__surftimeconst_2o0__vorticity_z__p_x.csv};
			\addlegendentry{$2\KolmogorovTimeScale$}
			\addplot[
				ColorSurf!75!ColorDuration,
				forget plot,
			] table [
				x expr={\thisrowno{0}},
				y expr={\thisrowno{1}},
				col sep=comma,
				comment chars=\#,
				unbounded coords=discard,
			] {data/surfers__flow__n_128__re_250/surfer__vs_1o0__surftimeconst_2o0__vorticity_z__model_p_x.csv};
			% tau 4.0
			\addplot[
				ColorSurf!100!ColorDuration,
				mark=*,
				only marks,
			] table [
				x expr={\thisrowno{0}},
				y expr={\thisrowno{1}},
				col sep=comma, 
				comment chars=\#,
				unbounded coords=discard,
			] {data/surfers__flow__n_128__re_250/surfer__vs_1o0__surftimeconst_4o0__vorticity_z__p_x.csv};
			\addlegendentry{$4\KolmogorovTimeScale$}
			\addplot[
 				ColorSurf!100!ColorDuration,
 				forget plot,
 			] table [
 				x expr={\thisrowno{0}},
 				y expr={\thisrowno{1}},
 				col sep=comma,
 				comment chars=\#,
 				unbounded coords=discard,
 			] {data/surfers__flow__n_128__re_250/surfer__vs_1o0__surftimeconst_4o0__vorticity_z__model_p_x.csv};		 
			\addlegendimage{empty legend}\addlegendentry{\NameBh, $\ReorientationTime=$}
			% reorientationtime 0.5
			\addplot[
				ColorBh!50!ColorDuration,
				mark=triangle,
				only marks,
			] table [
				x expr={\thisrowno{0}},
				y expr={\thisrowno{1}},
				col sep=comma, 
				comment chars=\#,
				unbounded coords=discard,
			] {data/control_surfers__flow__n_128__re_250/spherical_riser__vs_1o0__reorientationtime_0o5__vorticity_z__p_x.csv};
			\addlegendentry{$\KolmogorovTimeScale/2$}
			\addplot[
				ColorBh!50!ColorDuration,
				forget plot,
			] table [
				x expr={\thisrowno{0}},
				y expr={\thisrowno{1}},
				col sep=comma, 
				comment chars=\#,
				unbounded coords=discard,
			] {data/control_surfers__flow__n_128__re_250/spherical_riser__vs_1o0__reorientationtime_0o5__vorticity_z__model_p_x.csv};
			% reorientationtime 1.0
			\addplot[
				ColorBh!75!ColorDuration,
				mark=square,
				only marks,
			] table [
				x expr={\thisrowno{0}},
				y expr={\thisrowno{1}},
				col sep=comma, 
				comment chars=\#,
				unbounded coords=discard,
			] {data/control_surfers__flow__n_128__re_250/spherical_riser__vs_1o0__reorientationtime_1o0__vorticity_z__p_x.csv};
			\addlegendentry{$\KolmogorovTimeScale$}
			\addplot[
				ColorBh!75!ColorDuration,
				forget plot,
			] table [
				x expr={\thisrowno{0}},
				y expr={\thisrowno{1}},
				col sep=comma, 
				comment chars=\#,
				unbounded coords=discard,
			] {data/control_surfers__flow__n_128__re_250/spherical_riser__vs_1o0__reorientationtime_1o0__vorticity_z__model_p_x.csv};
			% reorientationtime 2.0
			\addplot[
				ColorBh!100!ColorDuration,
				mark=o,
				only marks,
			] table [
				x expr={\thisrowno{0}},
				y expr={\thisrowno{1}},
				col sep=comma, 
				comment chars=\#,
				unbounded coords=discard,
			] {data/control_surfers__flow__n_128__re_250/spherical_riser__vs_1o0__reorientationtime_2o0__vorticity_z__p_x.csv};
			\addlegendentry{$2\KolmogorovTimeScale$}
			\addplot[
				ColorBh!100!ColorDuration,
				forget plot,
			] table [
				x expr={\thisrowno{0}},
				y expr={\thisrowno{1}},
				col sep=comma, 
				comment chars=\#,
				unbounded coords=discard,
			] {data/control_surfers__flow__n_128__re_250/spherical_riser__vs_1o0__reorientationtime_2o0__vorticity_z__model_p_x.csv};
		% n_{surf, x}
		\nextgroupplot[
			% x
			xlabel=$\FlowVorticityScalar_y \KolmogorovTimeScale$,
			xmin=-2,
			xmax=2,
			% y
			ylabel=$\langle \SwimmingDirection_{x} \rangle$,
			ymin=-1,
			ymax=1,
			ytick={-1,0,1},
		]
			% tau 1.0
			\addplot[
				ColorSurf!50!ColorDuration,
				mark=triangle*,
				only marks,
			] table [
				x expr={\thisrowno{0}},
				y expr={\thisrowno{1}},
				col sep=comma, 
				comment chars=\#,
				unbounded coords=discard,
			] {data/surfers__flow__n_128__re_250/surfer__vs_1o0__surftimeconst_1o0__vorticity_z__p_y.csv};
			\addplot[
				ColorSurf!50!ColorDuration,
				forget plot
			] table [
				x expr={\thisrowno{0}},
				y expr={\thisrowno{1}},
				col sep=comma, 
				comment chars=\#,
				unbounded coords=discard,
			] {data/surfers__flow__n_128__re_250/surfer__vs_1o0__surftimeconst_1o0__vorticity_z__model_p_y.csv};
			% tau 2.0
			\addplot[
				ColorSurf!75!ColorDuration,
				mark=square*,
				only marks,
			] table [
				x expr={\thisrowno{0}},
				y expr={\thisrowno{1}},
				col sep=comma, 
				comment chars=\#,
				unbounded coords=discard,
			] {data/surfers__flow__n_128__re_250/surfer__vs_1o0__surftimeconst_2o0__vorticity_z__p_y.csv};
			\addplot[
				ColorSurf!75!ColorDuration,
				forget plot
			] table [
				x expr={\thisrowno{0}},
				y expr={\thisrowno{1}},
				col sep=comma, 
				comment chars=\#,
				unbounded coords=discard,
			] {data/surfers__flow__n_128__re_250/surfer__vs_1o0__surftimeconst_2o0__vorticity_z__model_p_y.csv};
			% tau 4.0
			\addplot[
				ColorSurf!100!ColorDuration,
				mark=*,
				only marks,
			] table [
				x expr={\thisrowno{0}},
				y expr={\thisrowno{1}},
				col sep=comma, 
				comment chars=\#,
				unbounded coords=discard,
			] {data/surfers__flow__n_128__re_250/surfer__vs_1o0__surftimeconst_4o0__vorticity_z__p_y.csv};
			\addplot[
				ColorSurf!100!ColorDuration,
				forget plot
			] table [
				x expr={\thisrowno{0}},
				y expr={\thisrowno{1}},
				col sep=comma, 
				comment chars=\#,
				unbounded coords=discard,
			] {data/surfers__flow__n_128__re_250/surfer__vs_1o0__surftimeconst_4o0__vorticity_z__model_p_y.csv};
			% reorientationtime 0.5
			\addplot[
				ColorBh!50!ColorDuration,
				mark=triangle,
				only marks,
			] table [
				x expr={\thisrowno{0}},
				y expr={\thisrowno{1}},
				col sep=comma, 
				comment chars=\#,
				unbounded coords=discard,
			] {data/control_surfers__flow__n_128__re_250/spherical_riser__vs_1o0__reorientationtime_0o5__vorticity_z__p_y.csv};
			\addplot[
				ColorBh!50!ColorDuration,
				forget plot,
			] table [
				x expr={\thisrowno{0}},
				y expr={\thisrowno{1}},
				col sep=comma, 
				comment chars=\#,
				unbounded coords=discard,
			] {data/control_surfers__flow__n_128__re_250/spherical_riser__vs_1o0__reorientationtime_0o5__vorticity_z__model_p_y.csv};
			% reorientationtime 1.0
			\addplot[
				ColorBh!75!ColorDuration,
				mark=square,
				only marks,
			] table [
				x expr={\thisrowno{0}},
				y expr={\thisrowno{1}},
				col sep=comma, 
				comment chars=\#,
				unbounded coords=discard,
			] {data/control_surfers__flow__n_128__re_250/spherical_riser__vs_1o0__reorientationtime_1o0__vorticity_z__p_y.csv};
			\addplot[
				ColorBh!75!ColorDuration,
				forget plot,
			] table [
				x expr={\thisrowno{0}},
				y expr={\thisrowno{1}},
				col sep=comma, 
				comment chars=\#,
				unbounded coords=discard,
			] {data/control_surfers__flow__n_128__re_250/spherical_riser__vs_1o0__reorientationtime_1o0__vorticity_z__model_p_y.csv};
			% reorientationtime 2.0
			\addplot[
				ColorBh!100!ColorDuration,
				mark=o,
				only marks,
			] table [
				x expr={\thisrowno{0}},
				y expr={\thisrowno{1}},
				col sep=comma, 
				comment chars=\#,
				unbounded coords=discard,
			] {data/control_surfers__flow__n_128__re_250/spherical_riser__vs_1o0__reorientationtime_2o0__vorticity_z__p_y.csv};
			\addplot[
				ColorBh!100!ColorDuration,
				forget plot,
			] table [
				x expr={\thisrowno{0}},
				y expr={\thisrowno{1}},
				col sep=comma, 
				comment chars=\#,
				unbounded coords=discard,
			] {data/control_surfers__flow__n_128__re_250/spherical_riser__vs_1o0__reorientationtime_2o0__vorticity_z__model_p_y.csv};
		% % angle_{surf, y}
		% \nextgroupplot[
			% % x
			% xlabel=$\FlowVorticityScalar_y \KolmogorovTimeScale$,
			% xmin=-2,
			% xmax=2,
			% % y
			% ylabel=$\theta _{\langle \SwimmingDirection_{\mathrm{\NameSurfShort}} \rangle, y}$,
			% ymin=-0.5*pi,
			% ymax=0.5*pi,
			% ytick={-0.5*pi,0,0.5*pi},
			% yticklabels={$-\pi/2$,0,$\pi/2$},
		% ]
			% % tau 1.0
			% \addplot[
				% ColorSurf!50!ColorDuration,
				% mark=triangle*,
			% ] table [
				% x expr={\thisrowno{0}},
				% y expr={\thisrowno{1}},
				% col sep=comma,
				% comment chars=\#,
				% unbounded coords=discard,
			% ] {data/control_surfers__flow__n_128__re_250/control_surfer__vs_1o0__surftimeconst_1o0__omegamax_1o0__vorticity_z__angle_z.csv};
			% % tau 2.0
			% \addplot[
				% ColorSurf!75!ColorDuration,
				% mark=square*,
			% ] table [
				% x expr={\thisrowno{0}},
				% y expr={\thisrowno{1}},
				% col sep=comma,
				% comment chars=\#,
				% unbounded coords=discard,
			% ] {data/control_surfers__flow__n_128__re_250/control_surfer__vs_1o0__surftimeconst_2o0__omegamax_1o0__vorticity_z__angle_z.csv};
			% % tau 4.0
			% \addplot[
				% ColorSurf!100!ColorDuration,
				% mark=*,
			% ] table [
				% x expr={\thisrowno{0}},
				% y expr={\thisrowno{1}},
				% col sep=comma,
				% comment chars=\#,
				% unbounded coords=discard,
			% ] {data/control_surfers__flow__n_128__re_250/control_surfer__vs_1o0__surftimeconst_4o0__omegamax_1o0__vorticity_z__angle_z.csv};
			% % reorientationtime
			% \addplot[
				% ColorBh!50!ColorDuration,
				% mark=triangle,
			% ] table [
				% x expr={\thisrowno{0}},
				% y expr={\thisrowno{1}},
				% col sep=comma,
				% comment chars=\#,
				% unbounded coords=discard,
			% ] {data/control_surfers__flow__n_128__re_250/spherical_riser__vs_1o0__reorientationtime_0o5__vorticity_z__angle_z.csv};
			% \addplot[
				% ColorBh!75!ColorDuration,
				% mark=square,
			% ] table [
				% x expr={\thisrowno{0}},
				% y expr={\thisrowno{1}},
				% col sep=comma,
				% comment chars=\#,
				% unbounded coords=discard,
			% ] {data/control_surfers__flow__n_128__re_250/spherical_riser__vs_1o0__reorientationtime_1o0__vorticity_z__angle_z.csv};
			% \addplot[
				% ColorBh!100!ColorDuration,
				% mark=o,
			% ] table [
				% x expr={\thisrowno{0}},
				% y expr={\thisrowno{1}},
				% col sep=comma,
				% comment chars=\#,
				% unbounded coords=discard,
			% ] {data/control_surfers__flow__n_128__re_250/spherical_riser__vs_1o0__reorientationtime_2o0__vorticity_z__angle_z.csv};
		% % angle_{surf, y}
		% \nextgroupplot[
			% % x
			% xlabel=$(\partial \FlowVelocityScalar_{\DirectionScalar}/ \partial \FlowVelocityScalar_{x}) \KolmogorovTimeScale$,
			% xmin=-2,
			% xmax=2,
			% % y
			% %ylabel=$\theta _{\langle \SwimmingDirection_{\mathrm{\NameSurfShort}, y} \rangle}$,
			% ymin=-0.5*pi,
			% ymax=0.5*pi,
			% ytick={-0.5*pi,0,0.5*pi},
			% yticklabels={$-\pi/2$,0,$\pi/2$},
		% ]
			% % tau
			% \addplot[
				% ColorSurf!50!ColorDuration,
				% mark=triangle*,
			% ] table [
				% x expr={\thisrowno{0}},
				% y expr={\thisrowno{1}},
				% col sep=comma,
				% comment chars=\#,
				% unbounded coords=discard,
			% ] {data/control_surfers__flow__n_128__re_250/control_surfer__vs_1o0__surftimeconst_1o0__omegamax_1o0__duxdy__angle_z.csv};
			% \addplot[
				% ColorSurf!75!ColorDuration,
				% mark=square*,
			% ] table [
				% x expr={\thisrowno{0}},
				% y expr={\thisrowno{1}},
				% col sep=comma,
				% comment chars=\#,
				% unbounded coords=discard,
			% ] {data/control_surfers__flow__n_128__re_250/control_surfer__vs_1o0__surftimeconst_2o0__omegamax_1o0__duxdy__angle_z.csv};
			% \addplot[
				% ColorSurf!100!ColorDuration,
				% mark=*,
			% ] table [
				% x expr={\thisrowno{0}},
				% y expr={\thisrowno{1}},
				% col sep=comma,
				% comment chars=\#,
				% unbounded coords=discard,
			% ] {data/control_surfers__flow__n_128__re_250/control_surfer__vs_1o0__surftimeconst_4o0__omegamax_1o0__duxdy__angle_z.csv};
			% % reorientationtime
			% \addplot[
				% ColorBh!50!ColorDuration,
				% mark=triangle,
			% ] table [
				% x expr={\thisrowno{0}},
				% y expr={\thisrowno{1}},
				% col sep=comma,
				% comment chars=\#,
				% unbounded coords=discard,
			% ] {data/control_surfers__flow__n_128__re_250/spherical_riser__vs_1o0__reorientationtime_0o5__duxdy__angle_z.csv};
			% \addplot[
				% ColorBh!75!ColorDuration,
				% mark=square,
			% ] table [
				% x expr={\thisrowno{0}},
				% y expr={\thisrowno{1}},
				% col sep=comma,
				% comment chars=\#,
				% unbounded coords=discard,
			% ] {data/control_surfers__flow__n_128__re_250/spherical_riser__vs_1o0__reorientationtime_1o0__duxdy__angle_z.csv};
			% \addplot[
				% ColorBh!100!ColorDuration,
				% mark=o,
			% ] table [
				% x expr={\thisrowno{0}},
				% y expr={\thisrowno{1}},
				% col sep=comma,
				% comment chars=\#,
				% unbounded coords=discard,
			% ] {data/control_surfers__flow__n_128__re_250/spherical_riser__vs_1o0__reorientationtime_2o0__duxdy__angle_z.csv};
	\end{groupplot}
\end{tikzpicture}

	\caption[Evaluation of the proposed experimental cues in simulated turbulence.]{
		Evaluation of the proposed experimental cues in simulated turbulence for both simulated surfers and bottom-heavy swimmers.
		Note how these cues would let us differentiate an active surfing-like behavior from a passive bottom-heavy behavior.
	}
	\label{fig:experimental_cues}
\end{figure}

\section{Surfers}

As presented throughout the study, in contrast to bottom-heavy swimmers, surfers tend to oppose to vorticity when reorienting.
This discrepency should then be observable on these previously described cues.

As presented above, the surfing strategy can be approximated by the following expression
\begin{equation}
	\ControlDirectionOpt(\TimeHorizon) = \frac{\ControlDirectionOptNN}{\norm{\ControlDirectionOptNN}} \quad \text{with} \quad \ControlDirectionOptNN = \Direction + \TimeHorizon \vec{\nabla} \FlowVelocityScalar_z
\end{equation}
Based on this expression we can then estimate the expected orientation off surfers given a value of $\FlowVorticityScalar_y$
\begin{subequations}
	\begin{align}
		\left\langle \ControlDirection_{\mathrm{\NameSurfShort}, x} \right\rangle &= \int_{-\infty}^{+\infty} \int_{-\infty}^{+\infty} \int_{-\infty}^{+\infty} p \left(\frac{\partial \FlowVelocityScalar_x}{\partial z} \right) p \left( \frac{\partial \FlowVelocityScalar_z}{\partial y} \right) p \left( \frac{\partial \FlowVelocityScalar_z}{\partial z} \right) \frac{\TimeHorizon}{\norm{\ControlDirectionOptNN}} \left( \frac{\partial \FlowVelocityScalar_x}{\partial z} - \FlowVorticityScalar_y \right) \, d\frac{\partial \FlowVelocityScalar_x}{\partial z} \, d\frac{\partial \FlowVelocityScalar_z}{\partial y} \, d\frac{\partial \FlowVelocityScalar_z}{\partial z}\\
		\left\langle \ControlDirection_{\mathrm{\NameSurfShort}, z} \right\rangle &= \int_{-\infty}^{+\infty} \int_{-\infty}^{+\infty} \int_{-\infty}^{+\infty} p \left(\frac{\partial \FlowVelocityScalar_x}{\partial z} \right) p \left( \frac{\partial \FlowVelocityScalar_z}{\partial y} \right) p \left( \frac{\partial \FlowVelocityScalar_z}{\partial z} \right) \frac{1}{\norm{\ControlDirectionOptNN}} \left( 1 + \TimeHorizon \frac{\partial \FlowVelocityScalar_z}{\partial z} \right) \, d\frac{\partial \FlowVelocityScalar_x}{\partial z} \, d\frac{\partial \FlowVelocityScalar_z}{\partial y} \, d\frac{\partial \FlowVelocityScalar_z}{\partial z}\\
		\text{with} ~~ &\norm{\ControlDirectionOptNN} = \sqrt{\TimeHorizon^2 \left( \frac{\partial \FlowVelocityScalar_x}{\partial z} - \FlowVorticityScalar_y \right)^2 + \TimeHorizon^2 \left( \frac{\partial \FlowVelocityScalar_z}{\partial y} \right)^2 + \left(1 + \TimeHorizon \frac{\partial \FlowVelocityScalar_z}{\partial z}\right)^2}
	\end{align}
\end{subequations}
with $p$ denoting the corresponding probability density functions.
Note that in practice, the statistics of $\partial \FlowVelocityScalar_z / \partial y$ are not accessible in the experiment but can be estimated assuming isotropy $p(\partial \FlowVelocityScalar_z / \partial y) \approx p(\partial \FlowVelocityScalar_z / \partial x)$.
\begin{figure}
	\centering
	\begin{tikzpicture}
	\begin{groupplot}[
			group style={
				group size=2 by 1,
				%y descriptions at=edge left,
				horizontal sep=0.14\linewidth,
				%vertical sep=0.08\linewidth,
			},
			axis on top,
			% size
			width=0.47\textwidth,
			%ymode=log,
			% layers
			set layers ,
			% legend
			legend style={draw=none, fill=none, /tikz/every even column/.append style={column sep=4pt}, at={(1.0, 1.1)}, anchor=south},
			%legend pos=north east,
			legend cell align=right,
			legend columns=-1,
		]
		% n_{surf, z}
		\nextgroupplot[
			% x
			xlabel=$\FlowVorticityScalar_y \KolmogorovTimeScale$,
			xmin=-2,
			xmax=2,
			% y
			ylabel=$\langle \SwimmingDirection_{\DirectionScalar} \rangle$,
			ymin=-1,
			ymax=1,
			ytick={-1,0,1},
		]
			\addlegendimage{empty legend}\addlegendentry{\NameSurf, $\ReorientationTime =$}
			% rtime 0.0
			\addplot[
				ColorSurf!25!ColorDuration,
				mark=star,
			] table [
				x expr={\thisrowno{0}},
				y expr={\thisrowno{1}},
				col sep=comma, 
				comment chars=\#,
				unbounded coords=discard,
			] {data/surfers__flow__n_128__re_250/surfer__vs_1o0__surftimeconst_1o0__vorticity_z__p_x.csv};
			\addlegendentry{$0$}
			% rtime 0.5
			\addplot[
				ColorSurf!50!ColorDuration,
				mark=triangle*,
			] table [
				x expr={\thisrowno{0}},
				y expr={\thisrowno{1}},
				col sep=comma, 
				comment chars=\#,
				unbounded coords=discard,
			] {data/control_surfers__flow__n_128__re_250/control_surfer__vs_1o0__surftimeconst_1o0__omegamax_1o0__vorticity_z__p_x.csv};
			\addlegendentry{$\KolmogorovTimeScale/2$}
			% rtime 1.0
			\addplot[
				ColorSurf!75!ColorDuration,
				mark=square*,
			] table [
				x expr={\thisrowno{0}},
				y expr={\thisrowno{1}},
				col sep=comma, 
				comment chars=\#,
				unbounded coords=discard,
			] {data/control_surfers__flow__n_128__re_250/control_surfer__vs_1o0__surftimeconst_1o0__omegamax_0o5__vorticity_z__p_x.csv};
			\addlegendentry{$\KolmogorovTimeScale$}
			% tau 2.0
			\addplot[
				ColorSurf!100!ColorDuration,
				mark=*,
			] table [
				x expr={\thisrowno{0}},
				y expr={\thisrowno{1}},
				col sep=comma, 
				comment chars=\#,
				unbounded coords=discard,
			] {data/control_surfers__flow__n_128__re_250/control_surfer__vs_1o0__surftimeconst_1o0__omegamax_0o125__vorticity_z__p_x.csv};
			\addlegendentry{$2\KolmogorovTimeScale$}
		% n_{surf, x}
		\nextgroupplot[
			% x
			xlabel=$\FlowVorticityScalar_y \KolmogorovTimeScale$,
			xmin=-2,
			xmax=2,
			% y
			ylabel=$\langle \SwimmingDirection_{x} \rangle$,
			ymin=-1,
			ymax=1,
			ytick={-1,0,1},
		]
			% rtime 0.25
			\addplot[
				ColorSurf!25!ColorDuration,
				mark=star,
			] table [
				x expr={\thisrowno{0}},
				y expr={\thisrowno{1}},
				col sep=comma, 
				comment chars=\#,
				unbounded coords=discard,
			] {data/surfers__flow__n_128__re_250/surfer__vs_1o0__surftimeconst_1o0__vorticity_z__p_y.csv};
			% rtime 0.5
			\addplot[
				ColorSurf!50!ColorDuration,
				mark=triangle*,
			] table [
				x expr={\thisrowno{0}},
				y expr={\thisrowno{1}},
				col sep=comma, 
				comment chars=\#,
				unbounded coords=discard,
			] {data/control_surfers__flow__n_128__re_250/control_surfer__vs_1o0__surftimeconst_1o0__omegamax_1o0__vorticity_z__p_y.csv};
			% rtime 1.0
			\addplot[
				ColorSurf!75!ColorDuration,
				mark=square*,
			] table [
				x expr={\thisrowno{0}},
				y expr={\thisrowno{1}},
				col sep=comma, 
				comment chars=\#,
				unbounded coords=discard,
			] {data/control_surfers__flow__n_128__re_250/control_surfer__vs_1o0__surftimeconst_1o0__omegamax_0o5__vorticity_z__p_y.csv};
			% rtime 2.0
			\addplot[
				ColorSurf!100!ColorDuration,
				mark=*,
			] table [
				x expr={\thisrowno{0}},
				y expr={\thisrowno{1}},
				col sep=comma, 
				comment chars=\#,
				unbounded coords=discard,
			] {data/control_surfers__flow__n_128__re_250/control_surfer__vs_1o0__surftimeconst_1o0__omegamax_0o125__vorticity_z__p_y.csv};
	\end{groupplot}
\end{tikzpicture}

	\caption[Comparison of analytic models with the proposed experimental cues in simulated turbulence.]{
		Comparison of analytic models with the proposed experimental cues in simulated turbulence for both simulated surfers and bottom-heavy swimmers.
	}
	\label{fig:experimental_cues}
\end{figure}

\begin{table}
	\center
	\begin{tabular}{w{c}{0.15\linewidth}w{c}{0.15\linewidth}w{c}{0.15\linewidth}w{c}{0.15\linewidth}w{c}{0.15\linewidth}}
		\rowcolor{ColorTabularParameters}
		stage & $\PlankterSize$ (cm) & $\SwimmingVelocity$ (cm.s$^{-1}$) & $\ReorientationTimeSurf$ (s) & $\ReorientationTimeBh$ (s) \\
		\rowcolor{ColorTabularValues}
		early & 0.0488 & 0.18 & 0.09 & 0.12 \\
		\rowcolor{ColorTabularValues}
		late & 0.0788 & 0.33 & 0.08 & 0.08 \\
	\end{tabular}
	\caption{
		Parameters of the Crepidula Fornicata used.
	}
	\label{tab:snoopy_simulation_parameters}
\end{table}

\begin{figure}
	\centering
	\begin{tikzpicture}
	\begin{groupplot}[
			group style={
				group size=1 by 1,
				%y descriptions at=edge left,
				horizontal sep=0.14\linewidth,
				%vertical sep=0.08\linewidth,
			},
			axis on top,
			% size
			width=0.47\textwidth,
			% x
			%xlabel=$\SwimmingVelocity / \KolmogorovVelocityScale$,
			%xmin=1,
			%xmax=8,
			xlabel=$\epsilon \nu / \SwimmingVelocity^4$,
			xmode=log,
			xmin=0.0001,
			xmax=1,
			% y
			%ylabel=$\langle \FlowVelocityScalar_z(\ParticlePosition) \rangle_N / \KolmogorovVelocityScale$,
			ylabel=$\langle \FlowVelocityScalar_z(\ParticlePosition) \rangle_N / \SwimmingVelocity$,
			ymin=0,
			%ymax=5,
			ymax=2,
			% layers
			set layers ,
			% legend
			legend style={draw=none, fill=none, /tikz/every even column/.append style={column sep=4pt}, at={(0.5, 1.1)}, anchor=south},
			%legend pos=north east,
			legend cell align=right,
			legend columns=4,
		]
		% n_{surf, z}
		\nextgroupplot[
		]
			% tau 1.0
			\addplot[
				ColorSurf,
				mark=triangle*,
				only marks,
			] table [
				%x expr={\thisrowno{7}},
				x expr={1.0/\thisrowno{7}^4},
				%y expr={\thisrowno{1} / (0.066)},
				y expr={\thisrowno{1} / (0.066 * \thisrowno{7})},
				col sep=comma,
				comment chars=\#,
				unbounded coords=discard,
			] {data/surfers/surfer__max_average_sampled_flow_velocity.csv};
			%%% 95 CLI
			\addplot[name path=A, draw=none, forget plot] table [
				%x expr={\thisrowno{7}},
				x expr={1.0/\thisrowno{7}^4},
				%y expr={(\thisrowno{1} - \thisrowno{2}) / (0.066)}, %u_\eta = 0.066
				y expr={(\thisrowno{1} - \thisrowno{2}) / (0.066 * \thisrowno{7})}, %u_\eta = 0.066
				col sep=comma,
				comment chars=\#,
				unbounded coords=discard,
			] {data/surfers/surfer__max_average_sampled_flow_velocity.csv};
			\addplot[name path=B, draw=none, forget plot] table [
				%x expr={\thisrowno{7}},
				x expr={1.0/\thisrowno{7}^4},
				%y expr={(\thisrowno{1} + \thisrowno{2}) / (0.066)}, %u_\eta = 0.066
				y expr={(\thisrowno{1} + \thisrowno{2}) / (0.066 * \thisrowno{7})}, %u_\eta = 0.066
				col sep=comma,
				comment chars=\#,
				unbounded coords=discard,
			] {data/surfers/surfer__max_average_sampled_flow_velocity.csv};
			\addplot[ColorSurf, opacity=0.25, forget plot, on layer=axis background] fill between[of=A and B];
	\end{groupplot}
\end{tikzpicture}

	\caption[Evaluation of the proposed experimental cues in simulated turbulence.]{
		Evaluation of the proposed experimental cues in simulated turbulence for both simulated surfers and bottom-heavy swimmers.
		Note how these cues would let us differentiate an active surfing-like behavior from a passive bottom-heavy behavior.
	}
	\label{fig:experimental_cues}
\end{figure}

% If only the vorticity is accounted for, the surfing strategy can be expressed as
% \begin{equation}
	% \ControlDirectionOpt(\TimeHorizon) = \matr{R}_{\hat{\FlowVorticity}}(-\omega \TimeHorizon/2) \, \Direction,
% \end{equation}
% with $\matr{R}_{\hat{\FlowVorticity}}(-\FlowVorticityScalar \TimeHorizon/2)$ the rotation matrix of angle $-\FlowVorticityScalar \TimeHorizon/2$ with $\omega = \norm{\vec{\nabla} \times \FlowVelocity}$ and of axis $\hat{\FlowVorticity} = \FlowVorticity/\FlowVorticityScalar$, the normalized vorticity.
% 
% Then, similarly to the bottom-heavy case, the expression of both components of the expected orientation can be expressed as
% \begin{subequations}
	% \begin{align}
		% \left\langle \tan \theta_{\mathrm{\NameSurfShort}, y} \right\rangle_{N, t} &= \frac{1}{4 \pi^2} \int_{-\infty}^{+\infty} \int_0^{2\pi} \int_{-\pi/4}^{0} p(\FlowVorticityScalar_x) \frac{\cos \left( \phi_{x} - \phi_{\mathrm{\NameSurfShort}} \right) \sin \theta_{\mathrm{\NameSurfShort}}}{\cos \theta_{\mathrm{\NameSurfShort}}} \, d \phi_{\mathrm{\NameSurfShort}} \, d\phi_{x} \, d\FlowVorticityScalar_x\\
		% \text{with} \quad \sin \theta_{\mathrm{\NameSurfShort}} &= \TimeHorizon \sqrt{\FlowVorticityScalar_{x}^2 + \FlowVorticityScalar_{y}^2} \cos \phi_{\mathrm{\NameSurfShort}}
	% \end{align}
% \end{subequations}
