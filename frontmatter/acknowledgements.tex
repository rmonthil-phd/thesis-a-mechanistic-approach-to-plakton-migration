\begin{acknowledgements}

En premier lieu, un grand merci à mes deux directeurs de thèse, Christophe et Benjamin. Merci Christophe pour m'avoir introduit au monde fascinant du plancton et merci Benjamin pour ton accompagnement tout au long des divers obstacles numériques rencontrés.
Merci aussi à Aurore pour nos discussion fructueuses et ton rôle de quasi troisième directrice de thèse tout au long de ces trois années.

Ensuite, j'aimerais remercier les membres du jury, et en particulier les deux rapporteurs, Éric et Agnese, pour leur lecture attentive de mon manuscrit et pour nos échanges lors de ma soutenance. 
Ce fut un réel plaisir de discuter avec vous.

L'ensemble du personnel de l'Institut de Recherche sur les Phénomènes Hors Équilibre ont contribué, de près ou de loin, à ce travail de recherche. 
Parmi le personnel permanent, je remercie particulièrement Patrice pour nos échanges, souvent à la cafétéria et Frédéric pour son aide vis-à-vis de mes besoins en matière de réseau.

Malgré le Covid-19, pendant ces trois années, l'ensemble de l'équipe de postdocs, doctorants et stagiaires ont fait de ce laboratoire un réel lieu de vie et de travail !
En commençant par les générations précédentes, je les remercie pour leur accueil d'une grande bienveillance. 
Sans tous les citer (ils se reconnaîtront), je remercie particulièrement Amélie pour ses contributions essentielles à la vie du laboratoire.

Cette aventure aurait été des plus ardus sans l'avoir partagé avec les doctorants de ma génération:
Merci à Tristan pour son éternelle bonne humeur et nos matchs de lutte sur la plage. 
Merci à Ahmed pour sa gentillesse et pour nous avoir suivi au mont Ventoux !
Merci à Raùl, le plus courageux d'entre nous, pour les bières qu'on a partagé ainsi que notre passion partagé pour le fromage.
Enfin un grand merci à Tommaso pour m'avoir supporté pendant ces trois ans et pour m'avoir tant apporté.

Un grand merci également aux générations de doctorants et stagiaires suivantes (dont la liste est longue) qui ont pu supporter les nombreuses versions de mon jeu de société mais aussi les itinéraires approximatifs de nos treks. Un grand merci pour toutes nos discussions scientifiques (notamment de biomécanique, la physique des fronts réactifs ou même la pluie de sucres, etc...) et pour toute votre aide inestimable.

Ce travail n'aurait pas pu être réalisé sans le soutien de ma famille et de mes amis. Un grand merci pour nos voyages, nos soirées arrosées, nos treks annuels, nos sorties alpines et célestes, nos restaurants mensuels, nos LANs et parties de jeux de société. Mention spéciale aux survivants, Marie et Maxime, dont la présence à Marseille a été inestimable et à mes parents qui ont su me transmettre leurs passions ainsi que leur détermination pour venir à bout de ce travail.

Enfin, je ne pourrais jamais assez remercier Suzanne pour son soutien indéfectible pendant ces trois années, que ce soit pour la thèse ou pour tous les projets que j'ai pu entreprendre, sans quoi rien de tout ça n'aurait été possible. Je te dois tant, merci.

\end{acknowledgements}
