%% For tips on how to write a great abstract, have a look at
%%	-	https://www.cdc.gov/stdconference/2018/How-to-Write-an-Abstract_v4.pdf (presentation, start here)
%%	-	https://users.ece.cmu.edu/~koopman/essays/abstract.html
%%	-	https://search.proquest.com/docview/1417403858
%%  - 	https://www.sciencedirect.com/science/article/pii/S037837821830402X

\begin{abstract}
At the basis of the marine food chain, plankton play a key role in ocean ecology. 
Due to their daily vertical migration of several hundred meters, some of these organisms play an important role in the vertical mixing of the ocean.
Far from being passive, many planktonic organisms are motile and able to perceive their environment and react accordingly.

Focusing on the representative example of unidirectional migration, this thesis aims to demonstrate that these motor and sensory skills, useful for predation, are also advantageous for navigating their turbulent environment.

Limited by the measure of local variations of velocity, the inability of plankton to directly sense the flow velocity makes this navigation problem particularly challenging.
By considering non-inertial organisms swimming at constant speed and able to reorient instantaneously in response to their flow sensing, we propose a new strategy that derives from an optimal solution in a linear flow.
Taking advantage of nearby updrafts to ascend faster, this strategy is called ``surfing''.

This strategy is first characterized in various simple flows and then tested in turbulent flow simulations in which ``surfers'' are observed to migrate vertically twice as fast as simulated plankters that do not react actively to the local flow.
We then relax the assumptions of the model to demonstrate the robustness of this strategy to various plankter limitations before comparing it to various other navigation approaches, involving various optimization algorithms.
Finally we discuss of the benefit that such strategy could provide to real-life plankton and propose experimental approaches that would help finding out whether surfing is a realistic strategy for planktonic navigation before discussing the perspectives of the study.
\end{abstract}
