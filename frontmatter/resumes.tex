%% For tips on how to write a great abstract, have a look at
%%	-	https://www.cdc.gov/stdconference/2018/How-to-Write-an-Abstract_v4.pdf (presentation, start here)
%%	-	https://users.ece.cmu.edu/~koopman/essays/abstract.html
%%	-	https://search.proquest.com/docview/1417403858
%%  - 	https://www.sciencedirect.com/science/article/pii/S037837821830402X

\begin{resumes}
Constituant la base de la chaîne alimentaire marine, le plancton joue un rôle primordial pour l’écologie océanique. De par leurs migrations verticales journalières de plusieurs centaines de mètres, certains de ces organismes participent grandement au mélange vertical océanique.
Loin d’être passifs, les planctons sont, pour beaucoup, capables de nager, de percevoir leur environnement et d’y réagir.

Au travers de l'exemple de la migration verticale, cette thèse vise à démontrer que ces capacités motrices et sensorielles, utiles à la prédation, leur donnent également un avantage pour naviguer leur environnement turbulent.

Limité par la mesure des variations locales de vitesse, et non la mesure directe de la vitesse de l'écoulement, ce problème de navigation devient particulièrement difficile. 
En considérant des organismes non inertiels nageant à vitesse constante et pouvant se réorienter instantanément en réponse à leur mesure de l'écoulement, nous proposons une nouvelle stratégie qui découle d'une solution optimale dans un écoulement linéaire. 
Permettant d'atteindre les courants ascendants à proximité pour se faire porter par l'écoulement, cette stratégie est nommée le ``surf''.

Cette stratégie est tout d'abord caractérisée grâce à des écoulements simples avant d'être testée dans des simulations d'écoulements turbulents, dans lesquels les surfeurs parviennent à migrer verticalement deux fois plus vite que des planctons ne réagissant pas à l'écoulement.
Nous démontrons la robustesse de cette stratégie de navigation aux diverses limitations auxquelles les planctons sont soumis avant de comparer  le ``surf'' à d'autres méthodes de navigation impliquant divers algorithmes d'optimisation.

Enfin nous quantifions l'avantage que pourrait procurer cette stratégie à de vrais planctons dans leur habitat turbulent et proposons des approches expérimentales qui permettraient de vérifier la capacité du plancton à surfer, avant de discuter des perspectives de cette étude.
\end{resumes}
