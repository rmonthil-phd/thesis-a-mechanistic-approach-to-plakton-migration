\chapter{Temporary misc}\label{app:misc}

Misc appendix that includes all the preliminary work for the extended edition of this Ph.D. thesis.

\section{Filtered measure of $\Gradients$}

\subsection{Previous version}

One can then continue the analysis and account for the known energy spectrum of turbulence $E(k) \propto \epsilon^{2/3} k^{-5/3}$ in homogeneous isotropic turbulence [Eq.~\eqref{eq:kolmogorov_spectrum}, Chap.~\ref{chap:surfing_on_turbulence}, Sec.~\ref{sec:numeric_hit}].
This expression states that the characteristic velocity of flow corresponding to a scale $l = 1/k$ reads $\FlowVelocityScalar_l^2 \propto \epsilon^{2/3} l^{5/3}$.
By taking the spatial derivative of this expression we deduce that $\FlowVelocityScalar_l \nabla \FlowVelocityScalar_l \propto \epsilon^{2/3} l^{2/3}$
Finally for a given filter length $\FilterLength$, we expect the filtered gradient intensity to scale with
\begin{equation}\label{eq:scaling_filtered_gradient}
	\norm{\vec{\nabla} \FlowVelocityScalar_{\DirectionScalar}|_{\FilterLength}} \propto \epsilon^{1/3} \FilterLength^{-1/6}.
\end{equation}
Note that the intensity of the gradient is predicted decrease as function of $\FilterLength$.
This effect should reduce surfing performance, unless swimming velocity is large enough to pass seamlessly through the smaller scales of the flow.
The potential flow velocity that can be foraged should then scale with
\begin{equation}
	\FlowVelocityScalar_{\DirectionScalar} (\ParticlePosition) \propto \min (\FilterLength, \SwimmingVelocity \TimeHorizon_{\FilterLength,\mathrm{corr.}}) \norm{\vec{\nabla} \FlowVelocityScalar_{\DirectionScalar}|_{\FilterLength}} \propto \epsilon^{1/3} \min (\FilterLength, \SwimmingVelocity \TimeHorizon_{\FilterLength,\mathrm{corr.}}) \FilterLength^{-1/6}.
\end{equation}
Finally, given a swimming velocity $\SwimmingVelocity$, the optimal averaging length should scale with $\FilterLength^* \propto \SwimmingVelocity \TimeHorizon_{\FilterLength,\mathrm{corr.}}$ for which maximal preferential sampling is reached: $\FlowVelocityScalar_{\DirectionScalar} (\ParticlePosition) \propto \epsilon^{1/3}  (\FilterLength^*)^{5/6}$.
Now while the lifetime of Eulerian turbulence features of size $l$ are expected to scale with $\TimeHorizon_l \propto \epsilon^{1/3}  l^{-1/6}$ [same as Eq.~\eqref{eq:scaling_filtered_gradient}], the lifetime of Lagrangian features of size $l$, $\TimeHorizon_{l,\mathrm{corr.}}$, is less clear.
However using $\TimeHorizon_l$ as an estimate, under the limit of this assumption, we deduce that $\FilterLength^* \propto \epsilon^{2/7} \SwimmingVelocity^{6/7}$ which increases with swimming velocity as observed in Fig.~\ref{fig:surfing_filtered}\textbf{(a)}.
One may then estimate preferential sampling for the optimal filter length as a function of swimming velocity $\SwimmingVelocity$: $\FlowVelocityScalar_{\DirectionScalar} (\ParticlePosition) \propto \epsilon^{5/21} \SwimmingVelocity^{5/7}$.
Even though for high swimming velocities $\SwimmingVelocity$, flow filtering increases surfing performance, the overall preferential sampling with respect to swimming velocity $\FlowVelocityScalar_{\DirectionScalar} (\ParticlePosition) \SwimmingVelocity \propto \epsilon^{5/21} \SwimmingVelocity^{-2/7}$ decreases as swimming velocity increases.
This causes the an overall decrease of performance $\Performance / \SwimmingVelocity$ with swimming speed, regardless of the ability to filter the flow.

\subsection{New version}

One can then continue the analysis based on the theory of Kolmogorov presented in Chap.~\ref{chap:surfing_on_turbulence}, Sec.~\ref{sec:numeric_hit}.
One of the first step to derive Kolmogorov's theory is to link the turbulent dissipation rate $\epsilon$ to the velocity corresponding to length scale $\FilterLength$ by dimensional analysis
\begin{equation}
	\FlowVelocityScalar_\FilterLength \approx (\epsilon \FilterLength)^{1/3}.
\end{equation}
Similarly, one can estimate the value of the gradient by dividing this expression by $\FilterLength$
\begin{equation}
	\norm{\Gradients|_\FilterLength} \approx \epsilon^{1/3} \FilterLength^{-2/3}.
\end{equation}
Note that the intensity of the gradient is predicted to decrease as function of $\FilterLength$.
Therefore filtering should reduce surfing performance as it corresponds to a direct exploitation of the gradient.
However if the swimming velocity is large enough to reach the maximal velocity of that flow scale $\propto \FlowVelocityScalar_\FilterLength$ before that structure is dissipated, thus reversing that trend as $\FlowVelocityScalar_\FilterLength$ increases with $\FilterLength$.


Denoting $\TimeHorizon_{\FilterLength}$ the life time of a flow structure of length scale $\FilterLength$, the flow velocity that can be foraged should then scale as
\begin{equation}
	\FlowVelocityScalar_{\DirectionScalar} (\ParticlePosition) \propto \min (\FilterLength, \SwimmingVelocity \TimeHorizon_{\FilterLength}) \norm{\vec{\nabla} \FlowVelocityScalar_{\DirectionScalar}|_{\FilterLength}} \propto \min (\FilterLength, \SwimmingVelocity \TimeHorizon_{\FilterLength}) \epsilon^{1/3} \FilterLength^{-2/3}.
\end{equation}
Therefore, given a swimming velocity $\SwimmingVelocity$, the averaging length for which preferential sampling is maximal should scale with $\FilterLength^* \propto \SwimmingVelocity \TimeHorizon_{\FilterLength}$.
The vertical velocity sampled then scales as $\FlowVelocityScalar_{\DirectionScalar} (\ParticlePosition) \propto \epsilon^{1/3}  (\FilterLength^*)^{1/3}$.

Based on dimensional analysis, we expect the lifetime $\TimeHorizon_\FilterLength$ of turbulence features of size $\FilterLength$ are expected to scale with $\TimeHorizon_\FilterLength \propto 1 / \norm{\Gradients|_\FilterLength}$.
However note that the actual temporal correlations of the Lagrangian filtered velocity gradient tensor in turbulence remain unclear as much as that of the unfiltered gradients.
However using $1 / \norm{\Gradients|_\FilterLength}$ as an estimate we can deduce that the optimal averaging length would roughly scale as $\FilterLength^* \propto \SwimmingVelocity^{3} / \epsilon$.
As observed in Fig.~\ref{fig:surfing_filtered}\textbf{(a)}, we obtain that $\FilterLength^*$ increases with swimming velocity.

% Finally one may estimate preferential sampling for the optimal filter length $\FilterLength^*$ as a function of swimming velocity $\SwimmingVelocity$
% \begin{equation}
	% \FlowVelocityScalar_{\DirectionScalar} (\ParticlePosition) \propto \SwimmingVelocity
% \end{equation}

\section{Zermelo solution: simulation of a reactive front}
