\chapter{Misc}\label{app:misc}

Misc appendix that includes all the preliminary work for the extended edition of this Ph.D. thesis.

\section{Filtered measure of $\Gradients$}

One can then continue the analysis and account for the known energy spectrum of turbulence $E(k) \propto \epsilon^{2/3} k^{-5/3}$ in homogeneous isotropic turbulence [Eq.~\eqref{eq:kolmogorov_spectrum}, Chap.~\ref{chap:surfing_on_turbulence}, Sec.~\ref{sec:numeric_hit}].
This expression states that the characteristic velocity of flow corresponding to a scale $l = 1/k$ reads $\FlowVelocityScalar_l^2 \propto \epsilon^{2/3} l^{5/3}$.
By taking the spatial derivative of this expression we deduce that $\FlowVelocityScalar_l \nabla \FlowVelocityScalar_l \propto \epsilon^{2/3} l^{2/3}$
Finally for a given filter length $\FilterLength$, we expect the filtered gradient intensity to scale with
\begin{equation}\label{eq:scaling_filtered_gradient}
	\norm{\vec{\nabla} \FlowVelocityScalar_{\DirectionScalar}|_{\FilterLength}} \propto \epsilon^{1/3} \FilterLength^{-1/6}.
\end{equation}
Note that the intensity of the gradient is predicted decrease as function of $\FilterLength$.
This effect should reduce surfing performance, unless swimming velocity is large enough to pass seamlessly through the smaller scales of the flow.
The potential flow velocity that can be foraged should then scale with
\begin{equation}
	\FlowVelocityScalar_{\DirectionScalar} (\ParticlePosition) \propto \min (\FilterLength, \SwimmingVelocity \TimeHorizon_{\FilterLength,\mathrm{corr.}}) \norm{\vec{\nabla} \FlowVelocityScalar_{\DirectionScalar}|_{\FilterLength}} \propto \epsilon^{1/3} \min (\FilterLength, \SwimmingVelocity \TimeHorizon_{\FilterLength,\mathrm{corr.}}) \FilterLength^{-1/6}.
\end{equation}
Finally, given a swimming velocity $\SwimmingVelocity$, the optimal averaging length should scale with $\FilterLength^* \propto \SwimmingVelocity \TimeHorizon_{\FilterLength,\mathrm{corr.}}$ for which maximal preferential sampling is reached: $\FlowVelocityScalar_{\DirectionScalar} (\ParticlePosition) \propto \epsilon^{1/3}  (\FilterLength^*)^{5/6}$.
Now while the lifetime of Eulerian turbulence features of size $l$ are expected to scale with $\TimeHorizon_l \propto \epsilon^{1/3}  l^{-1/6}$ [same as Eq.~\eqref{eq:scaling_filtered_gradient}], the lifetime of Lagrangian features of size $l$, $\TimeHorizon_{l,\mathrm{corr.}}$, is less clear.
However using $\TimeHorizon_l$ as an estimate, under the limit of this assumption, we deduce that $\FilterLength^* \propto \epsilon^{2/7} \SwimmingVelocity^{6/7}$ which increases with swimming velocity as observed in Fig.~\ref{fig:surfing_filtered}\textbf{(a)}.
One may then estimate preferential sampling for the optimal filter length as a function of swimming velocity $\SwimmingVelocity$: $\FlowVelocityScalar_{\DirectionScalar} (\ParticlePosition) \propto \epsilon^{5/21} \SwimmingVelocity^{5/7}$.
Even though for high swimming velocities $\SwimmingVelocity$, flow filtering increases surfing performance, the overall preferential sampling with respect to swimming velocity $\FlowVelocityScalar_{\DirectionScalar} (\ParticlePosition) \SwimmingVelocity \propto \epsilon^{5/21} \SwimmingVelocity^{-2/7}$ decreases as swimming velocity increases.
This causes the an overall decrease of performance $\Performance / \SwimmingVelocity$ with swimming speed, regardless of the ability to filter the flow.
