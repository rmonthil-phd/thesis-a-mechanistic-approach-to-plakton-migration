\chapter{Surfing towards a target}

Most of the time, when studying navigation problems, the problem of reaching a target is often considered.

\section{Tracking a fixed Eulerian target}

\begin{wrapfigure}{L}{0.3\textwidth}
	\centering
	\def\svgwidth{0.3\textwidth}
	\input{chap_more/schemes/problem_target.pdf_tex}
  	\caption{Illustration of performance evaluation.}
\end{wrapfigure}
We formalize the problem as follows. We consider a plankter in homogeneous, isotropic turbulence initially placed at a distance $\TargetDistance(0) = \TargetDistance_0$ from the position of its target, $\TargetPosition$, of radius $\TargetRadius$. Its task is to reach a given fixed target as fast as possible. We model the plankter as an acttive particle with position $\ParticlePosition(t)$, swimming in direction $\SwimmingDirection(t)$ at constant speed $\SwimmingVelocity$ in a 3D flow velocity field $\FlowVelocity(\vec{x}, t)$ of vorticity $\FlowVorticity(\vec{x}, t) = \vec{\nabla} \times \FlowVelocity$. The plankter is assumed to be inertialess, neutrally buoyant, and small compared to the Kolmogorov scale $\KolmogorovScale$. It actively controls its orientatio by choosing a preferred direction $\ControlDirection$. We start by assuming that the swimming direction $\SwimmingDirection$ is always aligned with its preferred direction $\ControlDirection$ (instantaneous reorientation). Under these assumptions, the equations of motion are
\begin{subequations}\label{eq:surfing_motion}
	\begin{align}
		\frac{d \ParticlePosition}{dt} & =
		 \FlowVelocity (\ParticlePosition, t) + \SwimmingVelocity \, \SwimmingDirection, \label{eq:surfing_x_motion}\\
		\SwimmingDirection(t) & = \ControlDirection(t) .\label{eq:surfing_p_motion}
	\end{align}
\end{subequations}

We assume that the plankter senses the local flow velocity gradient $\Gradients$ and the direction towards the target $\TargetDirection = \TargetPosition/\TargetDistance$.
It responds to this information by choosing a preferred direction $\ControlDirection(\Gradients,\TargetDirection)$, without any memory.
Several metrics can be used to quantify the performance of plankters:
\begin{itemize}
	\item The median time $\ArrivalTime^{50\%}$, the time for half of the swimmers sample to reach the target. Any other proportion can also be used. All plankters may not reach the target by the end of a numerical experiment thus preventing us to use the average arrival time $\left\langle \ArrivalTime \right\rangle$ as a metric. Furthermore, this metric may be relevant for biophysical applications as many planktonic larvae die before reaching their adult state. Thus only the performance of the ``best'' ones matter.
	\item The proportion of plankters $p_T$ having reached the target after a given time $T$. If the minimal performance that ensures individual survival is known, maximizing the proportion of individuals meeting this performance criteria should maximize the species chances of survival.
	\item The steady state average distance $\left\langle \TargetDistance \right\rangle_{\infty}$ of the plankters from the target. If the swimming speed of plankters is not strong enough, $\left\langle \TargetDistance \right\rangle_{\infty}$ should be greater than $0$. This metric is particularly useful as it should not depend on the target size $a$ neither the initial distance $\TargetDistance_0$. 
\end{itemize}
\begin{wrapfigure}{R}{0.3\textwidth}
	\centering
	\def\svgwidth{0.3\textwidth}
	\input{chap_more/schemes/random_walk_step.pdf_tex}
  	\caption{Illustration of the effect of random walk on the distance to a target.}
\end{wrapfigure}

Before introducing swimming velocity, one may be interested of how passive particles would perform such a task. 
To do so, we can rely on the literature of particle dispersion in turbulence. \todo{Cite and explain litterature.}

For swimmers to reach a specific target, they need to overcome the dispersive effect of turbulence. 
To quantify the level of the challenge of reaching a specific target, we need to quantify the effect of dispersion as a function of distance $\TargetDistance$ and flow properties.
Dispersion problem are often tackled using random walk problems. \todo{Cite and explain litterature.}
We will use this approach to estimate the dispersion velocity that statistically pushes plankters away from their targets.
Assuming a flow moves passive particles at a velocity $\left\langle \norm{\FlowVelocity} \right\rangle$ in a randomly directed straight line for a time $\CorrelationTime$ leads to a random walk process of step stize $l = \left\langle \norm{\FlowVelocity} \right\rangle \CorrelationTime$.
Considering a plankter at a given distance $\TargetDistance$ from its target, the average distance to the target after a time $\CorrelationTime$ can be expressed as follows:
\begin{equation}
	\left\langle r(t + \CorrelationTime) \right\rangle = \frac{1}{\pi} \int_0^\pi \sqrt{\TargetDistance(t)^2 + l^2 + 2 \TargetDistance l \cos \theta }  \, d\theta \, , \\
\end{equation}
with $\theta$ the direction angle.
We can then deduce the expression of the dispersion velocity as a function of distance:
\begin{align}
	\DispersionVelocity(\TargetDistance) &= \frac{\left\langle \TargetDistance(t + \CorrelationTime) \right\rangle - \TargetDistance(t)}{\CorrelationTime} \\
	&= \frac{1}{\pi} \int_0^\pi \sqrt{( r/\CorrelationTime )^2 + \left\langle \norm{\FlowVelocity} \right\rangle^2 + 2 (r / \CorrelationTime) \left\langle \norm{\FlowVelocity} \right\rangle \cos \theta }  \, d\theta - \frac{r}{\CorrelationTime}
\end{align}
Note that $\DispersionVelocity(0) = \left\langle \FlowVelocity \right\rangle$ and $\forall \TargetDistance > 0,~\lim_{\CorrelationTime \to 0} \DispersionVelocity = 0$.

\begin{figure}%[H]
	\centering
	\begin{tikzpicture}
	\begin{axis} [
		% more
		axis on top,
		set layers,
		% size
		width=0.7\textwidth,
		height=0.62\textwidth,
		% y
		ylabel={$V_{\mathrm{drift}}$},
		y label style={yshift=-4pt},
		ymin=0,
		ymax=1.0,
		% x
		xlabel=$d_{\mathrm{target}} / L$,
		x label style={yshift=4pt},
		xmin=0,
		xmax=2,
		% legend
		legend style={draw=none, fill=none},
		legend pos=north east,
		legend cell align=left,
		legend columns=1,
	]
		% % tracers 95 CLI
		% \addplot[name path=A, draw=none, forget plot] table [
			% x index=0,
			% y expr={(\thisrowno{1} - \thisrowno{2})},
			% col sep=comma,
			% comment chars=\#,
			% unbounded coords=discard,
		% ]{data/tracers__re_250/tracer__average_dispersion_velocity.csv};
		% \addplot[name path=B, draw=none, forget plot] table [
			% x index=0,
			% y expr={(\thisrowno{1} + \thisrowno{2})},
			% col sep=comma,
			% comment chars=\#,
			% unbounded coords=discard,
		% ]{data/tracers__re_250/tracer__average_dispersion_velocity.csv};
		% \addplot[black, opacity=0.25, forget plot, on layer=axis background] fill between[of=A and B];
		% tracers
		\addplot
		[
		color=black,
		opacity=1.0,
		mark=*,
		only marks,
		]
		table[
			x index=0,
			y expr={\thisrowno{1}},
			col sep=comma, 
			comment chars=\#,
		]{data/tracers__flow__n_128__re_250/tracer__average_dispersion_velocity.csv};
		\addlegendentry{tracers}
		% theory
		\addplot
		[
		color=black,
		opacity=1.0,
		%mark=square*,
		%only marks,
		]
		table[
			x index=0,
			y expr={\thisrowno{1}},
			col sep=comma, 
			comment chars=\#,
		]{data/tracers__flow__n_128__re_250/theory__dispersion_velocity.csv};
		\addlegendentry{model}
	\end{axis}
\end{tikzpicture}

	\caption{
		Dispersion velocity test.
	}
	\label{fig:dispersion_velocity}
\end{figure}

\begin{figure}%[H]
	\centering
	\begin{tikzpicture}
	\begin{groupplot}[
			group style={
				group size=1 by 1,
				%y descriptions at=edge left,
				%x descriptions at=edge bottom,
				%horizontal sep=0.02\linewidth,
				%vertical sep=0.02\linewidth,
			},
			% size
			%width=0.55\textwidth,
			%height=0.45\textwidth,
			width=0.7\textwidth,
			height=0.62\textwidth,
			% more
			axis on top,
			set layers,
			% y
			ylabel={$\left\langle \TargetDistance \right\rangle$},
			y label style={yshift=-4pt},
			ymin=0,
			ymax=2,
			%ymode=log,
			% x
			xlabel=$t / \KolmogorovTimeScale$,
			x label style={yshift=4pt},
			xmin=0,
			xmax=120,
			% legend
			%legend style={draw=none, fill=none, at={(1.0, 1.05)}, anchor=south},
			legend style={draw=none, fill=none},
			legend pos=north west,
			legend cell align=left,
			legend columns=7,
		]
	%\node[anchor=south east] at (3.5,5.35) {$\SwimmingVelocity =$};
	\nextgroupplot[
	]
		% vs = 0.5
		%% 95 CLI
		\addplot[name path=A, draw=none, forget plot] table [
            x expr={\thisrowno{0} / 0.088},
            y expr={(\thisrowno{1} - \thisrowno{2}) / 1.0},
            col sep=comma,
            comment chars=\#,
        ]{data/straight_trackers_fixed_target__flow__n_128__re_250/straight_tracker__vs_0o5__average_distance.csv};
        \addplot[name path=B, draw=none, forget plot] table [
            x expr={\thisrowno{0} / 0.088},
            y expr={(\thisrowno{1} + \thisrowno{2}) / 1.0},
            col sep=comma,
            comment chars=\#,
        ]{data/straight_trackers_fixed_target__flow__n_128__re_250/straight_tracker__vs_0o5__average_distance.csv};
        \addplot[ColorBh!100!ColorVs, opacity=0.25, forget plot, on layer=axis background] fill between[of=A and B];
        %% average
		\addplot
		[
		color=ColorBh!100!ColorVs,
		opacity=1.0,
		mark=triangle*,
		%only marks,
		]
		table[
			x expr={\thisrowno{0} / 0.088},
			y expr={\thisrowno{1} / 1.0},
			col sep=comma, 
			comment chars=\#,
		]{data/straight_trackers_fixed_target__flow__n_128__re_250/straight_tracker__vs_0o5__average_distance.csv};
		\addlegendentry{$\KolmogorovVelocityScale / 2$}
		% vs = 1.0
		%% 95 CLI
		\addplot[name path=A, draw=none, forget plot] table [
            x expr={\thisrowno{0} / 0.088},
            y expr={(\thisrowno{1} - \thisrowno{2}) / 1.0},
            col sep=comma,
            comment chars=\#,
        ]{data/straight_trackers_fixed_target__flow__n_128__re_250/straight_tracker__vs_1o0__average_distance.csv};
        \addplot[name path=B, draw=none, forget plot] table [
            x expr={\thisrowno{0} / 0.088},
            y expr={(\thisrowno{1} + \thisrowno{2}) / 1.0},
            col sep=comma,
            comment chars=\#,
        ]{data/straight_trackers_fixed_target__flow__n_128__re_250/straight_tracker__vs_1o0__average_distance.csv};
        \addplot[ColorBh!80!ColorVs, opacity=0.25, forget plot, on layer=axis background] fill between[of=A and B];
        %% average
		\addplot
		[
		color=ColorBh!80!ColorVs,
		opacity=1.0,
		mark=square,
		%only marks,
		]
		table[
			x expr={\thisrowno{0} / 0.088},
			y expr={\thisrowno{1} / 1.0},
			col sep=comma, 
			comment chars=\#,
		]{data/straight_trackers_fixed_target__flow__n_128__re_250/straight_tracker__vs_1o0__average_distance.csv};
		\addlegendentry{$\KolmogorovVelocityScale$}
		% vs = 2.0
		%% 95 CLI
		\addplot[name path=A, draw=none, forget plot] table [
            x expr={\thisrowno{0} / 0.088},
            y expr={(\thisrowno{1} - \thisrowno{2}) / 1.0},
            col sep=comma,
            comment chars=\#,
        ]{data/straight_trackers_fixed_target__flow__n_128__re_250/straight_tracker__vs_2o0__average_distance.csv};
        \addplot[name path=B, draw=none, forget plot] table [
            x expr={\thisrowno{0} / 0.088},
            y expr={(\thisrowno{1} + \thisrowno{2}) / 1.0},
            col sep=comma,
            comment chars=\#,
        ]{data/straight_trackers_fixed_target__flow__n_128__re_250/straight_tracker__vs_2o0__average_distance.csv};
        \addplot[ColorBh!60!ColorVs, opacity=0.25, forget plot, on layer=axis background] fill between[of=A and B];
        %% average
		\addplot
		[
		color=ColorBh!60!ColorVs,
		opacity=1.0,
		mark=pentagon*,
		%only marks,
		]
		table[
			x expr={\thisrowno{0} / 0.088},
			y expr={\thisrowno{1} / 1.0},
			col sep=comma, 
			comment chars=\#,
		]{data/straight_trackers_fixed_target__flow__n_128__re_250/straight_tracker__vs_2o0__average_distance.csv};
		\addlegendentry{$2\KolmogorovVelocityScale$}
		% vs = 4.0
		%% 95 CLI
		\addplot[name path=A, draw=none, forget plot] table [
            x expr={\thisrowno{0} / 0.088},
            y expr={(\thisrowno{1} - \thisrowno{2}) / 1.0},
            col sep=comma,
            comment chars=\#,
        ]{data/straight_trackers_fixed_target__flow__n_128__re_250/straight_tracker__vs_4o0__average_distance.csv};
        \addplot[name path=B, draw=none, forget plot] table [
            x expr={\thisrowno{0} / 0.088},
            y expr={(\thisrowno{1} + \thisrowno{2}) / 1.0},
            col sep=comma,
            comment chars=\#,
        ]{data/straight_trackers_fixed_target__flow__n_128__re_250/straight_tracker__vs_4o0__average_distance.csv};
        \addplot[ColorBh!40!ColorVs, opacity=0.25, forget plot, on layer=axis background] fill between[of=A and B];
        %% average
		\addplot
		[
		color=ColorBh!40!ColorVs,
		opacity=1.0,
		mark=o,
		%only marks,
		]
		table[
			x expr={\thisrowno{0} / 0.088},
			y expr={\thisrowno{1} / 1.0},
			col sep=comma, 
			comment chars=\#,
		]{data/straight_trackers_fixed_target__flow__n_128__re_250/straight_tracker__vs_4o0__average_distance.csv};
		\addlegendentry{$4\KolmogorovVelocityScale$}
	\end{groupplot}
\end{tikzpicture}

	\caption{
		Average distance to the target as function of time for different swimming velocities.
	}
	\label{fig:tracking_distance}
\end{figure}

\begin{figure}%[H]
	\centering
	\begin{tikzpicture}
	\begin{groupplot}[
			group style={
				group size=1 by 1,
				%y descriptions at=edge left,
				%x descriptions at=edge bottom,
				%horizontal sep=0.02\linewidth,
				%vertical sep=0.02\linewidth,
			},
			% size
			%width=0.55\textwidth,
			%height=0.45\textwidth,
			width=0.7\textwidth,
			height=0.62\textwidth,
			% more
			axis on top,
			set layers,
			% y
			ylabel={$\left\langle \TargetDistance \right\rangle$},
			y label style={yshift=-4pt},
			ymin=0,
			ymax=2,
			%ymode=log,
			% x
			xlabel=$t / \KolmogorovTimeScale$,
			x label style={yshift=4pt},
			xmin=0,
			xmax=120,
			% legend
			%legend style={draw=none, fill=none, at={(1.0, 1.05)}, anchor=south},
			legend style={draw=none, fill=none},
			legend pos=north west,
			legend cell align=left,
			legend columns=7,
		]
	%\node[anchor=south east] at (3.5,5.35) {$\SwimmingVelocity =$};
	\nextgroupplot[
	]
		% surftimeconst = 1.0
		%% 95 CLI
		\addplot[name path=A, draw=none, forget plot] table [
            x expr={\thisrowno{0} / 0.088},
            y expr={(\thisrowno{1} - \thisrowno{2}) / 1.0},
            col sep=comma,
            comment chars=\#,
        ]{data/surfing_trackers_fixed_target__flow__n_128__re_250/surfing_tracker__vs_1o0__surftimeconst_0o0__average_distance.csv};
        \addplot[name path=B, draw=none, forget plot] table [
            x expr={\thisrowno{0} / 0.088},
            y expr={(\thisrowno{1} + \thisrowno{2}) / 1.0},
            col sep=comma,
            comment chars=\#,
        ]{data/surfing_trackers_fixed_target__flow__n_128__re_250/surfing_tracker__vs_1o0__surftimeconst_0o0__average_distance.csv};
        \addplot[ColorBh!100!ColorSurf, opacity=0.25, forget plot, on layer=axis background] fill between[of=A and B];
        %% average
		\addplot
		[
		color=ColorBh!100!ColorSurf,
		opacity=1.0,
		mark=o,
		%only marks,
		]
		table[
			x expr={\thisrowno{0} / 0.088},
			y expr={\thisrowno{1} / 1.0},
			col sep=comma, 
			comment chars=\#,
		]{data/surfing_trackers_fixed_target__flow__n_128__re_250/surfing_tracker__vs_1o0__surftimeconst_0o0__average_distance.csv};
		\addlegendentry{$0$}
		% surftimeconst = 1.0
		%% 95 CLI
		\addplot[name path=A, draw=none, forget plot] table [
            x expr={\thisrowno{0} / 0.088},
            y expr={(\thisrowno{1} - \thisrowno{2}) / 1.0},
            col sep=comma,
            comment chars=\#,
        ]{data/surfing_trackers_fixed_target__flow__n_128__re_250/surfing_tracker__vs_1o0__surftimeconst_1o0__average_distance.csv};
        \addplot[name path=B, draw=none, forget plot] table [
            x expr={\thisrowno{0} / 0.088},
            y expr={(\thisrowno{1} + \thisrowno{2}) / 1.0},
            col sep=comma,
            comment chars=\#,
        ]{data/surfing_trackers_fixed_target__flow__n_128__re_250/surfing_tracker__vs_1o0__surftimeconst_1o0__average_distance.csv};
        \addplot[ColorBh!75!ColorSurf, opacity=0.25, forget plot, on layer=axis background] fill between[of=A and B];
        %% average
		\addplot
		[
		color=ColorBh!75!ColorSurf,
		opacity=1.0,
		mark=*,
		%only marks,
		]
		table[
			x expr={\thisrowno{0} / 0.088},
			y expr={\thisrowno{1} / 1.0},
			col sep=comma, 
			comment chars=\#,
		]{data/surfing_trackers_fixed_target__flow__n_128__re_250/surfing_tracker__vs_1o0__surftimeconst_1o0__average_distance.csv};
		\addlegendentry{$\KolmogorovTimeScale$}
		% surftimeconst = 2.0
		%% 95 CLI
		\addplot[name path=A, draw=none, forget plot] table [
            x expr={\thisrowno{0} / 0.088},
            y expr={(\thisrowno{1} - \thisrowno{2}) / 1.0},
            col sep=comma,
            comment chars=\#,
        ]{data/surfing_trackers_fixed_target__flow__n_128__re_250/surfing_tracker__vs_1o0__surftimeconst_2o0__average_distance.csv};
        \addplot[name path=B, draw=none, forget plot] table [
            x expr={\thisrowno{0} / 0.088},
            y expr={(\thisrowno{1} + \thisrowno{2}) / 1.0},
            col sep=comma,
            comment chars=\#,
        ]{data/surfing_trackers_fixed_target__flow__n_128__re_250/surfing_tracker__vs_1o0__surftimeconst_2o0__average_distance.csv};
        \addplot[ColorBh!50!ColorSurf, opacity=0.25, forget plot, on layer=axis background] fill between[of=A and B];
        %% average
		\addplot
		[
		color=ColorBh!50!ColorSurf,
		opacity=1.0,
		mark=pentagon,
		%only marks,
		]
		table[
			x expr={\thisrowno{0} / 0.088},
			y expr={\thisrowno{1} / 1.0},
			col sep=comma, 
			comment chars=\#,
		]{data/surfing_trackers_fixed_target__flow__n_128__re_250/surfing_tracker__vs_1o0__surftimeconst_2o0__average_distance.csv};
		\addlegendentry{$2\KolmogorovTimeScale$}
		% surftimeconst = 3.0
		%% 95 CLI
		\addplot[name path=A, draw=none, forget plot] table [
            x expr={\thisrowno{0} / 0.088},
            y expr={(\thisrowno{1} - \thisrowno{2}) / 1.0},
            col sep=comma,
            comment chars=\#,
        ]{data/surfing_trackers_fixed_target__flow__n_128__re_250/surfing_tracker__vs_1o0__surftimeconst_3o0__average_distance.csv};
        \addplot[name path=B, draw=none, forget plot] table [
            x expr={\thisrowno{0} / 0.088},
            y expr={(\thisrowno{1} + \thisrowno{2}) / 1.0},
            col sep=comma,
            comment chars=\#,
        ]{data/surfing_trackers_fixed_target__flow__n_128__re_250/surfing_tracker__vs_1o0__surftimeconst_3o0__average_distance.csv};
        \addplot[ColorBh!25!ColorSurf, opacity=0.25, forget plot, on layer=axis background] fill between[of=A and B];
        %% average
		\addplot
		[
		color=ColorBh!25!ColorSurf,
		opacity=1.0,
		mark=square*,
		%only marks,
		]
		table[
			x expr={\thisrowno{0} / 0.088},
			y expr={\thisrowno{1} / 1.0},
			col sep=comma, 
			comment chars=\#,
		]{data/surfing_trackers_fixed_target__flow__n_128__re_250/surfing_tracker__vs_1o0__surftimeconst_3o0__average_distance.csv};
		\addlegendentry{$3\KolmogorovTimeScale$}
		% surftimeconst = 4.0
		%% 95 CLI
		\addplot[name path=A, draw=none, forget plot] table [
            x expr={\thisrowno{0} / 0.088},
            y expr={(\thisrowno{1} - \thisrowno{2}) / 1.0},
            col sep=comma,
            comment chars=\#,
        ]{data/surfing_trackers_fixed_target__flow__n_128__re_250/surfing_tracker__vs_1o0__surftimeconst_4o0__average_distance.csv};
        \addplot[name path=B, draw=none, forget plot] table [
            x expr={\thisrowno{0} / 0.088},
            y expr={(\thisrowno{1} + \thisrowno{2}) / 1.0},
            col sep=comma,
            comment chars=\#,
        ]{data/surfing_trackers_fixed_target__flow__n_128__re_250/surfing_tracker__vs_1o0__surftimeconst_4o0__average_distance.csv};
        \addplot[ColorBh!00!ColorSurf, opacity=0.25, forget plot, on layer=axis background] fill between[of=A and B];
        %% average
		\addplot
		[
		color=ColorBh!00!ColorSurf,
		opacity=1.0,
		mark=triangle,
		%only marks,
		]
		table[
			x expr={\thisrowno{0} / 0.088},
			y expr={\thisrowno{1} / 1.0},
			col sep=comma, 
			comment chars=\#,
		]{data/surfing_trackers_fixed_target__flow__n_128__re_250/surfing_tracker__vs_1o0__surftimeconst_4o0__average_distance.csv};
		\addlegendentry{$4\KolmogorovTimeScale$}
	\end{groupplot}
\end{tikzpicture}

	\caption{
		Average distance to the target as function of time for different surfing parameter $\TimeHorizon$.
	}
	\label{fig:surfing_distance}
\end{figure}

\begin{figure}%[H]
	\centering
	\begin{tikzpicture}
	\begin{groupplot}[
			group style={
				group size=2 by 1,
				y descriptions at=edge left,
				%x descriptions at=edge bottom,
				horizontal sep=0.02\linewidth,
				%vertical sep=0.02\linewidth,
			},
			% size
			width=0.55\textwidth,
			height=0.45\textwidth,
			% more
			axis on top,
			set layers,
			% y
			ylabel={$\sigma_{\hat{\vec{e}}_i} / (\left\langle \norm{\FlowVelocityScalar_i} \right\rangle \KolmogorovTimeScale)$},
			y label style={yshift=-4pt},
			ymin=0,
			ymax=20,
			% x
			xlabel=$t / \KolmogorovTimeScale$,
			x label style={yshift=4pt},
			xmin=0,
			xmax=120,
			% legend
			legend style={draw=none, fill=none, at={(1.0, 1.05)}, anchor=south},
			%legend pos=north west,
			legend cell align=left,
			legend columns=7,
		]
	\node[anchor=south east] at (3.5,5.35) {$\SwimmingVelocity =$};
	\node[anchor=center] at (1.0,4.6) {\textbf{(a)} $\hat{\vec{e}}_i = \Direction$};
	\node[anchor=center] at (8.1,4.6) {\textbf{(b)} $\hat{\vec{e}}_i \cdot \Direction = 0$};
	\nextgroupplot[
	]
		\addplot
		[
		color=ColorPassive,
		opacity=1.0,
		mark=asterisk,
		only marks,
		]
		table[
			x expr={\thisrowno{0} / 0.088},
			y expr={\thisrowno{1} / (0.63 * 0.088)},
			col sep=comma,
			comment chars=\#,
		]{data/tracers__flow__n_128__re_250/tracer__standard_deviation.csv};
		\addlegendentry{$0$}
		\addplot
		[
		color=ColorBh!100!ColorVs,
		opacity=1.0,
		mark=triangle*,
		only marks,
		]
		table[
			x expr={\thisrowno{0} / 0.088},
			y expr={\thisrowno{1} / (0.63 * 0.088)},
			col sep=comma, 
			comment chars=\#,
		]{data/risers__flow__n_128__re_250/riser__vs_0o5__standard_deviation.csv};
		\addlegendentry{$\KolmogorovVelocityScale / 2$}
		\addplot
		[
		color=ColorBh!80!ColorVs,
		opacity=1.0,
		mark=square,
		only marks,
		]
		table[
			x expr={\thisrowno{0} / 0.088},
			y expr={\thisrowno{1} / (0.63 * 0.088)},
			col sep=comma, 
			comment chars=\#,
		]{data/risers__flow__n_128__re_250/riser__vs_1o0__standard_deviation.csv};
		\addlegendentry{$\KolmogorovVelocityScale$}
		\addplot
		[
		color=ColorBh!60!ColorVs,
		opacity=1.0,
		mark=pentagon*,
		only marks,
		]
		table[
			x expr={\thisrowno{0} / 0.088},
			y expr={\thisrowno{1} / (0.63 * 0.088)},
			col sep=comma, 
			comment chars=\#,
		]{data/risers__flow__n_128__re_250/riser__vs_2o0__standard_deviation.csv};
		\addlegendentry{$2\KolmogorovVelocityScale$}
		\addplot
		[
		color=ColorBh!40!ColorVs,
		opacity=1.0,
		mark=o,
		only marks,
		]
		table[
			x expr={\thisrowno{0} / 0.088},
			y expr={\thisrowno{1} / (0.63 * 0.088)},
			col sep=comma, 
			comment chars=\#,
		]{data/risers__flow__n_128__re_250/riser__vs_4o0__standard_deviation.csv};
		\addlegendentry{$4\KolmogorovVelocityScale$}
		\addplot
		[
		color=ColorBh!20!ColorVs,
		opacity=1.0,
		mark=*,
		only marks,
		]
		table[
			x expr={\thisrowno{0} / 0.088},
			y expr={\thisrowno{1} / (0.63 * 0.088)},
			col sep=comma, 
			comment chars=\#,
		]{data/risers__flow__n_128__re_250/riser__vs_8o0__standard_deviation.csv};
		\addlegendentry{$8\KolmogorovVelocityScale$}
		\addplot
		[
		color=ColorBh!00!ColorVs,
		opacity=1.0,
		mark=star,
		only marks,
		]
		table[
			x expr={\thisrowno{0} / 0.088},
			y expr={\thisrowno{1} / (0.63 * 0.088)},
			col sep=comma, 
			comment chars=\#,
		]{data/risers__flow__n_128__re_250/riser__vs_16o0__standard_deviation.csv};
		\addlegendentry{$16\KolmogorovVelocityScale$}

	\nextgroupplot[
	]
		\addplot
		[
		color=ColorPassive,
		opacity=1.0,
		mark=asterisk,
		only marks,
		]
		table[
			x expr={\thisrowno{0} / 0.088},
			y expr={\thisrowno{3} / (0.63 * 0.088)},
			col sep=comma,
			comment chars=\#,
		]{data/tracers__flow__n_128__re_250/tracer__standard_deviation.csv};
		%\addlegendentry{$0$}
		\addplot
		[
		color=ColorBh!100!ColorVs,
		opacity=1.0,
		mark=triangle*,
		only marks,
		]
		table[
			x expr={\thisrowno{0} / 0.088},
			y expr={\thisrowno{3} / (0.63 * 0.088)},
			col sep=comma, 
			comment chars=\#,
		]{data/risers__flow__n_128__re_250/riser__vs_0o5__standard_deviation.csv};
		%\addlegendentry{$\KolmogorovVelocityScale / 2$}
		\addplot
		[
		color=ColorBh!80!ColorVs,
		opacity=1.0,
		mark=square,
		only marks,
		]
		table[
			x expr={\thisrowno{0} / 0.088},
			y expr={\thisrowno{3} / (0.63 * 0.088)},
			col sep=comma, 
			comment chars=\#,
		]{data/risers__flow__n_128__re_250/riser__vs_1o0__standard_deviation.csv};
		%\addlegendentry{$\KolmogorovVelocityScale$}
		\addplot
		[
		color=ColorBh!60!ColorVs,
		opacity=1.0,
		mark=pentagon*,
		only marks,
		]
		table[
			x expr={\thisrowno{0} / 0.088},
			y expr={\thisrowno{3} / (0.63 * 0.088)},
			col sep=comma, 
			comment chars=\#,
		]{data/risers__flow__n_128__re_250/riser__vs_2o0__standard_deviation.csv};
		%\addlegendentry{$2\KolmogorovVelocityScale$}
		\addplot
		[
		color=ColorBh!40!ColorVs,
		opacity=1.0,
		mark=o,
		only marks,
		]
		table[
			x expr={\thisrowno{0} / 0.088},
			y expr={\thisrowno{3} / (0.63 * 0.088)},
			col sep=comma, 
			comment chars=\#,
		]{data/risers__flow__n_128__re_250/riser__vs_4o0__standard_deviation.csv};
		%\addlegendentry{$4\KolmogorovVelocityScale$}
		\addplot
		[
		color=ColorBh!20!ColorVs,
		opacity=1.0,
		mark=*,
		only marks,
		]
		table[
			x expr={\thisrowno{0} / 0.088},
			y expr={\thisrowno{3} / (0.63 * 0.088)},
			col sep=comma, 
			comment chars=\#,
		]{data/risers__flow__n_128__re_250/riser__vs_8o0__standard_deviation.csv};
		%\addlegendentry{$8\KolmogorovVelocityScale$}
		\addplot
		[
		color=ColorBh!00!ColorVs,
		opacity=1.0,
		mark=star,
		only marks,
		]
		table[
			x expr={\thisrowno{0} / 0.088},
			y expr={\thisrowno{3} / (0.63 * 0.088)},
			col sep=comma, 
			comment chars=\#,
		]{data/risers__flow__n_128__re_250/riser__vs_16o0__standard_deviation.csv};
		%\addlegendentry{$16\KolmogorovVelocityScale$}
	\end{groupplot}
\end{tikzpicture}

	\caption{
		Dispersion as function of time for different swimming velocities.
	}
	\label{fig:dispersion_swimming_velocity}
\end{figure}

\begin{figure}%[H]
	\centering
	\begin{tikzpicture}
	\begin{groupplot}[
			group style={
				group size=2 by 1,
				%y descriptions at=edge left,
				%x descriptions at=edge bottom,
				horizontal sep=0.08\linewidth,
				%vertical sep=0.02\linewidth,
			},
			% size
			width=0.55\textwidth,
			height=0.45\textwidth,
			% more
			axis on top,
			set layers,
			% x
			xlabel=$\SwimmingVelocity / \KolmogorovVelocityScale$,
			x label style={yshift=4pt},
			xmin=0,
			xmax=16,
			% legend
			%legend style={draw=none, fill=none, at={(1.0, 1.05)}, anchor=south},
			legend style={draw=none, fill=none},
			legend pos=south west,
			legend cell align=left,
			legend columns=1,
		]
	\node[anchor=center] at (5.7,4.6) {\textbf{(a)} $\hat{\vec{e}}_i = \Direction$};
	\node[anchor=center] at (13.2,4.6) {\textbf{(b)} $\hat{\vec{e}}_i \cdot \Direction = 0$};
	\nextgroupplot[
		% y
		ylabel={$D/\nu$},
		y label style={yshift=-4pt},
		ymin=0,
		ymax=4,
	]
		\addplot
		[
		color=Color0,
		opacity=1.0,
		mark=square*,
		%only marks,
		]
		table[
			x expr={\thisrowno{0}},
			y expr={\thisrowno{1} / (1.0/250.0)},
			col sep=comma,
			comment chars=\#,
		]{data/risers__flow__n_128__re_250/riser__diffusion_coefficient.csv};
		\addlegendentry{$D_{\hat{\vec{e}}_i = \Direction}$}
		\addplot
		[
		color=Color2,
		opacity=1.0,
		mark=o,
		%only marks,
		]
		table[
			x expr={\thisrowno{0}},
			y expr={\thisrowno{3} / (1.0/250.0)},
			col sep=comma,
			comment chars=\#,
		]{data/risers__flow__n_128__re_250/riser__diffusion_coefficient.csv};
		\addlegendentry{$D_{\hat{\vec{e}}_i \cdot \Direction = 0}$}

	\nextgroupplot[
		% y
		ylabel={$\CorrelationTime / \KolmogorovTimeScale$},
		y label style={yshift=-4pt},
		ymin=0,
		ymax=3,
	]
		\addplot
		[
		color=Color0,
		opacity=1.0,
		mark=square*,
		%only marks,
		]
		table[
			x expr={\thisrowno{0}},
			y expr={6 * \thisrowno{1} / (0.6266*0.6266) / 0.088}, % time = 6 * D / U^2 and \tau_eta = 0.088
			col sep=comma,
			comment chars=\#,
		]{data/risers__flow__n_128__re_250/riser__diffusion_coefficient.csv};
		\addlegendentry{$\tau_{\mathrm{corr.}, \hat{\vec{e}}_i = \Direction}$}
		\addplot
		[
		color=Color2,
		opacity=1.0,
		mark=o,
		%only marks,
		]
		table[
			x expr={\thisrowno{0}},
			y expr={6 * \thisrowno{3} / (0.6266*0.6266) / 0.088}, % time = 6 * D / U^2 and \tau_eta = 0.088
			col sep=comma,
			comment chars=\#,
		]{data/risers__flow__n_128__re_250/riser__diffusion_coefficient.csv};
		\addlegendentry{$\tau_{\mathrm{corr.}, \hat{\vec{e}}_i \cdot \Direction = 0}$}
	\end{groupplot}
\end{tikzpicture}

	\caption{
		Dispersion coefficient.
	}
	\label{fig:dispersion_coefficient}
\end{figure}

\section{Tracking a passive Lagrangian target}

\section{Tracking a fleeing Lagrangian target}

% \section{Comparison to reinforcement learning literature}
% 
% It is interesting to compare the proposed surfing strategy to ``smart'' plankters trained by reinforcement learning.
% In a recent study, an agent was trained in a 3D turbulent flow to minimize the time to reach a fixed target using a local measure of vorticity \cite{Alageshan2020}.
% The best agent swims $1.19$ times faster than a naive agent that always swims towards the target.
% The surfing strategy can be adapted to solve the same problem by replacing directly in Eq. \eqref{eq:optimal_swimming_direction} the direction $\Direction$ by the instantaneous direction towards the fixed target.
% For the same dimensionless parameters ($\SwimmingVelocity = 1.5 u_\mathrm{rms}$, $\ReorientationTime = 0.5 \KolmogorovT$), initial distance and target radius, we find that surfers perform $1.25 \pm 0.02$ times better than naive bottom-heavy swimmers.
% Although this comparison is only indicative as the Reynolds number of the flow is different, it shows that the surfing strategy is competitive and could be used as a reference in reinforcement learning problems.

\section{Summary}
