\chapter{Towards the odor tracking problem}

Most of the time, when studying navigation problems, the problem of reaching a target is often considered.

\section{Odor plume characterization}

\begin{figure}
    \centering
    % Reynolds
\begin{tikzpicture}
	\begin{groupplot}[
			group style={
				group size=2 by 1,
				%y descriptions at=edge left,
				%x descriptions at=edge bottom,
				horizontal sep=0.14\linewidth,
				%vertical sep=0.02\linewidth,
			},
			% size
			width=0.49\textwidth,
			height=0.45\textwidth,
		]
	%\node[anchor=south east] at (1.4,4.3) {$t =$};
	\nextgroupplot[
		% y
		ylabel={$p(C = c)$},
		ymin=1e-9,
		ymax=1e-5,
		ymode=log,
		% x
		xlabel=$c$,
		xmin=0,
		xmax=1,
		% layers
		set layers,
		% legend
		legend columns=3,
		legend cell align=left,
		%legend style={draw=none, fill=none, at={(1.0, 1.1)}, anchor=south},
		legend style={draw=none, fill=none},
		legend pos=north west,
	]
		% plot
		\addplot
		[
		color=ColorSurf!100!black,
		opacity=1.0,
		%only marks,%solid,
		mark repeat=4,
		mark=triangle*
		]
		table[
			x expr={\thisrowno{0}}, 
			y expr={\thisrowno{1}},
			col sep=comma, 
			comment chars=\#,
			unbounded coords=discard,
			restrict expr to domain={\thisrowno{0}}{0.001:9.009},
		]{chap_more/data/diffuselet/tracer__time_0o032__sc_100o0__c_pdf.csv};
		\addlegendentry{$0.75 \KolmogorovTimeScale$ \quad}
		% plot
		\addplot
		[
		color=ColorSurf!80!black,
		opacity=1.0,
		%only marks,%solid,
		mark repeat=4,
		mark=square
		]
		table[
			x expr={\thisrowno{0}}, 
			y expr={\thisrowno{1}},
			col sep=comma, 
			comment chars=\#,
			unbounded coords=discard,
			restrict expr to domain={\thisrowno{0}}{0.001:9.009},
		]{chap_more/data/diffuselet/tracer__time_0o064__sc_100o0__c_pdf.csv};
		\addlegendentry{$1.5 \KolmogorovTimeScale$ \quad}
		% plot
		\addplot
		[
		color=ColorSurf!60!black,
		opacity=1.0,
		%only marks,%solid,
		mark repeat=4,
		mark=pentagon*
		]
		table[
			x expr={\thisrowno{0}}, 
			y expr={\thisrowno{1}},
			col sep=comma, 
			comment chars=\#,
			unbounded coords=discard,
			restrict expr to domain={\thisrowno{0}}{0.001:9.009},
		]{chap_more/data/diffuselet/tracer__time_0o128__sc_100o0__c_pdf.csv};
		\addlegendentry{$3 \KolmogorovTimeScale$ \quad}
		% plot
		\addplot
		[
		color=ColorSurf!40!black,
		opacity=1.0,
		%only marks,%solid,
		mark repeat=4,
		mark=o
		]
		table[
			x expr={\thisrowno{0}}, 
			y expr={\thisrowno{1}},
			col sep=comma, 
			comment chars=\#,
			unbounded coords=discard,
			restrict expr to domain={\thisrowno{0}}{0.001:9.009},
		]{chap_more/data/diffuselet/tracer__time_0o256__sc_100o0__c_pdf.csv};
		\addlegendentry{$6 \KolmogorovTimeScale$ \quad}
		% plot
		\addplot
		[
		color=ColorSurf!20!black,
		opacity=1.0,
		%only marks,%solid,
		mark repeat=4,
		mark=*
		]
		table[
			x expr={\thisrowno{0}},
			y expr={\thisrowno{1}},
			col sep=comma, 
			comment chars=\#,
			unbounded coords=discard,
			restrict expr to domain={\thisrowno{0}}{0.001:9.009},
		]{chap_more/data/diffuselet/tracer__time_0o512__sc_100o0__c_pdf.csv};
		\addlegendentry{$12 \KolmogorovTimeScale$ \quad}
		% plot
		\addplot
		[
		color=ColorSurf!0!black,
		opacity=1.0,
		%only marks,%solid,
		mark repeat=4,
		mark=star
		]
		table[
			x expr={\thisrowno{0}},
			y expr={\thisrowno{1}},
			col sep=comma, 
			comment chars=\#,
			unbounded coords=discard,
			restrict expr to domain={\thisrowno{0}}{0.001:9.009},
		]{chap_more/data/diffuselet/tracer__time_1o024__sc_100o0__c_pdf.csv};
		\addlegendentry{$24 \KolmogorovTimeScale$ \quad}

	\nextgroupplot[
		% y
		ylabel={$p(C = c)$},
		%ymin=1e-9,
		%ymax=1e-5,
		ymode=log,
		% x
		xlabel=$c$,
		xmin=0,
		xmax=1,
		% layers
		set layers,
		% legend
		legend columns=3,
		legend cell align=left,
		%legend style={draw=none, fill=none, at={(1.0, 1.1)}, anchor=south},
		legend style={draw=none, fill=none},
		legend pos=north west,
	]
		% plot
		\addplot
		[
		color=ColorSurf!100!black,
		opacity=1.0,
		%only marks,%solid,
		mark repeat=4,
		mark=triangle*
		]
		table[
			x expr={\thisrowno{0}/0.6}, 
			y expr={\thisrowno{1}},
			col sep=comma, 
			comment chars=\#,
			unbounded coords=discard,
			restrict expr to domain={\thisrowno{0}}{0.001:9.009},
		]{chap_more/data/diffuselet__exp/intensity_pdf__time_1o2121212121212122.csv};
		\addlegendentry{$1.2 \KolmogorovTimeScale$ \quad}
		% plot
		\addplot
		[
		color=ColorSurf!80!black,
		opacity=1.0,
		%only marks,%solid,
		mark repeat=4,
		mark=square
		]
		table[
			x expr={\thisrowno{0}/0.6}, 
			y expr={\thisrowno{1}},
			col sep=comma, 
			comment chars=\#,
			unbounded coords=discard,
			restrict expr to domain={\thisrowno{0}}{0.001:9.009},
		]{chap_more/data/diffuselet__exp/intensity_pdf__time_2o4242424242424243.csv};
		\addlegendentry{$2.4 \KolmogorovTimeScale$ \quad}
		% plot
		\addplot
		[
		color=ColorSurf!60!black,
		opacity=1.0,
		%only marks,%solid,
		mark repeat=4,
		mark=pentagon*
		]
		table[
			x expr={\thisrowno{0}/0.6}, 
			y expr={\thisrowno{1}},
			col sep=comma, 
			comment chars=\#,
			unbounded coords=discard,
			restrict expr to domain={\thisrowno{0}}{0.001:9.009},
		]{chap_more/data/diffuselet__exp/intensity_pdf__time_4o848484848484849.csv};
		\addlegendentry{$4.9 \KolmogorovTimeScale$ \quad}
		% plot
		\addplot
		[
		color=ColorSurf!40!black,
		opacity=1.0,
		%only marks,%solid,
		mark repeat=4,
		mark=o
		]
		table[
			x expr={\thisrowno{0}/0.6}, 
			y expr={\thisrowno{1}},
			col sep=comma, 
			comment chars=\#,
			unbounded coords=discard,
			restrict expr to domain={\thisrowno{0}}{0.001:9.009},
		]{chap_more/data/diffuselet__exp/intensity_pdf__time_9o696969696969697.csv};
		\addlegendentry{$9.7 \KolmogorovTimeScale$ \quad}
		% plot
		\addplot
		[
		color=ColorSurf!20!black,
		opacity=1.0,
		%only marks,%solid,
		mark repeat=4,
		mark=*
		]
		table[
			x expr={\thisrowno{0}/0.6}, 
			y expr={\thisrowno{1}},
			col sep=comma, 
			comment chars=\#,
			unbounded coords=discard,
			restrict expr to domain={\thisrowno{0}}{0.001:9.009},
		]{chap_more/data/diffuselet__exp/intensity_pdf__time_19o393939393939394.csv};
		\addlegendentry{$19 \KolmogorovTimeScale$ \quad}	
	\end{groupplot}
\end{tikzpicture}

    \caption{
    	Comparison of the diffuselet concept with experiments.
    	\textbf{(left)}: Parameters: $\mathit{Re}_\lambda = 418$, $\mathit{Sc_t} = 100$, $s_0 = \KolmogorovScale$ and $dA_0 = \KolmogorovScale^2$.
    	\textbf{(right)}: Plume exp.
    }
\end{figure}

\begin{figure}
    \centering
    % Reynolds
\begin{tikzpicture}
	\begin{groupplot}[
			group style={
				group size=2 by 1,
				%y descriptions at=edge left,
				%x descriptions at=edge bottom,
				horizontal sep=0.14\linewidth,
				%vertical sep=0.02\linewidth,
			},
			% size
			width=0.49\textwidth,
			height=0.45\textwidth,
		]
	%\node[anchor=south east] at (1.4,4.3) {$t =$};
	\nextgroupplot[
		% y
		ylabel={$\left\langle c^2 \right\rangle$},
		ymin=1e-9,
		ymax=1e-8,
		ymode=log,
		% x
		xlabel=$t / \KolmogorovTimeScale$,
		xmin=0,
		xmax=25,
		ymode=log,
		% layers
		set layers,
		% legend
		legend columns=3,
		legend cell align=left,
		%legend style={draw=none, fill=none, at={(1.0, 1.1)}, anchor=south},
		legend style={draw=none, fill=none},
		legend pos=north west,
	]
		% plot
		\addplot
		[
		color=ColorSurf!100!black,
		opacity=1.0,
		only marks,%solid,
		%mark repeat=4,
		mark=triangle*
		]
		table[
			x expr={\thisrowno{0} / 0.0424}, 
			y expr={\thisrowno{1}},
			col sep=comma, 
			comment chars=\#,
			unbounded coords=discard,
			restrict expr to domain={\thisrowno{0}}{0.001:9.009},
		]{chap_more/data/diffuselet/tracer__sc_100o0__time_c_var.csv};

	\nextgroupplot[
		% y
		ylabel={$\left\langle c^2 \right\rangle$},
		ymin=1e-4,
		ymax=1e-3,
		ymode=log,
		% x
		xlabel=$t / \KolmogorovTimeScale$,
		xmin=0,
		xmax=10,
		ymode=log,
		% layers
		set layers,
		% legend
		legend columns=3,
		legend cell align=left,
		%legend style={draw=none, fill=none, at={(1.0, 1.1)}, anchor=south},
		legend style={draw=none, fill=none},
		legend pos=north west,
	]
		% plot
		\addplot
		[
		color=ColorSurf!100!black,
		opacity=1.0,
		only marks,%solid,
		%mark repeat=4,
		mark=triangle*
		]
		table[
			x expr={\thisrowno{0}},
			y expr={\thisrowno{1}},
			col sep=comma, 
			comment chars=\#,
			unbounded coords=discard,
			restrict expr to domain={\thisrowno{0}}{0.001:9.009},
		]{chap_more/data/diffuselet__exp/time_intensity_variance.csv};
	\end{groupplot}
\end{tikzpicture}

    \caption{
    	Comparison of the diffuselet concept with experiments.
    	\textbf{(left)}: Parameters: $\mathit{Re}_\lambda = 418$, $\mathit{Sc_t} = 10^{-4} \nu$, $s_0 = \KolmogorovScale$ and $dA_0 = \KolmogorovScale^2$.
    	\textbf{(right)}: Plume exp.
    }
\end{figure}

\begin{figure}
    \centering
    \input{chap_more/plots/filament/diff_pdfs.tex}
    \caption{
		Simulating a filament. Probability density function of the initial distance $\varDelta_0$ (left) and the curvilinear distance $\varDelta_s$ of pair of points spaced of a distance less than the Kolmogorov scale $\varDelta_{\vec{X}} < \KolmogorovScale$.
    }
\end{figure}

\begin{figure}
    \centering
    % Reynolds
\begin{tikzpicture}
	\begin{groupplot}[
			group style={
				group size=1 by 1,
				%y descriptions at=edge left,
				%x descriptions at=edge bottom,
				horizontal sep=0.14\linewidth,
				%vertical sep=0.02\linewidth,
			},
			% size
			width=0.66\textwidth,
			height=0.62\textwidth,
		]
	%\node[anchor=south east] at (1.4,4.3) {$t =$};
	\nextgroupplot[
		% y
		ylabel={$p(\rho)$},
		ymin=1e-3,
		ymax=1e1,
		ymode=log,
		% x
		xlabel=$\rho$,
		xmin=-2.3,
		xmax=9.21,
		xtick={-2.3, 0, 2.3, 4.6, 6.9, 9.21},
		xticklabels={$10^{-1}$, $10^0$, $10^1$, $10^2$, $10^3$, $10^4$},
		%ymode=log,
		% layers
		set layers,
		% legend
		legend columns=3,
		legend cell align=left,
		%legend style={draw=none, fill=none, at={(1.0, 1.1)}, anchor=south},
		legend style={draw=none, fill=none},
		legend pos=north east,
	]
		% plot
		\addplot
		[
		color=ColorSurf!100!black,
		opacity=1.0,
		%only marks,%solid,
		mark repeat=4,
		mark=triangle*
		]
		table[
			x expr={\thisrowno{0}}, 
			y expr={\thisrowno{1}},
			col sep=comma, 
			comment chars=\#,
			unbounded coords=discard,
		]{chap_more/data/diffuselet/tracer__time_0o032__log_rho_pdf.csv};
		\addlegendentry{$0.75 \KolmogorovTimeScale$ \quad}
		% plot
		\addplot
		[
		color=ColorSurf!80!black,
		opacity=1.0,
		%only marks,%solid,
		mark repeat=4,
		mark=square
		]
		table[
			x expr={\thisrowno{0}}, 
			y expr={\thisrowno{1}},
			col sep=comma, 
			comment chars=\#,
			unbounded coords=discard,
		]{chap_more/data/diffuselet/tracer__time_0o064__log_rho_pdf.csv};
		\addlegendentry{$1.5 \KolmogorovTimeScale$ \quad}
		% plot
		\addplot
		[
		color=ColorSurf!60!black,
		opacity=1.0,
		%only marks,%solid,
		mark repeat=4,
		mark=pentagon*
		]
		table[
			x expr={\thisrowno{0}}, 
			y expr={\thisrowno{1}},
			col sep=comma, 
			comment chars=\#,
			unbounded coords=discard,
		]{chap_more/data/diffuselet/tracer__time_0o128__log_rho_pdf.csv};
		\addlegendentry{$3 \KolmogorovTimeScale$ \quad}
		% plot
		\addplot
		[
		color=ColorSurf!40!black,
		opacity=1.0,
		%only marks,%solid,
		mark repeat=4,
		mark=o
		]
		table[
			x expr={\thisrowno{0}}, 
			y expr={\thisrowno{1}},
			col sep=comma, 
			comment chars=\#,
			unbounded coords=discard,
		]{chap_more/data/diffuselet/tracer__time_0o256__log_rho_pdf.csv};
		\addlegendentry{$6 \KolmogorovTimeScale$ \quad}
		% plot
		\addplot
		[
		color=ColorSurf!20!black,
		opacity=1.0,
		%only marks,%solid,
		mark repeat=4,
		mark=*
		]
		table[
			x expr={\thisrowno{0}}, 
			y expr={\thisrowno{1}},
			col sep=comma, 
			comment chars=\#,
			unbounded coords=discard,
		]{chap_more/data/diffuselet/tracer__time_0o512__log_rho_pdf.csv};
		\addlegendentry{$12 \KolmogorovTimeScale$ \quad}
		% plot
		\addplot
		[
		color=ColorSurf!00!black,
		opacity=1.0,
		%only marks,%solid,
		mark repeat=4,
		mark=star
		]
		table[
			x expr={\thisrowno{0}}, 
			y expr={\thisrowno{1}},
			col sep=comma, 
			comment chars=\#,
			unbounded coords=discard,
		]{chap_more/data/diffuselet/tracer__time_1o024__log_rho_pdf.csv};
		\addlegendentry{$24 \KolmogorovTimeScale$ \quad}
	\end{groupplot}
\end{tikzpicture}

    \caption{
    	Probability density function of stretching rate.
    }
\end{figure}

\begin{figure}
    \centering
    % Reynolds
\begin{tikzpicture}
	\begin{groupplot}[
			group style={
				group size=2 by 1,
				%y descriptions at=edge left,
				%x descriptions at=edge bottom,
				horizontal sep=0.14\linewidth,
				%vertical sep=0.02\linewidth,
			},
			% size
			width=0.49\textwidth,
			height=0.45\textwidth,
		]
	%\node[anchor=south east] at (1.4,4.3) {$t =$};
	\nextgroupplot[
		% y
		ylabel={$\left\langle \lambda \right\rangle \, \KolmogorovTimeScale$},
		%ymin=1e-9,
		%ymax=1e-8,
		%ymode=log,
		% x
		xlabel=$t / \KolmogorovTimeScale$,
		xmin=0,
		xmax=25,
		%ymode=log,
		% layers
		set layers,
		% legend
		legend columns=3,
		legend cell align=left,
		%legend style={draw=none, fill=none, at={(1.0, 1.1)}, anchor=south},
		legend style={draw=none, fill=none},
		legend pos=north west,
	]
		% plot
		\addplot
		[
		color=ColorSurf!100!black,
		opacity=1.0,
		only marks,%solid,
		%mark repeat=4,
		mark=triangle*
		]
		table[
			x expr={\thisrowno{0} / 0.0424}, 
			y expr={\thisrowno{1} * 0.0424},
			col sep=comma, 
			comment chars=\#,
			unbounded coords=discard,
			restrict expr to domain={\thisrowno{0}}{0.001:9.009},
		]{chap_more/data/diffuselet/tracer__time_lyapunov_exponent.csv};

	\nextgroupplot[
		% y
		ylabel={$\left\langle V \right\rangle \, \KolmogorovTimeScale^2$},
		%ymin=1e-4,
		%ymax=1e-3,
		%ymode=log,
		% x
		xlabel=$t / \KolmogorovTimeScale$,
		xmin=0,
		xmax=25,
		%ymode=log,
		% layers
		set layers,
		% legend
		legend columns=3,
		legend cell align=left,
		%legend style={draw=none, fill=none, at={(1.0, 1.1)}, anchor=south},
		legend style={draw=none, fill=none},
		legend pos=north west,
	]
		% plot
		\addplot
		[
		color=ColorSurf!100!black,
		opacity=1.0,
		only marks,%solid,
		%mark repeat=4,
		mark=triangle*
		]
		table[
			x expr={\thisrowno{0} / 0.0424},
			y expr={\thisrowno{1} * 0.0424^2},
			col sep=comma, 
			comment chars=\#,
			unbounded coords=discard,
			restrict expr to domain={\thisrowno{0}}{0.001:9.009},
		]{chap_more/data/diffuselet/tracer__time_log_rho_var.csv};
	\end{groupplot}
\end{tikzpicture}

    \caption{
    	Average lyapunov coefficient and $\log \rho$ variance as a function of time.
    }
\end{figure}

\section{Summary}
