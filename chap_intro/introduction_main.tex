\chapter{Introduction}\label{chap:intro}

\section{What are plankton?}

The word ``plankton'' is derived from the ancient greek \textit{planktós}, that means \textit{wandering}.
This term regroups all organisms that live in water while being unfit to swim against currents.
This designation can refer to any kind of organisms in practice.
It includes many micro-organisms such as viruses, bacteria, algae, animal larvae and small crustaceans but also slowly swimming jellyfishes (Fig.~\ref{fig:marine_plankton}).
An individual organism of this group is called a plankter.
Plankton are generally separated into two groups: phytoplankton that describe the algae performing photosynthesis and zooplankton composed of animals that predate on other plankters.
\begin{figure}
	\centering
	\def\svgwidth{0.9\textwidth}
	\input{chap_intro/images/marine_plankton.pdf_tex}
  	\caption[Illustration of marine plankton.]{
  		Illustration of marine plankton. Picture of a part of the contents of a hand net. 
  		We can see a wide variety of plankters, ranging from phytoplankton (cyanobacteria, diatoms, ...) to zooplankton (copepods, dish eggs, crab larvae, worm larvae ...).
  		Adapted from \citet{nadeau2016microbial} \ccbysa ~ v4.0.
  	}
  	\label{fig:marine_plankton}
\end{figure}

\section{Why are plankton important?}

\subsection{Their role in the marine food-web}

Planktonic organisms are at the base of most of marine food-webs.
They constitute the food of many larger organisms that constitute themselves the food of even larger organisms. 
Moreover, they are the direct food source of some large mammals such as baleen whales.
Their role is then essential to sustain fisheries and marine ecology in general.

\subsection{The threat of climate change}

In a world where the climate is actively changing due to the global warming, understanding how planktonic organisms respond to it is primordial.
This necessity motivates numerous efforts to monitor plankton at a global scale \citep{brander2003use, batten2019global} and use them as a cue of the ocean ``health'' \citep{suthers2019importance}.
The climate change impacts on plankton ecology are numerous \citep{mckinnon2007vulnerability, hays2005climate}.
For instance, in addition to its direct physiological impact on plankters, water temperature influences the nutrient concentration of surface water \citep{bouman2003temperature, richardson2008hot, doney2006plankton}.
Low temperatures favour the ocean mixing that brings nutriments near the surface.
This leads to the development of large phytoplankton communities.
These conditions are particularly favourable for the development of large crustaceans.
Cold waters become then a rich environment that sustains life-dense seas.
On the contrary hotter water surface tends to stratify the upper ocean layers and prevent vertical mixing.
The absence of ocean mixing avoids sinking nutrients to get carried back to the surface.
This hinders the development of a phytoplankton communities that impact then the whole food-web.

Another consequence of climate change is the acidification of oceans. 
This is due to the growing amount of carbon dioxide released in the atmosphere that is now dissolving in the ocean.
\citet{caldeira2003anthropogenic} predict that the oceanic pH will drop by 0.3 units by 2100.
Among other effects, this acidification of the oceans especially affects organisms that rely on calcifying \citep{orr2005anthropogenic, flynn2012changes}.
This acidification causes the decrease of calcium carbonate saturation at ocean surfaces, that is needed by some plankton species (and corals) to form their external carbonate skeletons, necessary to their development.
Moreover, many plankton communities do not have the time to migrate as fast as these modifications occur.
As a consequence, some plankton species are at the verge of extinction \citep{trubovitz2020marine, lowery2020ecological}: a risk shared by all marine life that feeds on them.

\subsection{Their role on climate itself: the biological pump}

\begin{figure}
	\centering
	\def\svgwidth{0.9\textwidth}
	\input{chap_intro/schemes/biological_pump.pdf_tex}
	%\captionsetup{width=0.9\textwidth}
  	\caption[Illustration of the biological pump.]{
  		Illustration of the biological pump.
  		Adapted from \citet{ducklow2001upper}.
  		This cartoon lists the main processes of the biological pump. 
  		We focus in particular on the plankton dynamics that influence this physical process.
  		Starting from the top, carbone dioxide is dissolved into the ocean water through exchanges with the atmosphere. Inside water, CO$_2$ may react to form various carbonate species. This dissolved carbonates molecules may then be assimilated by planktonic organisms, either through photosynthesis or calcification for example. 
  		Through their movements in the flow, they actively participate to the dissemination of carbone in the water column.
  		By grazing on theme, other plankters further contribute to the transport of carbon molecules, in particular for those that perform daily vertical migrations.
  		At the same time, the excretion of zooplankton tends to aggregate with other marine debris forming larger particles called marine snow that sediment into the depth of the ocean.
  		Concentrating nutriments, this aggregates may attract hungry plankters that bring back some of the carbon molecules to the surface.
  		However the particles that reach the oceans can be trapped for huged amount of time.
  		The biological pump then contributes greatly to the global carbon trapping.
  	}
  	\label{fig:biological_pump}
\end{figure}
In addition to their impact on the marine food-web, plankton have an important role in the absorption of carbon dioxide from the atmosphere.
The plankton ecosystem acts as a ``biological pump'' that drags the carbon of the atmosphere and fix it on organic mater that sediments in the depth of the ocean.
This phenomenon entails a large variety of physical and biological processes briefly summarized here (Fig.~\ref{fig:biological_pump}).
The ocean absorbs carbon molecules through dissolution.
This phenomenon can be further enhanced by flow perturbations at the free surface, increasing air-water mixing.
The dissolved carbon molecules are then consumed by plankton species through photosynthesis but also through calcification to build outer shells and skeletons.
By grazing on them, other plankton species contribute to the vertical transport of carbon through their vertical migrations.
In addition, their excretion tend to aggregate with other ocean debris forming larger particles, called marine snow \citep{alldredge1988characteristics, turner2015zooplankton}, that settle.
During their sedimentation, these aggregates may break down due to their interactions with plankters that may feed on them.
The marine snow that makes it through and sediments in the ocean depths contributes to long term carbon trapping into solid sediments.
The oceanic flow comes into play on top of these processes.
It contributes to plankton transport and mixing, the formation of aggregates and their break up, as much as the direct advection of dissolved carbon.

The biological pump is then an essential part of the carbon cycle.
This process has led to the formation of carbon materials, such as chalk that is composed of prehistoric dead plankters \citep{farouk2020geochemical}.
More importantly, this process makes the ocean the largest carbon sink on earth \citep{lal2008carbon, hinge2020sustainability}.
The comprehension of the biological pump is then essential to model long term global climate. 
On top of that, plankton can also influence local weather.
They release organic matter in the atmosphere, such as organic sulfur molecules.
Acting as nucleation cores for water vapour, these particles eventually lead to the formation of clouds \citep{charlson1987oceanic, szyrmer1997biogenic}.
This phenomenon is particularly important when local high concentration of plankton occur, called plankton blooms \citep{behrenfeld2014resurrecting, park2017observational, creamean2019ice}.
Its impact on weather remains to be fully quantified \citep{quinn2011case} and motivates ongoing research on the topic.
For instance the Plankton, Aerosol, Cloud, ocean Ecosystem mission of the NASA \citep{werdell2019plankton} is expected to provide important satellite observations that will help understand and quantify these effects.

\section{Planktonic navigation problems}

Due to their implications in these various physical processes, modelling plankton is key to understand these important phenomena.
However modelling plankton is not trivial: (1) many plankters are motile and (2) react actively to their environment.
This motility contributes greatly to the processes described above.
Their active behavior and their responses to the environment are then key elements of plankton dynamics, that remain to be understood and accurately modelled \citep{franks2022oceanic}.
The objective of the current manuscript is to contribute to improve our understanding of the response of plankton to their flow environment. 

During their life, planktonic organisms have to face numerous survival tasks.
For instance, planktonic larvae have to disperse horizontally in the ocean to increases their chances to find a suitable settling habitat.
Plankters also have to forage for food and predate on other planktonic organisms, while having to escape their own predators.
Many zooplankton also perform diel vertical migrations: they travel to the oceans depths to avoid visual predators during the day.
Then, during the night when predators cannot see them, they get back to the ocean surface to feed on phytoplankton.
These previously described tasks involve reaching a target (either positional or directional) in a complex flow environment.
As such, these tasks can be considered as planktonic navigation problems.

It is not yet completely understood how plankters solve these problems.
Therefore navigation problems are the main interest of this study.
As we expect organisms to have evolved effective navigation strategies through natural selection, we investigate this issue using an optimality driven approach \citep{smith2011optimality}: we look for efficient navigation strategies in the context of plankton migration, in the aspiration it may lead to plankton behavior models in the future.

\subsection{How do plankton perceive their environment?}

\subsubsection{Plankton are riddled with inboard sensors!}

To address these planktonic navigation problems, the first question that arises is: how do plankters perceive their environment?
Many organisms, such as copepods [group of crustaceans sometime called the ``insects of the sea'' due to their abundance and variety \citep{schminke2007entomology}], are equipped with a primitive eye used to measure light intensity (Fig.~\ref{fig:copepod_picture}).
\begin{wrapfigure}[13]{R}[0.4\width]{0.4\textwidth}
	\vspace{-35pt}
	\centering
	\def\svgwidth{0.35\textwidth}
	\input{chap_intro/images/female_adult_acartia_clausi.pdf_tex}
	\captionsetup{width=0.35\textwidth}
  	\caption[Annotated picture of copepod]{
  		Annotated picture of copepod (Female adult \textit{Acartia Clausi}). 
  		Original picture by Minami Himemiya \ccbysa  ~ v3.0.
  	}
  	\label{fig:copepod_picture}
\end{wrapfigure}
This organ is however too underdeveloped to actually let them ``see'' their environment.
Therefore, to perceive their environment, copepods mostly rely on hair-like sensilia (sensory organs) on their antennules (antennae-shaped appendages).
Part of these sensilia are dedicated to flow measures, called setae, while others, called aesthetascs, enable chemical sensing \citep{heuschele2014chemical}.
Many planktonic larvae, such as oyster larvae, are also able to measure the direction of gravity thanks to their stotachists \citep{fuchs2015hydrodynamic}.
These organs function similarly to our internal ear that enable us to remain balanced.
Stotachists are sac-like organs containing a mobile mineralized component (statolith) that rolls in the organ when subject to acceleration or gravity (Fig.~\ref{fig:measure_vorticity}).
The statolith then triggers the setae (mechano-sensory cilia) that covers the organs internal walls.
Therefore, planktonic organisms equipped with this organ are able to measure their orientation with respect to gravity.

\subsubsection{Focus on flow sensing}\label{sec:intro_flow_sensing}

We now focus on the flow sensing of planktonic organisms. 
How does their perception of the flow differ from that of marine navigators on boats?
How does it differ from that of other animals such as flying insects, fishes or birds?

Marine navigators in charge of boat routing have access to weather forecasts and large scale ocean measures.
They can rely on this global information to plan their navigation route.
On the contrary, animals only react to the local information they can measure with their onboard sensors.
But planktonic organisms are even more limited than larger animals as they drift with the flow.
This property prevent them to directly measure the flow speed.
They are rather able to sense the difference of flow velocity between different parts of their body (gradients).

As humans, we are also subject to this phenomenon when we go swimming in the ocean.
It is easy to feel the impact of the waves and small scale currents on our body.
But if we do not visually pay enough attention to the shore, we hardly realize the large scale currents that may push us away from it.

For the readers that had the chance to take a ride in a hot air balloon, they may also have noticed they hardly felt any wind during their flight.
The only way to notice movement and apprehend the wind's presence is through the view of landscape passing below.
While flying insects, fishes and birds can assess their speed similarly using similar vision skills, most of planktonic organisms rely only on their measure of small flow perturbations (velocity gradients) due to their poor vision capabilities.
The measure of the flow speed and therefore, of their own total speed, is then out of reach.

Depending on their sensory organs, the flow information plankters have access to may be even more limited.
Even when equipped with setae (mechano-sensors) they would not be able to measure flow rotation (vorticity) as they would rotate with it as much as they drift with it.
This limits the part of the flow they can measure to pure strain (Fig.~\ref{fig:measure_strain}).
Moreover as sensing depends on the alignement of the atennules with shear, plankters orientation with respect to the flow strongly impacts sensing capabilities \citep{fields2010orientation}.
\begin{figure}
	\centering
	\begin{minipage}{0.5\textwidth}
	  	\centering
		\def\svgwidth{0.6\textwidth}
		\input{chap_turbulence/schemes/copepod_measure_strain.pdf_tex}
		\captionsetup{width=0.9\textwidth}
	  	\caption{Illustration of the measure process of the flow strain rate.}
	  	\label{fig:measure_strain}
	\end{minipage}%
	\begin{minipage}{0.5\textwidth}
		\centering
		\def\svgwidth{0.6\textwidth}
		\input{chap_turbulence/schemes/larval_measure_vorticity.pdf_tex}
		\captionsetup{width=0.9\textwidth}
	  	\caption{Illustration of the measure process of flow vorticity.}
	  	\label{fig:measure_vorticity}
	\end{minipage}
\end{figure}
If they are equipped with stotachists (gravity sensing) however, as illustrated in Fig~\ref{fig:measure_vorticity}, the flow vorticity (rotation rate of the flow) can be computed from their tilting angle with the vertical \citep{fuchs2015directional}.
Note that these organs may also let plankters react to strong flow accelerations.
Overall, when having only one or the other of these sensory organs, having access to the whole flow velocity gradients is challenging.

On the one hand, if one relies only on setae, like most copepods, one could overcome this difficulty with bottom-heaviness.
Having a heavy bottom causes immersed plankters to reorient and align with the vertical.
If the effect is strong enough, this may prevent them to rotate with the flow thus enabling them to measure flow vorticity.
However, this may limit their own active rotation capabilities.

On the other hand, if one only relies on statochists, having a non spherical shape could be helpful.
The rotation of non spherical particles is also influenced by the shear part of the flow.
However the challenge then lies in separating the part of rotation due to flow vorticity from the part due to pure strain\footnote{Note that \textbf{pure strain} (also called pure shear) is not to be confused with \textbf{simple shear} that contains a vortical part.}.

Despite these limitations, many plankton species, such as copepods, are known to use this partial flow information to detect mates, preys and predators \citep{kiorboe1999predator, kiorboe1999hydrodynamic, jiang2002hydrodynamic, kiorboe2018mechanistic}\footnote{The title of this thesis is an homage to the excellent book of \citet{kiorboe2018mechanistic}, that inspired greatly this whole study.}.
Oyster larvae are even known to be able to measure and forage underwater sounds \citep{williams2022oyster}. 
But, coming back to navigation, could plankters also use their flow sensing to travel faster towards their objective?

\subsection{Examples of planktonic navigation problems}

\begin{figure}
	\centering
	\def\svgwidth{0.9\textwidth}
	\input{chap_intro/schemes/planktonic_navigation_problems.pdf_tex}
	%\captionsetup{width=0.9\textwidth}
  	\caption{
  		Non exhaustive list and illustrations of planktonic navigation problems.
  	}
  	\label{fig:planktonic_navigation_problems}
\end{figure}
To answer this question, we need to define and formalize the planktonic navigation problem we address.
In practice a given problem depends on various parameters: the nature of the target to reach (or from which to escape), the sensory information available and the motivation to navigate.
Figure \ref{fig:planktonic_navigation_problems} lists and illustrates these different planktonic navigation problems that are described below.

\subsubsection{Close range navigation: predation and escape}

As a first example, we consider the case of prey hunting or predator escape when detection occurs thanks to local flow sensing.
The detection process controls the short range nature of the problem: from three to ten body lengths in calm water \citep{fields2010orientation}. 
This implies that hydrodynamical interactions, such as lubrification forces, need to be accounted for.
Then the behaviour of the prey or the predator also influences the problem. 
Is it possible to predict its behavior? 
Is stealth an option rather than running away?
Finally the motivation: either avoiding being eaten or feeding. 
To avoid death, the prey is motivated to react regardless of the energy used whereas the predator needs to ensure catching the prey is worth the energy consumed to catch it.

Similar navigation problems, accounting for various of the previously described effects, have already been addressed either using physics-based strategies and reinforcement learning models.
For instance, accounting for hydrodynamical interactions, \citet{zhu2022optimising} show that optimal control can be used to find the optimal navigation strategy that enables to catch a non-motile point. 
They also showed that efficient strategies can be found using reinforcement learning to catch finite-size preys and characterize the effect of the size of the prey on the navigation strategy.
\citet{borra2022reinforcement} additionally account for prey behavior and show that prey-predator systems can display complex dynamics.

However, these strategies do not account for an external flow gradient that could further modify navigation.
This is one of the aspects that could be explored in future research.
Note that from this problem also rise the questions of hydrodynamical detection strategies and hydrodynamical stealth strategies, also motivating ongoing research \citep{ren2021bluff}.

\subsubsection{Navigating to an odor source: finding a mate}

Another example consists of mate finding through chemical trails.
Particularly in the copepod group, the females of some plankton species leave a chemical trail behind as they swim \citep{weissburg1998following, bagoien2005blind, yen2010chemical}.
This odor trail is aimed to attract males that need to follow it to find their mate (phenomenon called chemotaxis).
This enables medium range sensing: up to $\approx 30$ body lengths \citep{bagoien2005blind} in quiescent water.

This complex navigation task is particularly challenging in the presence of a flow field that deforms that chemical trail.
Flow sensing can then be used to either try to find quicker path to the target but also to assess how the odor trail was deformed.
Note that tracking through chemical sensing is not exclusive to mate finding.
Experimental evidences show that odor sensing can be used by zooplankton to forage marine snow aggregates \citep{lombard2013copepods}.
The sperm cells of some marine invertebrates are also known to perform chemotaxis to reach egg cells \citep{lange2021sperm}.
On the contrary, many plankters display chemophobic responses to the detection of predator-mediated chemicals \citep{hay2009marine}.
This planktonic  navigation problem is similar, yet differs from odor tracking performed by flying insects \citep{carde2008navigational, willis2011role} due to the inability of most plankters to use visual references.

In different contexts, odor tracking has received increasing attention in the last years
These problems have been addressed using physics based models: \citet{vergassola2007infotaxis} proposes a very efficient strategy called \textit{infotaxis} to track the position of a scalar source without relying on local gradients.
The method however requires a spatial references and extensive memory.
This is relevant for complex organisms that can use spatial references using advanced vision to visualize their environment but this approach is out of reach for most plankters.
Furthermore this method does not account for flow sensing that could help to find the source.
On the contrary, \citet{lange2021sperm} proposes a simple chemotactic behavior based on gradient ascent that explains how external planktonic sperm cells find egg cells in the ocean.
Other studies rely on simple data-driven approaches. For instance \citet{koehl2007individual} deduced a simple on/off model parameterized with experiments.
This model was then used in simulations to understand how planktonic larvae increased their migration towards coral reefs in response to the chemical cues they emit.

Finally reinforcement learning methods are also a popular approach to address this problem \citep{lu2011learning, fischer2017odor, liberzon2018moth, jing2021recent, loisy2022searching, rigolli2022learning}.
Overall, reinforcement learning are shown to be a promising tool to solve this kind of navigation problems.
However most of these learning-based studies either consider simple flow environments (rather than chaotic environments such as turbulence) or consider large memory organisms with spatial reference (not applicable to the planktonic version of the problem).
Furthermore, most of these studies do not consider local flow sensing, then the question of how this information could be used to improve navigation remains to be addressed.

\subsubsection{Point to point navigation: hunting a bioluminescent target}

Yet another example of planktonic point to point navigation consists of target tracking sing basic light sensing.
The oceans contain numerous bioluminescent organisms (organisms that emit light): $76\%$ of the organisms recorded by \citet{martini2017quantification} had bioluminescent capabilities with very little variations with depth.
Despite their poor vision skills, many zooplankton react to these visual cues and trigger phototactic (attraction) or photophobic (repulsion) responses.
For instance, dinoflagellate flashing luminescence has been observed to trigger escape responses from certain species of copepods \citep{buskey1983behavioral}.
It has been suggested the dinoflagellates evolved to copy the warning signal of bioluminescent copepods to avoid being grazed \citep{buskey1985behavioral}.
On the contrary, many zooplankton species are observed to be phototactic and attracted by sunlight \citep{ringelberg1999photobehaviour}, artificial light \citep{jekely2008mechanism, krafft2021antarctic, stearns1984photosensitivity} or illuminated particles \citep{tanaka2019biased}.

Bioluminescence has an important role in marine ecology. 
It is used by marine organisms to predate, defend themselves and even to attract potential living habitats by bioluminescent bacteria  \citep{haddock2010bioluminescence}.
It also has a large impact on the carbon biological pump as bioluminescent organisms often aggregate in marine snow, then attracting organisms that feed on them.
In the context of planktonic navigation, bioluminescent organisms constitute detectable long range positional targets.

This navigation problem drawed attention in recent studies. For instance in the context of the flight of flies, \citet{fabian2018interception} showed that a proportional feedback control [called proportional navigation, also used to control the navigation of modern missiles \citep{shneydor1998missile}] can successfully model the interception of aerial targets by flies.
However no flow exploitation is considered.
Flow sensing could lead to the foraging of beneficial currents that could be used to reach (or escape) that visual target faster.
Reinforcement learning has also been shown to be a promising tool to solve this problems in simplified flow fields \citep{gunnarson2021learning} and using the information of the whole flow field \citep{biferale2019zermelo}.
However, these approaches are challenged in more complex flows such as time-dependant 3D turbulence \citep{alageshan2020machine, qiu2022active}, more representative of the complexity of plankton environment.

\subsubsection{Navigating to explore: plankton horizontal dispersion}

The population of many marine organisms is solely redistributed at their planktonic larvae stage.
As adults they fixate on surfaces preventing them to disperse anymore.
The dispersion of their larvae when a new generation births is then essential for the migration of these marine species.
While simple passive Lagrangian models are often used to model larval dispersion \citep{siegel2003lagrangian}, it has been shown that larval behavior can strongly affect the horizontal distribution of larvae \citep{naylor2006orientation, vikebo2007drift, morgan2021robotic}.
The horizontal dispersion problem consists then of a long range navigation problem where larvae need to maximize horizontal dispersion to ensure the largest settlement area is explored so the species gets the best chances of survival.

\subsubsection{Directional navigation: plankton vertical migrations}

Finally, we can consider the plankton vertical migrations.
It is yet another essential navigation problem that many plankters have to face.
The research on vertical migrations initiated two hundreds years ago when they were first observed by \citet{cuvier1817regne}.
Plankton vertical migration generally occurs either daily or seasonally \citep{bandara2021two}.

The former, called diel vertical migrations, are generally performed by grazing zooplankton that migrate to the surface during the night to graze on other plankters.\footnote{Note that all diel vertical migrations are not synchronous, and asynchronous migrations are also observed \citep{cottier2006unsynchronised}.}
Then they flee the light at dawn to avoid getting caught by visual predators (such as fishes).
These migrations are generally triggered by variations of light intensity \citep{richards1996diel, van2013diel}. 
More frequent migration can even be triggered due to cloud cover \citep{omand2021cloud}, moon light \citep{last2016moonlight} and eclipses \citep{adhikari2018effect}.

Seasonal vertical migration however, occur on much larger time scale.
Various planktonic organisms swim to the ocean depths at the beginning of winter to wait the end of the season in a dormant stage \citep{naess1991diapausing, kaartvedt1996habitat}.
The cold water of the depths keeps their metabolism slow while they wait for the time to migrate back to the surface when spring comes.
Seasonal vertical migration can also be part of larval dispersion \citep{mcmanus2012plankton, kim1994larval, vikebo2007drift}.
Some larvae may migrate near the surface to benefit from stronger currents that disperse over larger distances.

Overall vertical migration translates to a long range vertical navigation problem: Antarctic krill may migrate over one kilometer distance \citep{hamner1983behavior}, corresponding to one hundred thousand time their body length!

Overall this navigation problem shines by its simplicity: plankters just have migrate in a constant direction.
In addition, plankton vertical migration is an essential part of plankton dynamics and the biological pump.
Both of these reasons make of this problem, a good first candidate to start investigating planktonic navigation.

Note that, in different contexts, similar problems has already been addressed in the past. 
For instance solutions have been deduced in simple flows thanks to reinforcement learning \citep{colabrese2017flow, gustavsson2017finding}. 
The efficiency of reinforcement learning remains however to be demonstrated in more complex flows such as time-dependant 3D turbulence \citep{alageshan2020machine}.
Moreover the link between these navigation strategies and the modeling of plankton migration remains unclear due to the ``black box'' effect of learning methods that challenges their physical interpretation.
Therefore a physics-based approach is used in this study.

\section{Thesis structure}

Due to the essential importance of vertical migrations in plankton dynamics (and the biological pump) \citep{bianchi2013diel, archibald2019modeling}, the simplicity of the navigation problem and the lack of comparative physics-based model performing in complex flow environments, this thesis focuses on the navigation related to planktonic vertical migration.
The key question is to understand to what extent local flow information can be used to improve navigation through complex flow environments.

To address this problem, in Chap.~\ref{chap:the_surfing_strategy}, we first simplify the process of plankton vertical migrations to formalize the problem into an idealized navigation problem.
We then derive an optimal solution in linear flow, which we call the \textit{surfing strategy}, and assess its performance in simple non-linear flows.
In Chap.~\ref{chap:surfing_on_turbulence}, we introduce the physics and features of turbulence. 
We then assess the performance of the previously derived surfing strategy in turbulent environments.
In Chap.~\ref{chap:surfing_robustness}, we relax the previously made assumptions to assess how robust would the strategy remain to biological constraints.
Surfing is then compared to alternative navigation methods in Chap.~\ref{chap:navigation}.
This motivates a generalisation of the surfing strategy that can be applied with more flow information.
The surfing strategy is then discussed in the context of marine biology in Chap.~\ref{chap:bio_discussion}.
We demonstrate it to be beneficial over a wide range of plankton habitats and present potential experiments that could be performed to help verifying its relevance for actual plankters.
Finally we conclude and discuss of the various perspectives of this study in Chap.~\ref{chap:perspectives}.

% \subsection{Showing the Use of Acronyms}
% 
% In the early nineties, \acs{GSM} was deployed in many European countries. \ac{GSM} offered for the first time international roaming for mobile subscribers. The \acs{GSM}’s use of \ac{TDMA} as its communication standard was debated at length. And every now and then there are big discussion whether \ac{CDMA} should have been chosen over \ac{TDMA}.
% 
% If you want to know more about \acf{GSM}, \acf{TDMA}, \acf{CDMA} and other acronyms, just read a book about mobile communication. Just to mention it: There is another \ac{UA}, for testing.
