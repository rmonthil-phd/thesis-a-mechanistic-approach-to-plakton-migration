\chapter{Runge kutta method}\label{app:runge_kutta_method}

Numerical solvers are used to solve initial value problems of ordinary differential equations
\begin{equation}
	\frac{d \vec{y}}{dt} = \vec{f}(t, \vec{y}), \quad \vec{y}(t_0) = \vec{y}_0.
\end{equation}
Starting from $t = t_0$ for which $\vec{y} = \vec{y}_0$ is known, we can directly compute the slope $\vec{k}_1 = \vec{f}(t_0, \vec{y}_0)$ of this function.
Then approximating $\vec{y}$ to its linear approximation $\vec{y}(t) \approx \vec{y}_0 + (t - t_0) \, \vec{k}_1$, one can obtain the value $\vec{y}_1 = \vec{y}_0 + \varDelta t \, \vec{k} \approx \vec{y}(t + \varDelta t)$ for a prescribed time step $\varDelta t$.
Performing this operation repeatedly leads to the integration of $\vec{y}$ using the Euler method, defined by the following recursion equation
\begin{equation}
	\vec{y}_{n+1} = \vec{y}_n + \varDelta t \, f(y_n, t_n),
\end{equation}
where $\vec{y}_n = y(t + n \, \varDelta t)$ and $t_n = t + n \, \varDelta t$.
Despite the appeal due to its simplicity, this method remains an first-order numerical integration method, meaning the error is proportional to the step size $\varDelta t$.
Any error produced by the method may accumulate over iterations and lead to a large errors over time.
\begin{wraptable}[9]{L}[0.5\width]{0.45\textwidth}
	\vspace{-10pt}
	\center
	\begin{tabular}{ r | l l l l l }
		$c_1$ & & & & & \\
		$c_2$ & $a_{2,1}$ & & & & \\
		$c_3$ & $a_{3,1}$ & $a_{3,2}$ & & & \\
		$\vdots$ & $\vdots$ & & $\ddots$ & & \\
		$c_s$ & $a_{s,1}$ & $a_{s,2}$ & $\cdots$ & $a_{s,s-1}$ & \\
		\midrule
		 & $b_1$ & $b_2$ & $\cdots$ & $b_{s-1}$ & $b_s$
	\end{tabular}
	\captionsetup{width=0.4\textwidth}

	\caption{
		Butcher's table.
	}
	\label{tab:butcher_table}
\end{wraptable}

To overcome this issue, one can rely on higher orders methods such as the Runge-Kutta method, that can be seen as a generalization of the Euler method.
In this study,the equation of motion is integrated numerically using a $4^{\mathrm{th}}$ order explicit Runge-Kutta method.
This method relies on the following recursion law
\begin{equation}
	\vec{y}_{n+1} = \vec{y}_n + \varDelta t \sum_{i = 1}^s b_i \vec{k}_i
\end{equation}
where $b_i$ are numerical coefficients, called \textit{weights}, that are specific to each particular Runge-Kutta method.
The values $k_i$ correspond to the slope $\vec{f}$ evaluated for various intermediate values in time between $t_{n}$ and $t_{n+1}$ (Fig.~\ref{fig:runge_kutta}).
\begin{figure}
	\centering
	\def\svgwidth{0.6\textwidth}
	\input{chap_numeric/schemes/runge_kutta.pdf_tex}
	%\captionsetup{width=0.3\textwidth}
  	\caption{
  		Illustration of the Runge-Kutta method.
  	}
  	\label{fig:runge_kutta}
\end{figure}
These slope values are evaluated as follows
\begin{equation}
	k_i = f(t_n + c_i \varDelta t, \, y_n + \varDelta t \sum_{j=1}^{i-1} a_{i,j} k_j)
\end{equation}
with the coefficients $c_i$ called \textit{nodes} and the coefficients $a_{i,j}$ named the Runge-Kutta matrix.
All coefficients are generally summed up into the Butcher table to describe a method of the Runge-Kutta family (Tab.~\ref{tab:butcher_table}).
Note that the coefficients of the method need to be chosen carefully so that it converges consistently towards the solution.

In other words, while the Euler method only relies on a single slope $\vec{k}$ evaluated at $t = t_n$, the Runge-Kutta relies on $s$ measures of the slope $\vec{k}_i$ in between $t_n$ and $t_{n+1}$.
Moreover the Euler method corresponds to the Runge-Kutta case {\normalsize\begin{tabular*}{33pt}{ r | l }
	0 & \\
	\midrule
	 & 1
\end{tabular*}} with $s=1$.

In this study a fourth-order Runge-Kutta-Fehlberg method, characterized by the Butcher's table given in Tab.~\ref{tab:runge_kutta_fehlberg_table} \citep{fehlberg1968classical},
\begin{table}
	\center
	\begin{tabular}{ r | l l l l l l }
		0 & & & & & & \\
		1/4 & 1/4 & & & & & \\
		3/8 & 3/32 & 9/32 & & & & \\
		12/13 & 1932/2197 & 7200/2197 & 7296/2197 & & & \\
		1 & 439/216 & -8 & 	3680/513 & -845/4104 & & \\
		1/2 & -8/27 & 2 & -3544/2565 & 1859/4104 & -11/40 & \\
		\midrule
		 & 16/135 & 0 & 6656/12825 & 28561/56430 & -9/50 & 2/55 \\
		 & 25/216 & 0 & 1408/2565 & 2197/4104 & -1/5 & 0
	\end{tabular}

	\caption[Butcher's table of the Runge-Kutta-Fehlberg method.]{
		Butcher's table of the Runge-Kutta-Fehlberg method.
		The first row of $b_i$ coefficients at the bottom of the table gives a fifth-order method.
		The second row corresponds to a fourth-order method.
	}
	\label{tab:runge_kutta_fehlberg_table}
\end{table}
is used in our in-house open-source code \textit{SHELD0N} \footnote{Our in-house code is available at \url{http://www.github.com/C0PEP0D/sheld0n}.}, to integrate the equation of motion.
The code can be setup to integrate plankter trajectories by automatically querying various flow fields of John Hopkins turbulence database at plankton positions through simulations.
It is also able to use local flow databases generated from \textit{SNOOPY} simulations (or from any fluid solver with the same output format) by interpolating the flow field at plankter positions.
In this study, we used a fourth-order Lagrange polynomial interpolation to integrate trajectories in our own flow databases while a sixth-order interpolation is performed on query when using the Johns Hopkins turbulence database.
Implementation of various analytical flow velocity fields are also provided in which microswimmers trajectories can be integrated.
