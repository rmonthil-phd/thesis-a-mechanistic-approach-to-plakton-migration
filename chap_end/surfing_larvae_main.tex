\section{Surfing larvae}

\paragraph{Note} In the following, both orientation $\theta$ and vorticity $\omega$ are oriented positively clockwise.

\subsection{Expected orientation with respect to vorticity}

\paragraph{Bottom-heavy expected orientation.}
The orientation of a small spherical bottom-heavy plankter can be modeled as the following:
\begin{equation}\label{eq:spherical_bottom_heavy_orientation}
	\frac{d \SwimmingDirection}{d t}  =
	\frac{1}{2} \FlowVorticity (\ParticlePosition, t) \times \SwimmingDirection + \frac{1}{2 \ReorientationTime} \left[ \Direction - (\Direction \cdot \SwimmingDirection) \SwimmingDirection \right],
\end{equation},
with $\SwimmingDirection$ the swimming axis of the plankter, $\FlowVorticity = \vec{\nabla} \times \FlowVelocity$ the flow vorticity and $\ReorientationTime$ the characteristic alignment time towards the vertical $\Direction$ due to bottom-heaviness.
An equilibrium orientation exists if $1/\ReorientationTime > \FlowVorticity_{\perp \Direction}$ and can be formulated as follows:
\begin{equation}\label{eq:spherical_bottom_heavy_orientation}
	\theta_{\mathrm{\NameBhShort}} = \mathrm{arcsin} \, \ReorientationTime \FlowVorticity_{\perp \Direction}
\end{equation},
with $\theta$ the angle to the vertical and $\FlowVorticity_{\perp \Direction} = \FlowVorticity - (\FlowVorticity \cdot \Direction) \Direction$ the vorticity projected orthogonaly to $\Direction$.

\paragraph{Surfing expected orientation.}
In case of a pure vortex flow, the gradients matrix is skew symmetric.
As the matrix exponential of a skew symmetric matrix is rotation matrix, the surfing strategy can the be simplified to:
\begin{equation}
	\label{eq:optimal_swimming_direction_partial_asym}
	\SwimmingDirectionOpt(\TimeHorizon) = \matr{R}_{\hat{\FlowVorticity}}(-\omega \TimeHorizon/2) \, \Direction,
\end{equation}
with $\matr{R}_{\hat{\FlowVorticity}}(-\FlowVorticityScalar \TimeHorizon/2)$ the rotation matrix of angle $-\FlowVorticityScalar \TimeHorizon/2$ with $\omega = \norm*{\vec{\nabla} \times \FlowVelocity}$ and of axis $\hat{\FlowVorticity}$, the normalized vorticity.
The surfing direction's angle to the vertical can then be expressed as the following:
\begin{equation}\label{eq:surfing_orientation}
	\theta_{\mathrm{\NameSurfShort}} = -\frac{1}{2} \ReorientationTime \FlowVorticity_{\perp \Direction},
\end{equation}

\begin{figure}%[H]
	\centering
	\begin{tikzpicture}
	% gain as a function of the free parameter $\tau$
	\begin{axis} [
		axis on top,
		% size
		width=0.65\textwidth,
		% y
		ymin=-pi/2.0,
		ymax=pi/2.0,
		ylabel={$\theta$},
		ytick={-pi/2.0, -pi/4.0, 0, pi/4.0, pi/2.0},
		yticklabels={$-\frac{\pi}{2}$, $-\frac{\pi}{4}$, 0, $\frac{\pi}{4}$, $\frac{\pi}{2}$},
		% x
		xlabel={$\FlowVorticityScalar_{\perp \Direction}$},
		xmin=-2,
		xmax=2,
		% layers
		set layers,
		% legend
		legend style={
			at={(0.5,1.05)},
			anchor=south,
			draw=none,
			fill=none,
			%/tikz/every even column/.append style={column sep=8pt},
		},
		%legend pos=outer east,
		legend cell align=left,
		legend columns=-1,
	]
		% axes
		\addplot
		[
			color=black,
			opacity=0.2,
			solid, 
			on layer=axis background,
			domain=-2:2,
			forget plot,
		]{0.0};
		\addplot+[
			mark=none,
			color=black,
			opacity=0.2,
			solid, 
			on layer=axis background,
			forget plot,
		] coordinates {(0, -pi/2.0) (0, pi/2.0)};
		% surf
		\addplot
		[
			color=ColorSurf,
			opacity=1.0,
			solid, 
			domain=-2:2,
			samples=20,
			mark=square*,
		]{-x * 0.5 / 2.0};
		\addlegendentry{$\tau = 1/2$}
		\addplot
		[
			color=ColorSurf,
			opacity=1.0,
			domain=-2:2,
			mark=pentagon,
			samples=20,
		]{-x * 1.0 / 2.0};
		\addlegendentry{$\tau = 1$}
		\addplot
		[
			color=ColorSurf,
			opacity=1.0,
			domain=-2:2,
			mark=*,
			samples=20,
		]{-x * 2.0 / 2.0};
		\addlegendentry{$\tau = 2$}
		% bottom-heavy
		\addplot
		[
			color=ColorBh,
			opacity=1.0,
			solid, 
			domain=-2:2,
			samples=50,
			mark=o,
			samples=20,
		]{rad(asin(1.0/4.0 * x))};
		\addlegendentry{$\tau_{\mathrm{align.}}^{\mathrm{b-h}} = 1/4$}
		\addplot
		[
			color=ColorBh,
			opacity=1.0,
			solid, 
			domain=-2:2,
			mark=pentagon*,
			samples=20,
		]{rad(asin(1.0/2.0 * x))};
		\addlegendentry{$\tau_{\mathrm{align.}}^{\mathrm{b-h}} = 1/2$}
		\addplot
		[
			color=ColorBh,
			opacity=1.0,
			solid, 
			domain=-1:1,
			mark=square,
			samples=20,
		]{rad(asin(1.0 * x))};
		\addlegendentry{$\tau_{\mathrm{align.}}^{\mathrm{b-h}} = 1$}
	\end{axis}
\end{tikzpicture}

	\caption{
		Expected angle to the vertical $\theta$ as a function of flow vorticity for surfers ($\theta_{\mathrm{\NameSurfShort}}$) and for bottom-heavy swimmer ($\theta_{\mathrm{\NameBhShort}}$).
	}
	\label{fig:vorticity_angle_to_the_vertical}
\end{figure}

\subsection{Expected orientation with respect to shear}

\paragraph{Bottom-heavy expected orientation.}
We would expect bottom-heavy swimmers to orient with the underlying average vorticity.
In that case, bottom-heavy swimmers would orient preferentially towards their average equilibrium orientation (same as before):
\begin{equation}\label{eq:spherical_bottom_heavy_orientation}
	\theta_{\mathrm{\NameBhShort}} = \mathrm{arcsin} \, -\ReorientationTime \frac{\partial v}{\partial x}
\end{equation}

\paragraph{Surfing expected orientation.}
In case of a vertical simple shear flow, the matrix exponential of the gradients reduces to its two first terms, as a consequence surfing reduces to the following:
\begin{equation}
	\label{eq:optimal_swimming_direction_partial_asym}
	\SwimmingDirectionOpt(\TimeHorizon) = \frac{\SwimmingDirectionOptNN}{\norm{\SwimmingDirectionOptNN}} \quad \text{with} \quad \SwimmingDirectionOptNN = \Direction + \TimeHorizon \frac{\partial v}{\partial x} \, \vec{e}_x,
\end{equation}
The surfing direction's angle to the vertical can then be expressed as the following:
\begin{equation}\label{eq:surfing_orientation}
	\theta_{\mathrm{\NameSurfShort}} = \mathrm{arcsin} \, \frac{\TimeHorizon (\partial v / \partial x)}{\sqrt{1 + \TimeHorizon^2 (\partial v / \partial x)^2}}.
\end{equation}

\begin{figure}%[H]
	\centering
	\begin{tikzpicture}
	% gain as a function of the free parameter $\tau$
	\begin{axis} [
		axis on top,
		% size
		width=0.65\textwidth,
		% y
		ymin=-pi/2,
		ymax=pi/2,
		ylabel={$\theta$},
		ytick={-pi/2.0, -pi/4.0, 0, pi/4.0, pi/2.0},
		yticklabels={$-\frac{\pi}{2}$, $-\frac{\pi}{4}$, 0, $\frac{\pi}{4}$, $\frac{\pi}{2}$},
		% x
		xlabel={$\norm{\vec{\nabla}_{xy} \FlowVelocityScalar_{\DirectionScalar}}$},
		xmin=-2,
		xmax=2,
		% layers
		set layers,
		% legend
		legend style={
			at={(0.5,1.05)},
			anchor=south,
			draw=none,
			fill=none,
			%/tikz/every even column/.append style={column sep=8pt},
		},
		%legend pos=outer east,
		legend cell align=left,
		legend columns=-1,
	]
		% axes
		\addplot
		[
			color=black,
			opacity=0.2,
			solid, 
			on layer=axis background,
			domain=-2:2,
			forget plot,
		]{0.0};
		\addplot+[
			mark=none,
			color=black,
			opacity=0.2,
			solid, 
			on layer=axis background,
			forget plot,
		] coordinates {(0, -pi/2.0) (0, pi/2.0)};
		% surf
		\addplot
		[
			color=ColorSurf,
			opacity=1.0,
			solid, 
			domain=-2:2,
			samples=20,
			mark=square*,
		]{rad(asin(1.0/2.0 * x / sqrt(1 + (1.0/2.0 * x)^2)))};
		\addlegendentry{$\tau = 1/2$}
		\addplot
		[
			color=ColorSurf,
			opacity=1.0,
			domain=-2:2,
			mark=pentagon,
			samples=20,
		]{rad(asin(1.0 * x / sqrt(1 + (1.0 * x)^2)))};
		\addlegendentry{$\tau = 1$}
		\addplot
		[
			color=ColorSurf,
			opacity=1.0,
			domain=-2:2,
			mark=*,
			samples=20,
		]{rad(asin(2.0 * x / sqrt(1 + (2.0 * x)^2)))};
		\addlegendentry{$\tau = 2$}
		% bottom-heavy
		\addplot
		[
			color=ColorBh,
			opacity=1.0,
			solid, 
			domain=-2:2,
			samples=50,
			mark=o,
			samples=20,
		]{rad(asin(-1.0/4.0 * x))};
		\addlegendentry{$\tau_{\mathrm{align.}}^{\mathrm{b-h}} = 1/4$}
		\addplot
		[
			color=ColorBh,
			opacity=1.0,
			solid, 
			domain=-2:2,
			mark=pentagon*,
			samples=20,
		]{rad(asin(-1.0/2.0 * x))};
		\addlegendentry{$\tau_{\mathrm{align.}}^{\mathrm{b-h}} = 1/2$}
		\addplot
		[
			color=ColorBh,
			opacity=1.0,
			solid, 
			domain=-1:1,
			mark=square,
			samples=20,
		]{rad(asin(-1.0 * x))};
		\addlegendentry{$\tau_{\mathrm{align.}}^{\mathrm{b-h}} = 1$}
	\end{axis}
\end{tikzpicture}

	\caption{
		Expected angle to the vertical $\theta$ as a function of flow shear $\partial v/\partial x$ for surfers ($\theta_{\mathrm{\NameSurfShort}}$) and for bottom-heavy swimmer ($\theta_{\mathrm{\NameBhShort}}$).
	}
	\label{fig:shear_angle_to_the_vertical}
\end{figure}

\subsection{Expected orientation with respect to pure shear}

\paragraph{Bottom-heavy expected orientation.}
We would not expect any preferred spherical bottom-heavy swimmers orientation with pure shear.

\paragraph{Surfing expected orientation.}
In case of a vertical simple shear flow, the matrix exponential of the gradients reduces to its symmetric part, as a consequence, writing the surfing strategy in the eigen basis of the gradients ($\vec{e}_1, \vec{e}_2$) for which the respective eigen values are noted $\lambda_1, \lambda_2$ and are sorted in descending order $\lambda_1 > \lambda_2$:
\begin{equation}
	\label{eq:optimal_swimming_direction_partial_asym}
	\SwimmingDirectionOpt(\TimeHorizon) = \frac{\SwimmingDirectionOptNN}{\norm{\SwimmingDirectionOptNN}} \quad \text{with} \quad \SwimmingDirectionOptNN = 
	\begin{pmatrix}
		e^{\lambda_1 \tau} & 0\\
		0 & e^{\lambda_2 \tau}
	\end{pmatrix} \cdot
	\begin{pmatrix}
		\cos \psi \\
		\sin \psi
	\end{pmatrix},
\end{equation}
with $\psi$ the orientation angle $\vec{e}_1$ to the vertical.
Thus the surfing direction's angle to the vertical can then be expressed as the following:
\begin{equation}\label{eq:surfing_orientation}
	\theta_{\mathrm{\NameSurfShort}} = \mathrm{arccos} \, \frac{e^{\lambda_1 \TimeHorizon} \cos^2 \psi + e^{\lambda_2 \TimeHorizon} \sin^2 \psi}{\sqrt{e^{2\lambda_1 \TimeHorizon} \cos \psi + e^{2\lambda_2 \TimeHorizon} \sin \psi}}.
\end{equation}
Note that for $(\lambda_1 - \lambda_2) \TimeHorizon \gg 1$:
\begin{equation}\label{eq:surfing_orientation}
	\theta_{\mathrm{\NameSurfShort}} \approx \psi
\end{equation}
\quest{Would you get better correlations if you filtered your data to match that condition? (using the value $\TimeHorizon = 0.7 \KolmogorovTimeScale$ I have roughly fitted in the following)}
Anyway, even though that does not work better, it might still be interesting to discuss what are these larvae are actually measuring.

% \subsection{Preferential flow sampling}
% 
% Surfers exploit the turbulent flow by biasing the sampling of vertical flow velocities. This bias is illustrated in Fig.~\ref{fig:surfing_velocity_sampled},
% \begin{figure}%[H]
	% \centering
	% \begin{tikzpicture}
	\begin{groupplot}[
		group style={
			group size=2 by 2,
			y descriptions at=edge left,
			%x descriptions at=edge bottom,
			horizontal sep=0.06\linewidth,
			vertical sep=0.06\linewidth,
		},
		% size
		width=0.5\textwidth,
		% y
		ymode=log,
		% layers
		set layers,
		% legend
		legend style={draw=none, fill=none, /tikz/every even column/.append style={column sep=4pt}, at={(1.0, 1.05)}, anchor=south},
		%legend pos=north west,
   		legend cell align=left,
   		legend columns=-1,
	]
		\nextgroupplot[
			axis on top,
			% x
			xmin=-30,
			xmax=30,
			xtick={-30,-15,0,15,30},
			% y
			ylabel={$p(U = \FlowVelocityScalar)$},
			ymin=0.001,
			ymax=0.1,
		]
			\node[anchor=north west] at (axis cs:-30,0.1) {\textbf{(a)}: $\mathit{Re}_{\lambda}$ = 418};
			%% passive
			\addplot
			[
			color=black,
			opacity=1.0,
			%only marks,%solid
			mark=asterisk,
			mark repeat=10,
			]
			table[
				x expr={\thisrowno{0} / 0.066}, %u_\eta = 0.066
				y expr={\thisrowno{1} * 0.066},
				col sep=comma, 
				comment chars=\#,
			]{chap_turbulence/data/flow_sampled/pdf_u_0__pagent.csv};
			\addlegendentry{passive}
			%% us 1.0 straight
			\addplot
			[
			color=ColorBh,
			opacity=1.0,
			%only marks,%solid
			mark=o,
			mark repeat=10,
			]
			table[
				x expr={\thisrowno{0} / 0.066}, %u_\eta = 0.066
				y expr={\thisrowno{1} * 0.066},
				col sep=comma, 
				comment chars=\#,
			]{chap_turbulence/data/flow_sampled/pdf_u_0__agent__us_1o0__surftimeconst_0o0.csv};
			\addlegendentry{\NameBh}
			%% us 1.0 surf
			\addplot
			[
			color=ColorSurf,
			opacity=1.0,
			%only marks,%solid
			mark=square*,
			mark repeat=10,
			]
			table[
				x expr={\thisrowno{0} / 0.066}, %u_\eta = 0.066
				y expr={\thisrowno{1} * 0.066},
				col sep=comma, 
				comment chars=\#,
			]{chap_turbulence/data/flow_sampled/pdf_u_0__agent_full__us_1o0__surftimeconst_5o0.csv};
			\addlegendentry{\NameSurf}



		\nextgroupplot[
			axis on top,
			% x
			xmin=-30,
			xmax=30,
			xtick={-30,-15,0,15,30},
			% y
			ymin=0.001,
			ymax=0.1,
		]
			\node[anchor=north west] at (axis cs:-30,0.1) {\textbf{(b)}: $\mathit{Re}_{\lambda}$ = 418};
			%% passive
			\addplot
			[
			color=black,
			opacity=1.0,
			%only marks,%solid
			mark=asterisk,
			mark repeat=10,
			]
			table[
				x expr={\thisrowno{0} / 0.066}, %u_\eta = 0.066
				y expr={\thisrowno{1} * 0.066},
				col sep=comma, 
				comment chars=\#,
			]{chap_turbulence/data/flow_sampled/pdf_u_2__pagent.csv};
			%\addlegendentry{passive}
			%% us 1.0 straight
			\addplot
			[
			color=ColorBh,
			opacity=1.0,
			%only marks,%solid
			mark=o,
			mark repeat=10,
			]
			table[
				x expr={\thisrowno{0} / 0.066}, %u_\eta = 0.066
				y expr={\thisrowno{1} * 0.066},
				col sep=comma, 
				comment chars=\#,
			]{chap_turbulence/data/flow_sampled/pdf_u_2__agent__us_1o0__surftimeconst_0o0.csv};
			%\addlegendentry{\NameBh}
			%% us 1.0 surf
			\addplot
			[
			color=ColorSurf,
			opacity=1.0,
			%only marks,%solid
			mark=square*,
			mark repeat=10,
			]
			table[
				x expr={\thisrowno{0} / 0.066}, %u_\eta = 0.066
				y expr={\thisrowno{1} * 0.066},
				col sep=comma, 
				comment chars=\#,
			]{chap_turbulence/data/flow_sampled/pdf_u_2__agent_full__us_1o0__surftimeconst_5o0.csv};
			%\addlegendentry{\NameSurf}




		\nextgroupplot[
			axis on top,
			% x
			xlabel=$\FlowVelocity_\DirectionScalar / \KolmogorovVelocityScale$,
			xmin=-4,
			xmax=4,
			% y
			ylabel={$p(U = \FlowVelocityScalar)$},
			ymin=0.01,
			ymax=1,
		]
			\node[anchor=north west] at (axis cs:-4,1) {\textbf{(c)}: $\mathit{Re}_{\lambda}$ = 11};
			%% passive
			\addplot[
				ColorPassive,
				%only marks,
				mark=star,
			] table [
				x expr={\thisrowno{0} / 0.21},
				y expr={\thisrowno{1} * 0.21},
				col sep=comma, 
				comment chars=\#,
				unbounded coords=discard,
			] {data/tracers__flow__n_128__re_250/tracer__flow_velocity_sampled_pdfs.csv};
			%\addlegendentry{passive}
			%% bh
			\addplot[
				ColorBh,
				%only marks,
				mark=o,
			] table [
				x expr={\thisrowno{0} / 0.21},
				y expr={\thisrowno{1} * 0.21},
				col sep=comma, 
				comment chars=\#,
				unbounded coords=discard,
			] {data/surfers__flow__n_128__re_250/surfer__vs_1o0__surftimeconst_0o0__flow_velocity_sampled_pdfs.csv};
			%\addlegendentry{\NameBh}
			%% surfer
			\addplot[
				ColorSurf,
				%only marks,
				mark=square*,
			] table [
				x expr={\thisrowno{0} / 0.21},
				y expr={\thisrowno{1} * 0.21},
				col sep=comma, 
				comment chars=\#,
				unbounded coords=discard,
			] {data/surfers__flow__n_128__re_250/surfer__vs_1o0__surftimeconst_2o0__flow_velocity_sampled_pdfs.csv};
			%\addlegendentry{\NameSurf}




		
		\nextgroupplot[
			axis on top,
			% x
			xlabel=$\FlowVelocityScalar_y / \KolmogorovVelocityScale$,
			xmin=-4,
			xmax=4,
			% y
			ymin=0.01,
			ymax=1,
		]
			\node[anchor=north west] at (axis cs:-4,1) {\textbf{(d)}: $\mathit{Re}_{\lambda}$ = 11};
			%% passive
			\addplot[
				ColorPassive,
				%only marks,
				mark=star,
			] table [
				x expr={\thisrowno{0} / 0.21},
				y expr={\thisrowno{3} * 0.21},
				col sep=comma, 
				comment chars=\#,
				unbounded coords=discard,
			] {data/tracers__flow__n_128__re_250/tracer__flow_velocity_sampled_pdfs.csv};
			%\addlegendentry{passive}
			%% bh
			\addplot[
				ColorBh,
				%only marks,
				mark=o,
			] table [
				x expr={\thisrowno{0} / 0.21},
				y expr={\thisrowno{3} * 0.21},
				col sep=comma, 
				comment chars=\#,
				unbounded coords=discard,
			] {data/surfers__flow__n_128__re_250/surfer__vs_1o0__surftimeconst_0o0__flow_velocity_sampled_pdfs.csv};
			%\addlegendentry{\NameBh}
			%% surfer
			\addplot[
				ColorSurf,
				%only marks,
				mark=square*,
			] table [
				x expr={\thisrowno{0} / 0.21},
				y expr={\thisrowno{3} * 0.21},
				col sep=comma, 
				comment chars=\#,
				unbounded coords=discard,
			] {data/surfers__flow__n_128__re_250/surfer__vs_1o0__surftimeconst_2o0__flow_velocity_sampled_pdfs.csv};
			%\addlegendentry{\NameSurf}
	\end{groupplot}
\end{tikzpicture}

	% \caption{
		% Probability density function of the vertical flow velocity sampled along trajectories of passives particles, bottom-heavy swimmers and surfers.
		% Parameters: $\SwimmingVelocity = \KolmogorovVelocityScale$ and $\TimeHorizon_{\mathrm{\NameSurf}} = 5 \KolmogorovTimeScale \approx \TimeHorizonOpt$
	% }
	% \label{fig:surfing_velocity_sampled}
% \end{figure}
% where we show the distribution of the vertical velocity component of the turbulent flow sampled by surfers, bottom-heavy swimmers, and passive particles.
% One can see that the Gaussian distribution is not centered on zero but shifted toward positive values in the case of surfers.
% 
% As a consequence, the mean vertical velocity $\overline{v_f}$ sampled/interpolated along larva trajectories would be another nice interesting metric to assess if larvae are able to exploit the flow.

\subsection{Data renormalization and surfing parameter $\TimeHorizon$}

I did a quick renormalization test (Fig. \ref{fig:renormalization}) here to see how well the data renormalizes to the Kolmogorov time scale $\KolmogorovTimeScale$ (values in table \ref{tab:turbulence_data}).
Might be cleaner if the renormalization was done before binning the data?
I also did a quick fit to the surfing strategy. 
It would correspond to a surfing strategy of parameter $\TimeHorizon \approx 0.7 \KolmogorovTimeScale$.
\begin{table}
	\center
	\begin{tabular}{ l l l l }
		\toprule
		Turbulence level & Low & Medium & High \\
		 %& (mm) & (mm\,s$^{-1}$) & (s$^{-1}$) & (s$^{-1}$) \\[5pt]
		\midrule
		$\KolmogorovTimeScale$ (s) & 2.0 & 1.1 & 0.67 \\
		$\KolmogorovVelocityScale$ (cm/s) & 0.07 & 0.10 & 0.12 \\
		\bottomrule
	\end{tabular}
	\caption{
		Kolmogorov time and velocity scales.
	}
	\label{tab:turbulence_data}
\end{table}
Note that larvae swimming velocity of the order of the Kolmogorov velocity scale or smaller for all turbulence levels considered ($\SwimmingVelocity \lesssim \KolmogorovVelocityScale$).
This corresponds to the range of swimming velocity for which surfing is the most beneficial and the expected optimal parameter $\TimeHorizonOpt$ is the largest ($\TimeHorizonOpt \gtrsim 3 \KolmogorovTimeScale$. Figure \ref{fig:surfing_parameter_tau_larvae} shows an optimum for $\TimeHorizonOpt \approx 3 \KolmogorovTimeScale$ for $\mathit{Re}_{\lambda} = 11$ while the surfing on turbulence paper shows an optimum for $\TimeHorizonOpt \approx 4 \KolmogorovTimeScale$ for $\mathit{Re}_{\lambda} = 418$ so there is a small Reynolds dependance).
\begin{figure}%[H]
	\centering
	\begin{tikzpicture}
	\begin{groupplot}[
			group style={
				group size=2 by 1,
				y descriptions at=edge left,
				%x descriptions at=edge bottom,
				horizontal sep=0.04\linewidth,
				%vertical sep=0.02\linewidth,
			},
			% size
			width=0.55\textwidth,
			height=0.45\textwidth,
			% more
			axis on top,
			set layers,
			% x
			xlabel=$\FlowVorticityScalar \, \KolmogorovTimeScale$,
			xmin=-1,
			xmax=1,
			% y
			ylabel={$\theta$},
			ymin=-pi/4,
			ymax=pi/4,
			ytick={-pi/4, -pi/8, 0, pi/8, pi/4},
			yticklabels={$-\pi/4$, $-\pi/8$, 0, $\pi/8$, $\pi/4$},
			% legend
			legend style={draw=none, fill=none, at={(1.0, -0.25)}, anchor=north},
			%legend style={draw=none, fill=none},
			%legend pos=south west,
			legend cell align=left,
			legend columns=-1,
		]
	\nextgroupplot[
		title={Early}
	]
		\addplot
		[
		color=ColorTurbLow,
		opacity=1.0,
		mark=*,
		%only marks,
		] table[
			x expr={\thisrowno{0} * 2},
			y expr={\thisrowno{1}},
			col sep=comma,
			comment chars=\#,
		]{chap_end/data/early_low.csv};
		\addlegendentry{low}
		\addplot
		[
		color=ColorTurbMedium,
		opacity=1.0,
		mark=triangle*,
		%only marks,
		] table[
			x expr={\thisrowno{0} * 1.1},
			y expr={\thisrowno{1}},
			col sep=comma,
			comment chars=\#,
		]{chap_end/data/early_medium.csv};
		\addlegendentry{medium}
		\addplot
		[
		color=ColorTurbHigh,
		opacity=1.0,
		mark=square*,
		%only marks,
		] table[
			x expr={\thisrowno{0} * 0.67},
			y expr={\thisrowno{1}},
			col sep=comma,
			comment chars=\#,
		]{chap_end/data/early_high.csv};
		\addlegendentry{high}
		% surf
		\addplot
		[
			color=ColorSurf,
			opacity=1.0,
			solid, 
			domain=-1:1,
			samples=10,
			mark=pentagon,
		]{-x * 0.7 / 2.0};
		\addlegendentry{surf, $\TimeHorizon = 0.7 \KolmogorovTimeScale$}
	\nextgroupplot[
		title={Late}
	]
		\addplot
		[
		color=ColorTurbLow,
		opacity=1.0,
		mark=*,
		%only marks,
		] table[
			x expr={\thisrowno{0} * 2},
			y expr={\thisrowno{1}},
			col sep=comma,
			comment chars=\#,
		]{chap_end/data/late_low.csv};
		%\addlegendentry{low}
		\addplot
		[
		color=ColorTurbMedium,
		opacity=1.0,
		mark=triangle*,
		%only marks,
		] table[
			x expr={\thisrowno{0} * 1.1},
			y expr={\thisrowno{1}},
			col sep=comma,
			comment chars=\#,
		]{chap_end/data/late_medium.csv};
		%\addlegendentry{medium}
		\addplot
		[
		color=ColorTurbHigh,
		opacity=1.0,
		mark=square*,
		%only marks,
		] table[
			x expr={\thisrowno{0} * 0.67},
			y expr={\thisrowno{1}},
			col sep=comma,
			comment chars=\#,
		]{chap_end/data/late_high.csv};
		%\addlegendentry{high}
		% surf
		\addplot
		[
			color=ColorSurf,
			opacity=1.0,
			solid, 
			domain=-1:1,
			samples=10,
			mark=pentagon,
		]{-x * 0.7 / 2.0};
		%\addlegendentry{surf, $\TimeHorizon = 0.7 \KolmogorovTimeScale$}
	\end{groupplot}
\end{tikzpicture}

	\caption{
		Quick renormalization test.
	}
	\label{fig:renormalization}
\end{figure}
Note how small the fitted surfing parameter $\TimeHorizon = 0.7 \KolmogorovTimeScale$ on larvae data is small compared to the expected optimal value $\TimeHorizonOpt \gtrsim 3 \KolmogorovTimeScale$.
\begin{figure}%[H]
	\centering
	\begin{tikzpicture}
    % gain as a function of the free parameter $\tau$
    \begin{axis} [
        axis on top,
        % size
        width=0.66\textwidth,
        height=0.62\textwidth,
        % y
        ymin=0,
        ymax=2,
        ylabel={$\left\langle \Performance \right\rangle / \SwimmingVelocity$},
        y label style={yshift=-4pt},
        %extra y ticks={0.5, 1.5, 2.5},
        % x
        xlabel=$\TimeHorizon / \KolmogorovTimeScale$,
        x label style={yshift=4pt},
        xmin=0,
        xmax=8,
        % layers
        set layers,
        % legend
        legend style={
        	draw=none, 
        	fill=none, 
        	%/tikz/every even column/.append style={column sep=8pt},
        	xshift=30pt,
        },
        legend pos=south west,
        legend cell align=left,
        legend columns=4,
    ]
        \node[anchor=south west, yshift=7pt] at (rel axis cs:0.0,0.0) {$\SwimmingVelocity$ :};
        %% us 0.5
        %%% 95 CI
        \addplot[name path=A, draw=none, forget plot] table [
            x index=4,
            y expr={(\thisrowno{1} - \thisrowno{2}) / (\thisrowno{3} * 0.21)}, %u_\eta = 0.21
            col sep=comma, 
            comment chars=\#,
            restrict expr to domain={\thisrowno{3}}{0.5:0.5},
            unbounded coords=discard,
        ]{data/surfers__flow__n_128__re_250/surfer__merge_average_velocity_axis_0.csv};
        \addplot[name path=B, draw=none, forget plot] table [
            x index=4, 
            y expr={(\thisrowno{1} + \thisrowno{2}) / (\thisrowno{3} * 0.21)}, %u_\eta = 0.21
            col sep=comma,
            comment chars=\#,
            restrict expr to domain={\thisrowno{3}}{0.5:0.5},
            unbounded coords=discard,
        ]{data/surfers__flow__n_128__re_250/surfer__merge_average_velocity_axis_0.csv};
        \addplot[ColorSurf!100!ColorVs, opacity=0.25, forget plot, on layer=axis background] fill between[of=A and B];
        %%% average
        \addplot
        [
        color=ColorSurf!100!ColorVs,
        opacity=1.0,
        only marks,%solid
        mark=triangle*
        ]
        table[
            x index=4, 
            y expr={\thisrowno{1} / (\thisrowno{3} * 0.21)}, %u_\eta = 0.21
            col sep=comma, 
            comment chars=\#,
            restrict expr to domain={\thisrowno{3}}{0.5:0.5},
            unbounded coords=discard,
        ]{data/surfers__flow__n_128__re_250/surfer__merge_average_velocity_axis_0.csv};
        \addlegendentry{$\KolmogorovVelocityScale/2$}
        %%% fit
        \addplot
        [
        color=ColorSurf!100!ColorVs,
        opacity=1.0,
        solid,
        forget plot
        ]
        table[
            x index=0, 
            y expr={\thisrowno{3} / (0.5 * 0.21)}, %u_\eta = 0.21
            col sep=comma, 
            comment chars=\#,
            unbounded coords=discard,
        ]{data/surfers__flow__n_128__re_250/surfer__fits_average_velocity_axis_0.csv};
        %% us 1.0
        %%% 95 CI
        \addplot[name path=A, draw=none, forget plot] table [
            x index=4,
            y expr={(\thisrowno{1} - \thisrowno{2}) / (\thisrowno{3} * 0.21)}, %u_\eta = 0.21
            col sep=comma, 
            comment chars=\#,
            restrict expr to domain={\thisrowno{3}}{1.0:1.0},
            unbounded coords=discard,
        ]{data/surfers__flow__n_128__re_250/surfer__merge_average_velocity_axis_0.csv};
        \addplot[name path=B, draw=none, forget plot] table [
            x index=4, 
            y expr={(\thisrowno{1} + \thisrowno{2}) / (\thisrowno{3} * 0.21)}, %u_\eta = 0.21
            col sep=comma,
            comment chars=\#,
            restrict expr to domain={\thisrowno{3}}{1.0:1.0},
            unbounded coords=discard,
        ]{data/surfers__flow__n_128__re_250/surfer__merge_average_velocity_axis_0.csv};
        \addplot[ColorSurf!66!ColorVs, opacity=0.25, forget plot, on layer=axis background] fill between[of=A and B];
        %%% average
        \addplot
        [
        color=ColorSurf!66!ColorVs,
        opacity=1.0,
        only marks,%solid
        mark=square
        ]
        table[
            x index=4, 
            y expr={\thisrowno{1} / (\thisrowno{3} * 0.21)}, %u_\eta = 0.21
            col sep=comma, 
            comment chars=\#,
            restrict expr to domain={\thisrowno{3}}{1.0:1.0},
            unbounded coords=discard,
        ]{data/surfers__flow__n_128__re_250/surfer__merge_average_velocity_axis_0.csv};
        \addlegendentry{$\KolmogorovVelocityScale$}
        %%% fit
        \addplot
        [
        color=ColorSurf!66!ColorVs,
        opacity=1.0,
        solid,
        forget plot
        ]
        table[
            x index=0, 
            y expr={\thisrowno{1} / (1.0 * 0.21)}, %u_\eta = 0.21
            col sep=comma, 
            comment chars=\#,
            unbounded coords=discard,
        ]{data/surfers__flow__n_128__re_250/surfer__fits_average_velocity_axis_0.csv};
        %%%% model
        %\addplot
        %[
        %color=colortss!33!colorus,
        %opacity=1.0,
        %dashed,
        %forget plot
        %]
        %table[
        %    x index=0, 
        %    y expr={cos(deg(0.24 * (4.4 - \thisrowno{0}))) / cos(deg(0.24 * 4.4))}, %u_\eta = 0.21
        %    %y expr={cos(deg(\thisrowno{0}))}, %u_\eta = 0.21
        %    col sep=comma, 
        %    comment chars=\#,
        %    unbounded coords=discard,
        %]{data/jhtdb_more/fits_average_velocity_axis_0__agent.csv};
        %% us 4.0
        %%% 95 CI
        \addplot[name path=A, draw=none, forget plot] table [
            x index=4, 
            y expr={(\thisrowno{1} - \thisrowno{2}) / (\thisrowno{3} * 0.21)}, %u_\eta = 0.21
            col sep=comma, 
            comment chars=\#,
            restrict expr to domain={\thisrowno{3}}{4.0:4.0},
            unbounded coords=discard,
        ]{data/surfers__flow__n_128__re_250/surfer__merge_average_velocity_axis_0.csv};
        \addplot[name path=B, draw=none, forget plot] table [
            x index=4, 
            y expr={(\thisrowno{1} + \thisrowno{2}) / (\thisrowno{3} * 0.21)}, %u_\eta = 0.21
            col sep=comma, 
            comment chars=\#,
            restrict expr to domain={\thisrowno{3}}{4.0:4.0},
            unbounded coords=discard,
        ]{data/surfers__flow__n_128__re_250/surfer__merge_average_velocity_axis_0.csv};
        \addplot[ColorSurf!33!ColorVs, opacity=0.25, forget plot, on layer=axis background] fill between[of=A and B];
        %%% average
        \addplot
        [
        color=ColorSurf!33!ColorVs,
        opacity=1.0,
        only marks,%solid
        mark=pentagon*
        ]
        table[
            x index=4, 
            y expr={\thisrowno{1} / (\thisrowno{3} * 0.21)}, %u_\eta = 0.21
            col sep=comma, 
            comment chars=\#,
            restrict expr to domain={\thisrowno{3}}{4.0:4.0},
            unbounded coords=discard,
        ]{data/surfers__flow__n_128__re_250/surfer__merge_average_velocity_axis_0.csv};
        \addlegendentry{$4 \KolmogorovVelocityScale$}
        %%% fit
        \addplot
        [
        color=ColorSurf!33!ColorVs,
        opacity=1.0,
        solid,
        forget plot
        ]
        table[
            x index=0, 
            y expr={\thisrowno{2} / (4.0 * 0.21)}, %u_\eta = 0.21
            col sep=comma, 
            comment chars=\#,
            unbounded coords=discard,
        ]{data/surfers__flow__n_128__re_250/surfer__fits_average_velocity_axis_0.csv};
        %%%% model
        %\addplot
        %[
        %color=colortss!66!colorus,
        %opacity=1.0,
        %dashed,
        %forget plot
        %]
        %table[
        %    x index=0, 
        %    y expr={cos(deg(0.24 * (3.37 - \thisrowno{0}))) / cos(deg(0.24 * 3.37))}, %u_\eta = 0.21
        %    %y expr={cos(deg(\thisrowno{0}))}, %u_\eta = 0.21
        %    col sep=comma, 
        %    comment chars=\#,
        %    unbounded coords=discard,
        %]{data/jhtdb_more/fits_average_velocity_axis_0__agent.csv};
        %% us 8.0
        %%% 95 CI
        \addplot[name path=A, draw=none, forget plot] table [
            x index=4, 
            y expr={(\thisrowno{1} - \thisrowno{2}) / (\thisrowno{3} * 0.21)}, %u_\eta = 0.21
            col sep=comma, 
            comment chars=\#,
            restrict expr to domain={\thisrowno{3}}{8.0:8.0},
            unbounded coords=discard,
        ]{data/surfers__flow__n_128__re_250/surfer__merge_average_velocity_axis_0.csv};
        \addplot[name path=B, draw=none, forget plot] table [
            x index=4, 
            y expr={(\thisrowno{1} + \thisrowno{2}) / (\thisrowno{3} * 0.21)}, %u_\eta = 0.21
            col sep=comma, 
            comment chars=\#,
            restrict expr to domain={\thisrowno{3}}{8.0:8.0},
            unbounded coords=discard,
        ]{data/surfers__flow__n_128__re_250/surfer__merge_average_velocity_axis_0.csv};
        \addplot[ColorSurf!0!ColorVs, opacity=0.25, forget plot, on layer=axis background] fill between[of=A and B];
        %%% average
        \addplot
        [
        color=ColorSurf!0!ColorVs,
        opacity=1.0,
        only marks,%solid
        mark=o
        ]
        table[
            x index=4, 
            y expr={\thisrowno{1} / (\thisrowno{3} * 0.21)}, %u_\eta = 0.21
            col sep=comma, 
            comment chars=\#,
            restrict expr to domain={\thisrowno{3}}{8.0:8.0},
            unbounded coords=discard,
        ]{data/surfers__flow__n_128__re_250/surfer__merge_average_velocity_axis_0.csv};
        \addlegendentry{$8 \KolmogorovVelocityScale$}
        %%% fit
        \addplot
        [
        color=ColorSurf!0!ColorVs,
        opacity=1.0,
        solid,
        forget plot
        ]
        table[
            x index=0, 
            y expr={\thisrowno{4} / (8.0 * 0.21)}, %u_\eta = 0.21
            col sep=comma, 
            comment chars=\#,
            unbounded coords=discard,
        ]{data/surfers__flow__n_128__re_250/surfer__fits_average_velocity_axis_0.csv};
        %%%% model
        %\addplot
        %[
        %color=colortss!100!colorus,
        %opacity=1.0,
        %dashed,
        %forget plot
        %]
        %table[
        %    x index=0, 
        %    y expr={cos(deg(0.24 * (2.57 - \thisrowno{0}))) / cos(deg(0.24 * 2.57))}, %u_\eta = 0.21
        %    %y expr={cos(deg(\thisrowno{0}))}, %u_\eta = 0.21
        %    col sep=comma, 
        %    comment chars=\#,
        %    unbounded coords=discard,
        %]{data/jhtdb_more/fits_average_velocity_axis_0__agent.csv};
        %% y = x
        \addplot
        [
        color=gray!50!white,
        opacity=1.0,
        %line width=1pt, 
        solid, 
        on layer=axis background,
        domain=0:10,
        ]{1};
    \end{axis}
\end{tikzpicture}

	\caption{
		Surfing performance as a function of the surfing parameter $\TimeHorizon$ for various swimming speeds $\SwimmingVelocity$. $\mathit{Re}_{\lambda} = 11$.
	}
	\label{fig:surfing_parameter_tau_larvae}
\end{figure}
The surfing strategy has been derived and evaluated without considering the energetic cost of active orientation.
As a higher value of $\TimeHorizon$ would induce a stronger rotation against vorticity, we might expect higher values of $\TimeHorizon$ to be more energy consuming if the energetic rotation cost is considered.
This could be one way to explain why the fitted value of $\TimeHorizon$ is so small compared to the surfing optimum (if this value of $\TimeHorizon = 0.7 \KolmogorovTimeScale$ corresponds to an energetic optimum).
Another simple way to explain this difference would be a limitation of the active swimming torque (which seems obvious but maybe hard to evaluate).
